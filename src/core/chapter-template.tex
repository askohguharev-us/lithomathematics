% ==================================================================
% Chapter Template (Archetype-Invariant Fractal — Annals Standard)
% Depends on preamble providing: amsmath, amsthm, amssymb, mathtools, hyperref, cleveref
% If your preamble does not define theorem-like environments, see the example file.
% ==================================================================

\chapter{<Chapter Title>}
\label{chap:<label>}

% -------- Local micro-macros for archetype labels (safe if preamble lacks them)
\providecommand{\Orientation}{\par\noindent\textbf{Orientation.}\ }
\providecommand{\Objectives}{\par\noindent\textbf{Objectives.}\ }
\providecommand{\Invariants}{\par\noindent\textbf{Invariants/Assumptions.}\ }
\providecommand{\Statement}{\par\noindent\textbf{Statement.}\ }
\providecommand{\ErrorBounds}{\par\noindent\textbf{Error bounds.}\ }
\providecommand{\Optimality}{\par\noindent\textbf{Optimality.}\ }
\providecommand{\Audit}{\par\noindent\textbf{Audit/Dependencies.}\ }
\providecommand{\Closure}{\par\noindent\textbf{Closure.}\ }

% ---------------- Orientation Block ----------------
\section*{Orientation}
We fix the standing assumptions (SA.1--SA.4) from \Cref{sec:definitions}
and adopt the global notation therein. This chapter develops \emph{<main idea>},
building upon the geometric and spectral foundations established in the introduction.
All claims are scoped by the regularity and transversality hypotheses stated globally.

% ---------------- Foundational Objects ----------------
\section{Foundational Objects}

\begin{definition}[<Name of Object>]\label{def:<label>}
\Orientation Context and role of the object within the chapter and the global program.
\Objectives What is being defined and why this representation is chosen.
\Invariants Fixed parameters and regularity (e.g., $C^2$, positive reach, transversality).
\Statement Formal definition in concise, unambiguous terms (no theorems here).
\Audit Cross-links to \Cref{sec:definitions} and any earlier objects this depends on.
\Closure How this object will be used in lemmas/theorems below.
\end{definition}

\begin{definition}[<Optional Auxiliary Object>]\label{def:<aux-label>}
\Orientation Minimal context.
\Statement Formal definition.
\Audit Compatibility with Standing Assumptions.
\Closure Reason this is separated as a standalone object.
\end{definition}

% ---------------- Auxiliary Results ----------------
\section{Auxiliary Results}

\begin{lemma}[<Key Lemma>]\label{lem:<label>}
\Orientation Setup and location in the dependency graph.
\Objectives Precise claim needed for the main results.
\Invariants All assumptions explicitly re-stated or referenced.
\Statement Formal lemma statement (no proof sketch language inside the statement).
\ErrorBounds State constants and their dependence (dimension, reach, curvature).
\Audit References to definitions and prior results in this chapter.
\Closure How this lemma will be consumed in the main theorem(s).
\end{lemma}

\begin{proof}
Provide a complete proof with clear step structure, explicitly tracking
all constants and indicating where global hypotheses are used.
Avoid hidden appeals; cite precisely (definition numbers, lemmas, external refs).
\end{proof}

% ---------------- Main Results ----------------
\section{Main Results}

\begin{theorem}[<Main Theorem>]\label{thm:<label>}
\Orientation Exact scope (domain, regularity, boundary conditions).
\Objectives Central claim of the chapter.
\Invariants State assumptions compactly; reference global SA.* if used.
\Statement Formal statement with sharp exponents and explicit constants.
\ErrorBounds Remainder terms with quantified dependence (dimension, $\kappa$, reach).
\Optimality If known, record optimality or counterexamples; otherwise state as open.
\Audit Dependencies: \Cref{def:<label>,lem:<label>,def:<aux-label>}, etc.
\Closure Why this closes the chapter’s objective and how it plugs into the monograph.
\end{theorem}

\begin{proof}
Structured proof:
\begin{enumerate}
  \item \textit{Localization/parametrix.} …
  \item \textit{Uniform estimates and gluing.} …
  \item \textit{Remainder control and scale law.} …
\end{enumerate}
Track all constants and their provenance; ensure scale and unit consistency.
\end{proof}

% ---------------- Corollaries and Consequences ----------------
\section{Corollaries and Consequences}

\begin{corollary}[<Immediate Consequence>]\label{cor:<label>}
\Orientation Derived directly from \Cref{thm:<label>}.
\Statement Concise corollary statement.
\Audit Explicit reference to the theorem and any auxiliary lemma.
\Closure Note how this will be used in the next chapter.
\end{corollary}

% ---------------- Closing Invariant ----------------
\section*{Closing Invariant}
This chapter established <summary of results> under the global standing assumptions,
bridging \Cref{chap:<previous>} to \Cref{chap:<next>}. All definitions and constants
are harmonized with \Cref{sec:definitions}, ensuring scale-invariant statements and
consistent dependence on geometric complexity and reach.
