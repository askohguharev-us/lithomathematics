%==============================================================================
% macros.tex — project-wide macros for "lithomathematics" monograph
% Compatible with standard LaTeX toolchains and conservative journal styles.
%==============================================================================

% --- Core math packages (lightweight, widely supported) ----------------------
\usepackage{amsmath,amssymb,amsthm,mathtools}
\usepackage{bm}            % bold math symbols
\usepackage{enumitem}      % custom enumerate labels
\usepackage{xparse}        % flexible command args (standard in modern TeX)

% --- Safe spacing & micro-typography (optional; harmless if class disables) --
\providecommand{\microtypesetup}[1]{} % no-op if microtype not loaded

%------------------------------------------------------------------------------
% 0) Global toggles (style bridge Annals <-> internal)
%------------------------------------------------------------------------------
\newif\ifannalsstyle
\annalsstyletrue % keep true for Annals-like conservative look

% Heading style toggles (edit here to change look across the book)
\ifannalsstyle
  \newcommand{\HeadIt}[1]{\textit{#1}}
  \newcommand{\HeadSc}[1]{\textsc{#1}}
  \newcommand{\HeadBf}[1]{\textbf{#1}}
\else
  \newcommand{\HeadIt}[1]{\textit{#1}}
  \newcommand{\HeadSc}[1]{\textsc{#1}}
  \newcommand{\HeadBf}[1]{\textbf{#1}}
\fi

%------------------------------------------------------------------------------
% 1) Fractal–Annals Bridge headings (one-line inline labels)
%    Use in chapters as: \Orientation ... , \Objectives ... , etc.
%------------------------------------------------------------------------------
\newcommand{\Orientation}{\HeadIt{Overview.} }
\newcommand{\Objectives}{\HeadSc{Objectives.} }
\newcommand{\Invariants}{\HeadSc{Notation and standing assumptions.} }
\newcommand{\ErrorBounds}{\HeadBf{Error bounds.} }
\newcommand{\Optimality}{\HeadBf{Optimality of exponents.} }
\newcommand{\Verification}{\HeadSc{Verification checklist.} }
\newcommand{\Concluding}{\HeadIt{Concluding remarks.} }
\newcommand{\Literature}{\HeadIt{Position in the literature.} }

%------------------------------------------------------------------------------
% 2) Theorem-like environments
%------------------------------------------------------------------------------
\numberwithin{equation}{section}

\theoremstyle{plain}
\newtheorem{theorem}{Theorem}[section]
\newtheorem{lemma}[theorem]{Lemma}
\newtheorem{proposition}[theorem]{Proposition}
\newtheorem{corollary}[theorem]{Corollary}

\theoremstyle{definition}
\newtheorem{definition}[theorem]{Definition}
\newtheorem{assumption}[theorem]{Assumption}
\newtheorem{hypothesis}[theorem]{Hypothesis}
\newtheorem{example}[theorem]{Example}

\theoremstyle{remark}
\newtheorem{remark}[theorem]{Remark}
\newtheorem{noteenv}[theorem]{Note}

% Short aliases if needed
\newcommand{\Def}{\begin{definition}}
\newcommand{\EndDef}{\end{definition}}
\newcommand{\Thm}{\begin{theorem}}
\newcommand{\EndThm}{\end{theorem}}

%------------------------------------------------------------------------------
% 3) Paired delimiters and common math objects
%------------------------------------------------------------------------------
\DeclarePairedDelimiter{\abs}{\lvert}{\rvert}
\DeclarePairedDelimiter{\norm}{\lVert}{\rVert}
\DeclarePairedDelimiter{\ip}{\langle}{\rangle}
\DeclarePairedDelimiter{\braces}{\lbrace}{\rbrace}
\DeclarePairedDelimiter{\parens}{(}{)}
\DeclarePairedDelimiter{\angles}{\langle}{\rangle}

% Blackboard-bold sets
\newcommand{\R}{\mathbb{R}}
\newcommand{\C}{\mathbb{C}}
\newcommand{\N}{\mathbb{N}}
\newcommand{\Z}{\mathbb{Z}}
\newcommand{\Q}{\mathbb{Q}}
\newcommand{\T}{\mathbb{T}}
\newcommand{\bbS}{\mathbb{S}}

% Calligraphic / script
\newcommand{\calH}{\mathcal{H}}
\newcommand{\calL}{\mathcal{L}}
\newcommand{\calO}{\mathcal{O}}
\newcommand{\calE}{\mathcal{E}}
\newcommand{\calD}{\mathcal{D}}
\newcommand{\calA}{\mathcal{A}}
\newcommand{\calU}{\mathcal{U}}
\newcommand{\calV}{\mathcal{V}}

% Geometry and measures
\newcommand{\Hd}{\mathcal{H}}          % Hausdorff measure
\newcommand{\vol}{\mathrm{vol}}        % volume
\newcommand{\area}{\mathrm{area}}      % area
\newcommand{\length}{\mathrm{length}}  % length

% Differential / operators
\newcommand{\dd}{\,\mathrm{d}}
\newcommand{\grad}{\nabla}
\newcommand{\Lap}{\Delta}
\newcommand{\Lapg}{\Delta_g}
\newcommand{\Id}{\mathrm{Id}}

% Common operators (roman)
\DeclareMathOperator{\Tr}{Tr}
\DeclareMathOperator{\supp}{supp}
\DeclareMathOperator{\dist}{dist}
\DeclareMathOperator{\diam}{diam}
\DeclareMathOperator{\Ker}{Ker}
\DeclareMathOperator{\Ran}{Ran}
\DeclareMathOperator{\sgn}{sgn}
\DeclareMathOperator{\esssup}{ess\,sup}
\DeclareMathOperator{\essinf}{ess\,inf}

% Spectral-specific symbols
\newcommand{\Spec}{\mathrm{Spec}}
\newcommand{\Eproj}{E_\lambda}         % spectral projector
\newcommand{\HeatKer}{K}               % heat kernel

%------------------------------------------------------------------------------
% 4) Domain / fracture macros (consistent notation project-wide)
%------------------------------------------------------------------------------
\newcommand{\Domain}{\Omega}
\newcommand{\Boundary}{\partial\Omega}
\newcommand{\Fracture}{\Gamma}
\newcommand{\Metric}{g}
\newcommand{\II}{\mathrm{II}}          % second fundamental form on \Gamma
\newcommand{\kappag}{\kappa(\Gamma)}   % geometric complexity parameter
\newcommand{\KL}{K_{L}}                % litho-ratio
\newcommand{\KLstar}{K_{L}^{*}}        % ergodic limit of litho-ratio

%------------------------------------------------------------------------------
% 5) Asymptotic parameters and exponents
%------------------------------------------------------------------------------
\newcommand{\Tcut}{T}                  % spectral scale / cutoff
\newcommand{\mixrate}{\beta}           % exponential mixing rate
\newcommand{\thetaRP}{\theta}          % Ramanujan–Petersson bound (glossary clarifies)
\newcommand{\deltaPS}{\delta}          % power-saving exponent

% Canonical formula (used across chapters; keep identical everywhere)
\newcommand{\PowerSavingExponent}{%
  \deltaPS = \min\!\bigl(\tfrac{1}{2}-\thetaRP,\; \mixrate/4\bigr)%
}

%------------------------------------------------------------------------------
% 6) Structured lists for Goals / Invariants with stable labels
%------------------------------------------------------------------------------
\newlist{goals}{enumerate}{1}
\setlist[goals]{label=\textbf{G\arabic*}, leftmargin=2.3em, itemsep=0.25em}

\newlist{invariants}{enumerate}{1}
\setlist[invariants]{label=\textbf{I\arabic*}, leftmargin=2.3em, itemsep=0.25em}

\newlist{hypolist}{enumerate}{1}
\setlist[hypolist]{label=\textbf{H\arabic*}, leftmargin=2.3em, itemsep=0.25em}

%------------------------------------------------------------------------------
% 7) Safe referencing helpers (work even if cleveref is not loaded)
%------------------------------------------------------------------------------
\providecommand{\cref}[1]{\ref{#1}}
\providecommand{\Cref}[1]{\ref{#1}}

%------------------------------------------------------------------------------
% 8) Small utilities
%------------------------------------------------------------------------------
\newcommand{\ie}{i.e.\ }
\newcommand{\eg}{e.g.\ }
\newcommand{\as}{a.s.\ }
\newcommand{\wrt}{w.r.t.\ }

% Inline notes (kept silent by default; comment to show)
\newcommand{\todo}[1]{\ignorespaces} % make no-ops for submission

%==============================================================================
% End of macros.tex
%==============================================================================
