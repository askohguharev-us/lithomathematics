% ==================================================================
% 00-diamond-protocol.tex — Diamond Audit Standard v3.0
% This file establishes the quality protocol for the monograph.
% Pure English; Annals-level rigor; no metaphysical content.
% Loaded by main.tex after packages and macros.
% ==================================================================

\chapter*{Diamond Audit Protocol (v3.0)}
\addcontentsline{toc}{chapter}{Diamond Audit Protocol (v3.0)}

\noindent
\textbf{Purpose.} This protocol formalizes the quality, rigor, and formatting requirements for every part of the monograph. It implements three compulsory gates per unit (chapter or appendix): \emph{Spectral Closure}, \emph{Horizon Expansion}, and \emph{Diamond Recap}. Compliance is verified via an explicit checklist and cross-reference audit.

\medskip

\noindent
\textbf{Scope.} The protocol applies to:
\begin{itemize}
  \item Frontmatter (\emph{Executive Summary}, \emph{Reader's Roadmap}, \emph{Notation \& Glossary});
  \item Chapters 1–10 (core results);
  \item Appendices A–D (technical complements).
\end{itemize}

% ---------------------------------------------------------------
\section*{0. Global Principles}
\addcontentsline{toc}{section}{0. Global Principles}

\begin{enumerate}[label=\textbf{G\arabic*}.]
  \item \textbf{Annals-grade precision.} Every statement (definition, lemma, theorem, corollary) is fully quantified, with explicit hypotheses and constant dependencies.
  \item \textbf{No ambiguity.} All symbols are defined before use; no overloading unless explicitly stated in the \emph{Notation \& Glossary}.
  \item \textbf{Spectral completeness.} Each section must close all introduced estimates by stating remainder orders and parameter regimes.
  \item \textbf{Bibliographic integrity.} All external claims are cited in-text; the bibliography is complete and up to date.
  \item \textbf{Non-applied firewall.} No keys to costly applications and no pointers to unresolved conjectures beyond strictly mathematical discussion.
\end{enumerate}

% ---------------------------------------------------------------
\section*{1. Numbering, Environments, and Cross-References}
\addcontentsline{toc}{section}{1. Numbering, Environments, and Cross-References}

\paragraph{Numbering.}
Equations are numbered by section; theorems and lemmas by chapter. Figures and tables are numbered by chapter.

\paragraph{Environments.}
The following theorem-style environments are used throughout (defined in \texttt{macros/environments.tex}):
\begin{center}
\verb|\newtheorem{theorem}{Theorem}[chapter]|, \quad
\verb|\newtheorem{lemma}[theorem]{Lemma}|, \quad
\verb|\newtheorem{proposition}[theorem]{Proposition}|,\\
\verb|\newtheorem{corollary}[theorem]{Corollary}|, \quad
\verb|\theoremstyle{definition} \newtheorem{definition}[theorem]{Definition}|,\\
\verb|\theoremstyle{remark} \newtheorem{remark}[theorem]{Remark}|.
\end{center}

\paragraph{Labels.}
Every numbered item (equations, theorems, lemmas, figures) must carry a unique label:
\[
\texttt{\textbackslash label\{thm:ch6-ergodic\}},\quad
\texttt{\textbackslash label\{eq:3.14-local-trace\}},\quad
\texttt{\textbackslash label\{fig:5-spectrum\}}.
\]
Cross-referencing uses \verb|\cref|/\verb|\Cref| (cleveref package) and never hard-coded numbers.

% ---------------------------------------------------------------
\section*{2. Assumptions, Constants, and Remainders}
\addcontentsline{toc}{section}{2. Assumptions, Constants, and Remainders}

\paragraph{Assumption Registry.}
Each chapter begins with an \emph{Assumption Registry} environment enumerating standing hypotheses (regularity, compactness, rectifiability, mixing rates, etc.) with stable identifiers:
\[
\texttt{(H1) Geometric regularity},\quad \texttt{(H2) Fracture control},\quad \texttt{(H3) Mixing},\ \ldots
\]

\paragraph{Constants.}
For main results, state explicit dependency lists:
\[
C = C\!\left(\mathrm{Vol}(\Omega), \|\mathrm{Sec}\|_{L^\infty}, \|V\|_{L^\infty}, \alpha, \beta\right).
\]
Avoid opaque phrases (“universal constant”) unless independence is mathematically proved.

\paragraph{Remainders.}
Remainder terms must specify scale, regime, and order (e.g., $O(\lambda^{-\delta})$, $O(e^{-cT_0})$), with the parameter window ($\eta \ge \lambda^{-\theta}$, $0<\theta<\theta_0$) stated adjacent to the claim.

% ---------------------------------------------------------------
\section*{3. Literature and Priority}
\addcontentsline{toc}{section}{3. Literature and Priority}

Each chapter closes with a brief \emph{Literature Liaison} paragraph:
\begin{itemize}
  \item What the chapter proves beyond prior art (e.g., Bourdin–Francfort–Marigo on phase-field fracture; Dell’Antonio et al. on singular domains; Dal Maso/Braides on $\Gamma$-convergence).
  \item How constants/rates compare (polynomial vs.\ exponential decay, explicit dependencies).
\end{itemize}
Citations must appear in-text (author-year or numeric per bibliography style) and in \texttt{bib/references.bib}.

% ---------------------------------------------------------------
\section*{4. Figures, Data, and Reproducibility}
\addcontentsline{toc}{section}{4. Figures, Data, and Reproducibility}

\paragraph{Figures.}
All figures are vector PDFs sourced from \texttt{figures/}. Captions must state the mathematical content (not merely visual description) and cite any external data.

\paragraph{Reproducibility.}
If synthetic examples are included, a minimal specification of parameters accompanies the statement; any optional scripts reside under \texttt{scripts/} and are not required for the proofs.

% ---------------------------------------------------------------
\section*{5. Audit Template and Commands}
\addcontentsline{toc}{section}{5. Audit Template and Commands}

To ensure uniformity, we provide a lightweight environment and commands. If the project already defines richer styles in \texttt{macros/house-style.tex}, these commands \emph{gracefully degrade} via \verb|\providecommand|/\verb|\provideenvironment|.

\medskip

\noindent\textbf{Template environment.}
\begin{verbatim}
% In 00-diamond-protocol.tex (safe fallbacks):
\providecommand{\DiamondSection}[1]{\section*{#1}}
\providecommand{\DiamondItem}[1]{\par\noindent\textbullet\ #1}

\provideenvironment{diamondaudit}{\par\medskip\noindent
  \DiamondSection{Diamond Audit (v3.0)}\par\smallskip
}{\par\medskip}

% Usage at the end of each chapter/appendix:
\begin{diamondaudit}
  \DiamondSection{Spectral Closure}
  % Restate key results + bounds + hypotheses.

  \DiamondSection{Horizon Expansion}
  % Safe extensions without keys to external problems.

  \DiamondSection{Diamond Recap}
  % Checklist (numbering; xrefs; citations; Annals-level precision).
\end{diamondaudit}
\end{verbatim}

\noindent
Editors may replace these fallbacks with styled versions in \texttt{macros/house-style.tex}. The \emph{content} requirements are mandatory regardless of styling.

% ---------------------------------------------------------------
\section*{6. Per-Unit Compliance Checklist}
\addcontentsline{toc}{section}{6. Per-Unit Compliance Checklist}

Before merging any chapter/appendix, the following must hold:

\begin{enumerate}[label=\textbf{C\arabic*}.]
  \item All new symbols appear in the local preface or \emph{Notation \& Glossary}.
  \item Every theorem/lemma/proposition states hypotheses and constant dependencies.
  \item All remainders specify regimes and orders; no hidden parameters.
  \item Cross-references compile (\verb|cleveref| works; no broken links).
  \item Literature liaison present and cites post-2000 works where relevant.
  \item \emph{Diamond Audit (v3.0)} appears verbatim at the end, with non-empty content.
\end{enumerate}

% ---------------------------------------------------------------
\section*{7. Versioning and ArXiv Policy}
\addcontentsline{toc}{section}{7. Versioning and ArXiv Policy}

\paragraph{Versioning.}
Each PDF build embeds the git commit hash and date (via CI). Submissions to arXiv must include the exact commit tag referenced in \texttt{README.md}.

\paragraph{ArXiv classes.}
Primary: \texttt{math.AP} or \texttt{math.SP}; secondary as appropriate. The abstract and metadata must match the \emph{Executive Summary} claims (no inflated statements).

% ---------------------------------------------------------------
\section*{8. Non-Regression and Spectral Integrity}
\addcontentsline{toc}{section}{8. Non-Regression and Spectral Integrity}

No change may weaken a previously proved bound without an explicit note in the unit’s \emph{Diamond Recap}. Any shift in hypotheses must be tracked in the \emph{Assumption Registry} and propagated to cross-references.

% ---------------------------------------------------------------
\section*{9. Minimal Boilerplate for Chapters}
\addcontentsline{toc}{section}{9. Minimal Boilerplate for Chapters}

Each chapter begins with:
\begin{enumerate}[label=\textbf{B\arabic*}.]
  \item A two-paragraph \emph{Chapter Overview} stating goals and outputs.
  \item An \emph{Assumption Registry} enumerated as (H1)–(Hk).
  \item A \emph{Result Map} listing theorems/lemmas and where proofs are located.
\end{enumerate}
Each chapter ends with the \emph{Diamond Audit (v3.0)} block.

% ---------------------------------------------------------------
\section*{10. Closing Statement}
\addcontentsline{toc}{section}{10. Closing Statement}

The Diamond Audit Protocol (v3.0) is binding for this repository. It encodes the spectral closure principle, ensures Annals-level rigor, and preserves the monograph’s internal coherence across chapters and appendices. Any deviation must be documented and justified within the local \emph{Diamond Recap}.

\bigskip
\noindent
\emph{This file is normative. The build system treats missing audit blocks as errors.}
