% ===================== PART 1/2 (append PART 2/2 directly after) =====================

% =====================================================================
% Lithomathematics — Foundational Definitions and Conventions (polished)
% =====================================================================

% Macros (safe if already defined elsewhere)
\providecommand{\vol}{\operatorname{vol}}
\providecommand{\Dom}{\operatorname{Dom}}
\providecommand{\Spec}{\operatorname{Spec}}
\providecommand{\Tr}{\operatorname{Tr}}

\chapter*{Foundational Definitions and Conventions}\label{sec:definitions}
\addcontentsline{toc}{chapter}{Foundational Definitions and Conventions}

\paragraph{Orientation and scope.}
By \emph{lithomathematics} I mean the spectral–geometric analysis of Laplace–type operators on Riemannian manifolds sliced along internal $C^2$ hypersurfaces (interfaces) $\Gamma$.
This section fixes the standing geometric and functional conventions used throughout; all subsequent results are formulated relative to these conventions.
Whenever a statement is called “universal”, its scope is \emph{explicitly} restricted to the local, smooth, interior setting specified below, understood microlocally and away from the singular set $\partial\Gamma\cup(\Gamma\cap\partial M)$.
All asymptotic expansions are understood in the \emph{classical} sense as $\tau\downarrow 0$; in non-smooth (cornered/edged) settings, additional (possibly logarithmic) terms can occur at intermediate half-integer orders, which we do not use below.

% --- Standing Assumptions ---
\paragraph{Standing Assumptions (SA).}
Throughout we fix:
\begin{itemize}
  \item[(SA.1)] $(M,g)$ is a compact, connected, $d$-dimensional Riemannian manifold
  with boundary $\partial M$ of class $C^2$ (piecewise $C^2$ with finitely many Lipschitz corners allowed, in the sense of~\cite{Grisvard1985}).
  \item[(SA.2)] $\Gamma \subset M$ is a compact, embedded $C^2$ hypersurface, possibly with boundary.
  \item[(SA.3)] If $\partial\Gamma \neq \varnothing$, then $\partial\Gamma \subset \partial M$ and the intersection is transverse:
  \[
  T_p\Gamma + T_p(\partial M) = T_p M, \qquad p\in \partial\Gamma.
  \]
  In particular, the inclusion $\partial\Gamma\subset\partial M$ is transversal; hence $\partial\Gamma$ is a $C^1$ submanifold of codimension $2$ in $M$. If $\Gamma$ is closed (without boundary), this assumption is void.
  \item[(SA.4)] The relative reach of $\Gamma$ in $M$ is positive, $\mathrm{reach}_M(\Gamma)>0$, i.e. there exists $r>0$ such that every $x\in M$ with $\mathrm{dist}_g(x,\Gamma)<r$ has a unique nearest point in $\Gamma$.
  This guarantees the existence of tubular neighborhoods and the well-posedness of the normal exponential map (cf.\ Remark~\ref{rem:relative-reach}).
  While (SA.1) alone suffices for trace continuity, (SA.4) is essential for defining curvature integrals in $\kappa(\Gamma)$ and for avoiding edge self-focalization near $\partial\Gamma$ (not used for trace continuity; see Remark~\ref{rem:trace-continuity}).
\end{itemize}

\begin{remark}[Relative reach and normal exponential map]\label{rem:relative-reach}
The quantity $\mathrm{reach}_M(\Gamma)$ is defined intrinsically with the Riemannian distance on $M$.
Equivalently, $\mathrm{reach}_M(\Gamma)>0$ if and only if the normal exponential map
$\exp^\perp:\{(p,\nu)\in N\Gamma:\ |\nu|<r\}\to M$ is injective for some $r>0$; in particular,
$\mathrm{reach}_M(\Gamma)\le \mathrm{inj}^\perp(\Gamma)$.
For closed $C^2$ hypersurfaces, positivity follows from tubular neighborhood theorems;
for hypersurfaces with boundary, positivity is assumed to prevent edge self-focalization near $\partial\Gamma$ (cf.\ Federer’s reach in Euclidean ambient spaces).
\end{remark}

\begin{remark}[Federer’s reach: references]\label{rem:reach-ref}
In Euclidean ambient spaces the notion of reach was introduced by Federer \cite[Ann.\ of Math.\ (1959), §4.18]{Federer1959}.
For Riemannian manifolds, local charts reduce the definitions to the Euclidean case on coordinate patches; positivity of reach implies existence and $C^{1,1}$ regularity of the nearest-point projection on a tubular neighborhood, hence injectivity of $\exp^\perp$ on a uniform radius.
\end{remark}

% --- Notation ---
\paragraph{Notation.}
We write $\vol_d(\cdot)$ for $d$-dimensional Riemannian volume and
$\vol_{d-1}(\cdot)$ for $(d-1)$–dimensional Hausdorff measure on hypersurfaces, both induced by $g$.
We set $R:=\mathrm{diam}_g(M)$.
For (possibly disconnected) $\Gamma$, $N_{\mathrm{conn}}(\Gamma)$ denotes its number of connected components.

% --- Dimensional Convention ---
\paragraph{Dimensional Convention.}
All dimensional checks are performed relative to the following fixed units:
\[
[{\rm Length}]=L,\quad [\vol_d]=L^d,\quad [{\rm Curvature}]=L^{-1},\quad [\Delta]=L^{-2},\quad [\tau]=L^2,\quad [t]=L.
\]
Here $\tau$ denotes the \emph{heat–time parameter} (diffusion scaling, consistent with $\partial_\tau u=\Delta u$),
while $t$ denotes the \emph{geometric time parameter} for the geodesic flow.
All dimensional consistency checks below are carried out with this convention; in particular, the heat coefficients in Definition~\ref{def:heat-expansion} are written in the diffusion scale $\tau$, while the mixing hypothesis in Hypothesis~\ref{hyp:mixing} is stated in the geometric time $t$ for the reflecting flow.

% ------------------------------------------------------------
\subsection{(FA) Functional–analytic foundation}

\begin{definition}[Trace spaces and litho–Laplacian]\label{def:litho-Laplacian}
Let $\gamma_{\partial M}$ and $\gamma_\Gamma$ denote the trace operators on $\partial M$ and $\Gamma$, respectively (well-defined and continuous under (SA.1); see Remark~\ref{rem:trace-continuity}).
Define the Dirichlet Sobolev subspace
\[
H^1_0(M;\,\partial M\cup\Gamma)
:= \{\, u\in H^1(M) \;:\; \gamma_{\partial M}(u)=0,\ \gamma_\Gamma(u)=0 \,\}.
\]
The \emph{litho–Laplacian} $L_\Gamma$ is the unique self–adjoint operator associated with the closed quadratic form
\[
Q[u] = \int_M |\nabla u|^2\,d\vol_g,\qquad \Dom(Q)=H^1_0(M;\,\partial M\cup\Gamma).
\]
In other words, $L_\Gamma$ acts as the negative Laplace–Beltrami operator in the distributional sense on the open manifold $M\setminus\Gamma$:
\[
\Dom(L_\Gamma)=\{\,u\in H^1_0(M;\,\partial M\cup\Gamma):\ \Delta_g u \in L^2(M\setminus\Gamma)\,\},\qquad
L_\Gamma u=-\Delta_g u \quad \text{on } M\setminus\Gamma,
\]
where $\Delta_g u$ is taken in the distributional sense on $M\setminus\Gamma$ and represented by an $L^2$ function there.
Equivalently, writing $M\setminus\Gamma=\bigsqcup_{j=1}^J U_j$ into connected components, the condition $\Delta_g u\in L^2(M\setminus\Gamma)$ means that the distributional Laplacian of $u$ on each $U_j$ is represented by a function in $L^2(U_j)$; no transmission conditions across $\Gamma$ are imposed beyond the Dirichlet trace $\gamma_\Gamma u=0$.
In particular, functions in $\Dom(L_\Gamma)$ vanish on $\Gamma$ in the trace sense, so $\Gamma$ acts as an internal Dirichlet boundary: no transmission conditions across $\Gamma$ occur.
\end{definition}

\begin{remark}[Dirichlet Laplacian on the sliced domain]\label{rem:Dirichlet-on-cut}
The space $H^1_0(M;\,\partial M\cup\Gamma)$ imposes Dirichlet conditions on both $\partial M$ and $\Gamma$ (in the trace sense).
Equivalently, $L_\Gamma$ is the Dirichlet Laplacian on the sliced domain $M\setminus\Gamma$ with zero boundary data on $\partial M\cup \Gamma$.
By Kato's representation theorem, the closed, densely defined, semibounded form $Q$ yields a unique self–adjoint operator $L_\Gamma$.
Moreover, $Q[u]\ge 0$ on its domain, hence $L_\Gamma$ is a \emph{positive} self–adjoint operator; in particular, its spectrum is contained in $[0,\infty)$ and $\Tr e^{-\tau L_\Gamma}$ is well-defined for $\tau>0$.
Consequently, $e^{-\tau L_\Gamma}$ is a trace-class contraction on $L^2(M)$ for every $\tau>0$.
(See also Remark~\ref{rem:components} for the orthogonal decomposition across connected components of $M\setminus\Gamma$.)
\end{remark}

\begin{remark}[Continuity of trace operators]\label{rem:trace-continuity}
Trace continuity is a boundary regularity statement (Lipschitz suffices), independent of the tubular geometry controlled by $\mathrm{reach}_M(\Gamma)$.
Under (SA.1), the trace maps $\gamma_{\partial M}:H^1(M)\to L^2(\partial M)$ and
$\gamma_{\Gamma}:H^1(M\setminus\Gamma)\to L^2(\Gamma)$ are continuous; moreover, the Dirichlet subspace
$H^1_0(M;\,\partial M\cup\Gamma)$ coincides with the closure of $C^\infty(M)$–functions vanishing on $\partial M\cup\Gamma$ in the $H^1$–norm (see, e.g.,~\cite[Chs.~1–2]{Grisvard1985}).
Assumption (SA.4) plays a geometric role (tubular neighborhoods, normal exponential map) and is not required for trace continuity.
\end{remark}

\begin{remark}[Trace and Dirichlet closures: precise citations]\label{rem:grisvard-nums}
The continuity of trace operators and the identification
$H^1_0(M;\partial M\cup\Gamma)=\overline{C^\infty_0(M\setminus(\partial M\cup\Gamma))}^{\,H^1}$
follow from \cite[Thm.~1.5.1–1.5.2,\;§1.5]{Grisvard1985}.
On cornered domains we rely exclusively on the form domain $H^1_0$ (no $H^2$ up to the boundary is used; cf.\ \cite[Ch.~4]{Grisvard1985}).
\end{remark}

\begin{remark}[No hidden $H^2$ assumptions]\label{rem:noH2}
All variational and spectral statements below use only the closed form $Q$ on $H^1_0(M;\partial M\cup\Gamma)$ and the $L^2$–graph characterization of $L_\Gamma$ on $M\setminus\Gamma$; we never invoke $H^2$ boundary regularity.
\end{remark}

\begin{remark}[Spectral decomposition across components]\label{rem:components}
Let $\{U_j\}_{j=1}^J$ be the connected components of $M\setminus\Gamma$.
Then $L_\Gamma$ is unitarily equivalent to the orthogonal direct sum of Dirichlet Laplacians on the $(U_j,g|_{U_j})$, and $\Spec(L_\Gamma)$ is the union (with multiplicities) of the Dirichlet spectra of the $U_j$.
This identification will be used tacitly in several arguments (compare, e.g., Reed–Simon IV, Thm.~XIII.85).
\end{remark}

\begin{remark}[Trivial case $\Gamma=\varnothing$]\label{rem:empty}
If $\Gamma=\varnothing$, then $H^1_0(M;\,\partial M\cup\Gamma)=H^1_0(M;\,\partial M)$ and $L_\Gamma$ reduces to the Dirichlet Laplacian on $(M,g)$; all statements below specialize accordingly (in particular, $a_\Gamma=0$ in Definition~\ref{def:heat-expansion}).
\end{remark}

% ------------------------------------------------------------
\subsection{(GEO) Dimensionless geometric complexity}

\begin{definition}[Dimensionless geometric complexity]\label{def:complexity}
For a $C^2$ hypersurface $\Gamma\subset M$ with $\mathrm{reach}_M(\Gamma)>0$ we set
\[
\boxed{\;
\kappa(\Gamma)
:=\;
N_{\mathrm{conn}}(\Gamma)
\;+\;\frac{\vol_{d-1}(\Gamma)}{R^{d-1}}
\;+\; R^{3-d}\!\int_\Gamma |\vec H_\Gamma|^2\, d\vol_{d-1}\;,
}
\]
where $\vec H_\Gamma$ is the mean curvature \emph{vector}; the squared norm $|\vec H_\Gamma|^2$ is globally defined irrespective of orientability and equals $|H_\Gamma|^2$ in the two–sided case.
\end{definition}

\begin{remark}[Dimensional check and scaling]
Each summand in $\kappa(\Gamma)$ is dimensionless: $[\,\vol_{d-1}/R^{d-1}\,]=L^0$ and
$[\,R^{3-d}\!\int_\Gamma |H|^2 d\vol_{d-1}\,]=L^{3-d}\cdot L^{-2}\cdot L^{d-1}=L^0$.
Under homotheties $g\mapsto \lambda^2 g$, one has $R\mapsto \lambda R$, $\vol_{d-1}(\Gamma)\mapsto \lambda^{d-1}\vol_{d-1}(\Gamma)$ and $|H_\Gamma|\mapsto \lambda^{-1}|H_\Gamma|$, hence $\kappa(\Gamma)$ is scale–invariant.
\end{remark}

\begin{remark}[Choice of curvature invariant]
The use of the squared mean curvature $|\vec H_\Gamma|^2$ is natural for capturing the roundness of $\Gamma$ and is orientation–independent.
An alternative would be the squared norm of the second fundamental form $|A_\Gamma|^2$; the resulting $\kappa(\Gamma)$ would differ by a topological term (via Gauss–Bonnet in $d=3$) but remain scale-invariant and control the same geometric features.
Among curvature scalars, $|\vec H_\Gamma|^2$ is the lowest-order, scale-compatible choice; this makes it a minimal yet robust control parameter in $\kappa(\Gamma)$.
The term $N_{\mathrm{conn}}(\Gamma)$ encodes the topological fragmentation of the interface and supplies a dimensionless complexity counter, without invoking higher invariants such as genus or Euler characteristic.
\end{remark}

\begin{remark}[On $|A_\Gamma|^2$ vs $|\vec H_\Gamma|^2$ in $d=3$]\label{rem:gb-boundary}
In $d=3$, the identity $|\vec H|^2 = |A|^2 - 2K_\Gamma$ relates squared mean curvature and the second fundamental form via the intrinsic Gaussian curvature $K_\Gamma$.
For closed $\Gamma$ the integral of $K_\Gamma$ is topological by Gauss–Bonnet; with boundary, boundary geodesic curvature terms enter.
Thus replacing $|\vec H|^2$ by $|A|^2$ changes $\kappa(\Gamma)$ by a topological/boundary term without improving scale or robustness, hence our minimal choice.
\end{remark}

\begin{remark}[Integrability of the curvature term]\label{rem:H-L2}
Since $\Gamma$ is compact and $C^2$, one has $H_\Gamma\in L^\infty(\Gamma)$ and $\int_\Gamma |H_\Gamma|^2\,d\vol_{d-1}<\infty$; the curvature term is thus well-defined.
\end{remark}

\begin{remark}[Normalization and nondegeneracy of $\kappa(\Gamma)$]\label{rem:kappa-nondeg}
By definition $N_{\mathrm{conn}}(\Gamma)\ge 1$ whenever $\Gamma\neq\varnothing$, hence $\kappa(\Gamma)\ge 1$.
\end{remark}

\begin{remark}[On the choice of the length scale]\label{rem:R-choice}
Replacing $R=\mathrm{diam}_g(M)$ by any other global length scale comparable under homotheties (e.g.\ $\vol_d(M)^{1/d}$) yields an equivalent dimensionless quantity.
We prefer $R$ because it is monotone under taking disjoint unions and is compatible with the orthogonal direct sum decomposition across components.
\end{remark}

% ------------------------------------------------------------
\subsection{(UNI) Heat trace expansion and universality convention}

\begin{definition}[Heat trace expansion: scope and leading coefficients]\label{def:heat-expansion}
As $\tau\downarrow 0$ the heat trace admits the short–time expansion
\[
\Tr e^{-\tau L_\Gamma}
\;\sim\; a_0\,\tau^{-d/2}\;+\;\big(a_{1/2}+a_\Gamma\big)\,\tau^{-(d-1)/2}\;+\;O\!\big(\tau^{-(d-2)/2}\big).
\]
In the smooth regime ($\partial M$ and $\Gamma$ of class $C^2$, with $\Gamma\cap\partial M=\varnothing$), the leading coefficients are
\[
a_0=(4\pi)^{-d/2}\,\vol_d(M),\quad
a_{1/2}=-\tfrac14(4\pi)^{-(d-1)/2}\,\vol_{d-1}(\partial M),\quad
a_\Gamma=-\tfrac14(4\pi)^{-(d-1)/2}\,\vol_{d-1}(\Gamma).
\]
\end{definition}

\begin{remark}[Corners and intersections]\label{rem:corners}
If piecewise $C^2$ boundaries or transversal intersections $\Gamma\cap\partial M\neq\varnothing$ are present (the \emph{corners regime}),
additional corner/edge contributions appear at orders different from $\tau^{-(d-1)/2}$ (and may include logarithmic factors).
However, the \emph{interior} surface density of the $\tau^{-(d-1)/2}$ term remains universal and equals $-\tfrac14(4\pi)^{-(d-1)/2}$ away from corners.
Whenever explicit smooth–regime formulas are invoked in proofs, we tacitly assume $\Gamma\cap\partial M=\varnothing$ for that argument; the corners regime is handled separately where stated.
\end{remark}

\begin{remark}[Existence of the short-time expansion]\label{rem:existence-heat}
Under (SA.1)–(SA.4), the short–time heat trace asymptotics for $L_\Gamma$ follows from local parametrix constructions and known results on heat kernels for manifolds with (piecewise smooth) boundaries and internal interfaces; see, e.g., \cite{SafarovVassiliev1997,Gilkey1995} for smooth boundaries and \cite{Grisvard1985} for nonsmooth domains.
The constant $-1/4$ for the surface density arises already in the flat half-space model with a Dirichlet hyperplane, see \cite[Sec.~1.5]{Gilkey1995}.
In the corners regime, the full expansion may involve additional terms; our use of the leading surface density is confined to smooth points of $\Gamma$ away from $\partial M$.
See also \cite{Grieser2002} and related works on heat kernels in domains with corners; we use only the leading interior surface term away from singular sets.
\end{remark}

\begin{remark}[Locality of the universal surface density]\label{rem:local-universal-density}
The universality assertion for the $\tau^{-(d-1)/2}$ surface term is \emph{local} on $\Gamma$:
it holds at every point of $\Gamma$ where $\Gamma$ is $C^2$ and separated from $\partial M$; no contribution from curvature enters at this order.
At the next order $\tau^{-(d-2)/2}$ curvature invariants of $(M,g)$ and of $\Gamma$ do enter (cf.\ Minakshisundaram–Pleijel/Gilkey coefficients), which is consistent with the locality claim for the leading surface term.
Accordingly, the leading surface term is represented by a measure absolutely continuous with respect to $\vol_{d-1}\!\upharpoonright\Gamma$ and supported away from the singular set $\partial\Gamma\cup(\Gamma\cap\partial M)$.
Whenever we use the explicit smooth-regime coefficients in proofs, the argument is understood to be localized away from the singular set.
\end{remark}

\begin{remark}[Robustness of the universal coefficient]\label{rem:robust-coeff}
The coefficient $-\tfrac14(4\pi)^{-(d-1)/2}$ for the interior surface contribution is determined locally by the model heat kernel in a Dirichlet half–space and is therefore insensitive to ambient curvature to this order.
Equivalently, the surface density is the universal coefficient obtained from the Dirichlet half–space parametrix; stability under smooth perturbations follows from locality of heat invariants.
\end{remark}

\begin{remark}[Next order coefficients]\label{rem:seeley-dewitt}
Curvature invariants of $(M,g)$ and $\Gamma$ enter at order $\tau^{-(d-2)/2}$ via the Seeley–DeWitt / Minakshisundaram–Pleijel coefficients; see, e.g., \cite[Ch.~1]{Gilkey1995}.
This is consistent with the locality and curvature–independence of the leading interior surface density.
\end{remark}

\begin{convention}[Universality of the litho–term]\label{conv:universality}
Throughout, “universality of $a_\Gamma$” refers to the \emph{leading interior surface contribution} in Definition~\ref{def:heat-expansion}:
$a_\Gamma$ depends only on $\vol_{d-1}(\Gamma)$ and is independent of the ambient geometry away from $\Gamma$.
Higher–order coefficients may involve curvature invariants of $\Gamma$ and of $(M,g)$ near $\Gamma$.
\end{convention}

\begin{remark}[Diffusive time $\tau$ vs geometric time $t$]\label{rem:tau-vs-t}
Local heat coefficients are computed in the diffusive scale $\tau$.
Refined remainder estimates rely on Tauberian arguments linking $\Tr(e^{-\tau L_\Gamma})$ to oscillatory integrals controlled by the $t$–dynamics of the reflecting flow; thus $H_{\mathrm{mix}}$ is stated in the geometric time $t$.
\end{remark}

% ------------------------------------------------------------
\subsection{(DYN) Dynamical model for refined remainders}

\paragraph{Motivation for $H_{\mathrm{mix}}$.}
We invoke $H_{\mathrm{mix}}$ to quantify decay of correlations for the reflecting flow; this hypothesis underlies refined control of spectral remainders beyond the local heat coefficients, via Tauberian arguments linking dynamical mixing to spectral error terms.

\begin{hypothesis}[Exponential mixing for the reflecting model, $H_{\mathrm{mix}}$]\label{hyp:mixing}
Let $S^*(M\setminus\Gamma)$ denote the unit cotangent bundle of $M\setminus\Gamma$.
We consider two generalized geodesic flows on $T^*(M\setminus\Gamma)$:
\begin{enumerate}
  \item \emph{Reflecting model (conservative):} geodesic transport in the interior and \emph{specular reflections} at $\partial M\cup\Gamma$ (tangential covector preserved, normal covector reversed).
  This flow preserves the Liouville probability measure $\mu$ on $S^*(M\setminus\Gamma)$ (defined modulo a Liouville-negligible set of singular trajectories) and we normalize $\mu$ so that $\mu(S^*(M\setminus\Gamma))=1$.
  \item \emph{Absorbing model (dissipative):} specular reflections at $\partial M$ and termination at the first hitting time of $\Gamma$ (Dirichlet absorption).
\end{enumerate}
In the reflecting model, the billiard-type flow $\varphi^t$ is defined for $\mu$–almost every initial covector and preserves the Liouville probability measure $\mu$ (normalized by $\mu(S^*(M\setminus\Gamma))=1$).
The $\mu$–null singular set comprises trajectories that (i) hit corner points or $\partial\Gamma$ in finite time, or (ii) belong to the grazing set (covectors tangent to $\partial M\cup\Gamma$ leading to nontransversal contacts of infinite order); it also includes covectors that hit lower–dimensional strata of $\partial M\cup\Gamma$.
We define $\varphi^t$ on the complement of this $\mu$–null singular set; on this domain, $\varphi^t$ is $\mu$–measure preserving.
Existence a.e.\ and measure preservation for billiard-type flows in piecewise smooth settings are classical; we rely only on these a.e.\ properties here (see, e.g., standard references in dispersing billiards and billiards with singularities such as Chernov–Markarian).

% ===================== PART 2/2 (append directly after PART 1/2) =====================

% --- Hermetic closure of dynamics: phase space, observables, norms, constants ---

\subsection*{Appendix to (DYN): Hermetic specification of flow, observables, and mixing}

\begin{definition}[Regular phase space and singular set]\label{def:regular-phase}
Let $\Sing\subset S^*(M\setminus\Gamma)$ be the union of:
\begin{enumerate}
\item the \emph{grazing set} $\Gra := \{(x,\xi): \xi \text{ is tangent to } \partial M\cup\Gamma \text{ at the first hit}\}$;
\item all covectors whose billiard trajectory hits $\partial\Gamma$ or any corner/edge point of $\partial M\cup\Gamma$ in finite time (forward or backward);
\item all covectors that hit lower–dimensional strata created by the piecewise $C^2$ structure.
\end{enumerate}
Define the \emph{regular phase space} $S^*_{\reg}:=S^*(M\setminus\Gamma)\setminus \Sing$, and let $\mu$ be the Liouville probability measure on $S^*(M\setminus\Gamma)$, normalized by $\mu(S^*(M\setminus\Gamma))=1$.
By standard results for billiard-type flows with piecewise smooth boundary, $\mu(\Sing)=0$ and $S^*_{\reg}$ is $\varphi^t$–invariant a.e.
\end{definition}

\begin{definition}[Billiard flow on $S^*_{\reg}$]\label{def:billiard-flow}
The reflecting flow $\varphi^t:S^*_{\reg}\to S^*_{\reg}$ is defined by concatenation of unit–speed geodesic arcs and \emph{specular reflections} at $\partial M\cup\Gamma$:
if $n$ denotes the inward unit normal covector at a regular reflection point, then the outgoing covector is obtained by reversing the normal component and keeping the tangential component unchanged.
Along geodesic transport and at each regular reflection, the Liouville measure $\mu$ is preserved; hence $\varphi^t$ is $\mu$–measure preserving on $S^*_{\reg}$.
\end{definition}

\begin{remark}[Precise scope for correlations]\label{rem:scope-correlations}
All correlation integrals and norms below are taken on the \emph{regular} phase space $S^*_{\reg}$; the $\mu$–null singular set $\Sing$ is excluded by definition and does not affect correlation bounds.
All measurability statements are with respect to the Borel $\sigma$–algebra on $S^*_{\reg}$; composition $G\circ\varphi^t$ is defined $\mu$–a.e.\ by the a.e.\ definition of $\varphi^t$.
\end{remark}

\begin{definition}[Admissible observables and norms]\label{def:admissible}
Fix $r\in\mathbb{N}$ and $\eta\in(0,1]$.
Let $C^r(S^*_{\reg})$ denote the space of functions that are $C^r$ on $S^*_{\reg}$ in the sense of any fixed smooth atlas adapted to the piecewise $C^2$ structure and the reflection hypersurfaces; its norm is
\[
\|F\|_{C^r}:=\sum_{|\alpha|\le r}\sup_{S^*_{\reg}}|D^\alpha F|.
\]
Let $C^\eta(S^*_{\reg})$ be the Hölder class with the standard seminorms on such charts.
By compactness of $S^*(M\setminus\Gamma)$ and equivalence of adapted atlases on $S^*_{\reg}$, any two such norms are equivalent.
\end{definition}

\begin{remark}[Chart–independence]\label{rem:chart-independence}
All constants implicit in the norm equivalence depend only on the choice of a finite atlas subordinate to a fixed collar neighborhood of $\partial M\cup\Gamma$ and on uniform $C^2$ bounds of $g$ and of the second fundamental forms along the reflecting hypersurfaces.
Hence bounds stated below are intrinsic and stable under $C^2$–small perturbations of the geometry.
\end{remark}

\begin{definition}[Correlation functional]\label{def:correlation}
For $F,G\in L^2(S^*_{\reg},\mu)$ define the centered correlation
\[
\Corr_t(F,G):=\int_{S^*_{\reg}}\!\!\Big(F-\textstyle\int F\,d\mu\Big)\Big(G\circ\varphi^t-\textstyle\int G\,d\mu\Big)d\mu,
\qquad t\in\mathbb{R}.
\]
\end{definition}

\begin{hypothesis}[Hermetic exponential mixing $H_{\mix}^{\heartsuit}$]\label{hyp:mixing-hermetic}
There exist $r\in\mathbb{N}$, $\eta\in(0,1]$, constants $\alpha>0$ and $C=C(M,g,\Gamma,r,\eta)$ such that for all $F\in C^r(S^*_{\reg})$ and $G\in C^\eta(S^*_{\reg})$ with $\int F\,d\mu=\int G\,d\mu=0$ one has
\[
|\Corr_t(F,G)|\ \le\ C\,\|F\|_{C^r}\,\|G\|_{C^\eta}\,e^{-\alpha |t|},\qquad t\in\mathbb{R}.
\]
The constants $\alpha$ and $C$ may depend on $(M,g,\Gamma)$ through sectional curvature bounds on a fixed collar of $\partial M\cup\Gamma$, the $C^2$–norms of the second fundamental forms of $\partial M$ and $\Gamma$, and the complexity $\kappa(\Gamma)$; they are invariant under homotheties $g\mapsto \lambda^2 g$ after the natural time rescaling $t\mapsto \lambda t$.
\end{hypothesis}

\begin{remark}[Why $C^r$–vs–Hölder and robustness]\label{rem:anisotropic}
The pair $C^r$–$C^\eta$ is standard for billiard flows and suffices for Tauberian estimates of spectral remainders.
One could replace these by anisotropic Banach spaces à la Gouëzel–Liverani/Blank–Keller–Liverani; our arguments require only \emph{some} exponentially mixing dense algebra, which $C^r\cap C^\eta$ provides on $S^*_{\reg}$.
\end{remark}

\begin{remark}[Sufficient geometric scenarios]\label{rem:sufficient-geom}
Hypothesis~\ref{hyp:mixing-hermetic} holds in the following model settings (not used elsewhere except as sanity checks):
\begin{itemize}
\item \emph{Dispersing billiards in domains}: $(M,g)$ a Euclidean domain with strictly convex dispersing components of $\partial M\cup\Gamma$ and no cusps; exponential decay of correlations for Hölder observables is classical.
\item \emph{Anosov backgrounds}: If the geodesic flow on $(M,g)$ is Anosov and reflections occur on strictly convex $\partial M\cup\Gamma$ in the sense of Sinai, one expects exponential mixing on $S^*_{\reg}$; proofs exist in several codified frameworks.
\end{itemize}
These examples justify the non-emptiness of the hypothesis class; the present paper \emph{assumes} $H_{\mix}^{\heartsuit}$ and exploits only its stated consequence.
\end{remark}

\begin{lemma}[A.e.\ invariance and Liouville preservation]\label{lem:liouville}
On $S^*_{\reg}$ the flow $\varphi^t$ is defined for all $t\in\mathbb{R}$, $\mu$–a.e., and preserves $\mu$: for all Borel $E\subset S^*_{\reg}$, $\mu(\varphi^{-t}E)=\mu(E)$.
\end{lemma}

\begin{remark}[Alignment with the heat scale]\label{rem:tauberian-interface}
We use $H_{\mix}^{\heartsuit}$ only via Tauberian transference of oscillatory integral bounds into spectral remainder estimates for $\Tr(e^{-\tau L_\Gamma})$; the partition $S^*_{\reg}\sqcup\Sing$ ensures that grazing/corner interactions do not enter correlation estimates (they are $\mu$–null).
\end{remark}

% ------------------------------------------------------------
\paragraph{Interfaces as internal Dirichlet boundaries (recalled).}\label{rem:jump}
For $u\in H^1_0(M;\,\partial M\cup\Gamma)$ one has $\gamma_\Gamma(u)=0$; no transmission condition is imposed for the conormal derivative, and $\partial_\nu u$ may jump across $\Gamma$.
Thus $\Gamma$ acts as an internal Dirichlet wall in all variational identities used below.

\begin{remark}[No spurious boundary terms in Green identities]\label{rem:green}
All Green–type formulas are applied componentwise on $U_j$ and summed; the internal boundary $\Gamma$ contributes no boundary integrals because the trace of $u$ vanishes on $\Gamma$.
\end{remark}

% ------------------------------------------------------------
\paragraph{Bibliographic anchors.}
For boundary traces and elliptic problems in nonsmooth domains, see \cite[Chs.~1–2]{Grisvard1985}.
For heat kernel coefficients on smooth manifolds with boundary, see \cite{Gilkey1995,SafarovVassiliev1997};
the universal surface density $-\tfrac14\,(4\pi)^{-(d-1)/2}$ already appears in the flat half–space model \cite[Sec.~1.5]{Gilkey1995}.
For corner/edge contributions in polyhedral settings, see, e.g., \cite{Grieser2002}.
Classical references for billiard–type flows with singularities include Chernov–Markarian; we use only a.e.\ existence and measure preservation of the reflecting flow together with the stated exponential correlation decay hypothesis.

% =====================================================================
% End (paste PART 2/2 immediately after PART 1/2; no gaps, no edits)
% =====================================================================
