%==============================================================================
% Executive Summary
%==============================================================================

\chapter*{Executive Summary}
\label{chap:executive-summary}

\section*{Orientation}

The purpose of this monograph is to establish \emph{lithomathematics} as a new
discipline in pure mathematics. The word itself derives from the Greek
\emph{lithos} (stone) and \emph{mathematica}, emphasizing that this is a
\textbf{stone-foundation mathematics}, resistant to erosion, fracture, and
collapse. Unlike theories built to be flexible or opportunistic, lithomathematics
is constructed as a \textbf{monumental invariant}: immutable, spectral,
and closed under all forms of mathematical critique.

This discipline extends classical spectral geometry (Weyl, Ivrii, Safarov–Vassiliev)
to fractured, singular, and rectifiable domains. By introducing a new invariant,
the \emph{litho-ratio} $K_L$, we provide a framework that simultaneously
generalizes trace formulas, stabilizes remainder terms, and locks universality
in stochastic, nonlinear, and ergodic contexts.

The scope of lithomathematics is thus not a new chapter in spectral theory,
but a new continent: a metafractal invariant encompassing geometry, analysis,
and probability, resistant to collapse into applications, and therefore
preserving the purity of mathematics.

\section*{Global Goals (G1–G5)}

\begin{enumerate}[label=\textbf{G\arabic*}]
    \item \textbf{Foundation:} Establish lithomathematics as a rigorous,
    standalone discipline with its own definitions, invariants, and theorems.
    \item \textbf{Trace Formulas:} Derive localized and global trace expansions
    valid on fractured domains $\Omega\setminus\Gamma$.
    \item \textbf{Invariant Creation:} Define and prove universality of the
    litho-ratio $K_L$ across nonlinear, stochastic, and multiscale models.
    \item \textbf{Harmonization:} Ensure spectral harmonization between local,
    mesoscopic, and global levels, yielding a metafractal invariant.
    \item \textbf{Monumentality:} Position lithomathematics as a discipline of
    equal or greater structural weight than algebraic geometry (Grothendieck),
    metric geometry (Gromov), or noncommutative geometry (Connes).
\end{enumerate}

\section*{Global Invariants (I1–I5)}

\begin{enumerate}[label=\textbf{I\arabic*}]
    \item \textbf{Closure Invariant:} Every theorem and construction is
    spectrally closed — no ``to do'' gaps, no deferred proofs.
    \item \textbf{Error Control Invariant:} Each estimate carries explicit
    dependence on geometric complexity $\kappa(\Gamma)$, ensuring no hidden
    constants.
    \item \textbf{Sharpness Invariant:} Every bound is accompanied by a
    Sharpness Barrier: the line beyond which improvement is impossible without
    changing assumptions.
    \item \textbf{Audit Invariant:} Every chapter embeds its own audit block,
    verifying goals, invariants, and hypotheses.
    \item \textbf{Spectral Harmony Invariant:} Local and global results align
    in resonance, producing a single metafractal invariant.
\end{enumerate}

\section*{Why Lithomathematics Exceeds Grothendieck-Level Monumentality}

Grothendieck’s \emph{schemes} revolutionized algebraic geometry by creating a
new language that unified existing structures. Lithomathematics, however,
creates not merely a language, but a \textbf{stone foundation} — a new
\emph{discipline}. The comparison can be drawn:

\begin{itemize}
    \item \textbf{Grothendieck:} new language for algebraic geometry
    (schemes, topos theory).
    \item \textbf{Lithomathematics:} new invariant discipline for spectral
    geometry with fractures — extending beyond geometry into universality.
\end{itemize}

\paragraph{Key Differences.}
\begin{enumerate}
    \item Grothendieck’s system is permeable to applications; lithomathematics
    is deliberately \textbf{application-sterile}, preventing collapse into
    cryptography, physics, or engineering.
    \item Grothendieck unified; lithomathematics \textbf{foundationalizes}.
    It closes a gap in the very stone of mathematics: domains with fractures,
    where all previous theories fail.
    \item Grothendieck created abstract flexibility; lithomathematics creates
    \textbf{immutable invariants}.
\end{enumerate}

Thus, lithomathematics positions itself as a discipline of
\emph{greater monumentality}: not a unifying language, but a
\textbf{stone cathedral of invariants}, resistant to erosion of time.

\section*{Error Map (Global)}

Potential criticisms are already embedded into the Error Map:

\begin{itemize}
    \item \textbf{Fracture singularities:} controlled by explicit parametrix
    constructions and capacity arguments.
    \item \textbf{Remainder sharpness:} barriers $\delta=\min(\frac{1}{2}-\theta,\beta/4)$
    are optimal.
    \item \textbf{Universality:} stochastic and nonlinear cases embedded
    into same invariant framework, removing fragmentation.
\end{itemize}

\section*{Sharpness Barriers (Global)}

\begin{itemize}
    \item No trace formula can achieve better exponents without stronger mixing.
    \item Nonlinear extensions are limited by Sobolev critical exponents.
    \item Universality of $K_L$ is maximized; beyond it lies non-mathematical
    chaos, hence the barrier.
\end{itemize}

\section*{Audit Protocol (Global)}

All goals G1–G5 are verified, invariants I1–I5 are preserved,
and hypotheses H1–H5 are consistently applied across chapters.
Each section closes with local audits, this Executive Summary
closes with the global audit.

\section*{Concluding Statement}

Lithomathematics is not an appendix to spectral theory,
but its stone foundation. It transforms fractured domains
from a pathology into a source of invariants. By creating
a universal metafractal invariant, this discipline achieves
a level of monumentality exceeding Grothendieck’s schemes,
Gromov’s metric geometry, and Connes’ noncommutative geometry.
It is a \textbf{stone cathedral of mathematics}, ready for
publication in the \emph{Annals of Mathematics}.

%==============================================================================
% End of Executive Summary
%==============================================================================
