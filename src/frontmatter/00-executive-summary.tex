%==============================================================================
% Executive Summary
%==============================================================================

\chapter*{Executive Summary}
\label{chap:executive-summary}

\section*{Overview}

This monograph establishes the mathematical foundations of 
\emph{lithomathematics}, a discipline extending spectral geometry to domains 
with internal singular structures such as fractures, cracks, and rectifiable 
sets of codimension one. While classical spectral asymptotics---originating 
in the works of Weyl~\cite{Weyl1911}, Ivrii~\cite{Ivrii1980}, and 
Safarov--Vassiliev~\cite{SafarovVassiliev1997}---apply to smooth or piecewise 
smooth manifolds, they do not directly extend to fractured domains. 
Our goal is to develop a coherent analytic, microlocal, and probabilistic 
framework for such singular settings.

\section*{Principal Results}

\begin{theorem}[Localized Trace Formula on Fractured Domains]
Let $(\Omega,g)$ be a compact Riemannian manifold with boundary 
$\partial\Omega$ and an internal rectifiable fracture set $\Gamma$. 
For the Laplace operator with Dirichlet boundary conditions on 
$\Omega\setminus\Gamma$, one has
\[
    \mathrm{Tr}(g(\sqrt{-\Delta})) 
    = A_{\mathrm{vol}}(g) + A_{\partial\Omega}(g) + A_{\Gamma}(g) + \mathcal{R}(g),
\]
where the coefficients $A_{\mathrm{vol}}, A_{\partial\Omega}, A_{\Gamma}$ are 
explicitly determined by the geometry of $\Omega$ and $\Gamma$, and the 
remainder satisfies
\[
    |\mathcal{R}(g)| 
    \;\leq\; C\,\kappa(\Gamma)\,\|g\|_{C^{d+3}}
        \left( T^{d-2}\log(1+T) + e^{-cT}\right).
\]
\end{theorem}

\begin{theorem}[Geometric Complexity Parameter]
The influence of the fracture set $\Gamma$ on spectral asymptotics is 
captured by the geometric complexity parameter
\[
    \kappa(\Gamma) \;=\; 
        H^{d-1}(\Gamma) 
        + \int_\Gamma (1+|II(x)|^2)^{1/2}\,dH^{d-1}(x) 
        + N_{\mathrm{comp}}(\Gamma),
\]
where $H^{d-1}$ is the Hausdorff measure, $II(x)$ the second fundamental form, 
and $N_{\mathrm{comp}}(\Gamma)$ the number of connected components. 
All constants in the trace expansions depend polynomially on $\kappa(\Gamma)$.
\end{theorem}

\begin{theorem}[Power-Saving Refinements]
Assume exponential mixing in the geodesic flow on $\Omega\setminus\Gamma$. 
Then the spectral remainder admits a power-saving refinement:
\[
    |\mathcal{R}(g)| \;\leq\; C_\varepsilon\, T^{d-2-\delta+\varepsilon},
    \qquad 
    \delta = \min\!\left(\tfrac{1}{2}-\theta,\;\tfrac{\beta}{4}\right),
\]
where $\theta$ is the best known exponent toward the Ramanujan--Petersson 
conjecture and $\beta$ the mixing rate.
\end{theorem}

\begin{theorem}[Universality of the Litho-Ratio]
Define the litho-ratio $K_L$ as the normalized difference between bulk and 
fracture contributions in the trace formula. Under ergodic sampling of 
fractures, one has
\[
    K_L \;\to\; K_L^* 
    \quad\text{almost surely},
\]
with Gaussian fluctuations of order $O(N^{-1/2})$. 
The limit $K_L^*$ is universal across classes of fractured domains.
\end{theorem}

\section*{Methodological Innovations}

\begin{itemize}
    \item \textbf{Microlocal parametrix construction} for Laplacians on domains 
    with rectifiable singularities, extending Hörmander--Melrose theory.
    \item \textbf{Fracture coefficient $a_\Gamma(g)$} explicitly quantifying 
    spectral contributions of cracks.
    \item \textbf{Geometric complexity parameter $\kappa(\Gamma)$} 
    governing all error terms in trace expansions.
    \item \textbf{Stochastic homogenization and ergodic limits} 
    proving stability and universality of $K_L$.
\end{itemize}

\section*{Relation to Existing Literature}

\begin{itemize}
    \item Extends Weyl’s law~\cite{Weyl1911}, Ivrii’s boundary 
    asymptotics~\cite{Ivrii1980}, and the microlocal framework of 
    Safarov--Vassiliev~\cite{SafarovVassiliev1997}.
    \item Complements variational fracture models of 
    Bourdin--Francfort--Marigo~\cite{BourdinFrancfortMarigo2008} 
    by introducing spectral invariants.
    \item Connects to ergodic theorems (Lindenstrauss~\cite{Lindenstrauss2001}) 
    and homogenization (Cioranescu--Murat~\cite{CioranescuMurat1997}).
\end{itemize}

\section*{Implications}

The results establish lithomathematics as a coherent extension of spectral 
geometry to fractured media. They provide:
\begin{enumerate}
    \item Explicit trace formulas with controlled remainders.
    \item Geometric quantification via $\kappa(\Gamma)$.
    \item Universality of invariants under stochastic and ergodic limits.
\end{enumerate}

All statements are proven with explicit constants, sharpness barriers, and 
audited error maps, ensuring full reproducibility and mathematical rigor.

%==============================================================================
% End of Executive Summary
%==============================================================================
