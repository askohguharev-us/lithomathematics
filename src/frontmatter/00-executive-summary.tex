%==============================================================================
% Executive Summary
%==============================================================================

\chapter*{Executive Summary}
\label{chap:executive-summary}

\section*{Overview}

This monograph develops the analytic and microlocal foundations of spectral geometry 
on fractured domains, a framework we call \emph{lithomathematics}. 
It extends classical spectral asymptotics (Weyl, Ivrii, Safarov–Vassiliev) 
to compact Riemannian manifolds with internal singularities such as rectifiable 
fracture sets. The novelty lies in the explicit identification of fracture 
contributions in trace formulas, the introduction of a geometric complexity 
parameter, and universality results for spectral ratios.

\section*{Principal Results}

\begin{theorem}[A. Localized Trace Formula on Fractured Domains]
\label{thm:trace}
Let $(\Omega,g)$ be a compact $d$-dimensional Riemannian manifold with smooth boundary 
$\partial\Omega$ and an internal rectifiable fracture set $\Gamma$ of class $C^2$. 
Consider the Laplace operator with Dirichlet boundary conditions on $\Omega\setminus\Gamma$. 
For a smooth function $g$ with $\mathrm{supp}(g)\subset[-T,T]$, one has
\[
\operatorname{Tr}\!\big(g(\sqrt{-\Delta})\big)
  = A_{\mathrm{vol}}(g) + A_{\partial\Omega}(g) 
  + A_{\Gamma}(g) + \mathcal{R}(g),
\]
where:
\begin{itemize}
    \item $A_{\mathrm{vol}}(g)$ and $A_{\partial\Omega}(g)$ are the standard Weyl terms,
    \item $A_{\Gamma}(g)$ is an explicit integral over $\Gamma$ involving its 
    Hausdorff measure and second fundamental form,
    \item the remainder satisfies
    \[
    |\mathcal{R}(g)| \;\leq\; 
        C(\Omega,\Gamma)\,\kappa(\Gamma)\,
        \|g\|_{C^{d+3}}\big(T^{d-2}\log(1+T)+e^{-c(\Omega,\Gamma)T}\big).
    \]
\end{itemize}
\end{theorem}

\begin{definition}[Geometric Complexity Parameter]
\label{def:complexity}
The influence of the fracture set $\Gamma$ is quantified by
\[
\kappa(\Gamma) \;=\; 
    \mathcal{H}^{d-1}(\Gamma)
    + \int_\Gamma \!\!(1+|II(x)|^2)^{1/2}\,d\mathcal{H}^{d-1}(x) 
    + N_{\mathrm{comp}}(\Gamma),
\]
where $\mathcal{H}^{d-1}$ is the $(d-1)$-dimensional Hausdorff measure, 
$II(x)$ the second fundamental form on $\Gamma$, 
and $N_{\mathrm{comp}}(\Gamma)$ the number of connected components.
\end{definition}

\begin{proposition}[Polynomial Dependence]
\label{prop:polynomial}
All constants appearing in Theorem~\ref{thm:trace} can be bounded by a polynomial 
in $\kappa(\Gamma)$ of degree depending only on the dimension $d$.
\end{proposition}

\begin{theorem}[C. Power-Saving Refinements]
\label{thm:refinements}
Under the assumptions of Theorem~\ref{thm:trace}, assume further that the 
geodesic flow on $(\Omega\setminus\Gamma,g)$ is exponentially mixing with rate $\beta>0$. 
Then the remainder improves to
\[
|\mathcal{R}(g)| \;\leq\; C_\varepsilon T^{d-2-\delta+\varepsilon},
\]
for any $\varepsilon>0$, where $\delta>0$ is explicitly determined by $\beta$. 
This exponent is sharp under the stated assumptions.
\end{theorem}

\begin{theorem}[D. Universality of the Litho-Ratio]
\label{thm:litho-ratio}
Let $\{ \Gamma_i \}_{i=1}^N$ be an ergodic sample of admissible $C^2$ fracture sets 
with respect to a probability measure on the space of such subsets. 
Define the \emph{litho-ratio} $K_L$ as the spectral ratio measuring the 
relative fracture contribution in the localized trace expansion. 
Then
\[
K_L \;\to\; K_L^* \quad \text{almost surely as } N\to\infty,
\]
with Gaussian fluctuations at rate $O(N^{-1/2})$.
\end{theorem}

\section*{Methodological Innovations}

\begin{itemize}
    \item A microlocal parametrix construction adapted to rectifiable fracture sets, 
    incorporating diffraction effects at fracture edges.
    \item The geometric complexity parameter $\kappa(\Gamma)$ as a unifying measure 
    of spectral stability and error control.
    \item Power-saving error estimates under mixing assumptions, with optimal exponents.
    \item Probabilistic universality for spectral ratios, with Gaussian fluctuations.
\end{itemize}

\section*{Relation to Existing Literature}

This work extends the classical asymptotics of Weyl~\cite{weyl1911}, 
Ivrii~\cite{ivrii1980}, and Safarov–Vassiliev~\cite{safarov1997} 
to manifolds with internal singularities. 
Unlike variational approaches to fracture 
(Bourdin–Francfort–Marigo~\cite{bourdin2008},~\cite{bourdin2012}), 
the focus here is on spectral invariants and their universal features.

\section*{Structure of the Monograph}

Chapter~1 gives historical context.  
Chapters~2–4 introduce analytic and microlocal tools.  
Chapters~5–8 establish trace formulas, error estimates, and universality theorems.  
Chapters~9–10 discuss canonical examples and synthesize the theory.  
Appendices contain technical lemmas, extended proofs, and bibliographic notes.

\section*{Implications}

The results provide explicit spectral formulas for fractured domains, 
with rigorous error bounds and sharpness barriers. 
They establish lithomathematics as a coherent extension of spectral geometry, 
laying a foundation for future investigations into singular domains. 
All results are stated with explicit constants, ensuring full reproducibility.
