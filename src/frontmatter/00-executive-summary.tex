\chapter*{Executive Summary}
\label{chap:executive-summary}

\section*{Overview}

This monograph develops the mathematical foundations of 
\emph{lithomathematics}, a new discipline within spectral geometry 
focusing on domains with internal singularities (fractures).  
It establishes localized and global trace formulas, introduces 
geometric complexity parameters, and proves the universality of 
new spectral invariants under stochastic and ergodic limits.  

\section*{Principal Results}

\begin{theorem}[Localized Trace Formula on Fractured Domains]
\label{thm:localized-trace}
Let $\Omega$ be a compact Riemannian manifold of dimension $d$ 
with rectifiable fracture set $\Gamma \subset \Omega$. 
For the Laplace operator with Dirichlet conditions on $\Omega\setminus\Gamma$, 
the spectral trace satisfies
\[
\mathrm{Tr}\big(g(\sqrt{-\Delta})\big) 
= A_{\mathrm{vol}}(g) + A_{\partial\Omega}(g) + A_{\Gamma}(g) + \mathcal{R}(g),
\]
where
\begin{itemize}
    \item $A_{\mathrm{vol}}(g)$ is the Weyl bulk term, 
    \item $A_{\partial\Omega}(g)$ the Ivrii-type boundary contribution,
    \item $A_{\Gamma}(g)$ a new fracture coefficient depending explicitly on $\Gamma$,
\end{itemize}
and the remainder satisfies the quantitative bound
\[
|\mathcal{R}(g)| \leq C \, \kappa(\Gamma)\, \|g\|_{C^{d+3}}\, 
\big( T^{d-2}\log(1+T) + e^{-cT} \big),
\]
with constants depending only on macroscopic invariants of $\Omega$ and $\Gamma$.
\end{theorem}

\begin{theorem}[Power-Saving Refinement]
\label{thm:power-saving}
Under exponential mixing assumptions, the remainder in Theorem~\ref{thm:localized-trace} 
admits the sharper bound
\[
|\mathcal{R}(g)| \leq C \, \kappa(\Gamma)\, \|g\|_{C^{d+3}}\, T^{d-2-\delta},
\qquad 
\delta = \min\!\left(\tfrac{1}{2}-\theta, \tfrac{\beta}{4}\right),
\]
where $\theta$ is the best-known bound toward the Ramanujan–Petersson conjecture 
and $\beta$ the mixing exponent.
\end{theorem}

\begin{theorem}[Universality of the Litho-Ratio]
\label{thm:litho-ratio}
Define the litho-ratio $K_L$ as the normalized limit of fracture contributions.  
Then
\[
K_L \;\;\xrightarrow[N\to\infty]{\text{a.s.}}\;\; K_L^*
\]
under ergodic sampling of fractures, with Gaussian fluctuations:
\[
\sqrt{N}\,\big(K_L - K_L^*\big) \;\;\xrightarrow{d}\;\; \mathcal{N}(0,\sigma^2(\Gamma)).
\]
\end{theorem}

\section*{Methodological Innovations}

\begin{itemize}
    \item Construction of \textbf{microlocal parametrices} on domains with rectifiable singularities, extending Hörmander–Melrose techniques.  
    \item Definition of a \textbf{geometric complexity parameter} $\kappa(\Gamma)$ capturing both measure and curvature of fracture sets.  
    \item Development of \textbf{stochastic homogenization} and ergodic limit theorems for spectral invariants.  
\end{itemize}

\section*{Relation to Existing Literature}

This work extends classical results of Weyl \cite{Weyl1911}, Ivrii \cite{Ivrii1980}, 
and Safarov–Vassiliev \cite{SafarovVassiliev1997} to non-smooth domains with 
fractures.  
Unlike the variational approaches to fracture mechanics (Bourdin–Francfort–Marigo \cite{Bourdin2008}), 
the focus here is on spectral asymptotics, trace invariants, and universality.  

\section*{Applications and Implications}

The results establish fundamental limits for spectral geometry on fractured 
domains.  
They provide rigorous tools for analyzing wave propagation and homogenization 
in singular media, while explicitly quantifying error bounds and sharpness 
barriers.  
All theorems are formulated with explicit constants and controlled dependence 
on geometric invariants.

