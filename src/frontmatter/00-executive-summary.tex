%==============================================================================
% 00-executive-summary.tex
%==============================================================================

\chapter*{Executive Summary}
\label{chap:executive-summary}

\section*{Overview}

This monograph establishes the mathematical foundations of \emph{lithomathematics}, 
a framework extending spectral geometry to domains with internal singular structures. 
The theory addresses the analytic and microlocal behavior of Laplacians on fractured 
domains, providing explicit trace expansions, quantitative error bounds, and universal 
spectral invariants.

\section*{Principal Results}

\begin{theorem}[A. Localized Trace Formula on Fractured Domains] \label{thm:trace}
Let $(\Omega,g)$ be a compact Riemannian manifold with boundary $\partial\Omega$ and 
an internal rectifiable fracture set $\Gamma$ of class $C^2$. Consider the Laplace 
operator with Dirichlet boundary conditions on $\Omega\setminus\Gamma$. For a smooth 
function $g$ of compact support, one has
\[
    \mathrm{Tr}(g(\sqrt{-\Delta})) 
    = A_{\mathrm{vol}}(g) + A_{\partial\Omega}(g) + A_{\Gamma}(g) + \mathcal{R}(g),
\]
where the coefficients are explicitly determined by the geometry of $\Omega$ and $\Gamma$, 
and the remainder satisfies
\[
    |\mathcal{R}(g)| 
    \;\leq\; C(\Omega,\Gamma)\,\kappa(\Gamma)\,\|g\|_{C^{d+3}}
        \left( T^{d-2}\log(1+T) + e^{-c(\Omega,\Gamma) T}\right).
\]
\end{theorem}

\begin{theorem}[B. Geometric Complexity Parameter] \label{thm:complexity}
The influence of the fracture set $\Gamma$ on spectral asymptotics is captured by the 
geometric complexity parameter
\[
    \kappa(\Gamma) \;=\; 
        \mathcal{H}^{d-1}(\Gamma) 
        + \int_\Gamma (1+|II(x)|^2)^{1/2}\,d\mathcal{H}^{d-1}(x) 
        + N_{\mathrm{comp}}(\Gamma),
\]
where $\mathcal{H}^{d-1}$ is the $(d-1)$-dimensional Hausdorff measure, $II(x)$ the 
second fundamental form, and $N_{\mathrm{comp}}(\Gamma)$ the number of connected 
components. All constants in the trace expansions depend polynomially on $\kappa(\Gamma)$.
\end{theorem}

\begin{theorem}[C. Power-Saving Refinements] \label{thm:refinements}
Assume exponential mixing for the geodesic flow on $(\Omega\setminus\Gamma,g)$. Then 
the error exponent in the trace remainder satisfies
\[
    \delta \;=\; \min\!\left(\tfrac{1}{2}-\theta,\;\tfrac{\beta}{4}\right),
\]
where $\theta$ is the best known bound toward the Ramanujan--Petersson conjecture 
and $\beta$ quantifies the mixing rate. This exponent is sharp: improvements require 
stronger dynamical assumptions.
\end{theorem}

\begin{theorem}[D. Universality of the Litho-Ratio] \label{thm:universality}
Let $K_L$ denote the litho-ratio defined in Chapter~5. Under an ergodic probability 
measure on the space of admissible fracture sets $\Gamma$ (precisely defined in 
Chapter~6), one has almost sure convergence
\[
    K_L \;\to\; K_L^*,
\]
with Gaussian fluctuations at rate $O(N^{-1/2})$. The limit $K_L^*$ is universal, 
independent of microscopic randomness.
\end{theorem}

\section*{Methodological Innovations}

\begin{itemize}
    \item Microlocal parametrix construction adapted to rectifiable singularities
    \item Geometric complexity parameter $\kappa(\Gamma)$ quantifying fracture effects
    \item Stochastic homogenization of spectral invariants across scales
    \item Error maps and sharpness barriers explicitly embedded in all proofs
\end{itemize}

\section*{Relation to Literature}

This work extends the classical results of Weyl~\cite{Weyl1911}, Ivrii~\cite{Ivrii1980}, 
and Safarov--Vassiliev~\cite{SafarovVassiliev1997} to non-smooth domains with internal 
fractures. In contrast to variational approaches to fracture 
(Bourdin--Francfort--Marigo~\cite{Bourdin2008,FrancfortMarigo1998}), we focus on 
spectral invariants and their universal properties.

\section*{Structure of the Monograph}

The text is organized into ten chapters and eight appendices. 
Chapters~1--4 establish the analytic framework; Chapter~5 develops the localized and 
global trace formulas; Chapters~6--8 address ergodic, homogenization, and stochastic 
extensions; Chapter~9 presents canonical examples; Chapter~10 provides synthesis 
and outlook. Appendices contain technical proofs, notation tables, and expanded 
error maps.

\section*{Implications}

The results provide a comprehensive spectral framework for fractured domains, 
with explicit constants and sharpness barriers ensuring reproducibility. 
They establish lithomathematics as a coherent extension of spectral geometry, 
laying foundations for further developments in analysis on singular spaces.

%==============================================================================
% End of Executive Summary
%==============================================================================

