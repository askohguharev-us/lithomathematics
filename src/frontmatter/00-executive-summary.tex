%==============================================================================
% Executive Summary
%==============================================================================

\chapter*{Executive Summary}
\label{chap:executive-summary}

\section*{Overview}

This monograph establishes the analytic and microlocal foundations of 
\emph{lithomathematics}, a framework extending classical spectral geometry 
to domains with internal singularities such as fractures. 
The main results concern localized and global trace formulas, 
the introduction of a geometric complexity parameter $\kappa(\Gamma)$, 
and the universality of the litho-ratio $K_L$ under ergodic sampling. 

\section*{Principal Results}

\begin{theorem}[A. Localized Trace Formula on Fractured Domains] 
\label{thm:trace}
Let $(\Omega,g)$ be a compact Riemannian manifold with boundary $\partial\Omega$ 
and an internal rectifiable fracture set $\Gamma$ of class $C^2$. 
Consider the Laplace operator with Dirichlet boundary conditions on $\Omega\setminus\Gamma$. 
For a smooth test function $g:\mathbb{R}\to\mathbb{R}$ with compact support 
$\mathrm{supp}(g)\subset [-T,T]$ for some $T>0$, one has
\[
    \mathrm{Tr}\!\left(g(\sqrt{-\Delta})\right) 
    = A_{\mathrm{vol}}(g) + A_{\partial\Omega}(g) + A_{\Gamma}(g) + \mathcal{R}(g),
\]
where the coefficients $A_{\mathrm{vol}},A_{\partial\Omega},A_{\Gamma}$ are explicitly determined 
by the geometry of $\Omega$ and $\Gamma$, and the remainder satisfies
\[
    |\mathcal{R}(g)| 
    \;\leq\; C(\Omega,\Gamma)\,\kappa(\Gamma)\,\|g\|_{C^{d+3}}
        \Big( T^{d-2}\log(1+T) + e^{-c(\Omega,\Gamma) T}\Big).
\]
\end{theorem}

\begin{theorem}[B. Geometric Complexity Parameter] 
\label{thm:complexity}
The spectral contribution of the fracture set $\Gamma$ is quantified by the 
geometric complexity parameter
\[
    \kappa(\Gamma) \;=\; 
        \mathcal{H}^{d-1}(\Gamma) 
        + \int_\Gamma (1+|II(x)|^2)^{1/2}\,d\mathcal{H}^{d-1}(x) 
        + N_{\mathrm{comp}}(\Gamma),
\]
where $\mathcal{H}^{d-1}$ denotes the $(d-1)$-dimensional Hausdorff measure, 
$II(x)$ the second fundamental form, and $N_{\mathrm{comp}}(\Gamma)$ the number of connected components. 
All constants in trace expansions depend polynomially on $\kappa(\Gamma)$.
\end{theorem}

\begin{theorem}[C. Power-Saving Refinements] 
\label{thm:refinements}
Under the assumptions of Theorem~\ref{thm:trace}, assume exponential mixing 
of the geodesic flow on $(\Omega\setminus\Gamma,g)$ with rate $\beta>0$. 
Then for every $\varepsilon>0$ the remainder satisfies
\[
    |\mathcal{R}(g)| \;\leq\; 
    C_\varepsilon(\Omega,\Gamma)\,T^{d-2-\delta+\varepsilon},
\]
where 
\[
\delta \;=\; \min\!\Big(\tfrac{1}{2}-\theta,\;\tfrac{\beta}{4}\Big),
\]
$\theta$ is the best known bound toward the Ramanujan--Petersson conjecture, 
and $T$ is the support parameter from Theorem~\ref{thm:trace}. 
The exponent $\delta$ is sharp under the stated assumptions.
\end{theorem}

\begin{theorem}[D. Universality of the Litho-Ratio] 
\label{thm:universality}
Let $\{\,\Gamma_i\,\}_{i=1}^N$ be an ergodic sample of admissible fracture sets 
with respect to a probability measure on the space of $C^2$ rectifiable subsets. 
Then the litho-ratio $K_L$ converges almost surely to a universal limit $K_L^*$ as 
$N\to\infty$, with Gaussian fluctuations at rate $O(N^{-1/2})$, i.e.,
\[
    \sqrt{N}\,(K_L - K_L^*) \;\Longrightarrow\; \mathcal{N}(0,\sigma^2).
\]
\end{theorem}

\section*{Methodological Innovations}

\begin{itemize}
    \item Microlocal parametrix construction adapted to rectifiable singularities.
    \item Introduction of the geometric complexity parameter $\kappa(\Gamma)$.
    \item Extension of trace formulas to nonlinear operators, stochastic ensembles, 
    and homogenized limits.
\end{itemize}

\section*{Relation to Literature}

This work extends the classical results of Weyl~\cite{weyl1911}, 
Ivrii~\cite{ivrii1980}, and Safarov--Vassiliev~\cite{safarov1997} 
to domains with internal singularities. 
It contrasts with variational approaches to fracture 
(Bourdin--Francfort--Marigo~\cite{bourdin2000}, \cite{francfort1998}) 
by focusing on spectral invariants and universality phenomena.

\section*{Structure of the Monograph}

\begin{itemize}
    \item Chapter~1: Historical Context and Motivation.
    \item Chapter~2: Preliminaries and Framework.
    \item Chapters~3--5: Variational structures, microlocal analysis, trace formulas.
    \item Chapter~6: Ergodic limits and universality.
    \item Chapter~7: Homogenization and multiscale analysis.
    \item Chapter~8: Nonlinear and random extensions.
    \item Chapter~9: Canonical examples and sharpness.
    \item Chapter~10: Synthesis and outlook.
    \item Appendices A--H: Technical lemmas, expansions, bibliographic comparisons.
\end{itemize}

\section*{Implications}

The results include explicit constants, error bounds, and sharpness barriers. 
They establish a unified framework for spectral geometry on singular domains, 
with universality of $K_L^*$ across deterministic, stochastic, and ergodic settings. 
This provides a foundation for further developments in microlocal analysis, 
homogenization, and ergodic theory on non-smooth spaces.

%==============================================================================
% End of Executive Summary
%==============================================================================
