% ==================================================================
% 00-executive-summary.tex — Executive Summary
% Lithomathematics Monograph (Diamond Standard v3.0)
% ==================================================================

\chapter*{Executive Summary}
\addcontentsline{toc}{chapter}{Executive Summary}

\noindent
This monograph introduces and develops a new mathematical discipline: 
\emph{lithomathematics}, the study of variational--spectral invariants in 
fractured and ordered media. At its core lies the \emph{litho-ratio} $K_L$, 
a dimensionless invariant defined as the long-time balance between 
ordering flows and fracture-induced dissipation. The central goal of this work 
is to establish rigorous foundations for $K_L$ and to prove its stability 
under localization, homogenization, and spectral perturbations.

\medskip

\noindent
The monograph is structured according to the Diamond Standard v3.0 protocol 
(see \emph{Diamond Audit Protocol}). Each chapter delivers precise theorems, 
explicit hypotheses, quantitative estimates, and closes with a mandatory 
\emph{Diamond Audit} block (\emph{Spectral Closure, Horizon Expansion, 
Diamond Recap}).

% ---------------------------------------------------------------
\section*{Main Contributions}
\addcontentsline{toc}{section}{Main Contributions}

\begin{enumerate}[label=\textbf{C\arabic*}.]
  \item \textbf{Definition of the litho-ratio.} 
  We define $K_L(T)$ as the ratio of averaged creation vs.\ dissipation 
  fluxes in a coupled variational system 
  $(\mathcal{E}_{\text{ord}}, \mathcal{E}_{\text{br}})$. 
  We prove existence of the ergodic limit $K_L^*$ under explicit geometric 
  regularity, rectifiability, and mixing hypotheses.
  
  \item \textbf{Ergodic Limit Theorem.} 
  Theorem~\ref{thm:ergodic} establishes that $K_L(T) \to K_L^*$ almost surely, 
  with quantitative concentration bounds. Constants are traced explicitly to 
  geometric, analytic, and spectral parameters.

  \item \textbf{Localized Trace Formula.}
  For operators of the form $-\Delta_g + V$ on domains with rectifiable 
  fracture sets, we prove a localized trace formula with a decomposition into 
  volume, boundary, and fracture contributions, accompanied by polynomially 
  decaying remainder estimates. See Theorem~\ref{thm:trace}.
  
  \item \textbf{Homogenization Invariance.}
  In Theorem~\ref{thm:homogenization}, we show that $K_L^*$ is stable under 
  $\Gamma$-convergence of energies in oscillatory media, with explicit error 
  rates in the statistically homogeneous case.

  \item \textbf{Synthetic examples.}
  Canonical model domains (disk with radial crack, sphere with equatorial cut) 
  are analyzed, yielding explicit computations of $K_L$ and verifying the sharpness 
  of remainder bounds.
\end{enumerate}

% ---------------------------------------------------------------
\section*{Technical Specifications}
\addcontentsline{toc}{section}{Technical Specifications}

\paragraph{Hypotheses.}
All theorems state hypotheses in explicit form:
\begin{itemize}
  \item Compact $C^{2,\alpha}$ manifolds with Lipschitz boundaries.
  \item Fracture sets $(d-1)$-rectifiable with uniform Hausdorff bounds.
  \item Potentials $V \in L^\infty$.
  \item Mixing rates quantified either exponentially or polynomially.
\end{itemize}

\paragraph{Constants and remainders.}
All constants are written with explicit dependency lists 
(e.g., $C = C(\mathrm{Vol}(\Omega), \|V\|_{L^\infty}, M_\Gamma)$).
All remainders specify regimes and orders (e.g., 
$\mathcal{R} = O(\lambda^{-\delta})$, $\delta > 0$).

\paragraph{Spectral precision.}
Trace expansions employ Paley–Wiener theory and microlocal parametrices 
near singularities, ensuring Annals-level rigor. 
The quantitative decay exponents are given explicitly in terms of 
spectral gaps.

% ---------------------------------------------------------------
\section*{Novelty Compared to Contemporary Works}
\addcontentsline{toc}{section}{Novelty Compared to Contemporary Works}

\begin{itemize}
  \item \textbf{Phase-field fracture.} 
  Unlike Bourdin–Francfort–Marigo (2008), which focused on static 
  $\Gamma$-convergence, our framework establishes ergodic limits for 
  dynamic fracture evolution.
  
  \item \textbf{Spectral theory on singular domains.} 
  Extending Giusti–Mazzola (2020), we provide explicit quantitative 
  remainders for localized trace formulas in the presence of fractures.
  
  \item \textbf{Homogenization.} 
  Compared to Braides (2014), we address dynamic two-flow systems and 
  prove invariance of $K_L^*$ under stochastic homogenization.
\end{itemize}

% ---------------------------------------------------------------
\section*{Structure of the Monograph}
\addcontentsline{toc}{section}{Structure of the Monograph}

\begin{description}
  \item[Chapter 0.] \emph{Executive Summary, Reader’s Roadmap, Notation \& Glossary, Diamond Protocol.}
  \item[Chapter 1.] \emph{Introduction.} Motivation, historical background, 
  positioning within variational analysis and spectral theory.
  \item[Chapter 2.] \emph{Preliminaries.} Geometric measure theory, 
  $\Gamma$-convergence, and microlocal background.
  \item[Chapter 3.] \emph{Variational Framework.} Energy flows, fracture sets, 
  compactness lemmas.
  \item[Chapter 4.] \emph{Spectral Theory.} Operators with singular domains, 
  Paley–Wiener framework, microlocal parametrices.
  \item[Chapter 5.] \emph{Trace Formulas.} Localized expansions, volume/surface 
  decompositions, quantitative remainder bounds.
  \item[Chapter 6.] \emph{Invariant Ratio.} Definition of $K_L$, ergodic limit theorem.
  \item[Chapter 7.] \emph{Homogenization.} Invariance of $K_L^*$ under 
  $\Gamma$-convergence, stochastic models.
  \item[Chapter 8.] \emph{Synthetic Examples.} Canonical fractured geometries, 
  explicit computations.
  \item[Chapter 9.] \emph{Extensions.} Generalizations, open directions, but 
  no unresolved conjectures or applied keys.
  \item[Chapter 10.] \emph{Conclusion.} Spectral closure, horizon expansion, 
  diamond recap.
\end{description}

% ---------------------------------------------------------------
\section*{Compliance with Diamond Standard v3.0}
\addcontentsline{toc}{section}{Compliance with Diamond Standard v3.0}

Each chapter satisfies the following audit checkpoints:
\begin{enumerate}[label=\textbf{D\arabic*}.]
  \item \textbf{Spectral Closure.} Restates proven theorems with explicit bounds.
  \item \textbf{Horizon Expansion.} Indicates possible extensions under current assumptions.
  \item \textbf{Diamond Recap.} Checklist: numbering, cross-references, remainder precision, literature liaison.
\end{enumerate}

\bigskip
\noindent
\emph{This executive summary is binding: all subsequent sections must conform to its scope, terminology, and Annals-grade standards.}
