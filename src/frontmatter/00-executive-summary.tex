%==============================================================================
% File: 00-executive-summary.tex
%==============================================================================

\chapter*{Executive Summary}
\label{chap:executive-summary}

\section*{Overview}

This monograph develops the mathematical foundations of \emph{lithomathematics}, 
a framework extending spectral geometry to domains with internal singular 
structures. The primary motivation is that classical trace formulas and 
spectral asymptotics, as established by Weyl, Ivrii, and Safarov–Vassiliev, 
do not apply directly in the presence of internal fractures or singular sets. 
Our aim is to establish a coherent analytic and microlocal theory that 
quantifies spectral invariants in such settings, with explicit remainder 
estimates and universal limits.

\section*{Principal Results}

\begin{theorem}[A. Localized Trace Formula on Fractured Domains]
\label{thm:main-trace}
Let $(\Omega,g)$ be a compact $d$-dimensional Riemannian manifold with 
smooth boundary $\partial\Omega$ and an internal fracture set 
$\Gamma \subset \Omega$ of class $C^2$, rectifiable with finite 
$(d-1)$-dimensional Hausdorff measure. 
Consider the Laplace operator with Dirichlet boundary conditions on 
$\Omega\setminus\Gamma$. Then, for any $g \in C^\infty_c(\mathbb{R})$,
\[
    \mathrm{Tr}(g(\sqrt{-\Delta})) \;=\;
    A_{\mathrm{vol}}(g) \;+\; A_{\partial\Omega}(g) \;+\; A_{\Gamma}(g) \;+\; \mathcal{R}(g),
\]
where the coefficients are explicitly determined by the geometry of 
$\Omega$ and $\Gamma$, and the remainder satisfies
\[
    |\mathcal{R}(g)| \;\leq\;
    C(\Omega,\Gamma)\,\kappa(\Gamma)\,\|g\|_{C^{d+3}}\,
    \Big(T^{d-2}\log(1+T) + e^{-c(\Omega,\Gamma)T}\Big).
\]
\end{theorem}

\begin{theorem}[B. Geometric Complexity Parameter]
\label{thm:complexity}
The influence of the fracture set $\Gamma$ on spectral asymptotics is 
captured by the geometric complexity parameter
\[
    \kappa(\Gamma) \;=\;
        \mathcal{H}^{d-1}(\Gamma)
        + \int_\Gamma (1+|II(x)|^2)^{1/2}\,d\mathcal{H}^{d-1}(x)
        + N_{\mathrm{comp}}(\Gamma),
\]
where $\mathcal{H}^{d-1}$ denotes $(d-1)$-dimensional Hausdorff measure, 
$II(x)$ is the second fundamental form at $x\in\Gamma$, and 
$N_{\mathrm{comp}}(\Gamma)$ the number of connected components. 
All constants in the trace expansions depend polynomially on $\kappa(\Gamma)$.
\end{theorem}

\begin{theorem}[C. Power-Saving Refinements]
\label{thm:power-saving}
Assume the geodesic flow on $(\Omega\setminus\Gamma,g)$ is exponentially 
mixing with spectral gap parameter $\beta>0$, and let $\theta$ denote the 
Ramanujan--Petersson bound associated with relevant automorphic forms. 
Then the remainder exponent $\delta$ in Theorem~\ref{thm:main-trace} 
admits the power-saving estimate
\[
    \delta \;=\; \min\!\left(\tfrac{1}{2}-\theta,\;\tfrac{\beta}{4}\right).
\]
This exponent is sharp under the stated assumptions.
\end{theorem}

\begin{theorem}[D. Universality of the Litho-Ratio]
\label{thm:universality}
Define the litho-ratio $K_L$ as the normalized limit of fracture 
contributions in the trace expansion. 
Under an ergodic probability measure on the space of admissible 
fracture sets $\Gamma$, one has
\[
    K_L \;\to\; K_L^* \quad \text{almost surely},
\]
with Gaussian fluctuations at rate $O(N^{-1/2})$ for $N$ independent samples. 
The universal constant $K_L^*$ is independent of microscopic geometry.
\end{theorem}

\section*{Methodological Innovations}

\begin{itemize}
\item \textbf{Microlocal parametrix construction} adapted to domains with rectifiable singularities.
\item \textbf{Geometric complexity parameter} $\kappa(\Gamma)$ providing quantitative control over spectral remainders.
\item \textbf{Stochastic homogenization and ergodic limits} for fracture ensembles, proving universality of the litho-ratio $K_L$.
\end{itemize}

\section*{Relation to the Literature}

This work extends the classical spectral asymptotics of Weyl~\cite{Weyl1911}, 
Ivrii~\cite{Ivrii1980}, and Safarov–Vassiliev~\cite{SafarovVassiliev1997} 
to singular domains with internal discontinuities. 
Unlike variational fracture models (Bourdin–Francfort–Marigo~\cite{Bourdin2000,FrancfortMarigo1998}), 
our approach is analytic and spectral, focusing on invariants rather than 
energy minimization. Connections with homogenization theory 
(Cioranescu–Murat~\cite{CioranescuMurat1997}) and ergodic theorems 
(Lindenstrauss~\cite{Lindenstrauss2001}) are established.

\section*{Structure of the Monograph}

The work is organized as follows.  
Chapters~1–2 establish the analytic and variational framework.  
Chapters~3–5 derive the trace formulas, geometric invariants, and power-saving refinements.  
Chapters~6–8 extend the theory to ergodic limits, homogenization, and nonlinear settings.  
Chapter~9 presents canonical examples, and Chapter~10 synthesizes the results.  
Appendices contain technical lemmas, error maps, and bibliographic notes.

\section*{Implications}

The theory provides a rigorous foundation for spectral geometry in the 
presence of internal singularities. It yields explicit error bounds, 
sharpness barriers, and universal invariants, forming a coherent extension 
of spectral asymptotics into the setting of fractured domains. 

%==============================================================================
% End of 00-executive-summary.tex
%==============================================================================
