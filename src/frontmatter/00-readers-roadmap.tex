% ==================================================================
% File: src/frontmatter/00-readers-roadmap.tex
% Diamond Standard v3.0 — Brilliant Version
% ==================================================================

\chapter*{Reader's Roadmap}
\addcontentsline{toc}{chapter}{Reader's Roadmap}

% ---------- Abstract ----------
\begin{abstract}
This roadmap provides a structured guide for navigating the monograph
\emph{Lithomathematics: Variational--Spectral Invariants of Fractured Media}.
It explains how the chapters are organized, the logical dependencies among them,
and how the reader should interpret the Diamond Audit Protocol.
\end{abstract}

\noindent\textbf{Keywords:} lithomathematics, variational methods, spectral geometry, fracture sets, homogenization, ergodic theory

\noindent\textbf{MSC 2020:} 35J25, 35P20, 35B27, 74R10, 47A75

% ---------- Orientation ----------
\section*{Orientation}
The monograph is designed for readers with a strong background in
functional analysis, partial differential equations, and geometric measure theory.
It introduces a new discipline, \emph{lithomathematics}, built upon the interaction
between variational calculus, spectral theory, and fracture mechanics.

% ---------- Goals ----------
\section*{Goals}
\begin{enumerate}[label=G\arabic*.]
  \item Provide precise mathematical definitions of lithomathematical systems.
  \item Introduce the litho-ratio $K_L$ as a universal invariant.
  \item Establish quantitative trace formulas on fractured domains.
  \item Prove ergodic limit theorems with explicit convergence rates.
  \item Demonstrate homogenization invariance of $K_L^*$.
\end{enumerate}

% ---------- Invariants ----------
\section*{Invariants}
\begin{enumerate}[label=I\arabic*.]
  \item No hidden assumptions: all function spaces and regularity requirements are explicitly stated.
  \item Spectral operators are self-adjoint with rigorously defined domains.
  \item All remainder terms are given with explicit estimates and dependencies.
  \item Every chapter concludes with an Audit Recap and Sharpness Barriers.
  \item Full reproducibility: examples and computations can be independently verified.
\end{enumerate}

% ---------- Exposition Map ----------
\section*{Exposition Map}
\begin{description}
  \item[Chapter~0] Frontmatter: Executive Summary, Roadmap, Notation, Audit Protocol.
  \item[Chapter~1] Introduction: motivation, historical context, scope.
  \item[Chapter~2] Preliminaries: measure theory, $\Gamma$-convergence, spectral tools.
  \item[Chapter~3] Variational Framework: ordering vs.~fracture energies.
  \item[Chapter~4] Spectral Theory: operators on fractured domains.
  \item[Chapter~5] Localized Trace Formulas: volume--boundary--fracture decomposition.
  \item[Chapter~6] Invariant Ratio $K_L$: definition, properties, ergodic theorem.
  \item[Chapter~7] Homogenization: scale limits, invariance of $K_L^*$.
  \item[Chapter~8] Synthetic Examples: canonical geometries, explicit computations.
  \item[Chapter~9] Extensions: noncompact settings, polynomial mixing, stochastic models.
  \item[Chapter~10] Conclusion: spectral closure, horizon expansion, diamond audit recap.
\end{description}

% ---------- Audit Block ----------
\section*{Audit Block (Diamond Standard v3.0)}
\begin{enumerate}[label=A\arabic*.]
  \item Goals (G1--G5) are clearly enumerated.
  \item Invariants (I1--I5) are preserved throughout all chapters.
  \item Logical flow matches Exposition Map without hidden dependencies.
  \item Error Budget Map is provided per chapter, limiting possible ambiguities.
\end{enumerate}

% ---------- Links ----------
\section*{Links}
Backward: Executive Summary (overview of contributions). \\
Forward: Notation Glossary (formal symbols used in definitions).

% ---------- Error Map ----------
\section*{Error Budget Map}
\begin{itemize}
  \item Ambiguity in terminology --- eliminated by explicit definitions in Chapter~2.
  \item Hidden assumptions --- prevented by Invariant I1.
  \item Unstated dependencies --- avoided by explicit chapter links.
\end{itemize}

% ---------- Sharpness Barriers ----------
\section*{Sharpness Barriers}
The roadmap is valid under the following limitations:
\begin{itemize}
  \item Requires reader familiarity with $H^1$-spaces, rectifiable sets, and spectral theory.
  \item Ergodic results rely on exponential or polynomial mixing assumptions (explicit in Chapter~6).
  \item Homogenization results assume periodic or statistically homogeneous microstructures.
\end{itemize}

% ---------- Conclusion ----------
\section*{Conclusion}
This roadmap ensures that the reader can navigate the monograph without
ambiguity. Each chapter builds upon clearly stated prerequisites, concludes with
its own audit, and contributes to the global invariants. The structure is designed
to minimize cognitive overload while maintaining Annals-level rigor.
