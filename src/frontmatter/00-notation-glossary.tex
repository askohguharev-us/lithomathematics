%==============================================================================
% Notation and Glossary
%==============================================================================

\chapter*{Notation and Glossary}
\label{chap:notation}

This section collects the principal symbols, parameters, and conventions
used throughout the monograph. Cross-references to later sections are
provided where formal definitions appear.

\section*{Geometric Objects}

\begin{longtable}{p{0.18\textwidth} p{0.75\textwidth}}
$\Omega$ & Compact $d$-dimensional Riemannian manifold with boundary. \\

$\partial\Omega$ & Boundary of $\Omega$, assumed piecewise $C^\infty$. \\

$\Gamma$ & Internal fracture set, a compact rectifiable subset of $\Omega$ of class $C^2$, unless specified otherwise. \\

$g$ & Riemannian metric on $\Omega$. \\

$T^*\Omega$ & Cotangent bundle of $\Omega$; $S^*\Omega$ denotes the unit cotangent bundle. \\
\end{longtable}

\section*{Operators}

\begin{longtable}{p{0.18\textwidth} p{0.75\textwidth}}
$-\Delta$ & Laplace–Beltrami operator on $(\Omega \setminus \Gamma, g)$ with Dirichlet boundary conditions on $\partial\Omega \cup \Gamma$. \\

$\operatorname{Tr}$ & Trace functional. For the operators considered here, $\operatorname{Tr}(g(\sqrt{-\Delta}))$ is defined via the Schwartz kernel of the associated heat or wave semigroup. \\

$E_\lambda$ & Spectral projector associated with $-\Delta$. \\

$K(t,x,y)$ & Kernel of the heat semigroup $e^{t\Delta}$. \\
\end{longtable}

\section*{Parameters and Invariants}

\begin{longtable}{p{0.18\textwidth} p{0.75\textwidth}}
$T$ & Spectral scaling parameter. Asymptotic results are considered in the limit $T \to \infty$. The support of test functions $g$ is contained in $[-T,T]$. \\

$\kappa(\Gamma)$ & Geometric complexity parameter:
\[
\kappa(\Gamma) \;=\;
    \mathcal{H}^{d-1}(\Gamma) +
    \int_\Gamma (1+|II(x)|^2)^{1/2}\, d\mathcal{H}^{d-1}(x) +
    N_{\mathrm{comp}}(\Gamma).
\]
Here $\mathcal{H}^{d-1}$ is the Hausdorff measure, $II(x)$ the second fundamental form, and $N_{\mathrm{comp}}(\Gamma)$ the number of connected components. The integral term measures the total curvature (bending energy) of the fracture set. \\

$\beta$ & Exponential mixing rate for the geodesic flow on the unit cotangent bundle $S^*(\Omega \setminus \Gamma, g)$. \\

$\theta$ & Best known bound toward the Ramanujan–Petersson conjecture for cusp forms; defined precisely in Section~5.4. For practical purposes, $\theta = 0$ is always admissible. \\

$\delta$ & Power-saving exponent:
\[
\delta = \min\left( \tfrac{1}{2} - \theta,\; \tfrac{\beta}{4} \right).
\] \\

$K_L$ & Litho-ratio, a spectral invariant measuring the relative contribution of the fracture set $\Gamma$ in trace expansions. Defined precisely in Section~6.2. \\
\end{longtable}

\section*{Conventions}

\begin{itemize}
    \item \textbf{Constants.} Symbols $C(\Omega,\Gamma)$ denote constructive constants depending at most polynomially on $\kappa(\Gamma)$. 
    \item \textbf{Inequalities.} The notation $A \lesssim B$ means $A \leq C B$ for some constant $C$ depending only on $(\Omega,\Gamma)$.
    \item \textbf{Proof structure.} Longer proofs are divided into steps: setup, key estimate, iteration, conclusion. Technical lemmas are proved separately, and their logical role is summarized in remarks following proofs.
\end{itemize}

%==============================================================================
% End of Notation and Glossary
%==============================================================================
