%==============================================================================
% 00-notation-glossary.tex
%==============================================================================

\chapter*{Notation and Glossary}
\label{chap:notation-glossary}

This chapter collects the main notation, parameters, and conventions used throughout the monograph.  
All terms are defined consistently with the mathematical standards of spectral geometry, and each symbol is introduced once and used uniformly.  
Where appropriate, references are given to the chapters where the concept is developed in detail.

%------------------------------------------------------------------------------
\section*{Core Symbols and Parameters}

\begin{longtable}{|c|p{11cm}|}
\hline
$\Omega$ & Compact Riemannian manifold with smooth boundary $\partial\Omega$. \\
\hline
$\Gamma$ & Internal fracture set in $\Omega$, assumed to be rectifiable and of class $C^2$ unless otherwise specified. \\
\hline
$g$ & Test function, usually smooth and compactly supported; precise requirements (e.g., $g \in C_c^\infty(\mathbb{R})$ or $g \in \mathcal{S}(\mathbb{R})$) are stated where used. \\
\hline
$\Delta$ & Laplace–Beltrami operator on $\Omega$ with Dirichlet boundary conditions on $\partial\Omega$ and on the fracture set $\Gamma$. \\
\hline
$\operatorname{Tr}$ & Trace functional. For the operators considered here, $\operatorname{Tr}(g(\sqrt{-\Delta}))$ is defined via the Schwartz kernel of the heat semigroup or wave group. \\
\hline
$A_{\mathrm{vol}}(g)$ & Volume contribution in trace expansions (classical Weyl term). \\
\hline
$A_{\partial\Omega}(g)$ & Boundary contribution in trace expansions (Ivrii term). \\
\hline
$A_{\Gamma}(g)$ & Fracture contribution in trace expansions, expressed through integrals over $\Gamma$ involving its Hausdorff measure and second fundamental form. \\
\hline
$\mathcal{R}(g)$ & Remainder term in trace expansions. Bounds for $\mathcal{R}(g)$ are always given with explicit dependence on $T$, $g$, and $\kappa(\Gamma)$. \\
\hline
$T$ & Spectral parameter bounding the support of $g$: $\mathrm{supp}(g) \subset [-T,T]$. \\
\hline
$\theta$ & Best known bound toward the Ramanujan–Petersson conjecture for cusp forms; defined precisely in Section~5.4. For practical purposes, $\theta=0$ is always admissible. \\
\hline
$\beta$ & Exponential mixing rate for the geodesic flow on $(\Omega\setminus\Gamma,g)$. \\
\hline
$\delta$ & Error exponent in refined remainder bounds, defined by
\[
    \delta = \min\!\left(\tfrac{1}{2}-\theta,\;\tfrac{\beta}{4}\right).
\] \\
\hline
$\kappa(\Gamma)$ & Geometric complexity parameter of the fracture set $\Gamma$, defined as
\[
    \kappa(\Gamma) = 
    \mathcal{H}^{d-1}(\Gamma) 
    + \int_\Gamma (1+|II(x)|^2)^{1/2}\,d\mathcal{H}^{d-1}(x)
    + N_{\mathrm{comp}}(\Gamma),
\]
where $\mathcal{H}^{d-1}$ is the $(d-1)$-dimensional Hausdorff measure, $II(x)$ is the second fundamental form at $x \in \Gamma$, and $N_{\mathrm{comp}}(\Gamma)$ is the number of connected components of $\Gamma$. \\
\hline
$K_L$ & Litho-ratio: spectral invariant measuring the relative contribution of $\Gamma$ to the trace. Defined rigorously in Chapter~6. \\
\hline
$K_L^*$ & Universal ergodic limit of the litho-ratio, independent of microscopic randomness. \\
\hline
\end{longtable}

%------------------------------------------------------------------------------
\section*{Conventions}

\begin{itemize}
    \item \textbf{Constants.}  
    All constants $C(\Omega,\Gamma)$ are constructive and depend at most polynomially on $\kappa(\Gamma)$.  
    Dependence on dimension $d$ is always indicated explicitly when relevant.

    \item \textbf{Norms.}  
    We use $\| \cdot \|_{C^k}$ for the uniform $C^k$-norm of functions, and $\| \cdot \|_{L^p}$ for Lebesgue norms.

    \item \textbf{Measures.}  
    $\mathcal{H}^{d-1}$ denotes the $(d-1)$-dimensional Hausdorff measure. Integrals over $\Gamma$ are with respect to this measure unless otherwise specified.

    \item \textbf{Sharpness.}  
    Whenever an exponent or bound is claimed to be sharp, this is proved either by explicit examples (Chapter~9) or by general barrier arguments (Appendix~G).

    \item \textbf{Proof Strategy.}  
    Long proofs are structured in steps: setup, key estimate, iteration, conclusion. Technical lemmas are proved separately, often in appendices.

    \item \textbf{Cross-references.}  
    Each definition and symbol introduced here is cross-referenced in the main text. No symbol is used without prior definition.
\end{itemize}

%==============================================================================
% End of 00-notation-glossary.tex
%==============================================================================
