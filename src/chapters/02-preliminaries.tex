\chapter{Preliminaries: Geometry, Traces, Parametrices, and Spectral Tools on Manifolds with Internal Dirichlet Walls}
\label{ch:preliminaries}

% ============================================================
% Orientation (aligned with Chapter 0 and the Introduction)
% ============================================================

\section*{Orientation}
This chapter fixes the working geometric, functional, and microlocal framework for the entire monograph.
It is \emph{fully aligned} with Chapter~0:
we work on a compact Riemannian manifold $(M,g)$ with piecewise $C^2$ boundary $\partial M$ and a compact internal $C^2$ hypersurface $\Gamma\subset M$ of positive relative reach, on which a \emph{Dirichlet} constraint is imposed.
All operators are defined by closed Dirichlet forms on $H^1_0(M;\partial M\cup\Gamma)$ (no hidden $H^2$ assumptions).
Local parametrix constructions near $\Gamma$ use Fermi coordinates in a uniform tubular neighborhood (guaranteed by reach), together with a two–sided reflection scheme.
Propagation of singularities is specular at smooth points of $\partial M$ and $\Gamma$, and diffraction is confined to edge/corner strata; this does \emph{not} affect the leading interior surface density at order $\tau^{-(d-1)/2}$.

\paragraph{Structure of the chapter.}
\begin{enumerate}
  \item \S\ref{sec:geom-setting}: geometric setting, reach, tubular coordinates, measures.
  \item \S\ref{sec:traces}: trace operators and the Dirichlet space $H^1_0(M;\partial M\cup\Gamma)$.
  \item \S\ref{sec:litho-laplacian-prelim}: the litho–Laplacian via the Friedrichs form; spectral discreteness; componentwise splitting.
  \item \S\ref{sec:tubular-coarea}: coarea in the reach–tube; collar integrals used later for local heat coefficients.
  \item \S\ref{sec:microlocal-core}: minimal microlocal toolbox (WF, $\Psi$DOs, elliptic parametrices) adapted to an internal Dirichlet wall.
  \item \S\ref{sec:parametrix-gamma}: local Dirichlet parametrix near $\Gamma$ (two–sided reflection); error control uniform in the tube.
  \item \S\ref{sec:propagation}: specular reflection and propagation; grazing and singular sets; what diffracts and what does not.
  \item \S\ref{sec:heat-trace-prep}: heat kernel and trace-class bounds; local–to–global assembly; what is needed for Ch.~3.
  \item \S\ref{sec:localized-trace}: localized wave/trace expressions for Tauberian arguments (hooks to DYN in Ch.~6–7).
  \item \S\ref{sec:notation-audit}: consolidated notation, dependency map, and sharpness barriers.
\end{enumerate}

% ============================================================
\section{Geometric Setting, Measures, and Reach}\label{sec:geom-setting}

\subsection{Ambient manifold, boundary, and internal wall}
We adopt the Standing Assumptions (SA) of Chapter~0. For convenience we recall the core parts we \emph{use} here:
\begin{itemize}
  \item $(M,g)$ compact, connected, $\dim M=d\ge 2$, $\partial M$ of class piecewise $C^2$ with finitely many Lipschitz edges/corners.
  \item $\Gamma\subset M$ a compact embedded $C^2$ hypersurface (possibly with boundary), with $\partial\Gamma\subset\partial M$ and transverse intersection $T_p\Gamma+T_p(\partial M)=T_pM$ for $p\in\partial\Gamma$.
  \item \emph{Relative reach} $\mathrm{reach}_M(\Gamma)>0$ (see Ch.~0, Remark~\ref{rem:relative-reach}): the normal exponential map $\exp^\perp$ is injective on a uniform radius $r_*>0$.
\end{itemize}

\begin{remark}[Choice of global scale and invariance]
Whenever a global length $R$ is used (e.g.\ in $\kappa(\Gamma)$ of Ch.~0), it is taken to be $R:=\mathrm{diam}_g(M)$; any scale equivalent under homotheties (e.g.\ $\vol_d(M)^{1/d}$) would yield an equivalent normalization. Constants declared ``uniform in the tube'' are invariant under $g\mapsto \lambda^2 g$ after $t\mapsto \lambda t$ rescaling.
\end{remark}

\subsection{Measures and dimensional bookkeeping}
We write $\vol_d(\cdot)$ for the Riemannian $d$–volume and $\vol_{d-1}(\cdot)$ for the $(d-1)$–dimensional Hausdorff measure induced by $g$ on smooth hypersurfaces.
Dimensional units (cf.\ Ch.~0, \emph{Dimensional Convention}): $[\Delta]=L^{-2}$, $[\tau]=L^2$ (heat time), and $[t]=L$ (geometric time).
The domain scale $R:=\mathrm{diam}_g(M)$ will normalize certain dimensionless quantities (e.g.\ $\kappa(\Gamma)$ in Ch.~0).

\subsection{Tubular coordinates and Fermi charts}\label{subsec:fermi}
By $\mathrm{reach}_M(\Gamma)>0$ there exists $r_*\in(0,1]$ (rescaled to $R=1$ if needed) such that the normal exponential map
\[
\exp^\perp:\{(p,\nu)\in N\Gamma: |\nu|<r_*\}\longrightarrow \mathcal{T}_{r_*}(\Gamma):=\{x\in M: \mathrm{dist}_g(x,\Gamma)<r_*\}
\]
is a $C^{1,1}$-diffeomorphism.
In $\mathcal{T}_{r_*}(\Gamma)$ we use Fermi coordinates $(y,s)$, $y\in\Gamma$, $s\in(-r_*,r_*)$, where $s$ is the signed distance along the unit normal $\nu$ and $y$ are local coordinates on $\Gamma$.
The metric admits the standard expansion
\[
g(y,s)= g_\Gamma(y) \oplus \big(1-\mathcal{S}_y s\big)^2\,ds^2 + O(s^2),
\]
with $\mathcal{S}_y$ the shape operator at $y$; all estimates below are uniform for $|s|<r_*/2$.

\begin{lemma}[Jacobian in the reach–tube: uniform bounds]\label{lem:tube-jac}
In Fermi coordinates the volume element satisfies
\[
d\vol_d = J(y,s)\, d\vol_{d-1}(y)\, ds,\qquad
J(y,s)=1-s\,H_\Gamma(y)+O(s^2),
\]
with the $O(s^2)$ uniform for $|s|<r_*/2$.
The error constant depends only on $\|A_\Gamma\|_{C^0}$ in the tube and curvature bounds of $g$ therein.
\end{lemma}

\begin{proof}[Sketch]
Write $g(y,s)=g_\Gamma(y)+2s\,\mathrm{II}_\Gamma(y)+s^2 G_2(y,s)$ with $\mathrm{II}_\Gamma$ the second fundamental form and $G_2$ uniformly bounded in the tube.
Then $\det g(y,s)=\det g_\Gamma(y)\,\det(I-s\,\mathcal S_y+O(s^2))$ with $\mathcal S_y$ the shape operator.
Expanding $\det(I-s\,\mathcal S_y)=1-s\,\mathrm{tr}\,\mathcal S_y+O(s^2)=1-s\,H_\Gamma(y)+O(s^2)$ gives the claim, uniformly by reach-based tubular control.
\end{proof}

\begin{remark}[Minkowski content vs.\ surface measure]
In the $C^2$+reach regime the symmetric Minkowski content of $\Gamma$ exists and equals $\vol_{d-1}(\Gamma)$ up to the standard normalization; throughout the monograph we take $\vol_{d-1}$ as the \emph{primary} surface measure and use Minkowski language only heuristically.
\end{remark}

\subsection{Coarea in the tube and level sets of distance}\label{sec:tube-coarea}
Let $\rho(x):=\mathrm{dist}_g(x,\Gamma)$ in $\mathcal{T}_{r_*}(\Gamma)$ with the sign chosen by the normal.
Then $\rho$ is $C^{1,1}$ and $|\nabla\rho|=1$ a.e.\ in the tube.
For $0<\varepsilon<r_*/2$ and $f\in L^1(\mathcal{T}_{\varepsilon}(\Gamma))$ the coarea formula gives
\[
\int_{\mathcal{T}_\varepsilon(\Gamma)} f(x)\,d\vol_d(x)
\,=\, \int_{-\varepsilon}^{\varepsilon} \left(\int_{\{\rho=s\}} f\, d\vol_{d-1}\right)\,ds.
\]
We will apply this \emph{only} within the reach tube; outside we never integrate against distance levels.

\begin{remark}[Uniformity from reach]\label{rem:reach-uniformity}
The positivity $\mathrm{reach}_M(\Gamma)\ge r_*$ implies existence of Fermi charts on $\mathcal{T}_{r_*}(\Gamma)$ and uniform control of $J(y,s)$ in Lemma~\ref{lem:tube-jac}. In particular, for $|s|<r_*/2$,
\[
|J(y,s)-[1-sH_\Gamma(y)]|\le C\, s^2,\qquad C=C\big(\|A_\Gamma\|_{C^0},\|{\rm Rm}_g\|_{C^0},r_*^{-1}\big).
\]
\end{remark}

\subsection*{Mini–audit (Geometry)}
\begin{itemize}
  \item Uniform tubular geometry secured by reach; Fermi charts fixed.
  \item Measures normalized; dimensional bookkeeping consistent with Ch.~0.
  \item Coarea confined to $\mathcal{T}_{r_*}$ (no global use of distance levels).
\end{itemize}

% ============================================================
\section{Trace Operators and the Dirichlet Space}\label{sec:traces}

\subsection{Traces on $\partial M$ and on $\Gamma$}
Under the standing regularity (piecewise $C^2$ boundary, $\Gamma$ $C^2$) and compactness of $M$, the trace maps
\[
\gamma_{\partial M}:H^1(M)\to L^2(\partial M),\qquad
\gamma_\Gamma:H^1(M\setminus\Gamma)\to L^2(\Gamma)
\]
are continuous.
We define the Dirichlet space
\[
H^1_0(M;\partial M\cup\Gamma):=\{u\in H^1(M):\ \gamma_{\partial M}(u)=0,\ \gamma_\Gamma(u)=0\}.
\]

\begin{remark}[Precise trace references]\label{rem:grisvard-cites}
Continuity of $\gamma_{\partial M}$ and $\gamma_{\Gamma}$, and the identification
\[
H^1_0(M;\partial M\cup\Gamma)=\overline{C^\infty_0(M\setminus(\partial M\cup\Gamma))}^{\,H^1},
\]
follow from standard results on traces and closures in nonsmooth (piecewise $C^2$/Lipschitz) settings; see, e.g., \cite[Thm.~1.5.1–1.5.2, §1.5;\; Ch.~4]{Grisvard1985}.%
\footnote{We only use the form domain $H^1_0$; no global $H^2$ up to the boundary/wall is invoked anywhere in this chapter.}
\end{remark}

\begin{remark}[Closure by smooth functions vanishing on the wall]
$H^1_0(M;\partial M\cup\Gamma)$ is the closure of $C^\infty(M)$-functions vanishing on $\partial M\cup\Gamma$ in the $H^1$ norm.
We never require $H^2$ regularity up to $\partial M$ or $\Gamma$.
\end{remark}

\begin{remark}[Dependence of constants]\label{rem:constants}
Whenever a constant is declared ``uniform'' in this chapter, it depends only on $d$, curvature bounds of $g$ on a fixed collar of $\partial M\cup\Gamma$, the $C^2$-norms of $\partial M$ and $\Gamma$ there, and the tube radius $r_*$.
\end{remark}

\subsection*{Mini–audit (Traces)}
\begin{itemize}
  \item Trace continuity stated and used; no extra geometric hypotheses needed.
  ̇\item Dirichlet space fixed; all later variational identities are taken componentwise on $M\setminus\Gamma$.
\end{itemize}

% ============================================================
\section{The Litho–Laplacian via the Friedrichs Form}\label{sec:litho-laplacian-prelim}

\subsection{Closed form and self–adjoint operator}
Let
\[
Q[u]:=\int_M |\nabla u|_g^2\,d\vol_d,\qquad \Dom(Q):=H^1_0(M;\partial M\cup\Gamma).
\]
$Q$ is densely defined, nonnegative, and closed on $L^2(M)$.
By the representation theorem, there exists a unique nonnegative self–adjoint operator $L_\Gamma$ such that
\[
\langle L_\Gamma u, v\rangle=\int_M \langle \nabla u,\nabla v\rangle_g\,d\vol_d\quad \text{for all } u\in \Dom(L_\Gamma),\ v\in \Dom(Q).
\]
Equivalently, $L_\Gamma$ acts as $-\Delta_g$ on each connected component $U_j$ of $M\setminus\Gamma$ in the distributional sense, with $u|_{\partial M\cup\Gamma}=0$ in trace sense.

\begin{remark}[No hidden $H^2$ assumptions]
We never identify $\Dom(L_\Gamma)$ with $H^2$ up to the boundary/wall. All statements use the form method and the $L^2$ graph characterization on $M\setminus\Gamma$.
\end{remark}

\subsection{Compact resolvent and spectral discreteness}
Since $M$ is compact and $H^1_0(M;\partial M\cup\Gamma)\hookrightarrow L^2(M)$ is compact, the resolvent $(L_\Gamma+I)^{-1}$ is compact; hence the spectrum is discrete with finite multiplicities:
\[
0<\lambda_1\le \lambda_2\le \cdots \nearrow \infty,\qquad \{ \phi_k\}_{k\ge1}\ \text{orthonormal basis of }L^2(M).
\]

\subsection{Componentwise splitting}\label{subsec:component-split}
Let $M\setminus\Gamma=\bigsqcup_{j=1}^J U_j$ be the open decomposition into connected components. Then
\[
L^2(M)=\bigoplus_{j=1}^J L^2(U_j),\qquad
L_\Gamma \cong \bigoplus_{j=1}^J L^{(j)}_{\mathrm{Dir}},
\]
where $L^{(j)}_{\mathrm{Dir}}$ is the Dirichlet Laplacian on $U_j$ with Dirichlet on $(\partial M\cap\partial U_j)\cup(\Gamma\cap\partial U_j)$.

\subsection*{Mini–audit (Operator)}
\begin{itemize}
  \item $L_\Gamma$ defined by the closed form; positivity and self–adjointness established.
  \item Discreteness of spectrum recorded; no $H^2$-up-to-boundary claims.
  \item Orthogonal splitting across $M\setminus\Gamma$ fixed for later use.
\end{itemize}

% ============================================================
\section{Tubular Coarea and Collar Integrals}\label{sec:tubular-coarea}

Let $0<\varepsilon<r_*/2$.
For $f\in L^1(\mathcal{T}_\varepsilon(\Gamma))$, using \S\ref{sec:tube-coarea}:
\[
\int_{\mathcal{T}_\varepsilon(\Gamma)} f\,d\vol_d
=\int_{-\varepsilon}^{\varepsilon}\!\Big(\int_{\{\rho=s\}} f\, d\vol_{d-1}\Big)\,ds.
\]
For $f$ smooth and compactly supported in the tube, the Jacobian in Fermi coordinates admits a curvature expansion
\[
d\vol_d = J(y,s)\, d\vol_{d-1}(y)\, ds,\qquad J(y,s)=1-s\,H_\Gamma(y)+O(s^2),
\]
with $H_\Gamma$ the mean curvature (trace of the shape operator). We will only use the $s$–even/odd structure to show that curvature corrections enter at the next order $\tau^{-(d-2)/2}$ (Ch.~3).

\begin{remark}[Even–odd cancellation at leading order]
Terms odd in $s$ integrate to zero in symmetric collars; consequently the $\tau^{-(d-1)/2}$ surface density is curvature–independent, while curvature enters at $\tau^{-(d-2)/2}$ via even corrections and symbol terms (see also \cite[Ch.~1]{Gilkey1995}).
\end{remark}

\subsection*{Mini–audit (Collar calculus)}
\begin{itemize}
  \item Coarea and Jacobian expansions stated in the uniform tube.
  \item Even–odd split recorded for later cancellation arguments at the leading order.
\end{itemize}

% ============================================================
\section{Minimal Microlocal Toolkit}\label{sec:microlocal-core}

\subsection{Wavefront sets and microlocal smoothness}
For $u\in\mathcal{D}'(M)$ the wavefront set $\WF(u)\subset T^*M\setminus 0$ is defined in the usual way by localizing with $\varphi\in C_c^\infty$ and testing decay of the localized Fourier transform on conic neighborhoods.
We will use:
\begin{itemize}
  \item $\WF(u)=\varnothing \iff u\in C^\infty$;
  \item If $A\in\Psi^m$ elliptic at $(x,\xi)$ and $Au\in C^\infty$ near $x$, then $(x,\xi)\notin\WF(u)$.
\end{itemize}
\begin{example}
For $u(x)=|x|$ on $\mathbb{R}$ one has $\WF(u)=\{(0,\xi):\xi\neq 0\}$.
\end{example}

\subsection{Pseudodifferential operators and elliptic parametrices}
We use standard $\Psi$DO classes on manifolds; if $A\in\Psi^m$ is elliptic, there exists $B\in\Psi^{-m}$ and $R,R'\in \Psi^{-\infty}$ such that
\[
BA=I-R,\qquad AB=I-R'.
\]
All constructions are local and compatible with charts that respect the piecewise $C^2$ boundary; near $\Gamma$ we work in Fermi charts and microlocalize away from the $\mu$–null grazing set.

\begin{remark}[Chart–independence of microlocal norms]
All seminorms we use on symbols and amplitudes are taken in a finite atlas adapted to collars of $\partial M\cup\Gamma$; equivalence of such norms follows from compactness and uniform $C^2$ bounds in the tube (cf.\ Remark~\ref{rem:constants}).
\end{remark}

\begin{remark}[Terminology]
We use \emph{grazing set} synonymously with \emph{glancing set} (covectors tangent to $\partial M\cup\Gamma$ at first contact).
\end{remark}

\subsubsection*{Symbol classes and grazing cutoffs (working convention)}
We fix the standard symbol classes $S^{m}_{1,0}(T^*M)$: $a\in S^{m}_{1,0}$ if for every chart and all multiindices $\alpha,\beta$
\[
|\partial_x^\alpha\partial_\xi^\beta a(x,\xi)|\le C_{\alpha\beta}\,\langle\xi\rangle^{m-|\beta|},\qquad \langle\xi\rangle=(1+|\xi|^2)^{1/2}.
\]
Microlocalizations near $\Gamma$ are always taken with an angular cutoff eliminating glancing directions: for a unit normal $\nu$ (in a Fermi chart) we insert a smooth $\chi_\theta(\xi\cdot\nu)$ with $\chi_\theta=0$ on $|\xi\cdot\nu|\le \theta |\xi|$ and $\chi_\theta=1$ on $|\xi\cdot\nu|\ge 2\theta |\xi|$, for some fixed small $\theta>0$.
This convention keeps $\Psi$DO amplitudes uniformly controlled in the tube and avoids degeneracy of the reflected canonical relation at the grazing set; cf.\ \cite{HormanderI,HormanderIII}.

\subsection*{Addendum: references and functional setting for $\Psi$DOs}
We recall the standard $S^{m}_{1,0}$ symbol classes and $\Psi^m$-calculus on manifolds with piecewise $C^2$ boundary (charts adapted to collars of $\partial M\cup\Gamma$); see H{\"o}rmander~\cite{HormanderI, HormanderIII} for symbol classes and calculus, and Duistermaat–H{\"o}rmander~\cite{DuistermaatHormander} for propagation at real principal type. Boundary-compatible localizations follow the collar reduction; near corner strata we only microlocalize away from those sets.

\begin{proposition}[Elliptic parametrix implies microlocal regularity]\label{prop:elliptic-ml}
Let $A\in\Psi^m(M)$ be elliptic at $(x,\xi)\in T^*M\setminus 0$, with domain defined by the quadratic form near $\partial M\cup\Gamma$. If $Au\in C^\infty$ microlocally near $x$, then $(x,\xi)\notin \WF(u)$. Moreover, there exists $B\in\Psi^{-m}$ with $BA=I-R$, $AB=I-R'$ and $R,R'\in\Psi^{-\infty}$ supported away from the grazing set.
\end{proposition}

\begin{example}[Internal Dirichlet wall: absence of transmission]\label{ex:no-transmission}
In a Fermi chart across $\Gamma$ the principal symbol of $-\Delta_g$ is $p(x,\xi)=|\xi|_g^2$. Dirichlet reflection maps $(x,\xi_n,\xi')\mapsto (x,-\xi_n,\xi')$ at $s=0$; bicharacteristics reflect specularly and there is \emph{no} transmission component in the canonical relation. This underlies the absence of interface-dependent leading terms at order $\tau^{-(d-1)/2}$.
\end{example}

\begin{remark}[Propagation at real principal type]
Away from corner/edge strata and the grazing set, propagation follows Duistermaat–H{\"o}rmander along the Hamiltonian flow of $p(x,\xi)$, with reflections encoded by $\xi\mapsto \xi-2(\xi\cdot\nu)\nu$ at $\partial M\cup\Gamma$.
\end{remark}

\subsection*{Mini–audit (Microlocal core)}
\begin{itemize}
  \item $\WF$-calculus and elliptic parametrix available in our setting.
  \item Globalization via partition of unity within the tube and away from it.
\end{itemize}

% ============================================================
\section{Local Dirichlet Parametrix Near the Internal Wall}\label{sec:parametrix-gamma}

\subsection{Two–sided reflection in Fermi coordinates}
In a Fermi chart $(y,s)$ with $\Gamma=\{s=0\}$, the leading–order flat model is the Dirichlet heat kernel in a half–space, obtained by reflection:
\[
H^{\rm Dir}_{\R^d}(z,z';\tau)=H_{\R^d}(z-z';\tau)-H_{\R^d}(z-\tilde z';\tau),
\]
with $\tilde z'$ the mirror image across $\{s=0\}$.
On $(M,g)$, flattening the metric and using Levi parametrix yields, for $|s|,|s'|<\varepsilon\ll r_*$,
\[
H_\Gamma(x,x';\tau)=H^{\rm Dir}_{\mathrm{flat}}(x,x';\tau)+E(x,x';\tau),
\]
where $E$ is a smooth remainder enjoying Gaussian off–diagonal decay and uniform bounds as $\tau\downarrow 0$, with constants controlled by curvature bounds and the second fundamental form in the tube.

\begin{proposition}[Uniform remainder bounds in the tube]\label{prop:uniform-E}
Fix $\varepsilon\in(0,r_*/4]$. For all multi-indices $\alpha,\beta$ with $|\alpha|+|\beta|\le k_0$ there exist $C_{\alpha,\beta},c>0$ (depending only on $d$, $k_0$, $\|{\rm Rm}_g\|_{C^0}$, $\|A_\Gamma\|_{C^0}$, and $r_*^{-1}$) such that for $x,x'\in \mathcal{T}_\varepsilon(\Gamma)$ and $0<\tau\le 1$,
\[
\big|\partial_x^\alpha \partial_{x'}^\beta E(x,x';\tau)\big|
\;\le\; C_{\alpha,\beta}\,\tau^{-\frac{d-2+|\alpha|+|\beta|}{2}}
\,\exp\!\Big(-\,c\,\frac{d_g(x,x')^2}{\tau}\Big).
\]
The same bound holds with $d_g$ replaced by the Fermi-distance $(|y-y'|^2+|s-s'|^2)^{1/2}$ up to uniform equivalence of metrics in the tube.
\end{proposition}

\begin{proof}[Sketch]
Apply the Levi construction with coefficients frozen along normal geodesics and patch with a $\Psi$DO cutoff that removes glancing directions (the $\chi_\theta$ described above). Standard parametrix iteration yields a Neumann-series remainder with Gaussian kernel bounds; differentiation under the integral is controlled by symbol seminorms in $S^{m}_{1,0}$ and the tube-uniform equivalence of charts.
\end{proof}

\begin{remark}[Model provenance and literature anchors]
The interior Dirichlet parametrix is a two–sided variant of the half–space method of images combined with a Levi patching procedure; compare the boundary case in \cite[§1.5]{Gilkey1995} and the global assembly of local parametrices in \cite[Ch.~5]{SafarovVassiliev1997}. Corner/edge modifications follow the spirit of \cite{Grieser2002}; we use only smooth interior points here.
\end{remark}

\subsection{Consequences for local densities}
Integrating the pointwise deficit against $1$ in a symmetric collar and using the even–odd expansion in $s$ shows that the \emph{interior} surface density at order $\tau^{-(d-1)/2}$ equals
\[
-\tfrac14(4\pi)^{-(d-1)/2}\, \vol_{d-1}(\Gamma),
\]
independent of curvature at this order; curvature enters at the next order via the Jacobian and symbol corrections.

\begin{corollary}[Curvature enters one order later]
Under the standing assumptions, all curvature contributions to $\Tr e^{-\tau L_\Gamma}$ from the collar $\mathcal T_{r_*}(\Gamma)$ appear first at order $\tau^{-(d-2)/2}$; the $\tau^{-(d-1)/2}$ surface density is curvature-free and equals $-\tfrac14(4\pi)^{-(d-1)/2}\vol_{d-1}(\Gamma)$.
\end{corollary}

\subsection*{Mini–audit (Parametrix)}
\begin{itemize}
  \item Two–sided Dirichlet parametrix constructed in the uniform tube (reach).
  \item Error terms controlled uniformly; sufficient for assembling the global short–time expansion in Ch.~3.
\end{itemize}

% ============================================================
\section{Propagation, Reflection, and the Singular Set}\label{sec:propagation}

\subsection{Reflecting flow and \texorpdfstring{$\mu$}{mu}–null singularities}
Let $\varphi^t$ denote the billiard–type reflecting flow on $S^*(M\setminus\Gamma)$ defined for $\mu$–a.e.\ initial covector (Liouville measure).
The singular set (grazing/tangent hits, edges/corners, $\partial\Gamma$) is $\mu$–null; on the regular set $S^*_{\mathrm{reg}}$ the flow preserves $\mu$.

\begin{lemma}[A.e.\ invariance and Liouville preservation]\label{lem:liouville-preserve}
On $S^*_{\mathrm{reg}}$ the flow $\varphi^t$ is defined for all $t\in\R$, $\mu$–a.e., and preserves $\mu$: for all Borel $E\subset S^*_{\mathrm{reg}}$, $\mu(\varphi^{-t}E)=\mu(E)$.
\end{lemma}

\subsection{Specular reflection and no transmission}
At a smooth point of $\Gamma$ with normal $\nu$,
\[
(x,\xi)\mapsto \big(x,\xi-2(\xi\cdot\nu)\nu\big).
\]
There is \emph{no cross–interface transmission of singularities} (Dirichlet wall).
Diffraction occurs only at $\partial\Gamma\cup(\Gamma\cap\partial M)$ and at corner/edge strata of $\partial M$ but does not affect the leading interior surface density.

\begin{remark}[Scope of correlation/mixing statements]\label{rem:scope-cor}
All dynamical quantities in later Tauberian steps are computed on $S^*_{\mathrm{reg}}$; the grazing/corner strata are $\mu$–null and never enter correlation integrals. For billiard backgrounds see, e.g., \cite{ChernovMarkarian}.
\end{remark}

\subsection*{Mini–audit (Propagation)}
\begin{itemize}
  \item Reflecting flow defined $\mu$–a.e.; measure preservation recorded.
  \item Specular law; diffraction localized on lower–dimensional strata.
\end{itemize}

% ============================================================
\section{Heat Kernel and Trace–Class Preliminaries}\label{sec:heat-trace-prep}

\subsection{Trace class and basic bounds}
Since $L_\Gamma\ge 0$ has compact resolvent, $e^{-\tau L_\Gamma}$ is trace class for every $\tau>0$ and
\[
\Tr e^{-\tau L_\Gamma}=\sum_{k\ge1} e^{-\tau \lambda_k}.
\]
On-diagonal bounds in the tube and away from it follow from the parametrix and standard heat kernel estimates on compact manifolds with boundary (Gaussian bounds and derivative estimates).\footnote{See, e.g., \cite[Thm.~1.3.3]{Davies1989} for Gaussian upper bounds, boundary adaptations via reflection in \cite[§1.5]{Gilkey1995}, and polyhedral/corner refinements in \cite{Grieser2002}. We use only short-time near-diagonal control and finite-time off-diagonal decay.}

\subsection{Local–to–global assembly}
A partition of unity subordinate to (i) a collar of $\partial M$, (ii) a collar of $\Gamma$ within $\mathcal{T}_{r_*}$, and (iii) the interior region away from both, together with the corresponding model kernels, yields the short–time asymptotic expansion of the heat trace stated in Ch.~3.
The \emph{interior} surface density at $\tau^{-(d-1)/2}$ equals $-\frac14(4\pi)^{-(d-1)/2}\,\vol_{d-1}(\Gamma)$.

\begin{remark}[What we \emph{do not} need]
We never require global wave parametrics at glancing or exact diffraction analysis at corners for the leading interior term; those phenomena contribute at orders $\le \tau^{-(d-2)/2}$ and are addressed separately in Ch.~3.
\end{remark}

\paragraph{Forward hooks into Chapters 3 and 6–7.}
The collar assembly above feeds directly into: (i) the proof of the universal interior coefficient in Chapter~3 (main theorem on the $\tau^{-(d-1)/2}$ density), where Proposition~\ref{prop:uniform-E} provides the error control needed for stationary phase in the tube; and (ii) the localized wave/trace formulae of \S\ref{sec:localized-trace}, used for Abelian/Tauberian transfer in the dynamical remainder analysis of Chapters~6–7.

\subsection*{Mini–audit (Heat)}
\begin{itemize}
  \item Trace class established; kernel bounds adequate for asymptotics.
  \item Collar partition fixed; leading interior density isolated.
\end{itemize}

% ============================================================
\section{Localized Wave/Trace Expressions}\label{sec:localized-trace}

\subsection{Spectral calculus and wave kernel}
For $g\in\mathcal{S}(\mathbb{R})$ even with compactly supported $\widehat g$, the spectral multiplier $g(\sqrt{L_\Gamma})$ has kernel
\[
K_g(x,y)=\frac{1}{2\pi}\int_{\mathbb{R}}\widehat g(t)\,\cos(t\sqrt{L_\Gamma})(x,y)\,dt,
\]
and
\[
\Tr\,g(\sqrt{L_\Gamma})
=\frac{1}{2\pi}\int_{\mathbb{R}}\widehat g(t)\,\Tr\big(\cos(t\sqrt{L_\Gamma})\big)\,dt.
\]

\subsection{Cutoffs and microlocalization}
Let $\chi_\Gamma$ be a smooth cutoff supported in $\mathcal{T}_{r_*/4}(\Gamma)$, equal to $1$ on $\mathcal{T}_{r_*/8}(\Gamma)$.
Then
\[
\Tr\big(\chi_\Gamma\,g(\sqrt{L_\Gamma})\big)=\frac{1}{2\pi}\int \widehat g(t)\,\Tr\big(\chi_\Gamma\,\cos(t\sqrt{L_\Gamma})\big)\,dt,
\]
and the kernel is governed by the local two–sided parametrix, enabling extraction of the interior surface contribution.
Analogous cutoffs away from $\Gamma$ recover the boundary and bulk terms.

\begin{remark}[Hooks to Tauberian transfer]
The localized representation above is exactly what is needed to pass from short–time expansions to spectral counting via Abelian/Tauberian arguments à la \cite{SafarovVassiliev1997}.
\end{remark}

\subsection*{Mini–audit (Localized trace)}
\begin{itemize}
  \item Localized trace identities set; cutoffs arranged by regions.
  \item Ready–to–use hooks for Tauberian arguments in Ch.~6–7 (DYN).
\end{itemize}

% ============================================================
\section{Notation, Dependencies, and Sharpness Barriers}\label{sec:notation-audit}

\subsection{Notation table}
\begin{center}
\renewcommand{\arraystretch}{1.12}
\begin{tabular}{|l|l|}
\hline
Symbol & Meaning \\
\hline
$\vol_d$, $\vol_{d-1}$ & Riemannian $d$–volume and $(d\!-\!1)$–area \\
$R$ & $\mathrm{diam}_g(M)$ \\
$r_*$ & reach–tube radius for $\Gamma$ (injectivity of $\exp^\perp$) \\
$\rho$ & signed distance to $\Gamma$ in the reach tube \\
$H_\Gamma$ & mean curvature of $\Gamma$ (trace of shape operator) \\
$A_\Gamma$ & second fundamental form of $\Gamma$ \\
$H^1_0(M;\partial M\cup\Gamma)$ & Dirichlet Sobolev space with zero trace on $\partial M\cup\Gamma$ \\
$Q[u]$ & Dirichlet form $\int_M|\nabla u|^2\,d\vol_d$ on $H^1_0$ \\
$L_\Gamma$ & litho–Laplacian, positive self–adjoint by Friedrichs \\
$U_j$ & connected components of $M\setminus\Gamma$ \\
$S^*(X)$ & unit cotangent bundle of $X$; $\mu$ Liouville probability \\
$S^*_{\reg}$ & regular phase space (grazing/corner sets removed, $\mu$–null) \\
\hline
\end{tabular}
\end{center}

\subsection{Dependency map}
\begin{itemize}
  \item \emph{Geometry} (\S\ref{sec:geom-setting}) $\Rightarrow$ \emph{Traces} (\S\ref{sec:traces}) $\Rightarrow$ \emph{Form/Operator} (\S\ref{sec:litho-laplacian-prelim}).
  \item \emph{Reach/Fermi} (\S\ref{subsec:fermi}, \S\ref{sec:tube-coarea}) $\Rightarrow$ \emph{Local parametrix} (\S\ref{sec:parametrix-gamma}).
  \item \emph{Microlocal core} (\S\ref{sec:microlocal-core}) $+$ \emph{Propagation} (\S\ref{sec:propagation}) $\Rightarrow$ \emph{Localized trace} (\S\ref{sec:localized-trace}).
\end{itemize}

\subsection{Sharpness barriers}
\begin{itemize}
  \item If $\mathrm{reach}_M(\Gamma)=0$ (highly oscillatory or self–approaching walls), the uniform Fermi tube fails; local parametrix may not globalize and the clean surface density requires separate analysis.
  \item If $\Gamma$ is less than $C^{1,1}$, the signed distance lacks $C^{1,1}$ regularity; coarea along levels of $\rho$ becomes delicate; leading density is expected locally but error control deteriorates.
  \item At $\partial\Gamma\cup(\Gamma\cap\partial M)$ diffractive phenomena enter; they do \emph{not} contribute at order $\tau^{-(d-1)/2}$ but create subleading corner/edge terms (tracked in Ch.~3).
\end{itemize}

% ============================================================
\section*{Contrast Box: Why We Do \emph{Not} Use Transmission Here}
\addcontentsline{toc}{section}{Contrast Box: Why We Do \emph{Not} Use Transmission Here}
Transmission/impedance models impose continuity/jump conditions across $\Gamma$
\[
[u]_\Gamma=0,\qquad [\partial_\nu u]_\Gamma=\kappa\, u|_\Gamma,
\]
yielding leading coefficients that \emph{depend on interface parameters} $\kappa$; a parameter–free universal surface law is lost by construction.
Lithomathematics, as developed here, places a \emph{non–transmitting Dirichlet wall} at the core: $u|_\Gamma=0$.
This produces a curvature–independent interior coefficient $a_\Gamma=-\frac14(4\pi)^{-(d-1)/2}\vol_{d-1}(\Gamma)$ at order $\tau^{-(d-1)/2}$.
Transmission thus serves only as a \emph{counterfactual baseline}: it explains why universality would be impossible if cross–interface coupling were allowed.

% ============================================================
\section*{Original Contributions of This Chapter}
\addcontentsline{toc}{section}{Original Contributions of This Chapter}
\begin{itemize}
  \item A two–sided Dirichlet parametrix in a \emph{reach–uniform} Fermi tube with explicit control of the error kernel (Proposition~\ref{prop:uniform-E}), tailored to internal walls.
  \item A collar coarea framework \emph{restricted to the reach tube} that isolates even/odd cancellations, making curvature enter only at order $\tau^{-(d-2)/2}$.
  \item A clean orthogonal decomposition across $M\setminus\Gamma$ tied to the form domain, ensuring trace identities are applied componentwise without hidden regularity.
\end{itemize}

% ============================================================
\section*{Chapter Audit (Preliminaries)}
\addcontentsline{toc}{section}{Chapter Audit (Preliminaries)}

\begin{itemize}
  \item \textbf{Geometry secured:} compact $(M,g)$, piecewise $C^2$ boundary, $C^2$ wall $\Gamma$, $\mathrm{reach}_M(\Gamma)>0$; Fermi charts and tubular coarea available with uniform constants.
  \item \textbf{Functional foundation:} traces continuous; $H^1_0(M;\partial M\cup\Gamma)$ fixed; $L_\Gamma$ defined by a closed form (no $H^2$ up to the wall/boundary).
  \item \textbf{Spectral discreteness:} compact resolvent, orthogonal splitting across components of $M\setminus\Gamma$.
  \item \textbf{Microlocal readiness:} $\Psi$DO calculus and elliptic parametrices in charts; two–sided Dirichlet parametrix near $\Gamma$ with reach–uniform control; diffraction localized on lower–dimensional strata.
  \item \textbf{Heat/trace hooks:} localized multipliers and cutoffs set; assembly scheme toward Ch.~3 ready; dependence on curvature postponed to the next order.
  \item \textbf{Sharpness transparent:} explicit barriers (reach, regularity, corners) recorded with precise scope of use.
\end{itemize}

\noindent\textbf{Spectral closure.}
Chapter~\ref{ch:preliminaries} provides a complete, geometry– and form–based platform for the universal interior surface law and for all subsequent arguments in Chapters~3–7.
All definitions, constants, and dependencies are explicit and compatible with the SA$\to$FA$\to$GEO$\to$UNI$\to$DYN pipeline of the monograph.

% =========================
% Bibliography placeholders
% =========================
% \bibliographystyle{plain}
% \begin{thebibliography}{10}
% \bibitem{Grisvard1985} P.~Grisvard, \emph{Elliptic Problems in Nonsmooth Domains}, Pitman, 1985.
% \bibitem{Gilkey1995} P.~B.~Gilkey, \emph{Invariance Theory, the Heat Equation, and the Atiyah–Singer Index Theorem}, 2nd ed., CRC Press, 1995.
% \bibitem{SafarovVassiliev1997} Y.~Safarov and D.~Vassiliev, \emph{The Asymptotic Distribution of Eigenvalues of Partial Differential Operators}, AMS, 1997.
% \bibitem{Grieser2002} D.~Grieser, Basics of the b-calculus, in: \emph{Approaches to Singular Analysis}, Birkhäuser, 2001/2002.
% \bibitem{Davies1989} E.~B.~Davies, \emph{Heat Kernels and Spectral Theory}, Cambridge Univ.\ Press, 1989.
% \bibitem{HormanderI} L.~H{\"o}rmander, \emph{Analysis of Linear Partial Differential Operators I}, Springer.
% \bibitem{HormanderIII} L.~H{\"o}rmander, \emph{Analysis of Linear Partial Differential Operators III}, Springer.
% \bibitem{DuistermaatHormander} J.~J.~Duistermaat, L.~H{\"o}rmander, Fourier integral operators II, Acta Math.\ (1972).
% \bibitem{MelroseAPS} R.~B.~Melrose, \emph{The Atiyah–Patodi–Singer Index Theorem}, A.~K.~Peters, 1993.
% \bibitem{ChernovMarkarian} N.~Chernov, R.~Markarian, \emph{Chaotic Billiards}, AMS, 2006.
% \end{thebibliography}
