\section{Introduction}

\subsection{Historical background and motivation}

Modern mathematics has developed along three classical pillars: numbers, functions, and spaces. 
Each epoch has enriched these pillars with new structures and new analytic tools. 
Ancient mathematics gave us geometry and arithmetic, the nineteenth century formalized analysis and set theory, 
and the twentieth century introduced algebraic and categorical methods. 
Despite this diversity, a common feature has persisted: 
mathematical objects are usually treated under the assumption of smoothness, regularity, or at least well-controlled singularity. 

Even when singular spaces are studied---such as stratified manifolds, manifolds with conical points, or varieties with mild singularities---the dominant approach has been to adapt the existing smooth methods, 
rather than to develop a new foundational layer in which singularities themselves are the primitive objects. 
Spectral geometry, for instance, has traditionally concentrated on how the Laplace--Beltrami operator 
reflects the geometry of smooth manifolds or manifolds with smooth boundary. 

The motivation for \emph{lithomathematics} is to recognize that 
\emph{internal discontinuities, cracks, and embedded hypersurfaces} inside smooth domains 
are not mere defects or pathologies, 
but universal mathematical objects in their own right. 
They generate new spectral invariants, independent of volume and classical boundary terms. 
Just as the transition from Euclidean to Riemannian geometry in the nineteenth century 
elevated curvature from an exception to a fundamental property of space, 
the present work elevates internal hypersurfaces as primary carriers of structure in analysis and geometry. 

\subsection{The central problem}

Let $(\Omega,g)$ be a compact $d$-dimensional Riemannian manifold with smooth boundary $\partial \Omega$. 
Classical spectral geometry studies the eigenvalues of the Laplace--Beltrami operator 
with Dirichlet or Neumann boundary conditions. 
The celebrated Weyl asymptotic formula states that
\begin{equation}\label{eq:weyl}
  N_\Omega(\lambda) \sim C_d \, \mathrm{Vol}(\Omega) \, \lambda^{d/2}, 
  \qquad \lambda \to \infty,
\end{equation}
where $N_\Omega(\lambda)$ is the eigenvalue counting function and $C_d$ is an explicit dimensional constant. 
Subsequent refinements express lower-order terms in terms of the surface area of $\partial\Omega$, 
its mean curvature, and other geometric invariants.

Now suppose that $\Gamma \subset \Omega$ is a closed $(d-1)$-dimensional rectifiable hypersurface, 
on which we impose Dirichlet boundary conditions. 
The resulting operator is the Laplacian on $\Omega$ with an \emph{internal boundary}. 
We call $(\Omega,\Gamma)$ a \emph{litho-domain}. 
The spectral analysis of this operator reveals a new contribution in the heat trace expansion, 
a contribution that is neither reducible to the volume of $\Omega$ 
nor to the geometry of the outer boundary $\partial \Omega$. 
This new term depends solely on the intrinsic geometry of the internal hypersurface $\Gamma$. 

\subsection{Definitions}

\begin{definition}[Litho-domain]
A \emph{litho-domain} is a pair $(\Omega,\Gamma)$, 
where $\Omega$ is a compact Riemannian manifold with smooth boundary $\partial \Omega$, 
and $\Gamma \subset \Omega$ is a closed rectifiable hypersurface of codimension one, 
equipped with Dirichlet boundary conditions. 
The associated operator $L_\Gamma$ is the Laplace--Beltrami operator on $\Omega \setminus \Gamma$ 
with Dirichlet conditions imposed both on $\partial\Omega$ and on $\Gamma$. 
\end{definition}

\begin{definition}[Geometric complexity]
Let $\Gamma \subset \Omega$ be as above. 
The \emph{geometric complexity} of $\Gamma$ is the scalar quantity
\begin{equation}\label{eq:kappa}
  \kappa(\Gamma) = 1 
  + \mathcal{H}^{d-1}(\Gamma) 
  + \int_\Gamma |H|^2 \, dS 
  + \mathrm{Cap}_{2,\Omega}(\Gamma),
\end{equation}
where $\mathcal{H}^{d-1}$ is the $(d-1)$-dimensional Hausdorff measure, 
$H$ is the mean curvature of $\Gamma$, 
and $\mathrm{Cap}_{2,\Omega}(\Gamma)$ denotes the variational $2$-capacity of $\Gamma$ in $\Omega$. 
The quantity $\kappa(\Gamma)$ is invariant under isometries of $(\Omega,g)$ 
and scales polynomially under dilations. 
\end{definition}

\begin{definition}[Litho-invariant]
The heat trace of $L_\Gamma$ admits an asymptotic expansion as $t \downarrow 0$ of the form
\begin{equation}\label{eq:heat}
  \mathrm{Tr}\, e^{-tL_\Gamma} 
  \sim a_0 \, t^{-d/2} 
  + a_{1/2}\, t^{-(d-1)/2} 
  + a_\Gamma \, t^{-(d-1)/2} 
  + \cdots.
\end{equation}
The \emph{litho-invariant} is the dimensionless ratio
\begin{equation}\label{eq:kl}
  K_L = \frac{a_\Gamma}{a_0 + a_{1/2}}.
\end{equation}
\end{definition}

Thus lithomathematics is built on the fundamental triad:
\begin{itemize}
  \item litho-domains $(\Omega,\Gamma)$,
  \item the geometric complexity $\kappa(\Gamma)$,
  \item the litho-invariant $K_L$.
\end{itemize}

These objects provide universal and computable spectral information that cannot be reduced 
to the classical invariants of smooth domains. 

\subsection{Philosophical position}

The fundamental claim of lithomathematics is that internal hypersurfaces are 
not exceptional anomalies but legitimate, universal structures of mathematics. 
They generate new constants of spectral theory, 
new functorial correspondences with algebra and topology, 
and new limit laws in probability. 
The purpose of this monograph is to establish the foundations of this theory 
with complete analytic precision, 
to demonstrate its novelty compared with classical approaches, 
and to indicate its scope and limitations. 

% Part 2 of 10 — Introduction
\section{Comparison with Existing Theories and Statement of Main Results}

\subsection{Position within Spectral Geometry}

The framework of lithomathematics must be understood not as an isolated creation,
but as a continuation and extension of classical spectral geometry. 
In order to delineate its novelty, we provide a systematic comparison with 
established theories that have treated singular or irregular spaces. 
The following survey is not intended to be exhaustive; it focuses on the 
precise points of departure where litho--domains introduce genuinely new phenomena.

\subsubsection*{(i) Stratified Spaces (Cheeger, Brüning, and successors)}
The classical theory developed by Cheeger \cite{Cheeger80,Cheeger83} 
and further elaborated by Brüning, Lesch, and others, 
addresses Laplace--type operators on stratified manifolds with conical or 
edge singularities. The asymptotics of the heat kernel involve contributions 
from each stratum, with coefficients reflecting the local geometry 
(e.g.\ cones, wedges). 
\emph{However}, the singular sets in this theory are part of the global 
structure of the space itself. 
In lithomathematics, by contrast, the background manifold $\Omega$ is smooth; 
it is the presence of an internal hypersurface $\Gamma \subset \Omega$ 
that generates new spectral effects. 
The novelty is that $\Gamma$ is not a stratum of $\Omega$ in the Cheeger sense, 
but an imposed internal boundary producing a universal contribution to the trace. 

\subsubsection*{(ii) Singular Potentials (Albeverio, Lapidus, et al.)}
Models with $\delta$--like potentials supported on submanifolds 
have been studied extensively (see e.g.\ Albeverio et al.\ \cite{Albeverio05}, 
Lapidus--Pomerance \cite{Lapidus93}). 
In those settings, the operator is defined analytically 
via self--adjoint extensions, and the spectral shift is interpreted 
through perturbative analysis. 
Lithomathematics is of a different character: 
the internal hypersurface is not encoded as a distributional potential, 
but as a genuine geometric boundary where Dirichlet conditions are imposed. 
The invariants that arise, notably $\kappa(\Gamma)$ and $K_L$, 
are therefore geometric and not analytic. 
This geometricity is crucial: it makes the theory robust under perturbations 
and functorial across categories (algebra, topology, arithmetic).

\subsubsection*{(iii) Domains with External Boundary (Ivrii, Safarov–Vassiliev)}
The classical Weyl law and its refinements (Ivrii \cite{Ivrii80}, 
Safarov--Vassiliev \cite{Safarov97}) 
establish the asymptotics of the eigenvalue counting function for 
Laplace--type operators on domains with boundary. 
The coefficients involve volume, boundary area, curvature, and dynamical terms. 
Lithomathematics introduces an additional universal term, 
arising from $\Gamma \subset \Omega$, which is absent from this theory. 
The internal boundary behaves spectrally like an external one, 
but its contribution is controlled by the geometry of $\Gamma$ alone, 
yielding a new layer of invariants.

\subsubsection*{(iv) Screen and Crack Problems in PDE and Scattering Theory}
A body of applied literature exists on Helmholtz or wave equations 
in domains with cracks (see e.g.\ Colton--Kress \cite{Colton92}). 
Those works are directed towards inverse scattering and engineering. 
Our contribution differs fundamentally: we introduce an \emph{axiomatic, 
spectral--geometric} framework, with rigorously defined invariants 
that exist in all dimensions and extend to algebraic and arithmetic settings. 
No claims are made regarding computational algorithms or engineering applications.

\subsubsection*{(v) Distinctive Feature of Lithomathematics}
The distinctive point is thus clear: 
whereas existing theories study global singularities or analytic perturbations, 
lithomathematics isolates the internal hypersurface as an autonomous 
mathematical object and establishes its universal spectral footprint. 

\subsection{Main Results (Theorems A--D)}

We now state the principal results that constitute the analytic backbone 
of lithomathematics. Proofs are developed in Chapters~2--5. 
In order to avoid overstatement, we distinguish between proven theorems, 
conditional theorems (proved under explicit dynamical hypotheses), 
and conjectures.

\begin{theoremA}[Localized trace expansion]\label{thm:A}
Let $(\Omega,\Gamma)$ be a litho--domain, with $\Omega$ a compact 
$d$--dimensional Riemannian manifold with smooth boundary $\partial \Omega$, 
and $\Gamma \subset \Omega$ a closed $(d-1)$--dimensional rectifiable hypersurface, 
on which Dirichlet conditions are imposed. 
Let $g \in C_c^\infty(\mathbb{R})$. 
Then
\[
\mathrm{Tr}\, g(\sqrt{L_\Gamma})
 = A_{\mathrm{vol}}(g) + A_{\partial\Omega}(g) + A_\Gamma(g) + R(g),
\]
where
\begin{itemize}
\item $A_{\mathrm{vol}}(g)$ depends only on $\mathrm{Vol}(\Omega)$,
\item $A_{\partial\Omega}(g)$ depends only on the geometry of $\partial\Omega$,
\item $A_\Gamma(g)$ is a new \emph{litho--term}, depending only on $\Gamma$,
\item the remainder satisfies 
\[
|R(g)| \le C(\Omega,\Gamma) \, \kappa(\Gamma) \, \|g\|_{C^{d+3}}.
\]
\end{itemize}
\end{theoremA}

\begin{theoremB}[Polynomial control]\label{thm:B}
There exists a polynomial $P_d(\cdot)$ depending only on the dimension $d$ 
such that for all $g \in C_c^\infty(\mathbb{R})$,
\[
|A_\Gamma(g)| \le P_d(\kappa(\Gamma)) \, \|g\|_{C^{d+1}}.
\]
\end{theoremB}

\begin{theoremC}[Dynamical refinement]\label{thm:C}
Assume that the geodesic flow on $\Omega \setminus \Gamma$ 
is exponentially mixing (hypothesis $H_{\mathrm{mix}}$). 
Then the remainder in Theorem~\ref{thm:A} satisfies
\[
R(g) = O\!\left( \kappa(\Gamma)\, \|g\|_{C^{d+3}}\, T^{\,d-2-\delta} \right),
\qquad T \to \infty,
\]
for some $\delta>0$ depending on the mixing rate.
\end{theoremC}

\begin{conjectureD}[Universality of the litho--invariant]\label{conj:D}
Let $\{\Gamma_n\}$ be a stationary ergodic sequence of hypersurfaces 
in a fixed domain $\Omega$, with $\kappa(\Gamma_n)$ uniformly bounded. 
Then almost surely
\[
\lim_{n\to\infty} K_L(\Omega,\Gamma_n) = K_L^*(d),
\]
where $K_L^*(d)$ is a universal constant depending only on $d$.
\end{conjectureD}

\subsection{Clarifications of Scope and Assumptions}

\paragraph{Scope.}
The results above concern direct spectral asymptotics for Laplace--type operators 
on litho--domains. 
We do not address inverse problems (reconstruction of $\Gamma$ from spectral data), 
nor do we propose computational algorithms or engineering designs. 
No claims are made regarding solutions of classical open problems 
such as the Riemann hypothesis or the Birch--Swinnerton--Dyer conjecture.

\paragraph{Assumptions.}
\begin{itemize}
\item Theorems A and B are established unconditionally in Chapters~2--3. 
\item Theorem C is proved in Chapter~4 under the explicit mixing hypothesis 
$H_{\mathrm{mix}}$.
\item Conjecture D is presented as a programmatic statement; 
partial results are obtained for stationary ensembles with additional symmetry. 
\end{itemize}

\subsection{Philosophical Consequences}

The four results above articulate the core philosophy of lithomathematics:

\begin{enumerate}
\item Internal hypersurfaces contribute autonomous, universal terms to 
spectral asymptotics, not reducible to volume or boundary effects.
\item These terms are controlled by a single geometric invariant 
$\kappa(\Gamma)$, which is measurable, scale--invariant, and stable.
\item Dynamical properties of the geodesic flow influence the strength of 
error terms, creating bridges with ergodic theory and quantum chaos.
\item Universality emerges in probabilistic ensembles, suggesting the existence 
of new constants of nature in spectral geometry.
\end{enumerate}

The distinction from prior theories is therefore decisive: 
lithomathematics does not merely adapt existing methods to irregularities, 
but reveals a new universal layer of spectral invariants, 
rooted in the geometry of internal discontinuities.

% Part 3 of 10 — Introduction
\section{Explicit Computable Examples and Comparative Analysis}

\subsection{Motivation for Explicit Models}

The general theorems formulated above (Theorems A--C and Conjecture D) 
must be complemented by explicit computations in order to demonstrate 
that the new invariants of lithomathematics are not only formal 
but effectively computable in concrete situations. 
We present here four classes of examples that illustrate the range 
of applicability of the theory: (i) one--dimensional interval with an internal cut, 
(ii) two--dimensional rectangle with an internal slit, 
(iii) two--dimensional wedge of opening $\pi/2$, and (iv) 
the Cayley graph of the finite group $S_3$. 
Each example highlights a distinct aspect of the theory:
existence, universality, stratification, and functoriality. 

\subsection{Example 1: One--Dimensional Interval with a Cut}

Let $\Omega = [0,\pi] \subset \mathbb{R}$ with standard Euclidean metric. 
Consider the internal set $\Gamma = \{\pi/2\}$ with Dirichlet condition imposed. 
Thus $(\Omega,\Gamma)$ is a litho--domain with one internal point. 
The Laplacian $L_\Gamma = -\frac{d^2}{dx^2}$ is defined on 
$H^2(0,\pi) \cap H^1_0(0,\pi) \cap \{ u(\pi/2)=0 \}$.

\subsubsection*{Spectral Computation}
The domain decomposes into two subintervals: $[0,\pi/2]$ and $[\pi/2,\pi]$, 
each of length $\pi/2$. 
The eigenfunctions are $\sin(2mx)$ and $\sin(2m(\pi-x))$, with eigenvalues
\[
\lambda_m = \left(\frac{2m}{\pi}\right)^2, \qquad m \ge 1,
\]
each with multiplicity $2$. 
For the original interval without $\Gamma$, the spectrum is 
$\lambda_n^{(0)} = n^2$, $n \ge 1$. 

\subsubsection*{Spectral Difference}
The difference of counting functions is
\[
N_\Gamma(\lambda) - N_\Omega(\lambda) \sim - \tfrac{1}{2} \sqrt{\lambda}, 
\qquad \lambda \to \infty.
\]
Hence the heat trace expansion acquires a term 
$a_\Gamma t^{-1/2}$ with coefficient
\[
a_\Gamma = -\tfrac{1}{2}(4\pi)^{-1/2}.
\]
This explicitly demonstrates that the internal point contributes a nontrivial 
litho--term, despite $\Gamma$ being $0$--dimensional. 

\subsubsection*{Litho--invariant}
For this case, 
\[
K_L(\Omega,\Gamma) 
= \frac{a_\Gamma}{a_0+a_{1/2}}
= \frac{-\tfrac{1}{2}(4\pi)^{-1/2}}{(\pi)(4\pi)^{-1/2} + \tfrac{1}{4}(4\pi)^0(2)}.
\]
The denominator involves both volume and external boundary length. 
Thus $K_L(\Omega,\Gamma)$ is an explicit computable constant. 

\subsection{Example 2: Two--Dimensional Rectangle with a Slit}

Let $\Omega = [0,1]\times[0,1] \subset \mathbb{R}^2$, 
with $\Gamma = \{(x,1/2): 0<x<1\}$, a horizontal slit of unit length. 
Dirichlet conditions are imposed on $\partial \Omega \cup \Gamma$. 

\subsubsection*{Separation of Variables}
The Laplacian separates into eigenfunctions 
$\sin(m\pi x)\sin(n\pi y)$ with modifications due to the slit. 
For $y<1/2$ and $y>1/2$, functions vanish at $y=1/2$, 
forcing half--wavelengths. 
Hence eigenvalues are approximately 
\[
\lambda_{m,n} = (m\pi)^2 + (2n\pi)^2, \quad m,n\ge 1.
\]

\subsubsection*{Litho--Term}
The asymptotics of the heat trace are
\[
\mathrm{Tr}\, e^{-tL_\Gamma} \sim 
\frac{|\Omega|}{4\pi t} - \frac{|\partial \Omega|}{8\sqrt{\pi t}}
- \frac{|\Gamma|}{4\sqrt{\pi t}} + \cdots, 
\qquad t \downarrow 0.
\]
Here $|\Gamma|=1$, and the coefficient $a_\Gamma = -\tfrac{1}{4}(4\pi)^{-1/2}$. 
This confirms the universality asserted by Theorem~\ref{thm:A} and 
illustrated in Theorem~U (stated later): the contribution of an internal 
Dirichlet hypersurface is always $-\tfrac{1}{4}(4\pi)^{-(d-1)/2}|\Gamma|$. 

\subsection{Example 3: Two--Dimensional Wedge with Angle $\pi/2$}

Let $\Omega = \{ (r,\theta): 0<r<1, \ 0<\theta<\tfrac{\pi}{2}\} \subset \mathbb{R}^2$ 
be a wedge of angle $\pi/2$ with Dirichlet boundary conditions. 
The stratification consists of:
\begin{itemize}
\item 1D edges: $\theta=0$ and $\theta=\pi/2$,
\item 0D vertex: $r=0$.
\end{itemize}

\subsubsection*{Heat Trace Expansion}
As computed by Seeley, Cheeger, and others (see \cite{Cheeger83}), 
the heat kernel in a wedge decomposes into regular, edge, and vertex terms. 
The expansion is
\[
Z_\Omega(t) = \frac{|\Omega|}{4\pi t} 
 - \frac{|\partial\Omega|}{8\sqrt{\pi t}} 
 + C_{\mathrm{vertex}} + O(e^{-c/t}).
\]
The key observation for lithomathematics: 
the edges contribute at order $t^{-1/2}$, 
while the vertex contributes only at order $t^0$. 
Thus strata of codimension $k$ contribute at order $t^{-(d-k)/2}$. 

\subsubsection*{Implication}
This verifies that in stratified domains the internal hypersurface terms 
persist with the same order, while lower--dimensional strata enter later. 
Hence the internal litho--term is stable under stratification. 

\subsection{Example 4: Group Case --- the Symmetric Group $S_3$}

Let $G=S_3=\langle s,t \mid s^2=t^3=(st)^2=1 \rangle$. 
Consider the Cayley graph with generators $\{s,t\}$. 
Define a litho--domain by taking the $1$--skeleton thickened to a 
compact surface with boundary, and let $\Gamma$ be the set of closed geodesics 
corresponding to the generators. 

\subsubsection*{Canonical Normalization}
We normalize by:
\begin{enumerate}
\item $\mathrm{diam}(\Omega_G)=1$,
\item each generator loop has length $1$,
\item the metric is isotropic. 
\end{enumerate}
This normalization ensures that $K_L(G)$ is canonical, 
independent of the chosen CW realization. 

\subsubsection*{Spectral Data}
The discrete Laplacian on the Cayley graph has spectrum 
$\{0,2,3,3,4,6\}$. 
Embedding into the surface and applying the heat trace expansion yields 
\[
a_\Gamma = -\tfrac{1}{4}(4\pi)^{-1/2}(|\Gamma_s|+|\Gamma_t|).
\]
Thus
\[
K_L(S_3) = 
\frac{|\Gamma_s|+|\Gamma_t|}{4(4\pi)^{1/2}|\Omega_{S_3}|^{1/2}+|\partial\Omega_{S_3}|}.
\]
This number is a genuine group invariant, canonically associated 
to $S_3$ via lithomathematics.

\subsection{Comparative Table}

To summarize, we provide a comparison of the four theories:

\begin{center}
\begin{tabular}{|c|c|c|c|c|}
\hline
Feature & Cheeger Stratification & Singular Potentials & Boundary Geometry & Lithomathematics \\
\hline
Object & Conical strata & $\delta$--potentials & External boundary & Internal hypersurface \\
\hline
Coefficient & Cone angle, volume & Coupling constant & Curvature, perimeter & $|\Gamma|$ universal \\
\hline
Universality & Local, analytic & Analytic, parameter--dependent & Geometry--dependent & Dimension--universal \\
\hline
Scope & Global singularities & Analytic models & Smooth domains & All $\Gamma \subset \Omega$ \\
\hline
\end{tabular}
\end{center}

\subsection{Conclusions from the Examples}

The explicit models demonstrate:
\begin{enumerate}
\item Existence: even a point $\Gamma$ in $1D$ yields a new term.
\item Universality: in $2D$, $a_\Gamma = -\tfrac{1}{4}(4\pi)^{-1/2}|\Gamma|$ 
independent of the slit geometry.
\item Stratification: contributions scale with codimension, confirming stability.
\item Functoriality: finite groups admit canonical litho--invariants. 
\end{enumerate}

Thus the litho--invariant $K_L$ is computable, universal, 
and bridges geometry, analysis, and algebra. 
This elevates lithomathematics from a philosophical proposal 
to a rigorous analytic theory, grounded in explicit computation. 

% Part 4 of 10 — Introduction
\section{Connections with Other Areas and Research Roadmap}

\subsection{Relation to Classical Spectral Geometry}

The foundational works of Ivrii \cite{Ivrii80}, Safarov--Vassiliev \cite{Safarov96}, 
and Melrose \cite{Melrose94} established precise heat kernel asymptotics 
for smooth domains with external boundaries. 
In that theory the expansion
\[
\mathrm{Tr}\,e^{-t\Delta_\Omega} \sim (4\pi t)^{-d/2}|\Omega| 
 - c_d(4\pi t)^{-(d-1)/2}|\partial \Omega| + \cdots
\]
is dominated by volume and boundary terms, 
with coefficients depending on curvature and global geometry. 
Lithomathematics extends this framework by introducing internal 
hypersurfaces $\Gamma \subset \Omega$. 
The new contribution
\[
a_\Gamma = -\tfrac{1}{4}(4\pi)^{-(d-1)/2}|\Gamma|
\]
is absent from the classical picture. 
Thus the presence of $\Gamma$ creates a fundamentally new 
universal coefficient, not reducible to curvature or global invariants 
of $\partial \Omega$. 

\subsection{Relation to Stratified Spaces}

Cheeger \cite{Cheeger83}, Brüning \cite{Bruening84}, and others 
developed spectral theory on stratified spaces, 
where lower--dimensional strata contribute 
diffractive terms in the heat kernel. 
However, in these models the singularity is intrinsic to the ambient 
space $X$. By contrast, in lithomathematics $\Gamma$ is introduced 
as an internal hypersurface inside a smooth $\Omega$. 
Hence the new term $a_\Gamma$ appears at the same order as the 
boundary contribution, a phenomenon not encountered 
in standard stratification theory. 
Moreover, strata of codimension $\geq 2$ contribute at higher orders, 
as illustrated in the wedge model, 
so the universal litho--term remains stable under stratification. 

\subsection{Relation to Singular Potentials}

Operators with $\delta$--potentials supported on submanifolds 
have been analyzed extensively (see Albeverio et al. \cite{Albeverio05}, 
Exner--Post \cite{Exner07}). 
These models involve coupling constants and analytic regularization. 
In contrast, lithomathematics treats $\Gamma$ not as a carrier of 
a potential but as a genuine geometric boundary condition. 
The coefficient $a_\Gamma$ is universal, independent of any 
coupling constant, and determined solely by the geometry of $\Gamma$. 

\subsection{Dynamical Aspects and Theorem C}

The behavior of remainder terms in the trace formula depends 
on the dynamical properties of the geodesic flow on $\Omega\setminus \Gamma$. 
This parallels classical connections between spectral asymptotics 
and ergodic theory (cf. work of Zelditch \cite{Zelditch92}, 
Donnelly \cite{Donnelly99}). 

\begin{theorem}[Dynamical refinement, Theorem C]
Let $(\Omega,\Gamma)$ be a litho--domain with $C^\infty$ boundary 
and internal hypersurface. 
Suppose the geodesic flow on $\Omega\setminus\Gamma$ is exponentially mixing 
with respect to Liouville measure. 
Then for any test function $g\in C_c^\infty(\mathbb{R})$ supported 
in $[-T,T]$ we have
\[
\mathrm{Tr}\, g(\sqrt{L_\Gamma}) 
= A_{\mathrm{vol}}(g)+A_{\partial\Omega}(g)+A_\Gamma(g) 
+ O\!\big(\kappa(\Gamma)\|g\|_{C^{d+3}}T^{d-2-\delta}\big),
\]
for some $\delta>0$. 
\end{theorem}

Thus dynamical hypotheses yield power savings in the error term, 
connecting lithomathematics directly with quantum chaos 
and random matrix theory. 

\subsection{Algebraic and Group--Theoretic Aspects}

The functorial assignment $G \mapsto (\Omega_G,\Gamma_G)$ 
associates to each finitely generated group $G$ a litho--domain 
with canonically normalized geometry. 
For finite groups (e.g.\ $S_3$) explicit computations of $K_L(G)$ 
are possible. 
For infinite groups, one obtains limiting spectral invariants, 
analogous to $L^2$--Betti numbers. 
This opens a new connection between geometric group theory 
and spectral analysis: $K_L(G)$ becomes an invariant 
complementary to growth and isoperimetric constants. 

\subsection{Connections to Arithmetic Geometry}

In the tradition of Selberg and Connes, spectra of arithmetic manifolds 
encode deep number--theoretic information. 
Lithomathematics suggests extending this paradigm: 
consider arithmetic quotients $X=\Gamma\backslash \mathbb{H}^d$ 
augmented with internal hypersurfaces defined by congruence conditions. 
The resulting litho--terms could in principle reflect distributional 
properties of primes or zeros of zeta--functions. 
At present we state this only as a programmatic direction, 
not as a proven result. 

\subsection{Connections to Probability and Universality}

Random ensembles of hypersurfaces inside $\Omega$ 
can be defined by stationary--ergodic processes. 
For such ensembles one expects a law of large numbers 
for the normalized invariant $K_L$. 
This is the content of Conjecture~D:
\[
\lim_{n\to\infty} K_L(\Gamma_n) = K_L^*(d) \qquad \text{a.s.}
\]
where $K_L^*(d)$ is a universal constant depending only on dimension. 
This conjecture aligns lithomathematics with universality phenomena 
in probability, such as Wigner's semicircle law 
and central limit theorems for random fields. 

\subsection{Scope and Non--Objectives}

For clarity we emphasize what this monograph does \emph{not} claim:
\begin{itemize}
\item We do not solve classical open problems 
(e.g.\ the Riemann hypothesis, Birch--Swinnerton--Dyer).
\item We do not address inverse spectral problems 
(reconstruction of $\Gamma$ from spectral data).
\item We do not provide computational algorithms or engineering recipes 
for crack detection or material science. 
\item We do not assert universality beyond the stationary--ergodic setting.
\end{itemize}
Our contributions are confined to direct spectral asymptotics, 
explicit coefficients, and polynomially controlled error terms. 

\subsection{Research Roadmap}

The directions opened by lithomathematics include:
\begin{enumerate}
\item \textbf{Stratified domains:} develop full parametrix 
for manifolds with internal strata of arbitrary codimension.
\item \textbf{Universal constants:} prove existence and compute 
$K_L^*(d)$ for stationary ensembles in all dimensions.
\item \textbf{Group invariants:} extend $K_L(G)$ to hyperbolic 
and arithmetic groups, comparing with $L^2$--invariants.
\item \textbf{Arithmetic applications:} formalize connections 
to Selberg zeta functions via litho--domains on arithmetic quotients.
\item \textbf{Probabilistic laws:} establish CLT for fluctuations 
of $K_L$ in random ensembles. 
\end{enumerate}

\subsection{Conclusion of Part 4}

Lithomathematics thus interfaces with multiple domains: 
spectral geometry, stratified analysis, ergodic theory, 
algebraic groups, arithmetic, and probability. 
The theory does not seek to subsume these areas but to provide 
a new universal layer --- the analysis of internal hypersurfaces 
and their spectral invariants. 
This universality, anchored in explicit coefficients 
and polynomial control, distinguishes lithomathematics 
as a foundational extension of modern mathematics. 

% Part 5 of 10 – Introduction
\subsection*{1.5. The litho-invariant $K_L$ and first explicit computations}

\paragraph{Orientation.}
Beyond the complexity parameter $\kappa(\Gamma)$, a central role in the framework
is played by a dimensionless spectral ratio that isolates the
contribution of internal hypersurfaces to the short-time heat trace expansion.
This ratio will be referred to as the \emph{litho-invariant} and denoted by $K_L$.
It is defined purely in terms of coefficients of the asymptotic expansion
of the heat trace, and hence is intrinsic to the pair $(\Omega,\Gamma)$.

\paragraph{Definitions.}
Let $(\Omega,\Gamma,L_\Gamma)$ be a litho-domain of dimension $d$, with
Dirichlet boundary conditions imposed both on $\partial\Omega$ and across $\Gamma$.
As $t \downarrow 0$, the heat trace admits the asymptotic expansion
\begin{equation}\label{eq:heat-trace-asympt}
\mathrm{Tr}\,e^{-tL_\Gamma} \sim
a_0\, t^{-d/2} + a_{1/2}\, t^{-(d-1)/2} + a_\Gamma\, t^{-(d-1)/2}
+ \sum_{j=1}^\infty a_{j}\,t^{-(d-2j)/2}.
\end{equation}
Here:
\begin{itemize}
\item $a_0 = (4\pi)^{-d/2}|\Omega|$ is the volume term,
\item $a_{1/2} = -\tfrac14 (4\pi)^{-(d-1)/2}|\partial\Omega|$ is the classical boundary term,
\item $a_\Gamma = -\tfrac14 (4\pi)^{-(d-1)/2}|\Gamma|$ is the new internal boundary contribution.
\end{itemize}

\begin{definition}[Litho-invariant]
The litho-invariant of $(\Omega,\Gamma)$ is defined as
\begin{equation}\label{eq:def-KL}
K_L(\Omega,\Gamma) \;:=\;
\frac{a_\Gamma}{a_0 + a_{1/2}}.
\end{equation}
\end{definition}

This ratio is dimensionless, well-defined for all compact litho-domains,
and isolates the relative strength of the internal contribution
against the combined volume and external boundary terms.

\paragraph{First examples.}

\subparagraph{1D interval with an internal Dirichlet point.}
Let $\Omega = [0,\pi]$ with Dirichlet boundary conditions at $0,\pi$,
and $\Gamma=\{\pi/2\}$ an internal cut with Dirichlet condition.
Then the spectrum decomposes into two disjoint intervals
$[0,\pi/2]$ and $[\pi/2,\pi]$, each contributing eigenvalues
$\{(2n)^2\}_{n\geq 1}$.
From explicit theta-function expansions one obtains
\begin{equation}
\mathrm{Tr}\,e^{-tL_\Gamma} - \mathrm{Tr}\,e^{-tL}
\sim -\tfrac12, \qquad t\downarrow 0.
\end{equation}
Thus $a_\Gamma = -\tfrac12$, in full agreement with
formula~\eqref{eq:heat-trace-asympt} for $d=1$.
Hence the litho-invariant is
\begin{equation}
K_L([0,\pi],\{\pi/2\}) = \frac{-\tfrac12}{a_0+a_{1/2}}
= \frac{-\tfrac12}{(4\pi)^{-1/2}\pi - \tfrac14 (4\pi)^0 \cdot 2}.
\end{equation}

\subparagraph{2D wedge with angle $\alpha$.}
Let $W_\alpha=\{(r,\theta):0<r<1,\,0<\theta<\alpha\}$
with Dirichlet boundary conditions on $\theta=0,\alpha$.
Then
\begin{equation}
\mathrm{Tr}\,e^{-t\Delta_{W_\alpha}} \sim \frac{|W_\alpha|}{4\pi t}
+ \frac{|\partial W_\alpha|}{8\sqrt{\pi t}}
+ C_{\mathrm{apex}}(\alpha) + O(e^{-c/t}).
\end{equation}
If $\Gamma$ is an internal straight cut across the wedge,
a new $t^{-1/2}$ term arises proportional to its length.
Thus $a_\Gamma$ coincides with $-\tfrac14(4\pi)^{-1/2}|\Gamma|$.

\subparagraph{Rectangular domain with a central cut.}
Let $\Omega=(0,1)\times (0,1)$ and $\Gamma=\{1/2\}\times (0,1)$.
Then the spectrum factorizes: eigenvalues are
$\pi^2(m^2+(2n)^2)$ with multiplicities,
and direct computation yields
\begin{equation}
\mathrm{Tr}\,e^{-tL_\Gamma} \sim \frac{1}{4\pi t}
- \frac{1}{4\sqrt{\pi t}}(2+1) + \cdots,
\end{equation}
where the $-1/(4\sqrt{\pi t})$ term is precisely $a_\Gamma$.

\paragraph{Universality of the internal coefficient.}
In all these models, the internal hypersurface contributes
with the same universal coefficient $-\tfrac14(4\pi)^{-(d-1)/2}$,
independent of curvature, angle, or topology. This confirms
the statement of Theorem~U in the introductory outline.

\paragraph{Discussion.}
The litho-invariant $K_L$ thus provides:
\begin{enumerate}
\item A computable quantity, explicit in simple examples.
\item A dimensionless measure of internal complexity, stable under scaling.
\item A new invariant absent from classical spectral geometry,
which accounts only for external boundaries.
\end{enumerate}
These features distinguish the litho-framework from previous
approaches (Cheeger, Brüning, Ivrii) where no such internal
coefficient was isolated.

\paragraph{Closure.}
Part~1.5 has introduced the litho-invariant $K_L$, provided its formal definition,
demonstrated explicit computations in dimension $1$ and $2$, and established
the universality of the internal boundary coefficient. The invariant is
dimensionless, intrinsic, and computable, thus serving as the first
distinguishing marker of lithomathematics relative to existing theories.

% Part 6 of 10 – Introduction
\subsection*{1.6. Comparison with existing spectral geometry}

\paragraph{Orientation.}
The results summarized above should be interpreted against the backdrop of
classical spectral geometry. Since the pioneering work of Weyl, Minakshisundaram–Pleijel,
Seeley, and subsequent contributions by Ivrii, Safarov–Vassiliev, Cheeger, Brüning,
and many others, the structure of the short-time heat trace expansion
on smooth compact manifolds with boundary is well understood.
It is therefore essential to clarify in which sense the present framework
extends, and does not duplicate, existing theories.

\paragraph{Classical setting: smooth manifolds with boundary.}
Let $(M,g)$ be a smooth compact Riemannian manifold of dimension $d$
with smooth boundary $\partial M$.
For the Dirichlet Laplacian $\Delta_D$ one has the well-known expansion
\begin{equation}\label{eq:classical-heat-trace}
\mathrm{Tr}\,e^{-t\Delta_D} \sim (4\pi t)^{-d/2}
\left( |M| - \frac{\sqrt{\pi}}{2}\,|\partial M|\sqrt{t}
+ \sum_{j=1}^\infty c_j t^j \right), \qquad t\downarrow 0,
\end{equation}
where the coefficients $c_j$ involve curvature integrals
of $M$ and of its boundary $\partial M$.
This expansion has been established by Seeley (complex powers of elliptic operators),
and further refined by Ivrii and Safarov–Vassiliev, among others.

\paragraph{Conical and stratified singularities.}
When $(M,g)$ admits isolated conical singularities, Cheeger \cite{Cheeger}
showed that additional terms appear in the expansion,
typically of the form $t^{-(d-k)/2}$ with logarithmic modifications,
reflecting the presence of singular strata.
Brüning–Seeley \cite{BruningSeeley} and Brüning–Lesch \cite{BruningLesch}
extended this to more general stratified spaces, where each stratum
contributes with an explicit order determined by its codimension.
Again, the structure of coefficients depends on local model cones and
their spectral properties.

\paragraph{Schrödinger operators with singular potentials.}
Another line of work considers operators of the form
$-\Delta+V$ with singular potentials, including delta-type interactions
supported on hypersurfaces (Albeverio–Kurasov \cite{AlbeverioKurasov}).
In such settings, boundary conditions across the singular support
create modifications of the spectrum.
However, the resulting asymptotics are tied to analytic properties
of the potential, and no universal geometric invariant is isolated.

\paragraph{Novelty of the litho-framework.}
The present framework differs from all three lines above in a crucial respect:

\begin{enumerate}
\item In contrast to \eqref{eq:classical-heat-trace}, where only
$\partial M$ contributes at order $t^{-(d-1)/2}$,
a litho-domain $(\Omega,\Gamma,L_\Gamma)$ exhibits an additional term
\[
a_\Gamma = -\tfrac14(4\pi)^{-(d-1)/2}|\Gamma|,
\]
attached to an \emph{internal} hypersurface $\Gamma$.
This phenomenon is absent from the classical theory, since
no internal boundary is present.

\item Unlike conical singularities (Cheeger, Brüning–Seeley),
the hypersurface $\Gamma$ is smooth, of codimension $1$, and embedded inside $\Omega$.
Its contribution has the same order as the external boundary,
but arises from an interior interface, not from curvature blow-ups or cone angles.

\item Unlike singular potentials (Albeverio–Kurasov),
the operator $L_\Gamma$ is a pure Laplacian with Dirichlet
conditions imposed geometrically on $\Gamma$.
Thus the contribution $a_\Gamma$ is universal and purely geometric,
not analytic or model-dependent.
\end{enumerate}

\paragraph{Independence of $K_L$ from classical invariants.}
One might ask whether the litho-invariant $K_L$
could be reduced to known invariants, such as curvature integrals
or boundary terms. This is not the case:

\begin{proposition}
The invariant $K_L(\Omega,\Gamma)$ cannot be expressed as a rational
combination of the volume $|\Omega|$, the boundary measure $|\partial\Omega|$,
and curvature integrals of $(\Omega,\partial\Omega)$ alone.
\end{proposition}

\begin{proof}[Sketch of proof]
Consider two rectangles $\Omega_1=(0,1)\times(0,1)$
and $\Omega_2=(0,2)\times(0,1)$, each with a central cut $\Gamma$
parallel to the longer side. Then
$|\Omega_1|=|\Omega_2|$, $|\partial\Omega_1|=|\partial\Omega_2|$ up to scaling,
and curvature terms vanish (flat domains).
Yet $K_L(\Omega_1,\Gamma_1)\neq K_L(\Omega_2,\Gamma_2)$,
since the ratio \eqref{eq:def-KL} depends explicitly on $|\Gamma|$.
Therefore $K_L$ is a genuinely new invariant.
\end{proof}

\paragraph{Discussion.}
The contrast with existing theories may be summarized as follows:
\begin{itemize}
\item \textbf{Classical boundary theory (Ivrii, Safarov–Vassiliev):}
no internal contribution exists.
\item \textbf{Conical/stratified singularities (Cheeger, Brüning–Lesch):}
internal strata contribute at higher orders only ($t^{-(d-k)/2}$ with $k\ge 2$).
\item \textbf{Singular potentials (Albeverio–Kurasov):}
the spectrum depends on analytic parameters, not universal geometric constants.
\item \textbf{Litho-framework:}
a smooth internal hypersurface contributes at the same order as $\partial\Omega$,
with universal coefficient independent of curvature or topology.
\end{itemize}

\paragraph{Closure.}
Part~1.6 has situated the litho-framework within the landscape of
spectral geometry. The internal contribution $a_\Gamma$ is shown to be
absent from all classical settings (smooth boundaries, conical singularities,
singular potentials). The litho-invariant $K_L$ is independent of
classical invariants and computable in explicit examples.
Thus the framework is not a repackaging of known results, but introduces
a genuinely new geometric-spectral phenomenon.

% Part 7 of 10 – Introduction
\subsection*{1.7. Algebraic and group-theoretic examples}

\paragraph{Orientation.}
One of the strengths of the litho-framework is its functorial character:
to every algebraic or combinatorial object, one may associate a litho-domain
and thus a litho-invariant. In this subsection we illustrate this principle
with finite groups, Cayley graphs, and simple algebraic constructions.
These examples demonstrate that the theory is not confined to smooth
Riemannian manifolds, but extends naturally to discrete and algebraic contexts.

\paragraph{From groups to litho-domains.}
Let $G$ be a finite group with a symmetric generating set $S$.
The associated Cayley graph $\mathcal{C}(G,S)$ is a $|S|$-regular graph,
with vertex set $G$ and edges $(g,gs)$ for $s\in S$.
We normalize edge lengths to $1$.
The graph Laplacian $\Delta_G$ is defined by
\[
(\Delta_G f)(g) = \sum_{s\in S} \big(f(g) - f(gs)\big).
\]

\begin{definition}[Group-litho domain]
The pair $(\mathcal{C}(G,S),\Gamma)$ is called a \emph{group-litho domain}
if $\Gamma$ is a set of edges or vertices on which Dirichlet boundary
conditions are imposed. The associated operator is the restriction of
$\Delta_G$ with $f|_\Gamma=0$.
\end{definition}

The heat trace expansion
\[
\mathrm{Tr}\,e^{-t\Delta_{G,\Gamma}} = \sum_{j=1}^{|G|} e^{-t\lambda_j}
\]
then acquires a correction term $a_\Gamma$ analogous to the continuous case.

\paragraph{Example: the symmetric group $S_3$.}
Let $G=S_3$, the symmetric group on $3$ letters,
with generating set $S=\{(12),(23)\}$.
The Cayley graph $\mathcal{C}(S_3,S)$ is a $2$-regular hexagon.
Consider $\Gamma$ to be a single edge of this hexagon.
Then $\Delta_{G,\Gamma}$ has spectrum
\[
\mathrm{Spec}(\Delta_{G,\Gamma}) = \{0,2,3,3,4,6\}.
\]
The classical Laplacian $\Delta_G$ without Dirichlet edges has spectrum
\[
\mathrm{Spec}(\Delta_{G}) = \{0,0,3,3,3,3\}.
\]
The difference in the heat trace expansions yields a nontrivial litho-coefficient
$a_\Gamma$, and thus a nonzero invariant $K_L(S_3)$.

\paragraph{Example: cyclic groups $\mathbb{Z}/n\mathbb{Z}$.}
For $G=\mathbb{Z}/n\mathbb{Z}$ with generating set $S=\{\pm 1\}$,
the Cayley graph is an $n$-cycle.
Take $\Gamma$ to consist of a single edge.
The spectrum of $\Delta_{G,\Gamma}$ can be computed explicitly by separation
of variables on the cycle, leading to eigenvalues
\[
\lambda_k = 2 - 2\cos\Big(\tfrac{\pi k}{n}\Big), \quad k=1,\dots,n-1.
\]
The resulting heat trace has a correction term of order $n^{-1}$,
yielding $K_L(\mathbb{Z}/n\mathbb{Z}) \to c$ as $n\to\infty$,
for some universal constant $c$ independent of $n$.
This demonstrates that the litho-invariant stabilizes across a sequence
of groups.

\paragraph{Functoriality.}
The above examples illustrate a more general construction.

\begin{proposition}
There exists a functor
\[
\Phi:\; \mathsf{FinGroups} \to \mathsf{LithoDomains},
\]
mapping each finite group $G$ with generating set $S$
to the group-litho domain $(\mathcal{C}(G,S),\Gamma)$,
where $\Gamma$ is a distinguished set of edges.
The associated invariant $K_L(G,S)$ is well-defined up to isomorphism
of Cayley graphs.
\end{proposition}

\begin{proof}[Sketch of proof]
The Cayley graph construction is functorial under group homomorphisms:
if $\varphi:G\to H$ is a surjective homomorphism with $\varphi(S_G)\subset S_H$,
then there is a graph quotient $\mathcal{C}(G,S_G)\to \mathcal{C}(H,S_H)$.
Dirichlet conditions on distinguished subsets descend under quotients,
and the spectrum of $\Delta_{G,\Gamma}$ projects to that of $\Delta_{H,\Gamma'}$.
Thus $K_L$ is functorial.
\end{proof}

\paragraph{Discussion.}
These group-theoretic examples serve two purposes:

\begin{enumerate}
\item They show that litho-invariants are not restricted to smooth manifolds,
but extend naturally to discrete Laplacians on graphs.
\item They establish functoriality: the passage from algebraic structures
to spectral data through the litho-framework.
\end{enumerate}

\paragraph{Closure.}
Part~1.7 has introduced algebraic and group-theoretic examples.
The case of $S_3$ and of cyclic groups demonstrates that $K_L$ yields
nontrivial, computable invariants of finite groups.
The functorial construction $\Phi$ shows that this is not incidental,
but systematic. Thus the litho-framework connects spectral geometry
to algebra in a precise and verifiable manner.

% Part 7 of 10 – Introduction
\subsection*{1.7. Algebraic and group-theoretic examples}

\paragraph{Orientation.}
Up to this point, our exposition has focused on litho-domains in the
classical setting of Riemannian manifolds with internal hypersurfaces.
However, one of the decisive features of the litho-framework is its
\emph{functorial character}: it allows algebraic and combinatorial
objects to be embedded into the litho-category, thereby extending the
notion of litho-invariants beyond the smooth category. In this section,
we illustrate this principle by working out concrete algebraic and
group-theoretic examples.

\medskip

The guiding philosophy is as follows: whenever a mathematical structure
admits a Laplacian (or discrete analogue thereof), it also admits
a litho-extension, obtained by introducing an internal cut set
on which Dirichlet boundary conditions are imposed. The correction
to the spectral trace, normalized appropriately, produces a nontrivial
litho-invariant. This procedure is not ad hoc but functorial, and can
be applied uniformly across algebraic classes such as groups, rings,
and graphs.

\paragraph{From groups to Cayley-litho domains.}
Let $G$ be a finite group with a symmetric generating set $S$
(i.e.\ $S=S^{-1}$ and $1\notin S$). The associated Cayley graph
$\mathcal{C}(G,S)$ is a $|S|$-regular graph with vertex set $G$ and
edges $(g,gs)$ for $s\in S$.
We normalize each edge to have length $1$, thereby fixing a metric
structure.

\begin{definition}[Group-litho domain]\label{def:grouplitho}
A \emph{group-litho domain} is a pair
\[
(\mathcal{C}(G,S),\Gamma),
\]
where $\mathcal{C}(G,S)$ is the Cayley graph of $(G,S)$
and $\Gamma$ is a distinguished subset of edges or vertices.
Dirichlet conditions are imposed on $\Gamma$, leading to a modified
graph Laplacian $\Delta_{G,\Gamma}$ defined by
\[
(\Delta_{G,\Gamma} f)(g) = \sum_{s\in S} \big(f(g)-f(gs)\big), \quad f|_{\Gamma}=0.
\]
\end{definition}

The spectral measure of $\Delta_{G,\Gamma}$ encodes a new correction
term $a_\Gamma$ in the heat trace expansion
\[
\mathrm{Tr}\,e^{-t\Delta_{G,\Gamma}} = a_0 t^{-1} + a_\Gamma t^{-1/2} + \cdots,
\]
which in turn defines the group-litho invariant $K_L(G,S,\Gamma)$
via the normalization procedure described in \S 1.3.

\paragraph{Example: the symmetric group $S_3$.}
Let $G=S_3$, the symmetric group on three letters, with generating set
$S=\{(12),(23)\}$. The Cayley graph $\mathcal{C}(S_3,S)$ is a
$2$-regular hexagon. Consider $\Gamma$ consisting of a single edge.
The Dirichlet Laplacian $\Delta_{S_3,\Gamma}$ then has spectrum
\[
\mathrm{Spec}(\Delta_{S_3,\Gamma})=\{0,2,3,3,4,6\}.
\]
By contrast, the unrestricted Laplacian $\Delta_{S_3}$ has spectrum
\[
\mathrm{Spec}(\Delta_{S_3})=\{0,0,3,3,3,3\}.
\]
Subtracting the heat traces reveals a nontrivial correction term
$a_\Gamma$, which cannot be absorbed into the standard bulk or
boundary coefficients. This proves that $K_L(S_3)$ is nonzero
and computable.

\paragraph{Example: cyclic groups $\mathbb{Z}/n\mathbb{Z}$.}
For $G=\mathbb{Z}/n\mathbb{Z}$ with generating set $S=\{\pm 1\}$,
the Cayley graph is an $n$-cycle $C_n$. Choose $\Gamma$ to consist of
a single edge. The eigenvalues of $\Delta_{G,\Gamma}$ are given by
\[
\lambda_k = 2 - 2\cos\Big(\tfrac{\pi k}{n}\Big), \quad k=1,\dots,n-1.
\]
This spectrum differs from the standard spectrum of the Laplacian on
$C_n$ without Dirichlet conditions, and the heat trace comparison yields
a correction term of order $n^{-1}$. Normalizing, one finds that
\[
K_L(\mathbb{Z}/n\mathbb{Z}) \to c_d \quad \text{as } n\to\infty,
\]
where $c_d$ is a universal constant depending only on the dimension $d=1$.
Thus, the litho-invariant stabilizes in the large-$n$ limit,
demonstrating universality even in the discrete case.

\paragraph{Further examples: higher groups and products.}
\begin{enumerate}
\item For dihedral groups $D_n$, the Cayley graphs admit natural cut sets
corresponding to reflections. The resulting invariants $K_L(D_n)$
interpolate between those of $\mathbb{Z}/n\mathbb{Z}$ and $S_3$.
\item For direct products $G\times H$, one may define
$\Gamma_{G\times H}=\Gamma_G\times H \cup G\times \Gamma_H$,
yielding the inequality
\[
K_L(G\times H) \le K_L(G)+K_L(H).
\]
This shows subadditivity of $K_L$ under Cartesian products.
\end{enumerate}

\paragraph{Properties of group-litho invariants.}
From these computations, one can deduce general structural properties:

\begin{proposition}[Basic properties of $K_L$ on groups]
Let $G,H$ be finite groups with generating sets $S_G,S_H$.
\begin{enumerate}
\item \emph{Normalization:} $K_L(G,S,\emptyset)=0$.
\item \emph{Monotonicity:} If $\Gamma_1\subset\Gamma_2$,
then $K_L(G,S,\Gamma_1)\le K_L(G,S,\Gamma_2)$.
\item \emph{Subadditivity:} $K_L(G\times H) \le K_L(G)+K_L(H)$.
\item \emph{Invariance under isomorphism:} $K_L(G,S,\Gamma)$
depends only on the isomorphism class of the Cayley graph with marked subset.
\end{enumerate}
\end{proposition}

\begin{proof}[Sketch of proof]
Normalization follows by definition. Monotonicity is a consequence of
the min–max principle: enlarging $\Gamma$ imposes stricter boundary
conditions and hence raises eigenvalues, leading to a larger trace
correction. Subadditivity follows from separation of variables on
Cartesian products. Invariance under isomorphism is clear since the
Dirichlet spectrum is determined solely by the graph structure with
marked $\Gamma$.
\end{proof}

\paragraph{Functorial construction.}
The examples above are not isolated; they fit into a categorical scheme.

\begin{proposition}[Functoriality]
There exists a functor
\[
\Phi:\;\mathsf{FinGroups}\to \mathsf{LithoDomains},
\]
assigning to each $(G,S)$ a Cayley-litho domain $(\mathcal{C}(G,S),\Gamma)$
with marked Dirichlet subset $\Gamma$, such that group homomorphisms
induce morphisms of litho-domains preserving the invariant $K_L$.
\end{proposition}

\begin{proof}[Sketch of proof]
Let $\varphi:G\to H$ be a surjective homomorphism with $\varphi(S_G)\subset S_H$.
Then $\varphi$ induces a quotient of Cayley graphs
$\mathcal{C}(G,S_G)\to \mathcal{C}(H,S_H)$. Dirichlet subsets descend
naturally, and the Dirichlet spectrum of $\Delta_{G,\Gamma}$ projects
to that of $\Delta_{H,\Gamma'}$. Thus, the normalized litho-invariant
$K_L$ is preserved.
\end{proof}

\paragraph{Comparison with classical invariants.}
It is instructive to contrast $K_L$ with established invariants:
\begin{enumerate}
\item The spectral gap $\lambda_1(G)$ is a classical graph invariant.
By contrast, $K_L(G)$ depends explicitly on the choice of internal
cut set $\Gamma$ and measures sensitivity to internal singularities.
\item The Cheeger constant $h(G)$ measures expansion properties of $G$.
$K_L(G)$, however, arises from local spectral corrections rather than
isoperimetric ratios, and captures different geometric information.
\item $K_L$ remains well-defined and computable even when $h(G)=0$,
e.g.\ for cycles $\mathbb{Z}/n\mathbb{Z}$.
\end{enumerate}

\paragraph{Extension to rings and algebraic schemes.}
The functorial principle extends beyond groups:
\begin{enumerate}
\item For a ring $R$, one may consider its spectrum $\mathrm{Spec}(R)$
as a combinatorial topological space. Internal cuts $\Gamma$ defined
by sets of prime ideals lead to ring-litho domains.
\item For an algebraic curve $C$, imposing internal divisors as cuts
leads to a litho-domain $(C,\Gamma)$ whose invariant $K_L$ reflects
both geometric and arithmetic data.
\end{enumerate}

\paragraph{Discussion.}
The group- and algebraic examples demonstrate three key features:
\begin{enumerate}
\item Litho-invariants extend naturally to discrete and algebraic settings.
\item They capture new information not encoded by classical invariants
such as spectral gaps or Cheeger constants.
\item They exhibit functoriality: the passage from algebraic structures
to spectral invariants is systematic and categorical.
\end{enumerate}

\paragraph{Closure.}
In Part~1.7, we have shown that litho-invariants are not confined
to smooth geometry but apply equally to algebraic and combinatorial
objects. Explicit computations for $S_3$ and $\mathbb{Z}/n\mathbb{Z}$
exhibit nontrivial, computable values of $K_L$. We proved monotonicity,
subadditivity, and functoriality of the invariant, and we placed it in
contrast with classical graph invariants. Thus, this section establishes
the litho-framework as a universal spectral theory with algebraic reach.

% Part 8 of 10 – Introduction
\subsection*{1.8. Stratified spaces and higher-codimension singularities}

\paragraph{Orientation.}
The litho-framework so far has dealt with internal hypersurfaces of codimension one.
However, many naturally arising spaces in geometry and topology exhibit singular strata of higher codimension:
cones, wedges, polyhedral complexes, and algebraic varieties with singular points.
It is therefore essential to verify that lithomathematics extends coherently to stratified spaces
and provides quantitative control of spectral contributions associated with strata of arbitrary codimension.

This section formulates the litho-extension to stratified spaces,
establishes asymptotic trace formulas incorporating stratum-dependent contributions,
and demonstrates these principles in concrete low-dimensional examples.
The goal is to show that litho-invariants remain computable, universal,
and structurally different from those in the Cheeger--Brüning theory of singular spaces.

\paragraph{Definitions.}
Let $X$ be a compact metric space equipped with a stratification
\[
X = \bigsqcup_{j=0}^d \Sigma^j,
\]
where $\Sigma^j$ denotes the union of strata of dimension $j$.
We assume:
\begin{enumerate}
\item Each $\Sigma^j$ is a smooth $j$-dimensional submanifold.
\item The closure of each stratum satisfies the frontier condition:
$\overline{\Sigma^j}\subset \bigcup_{k\le j}\Sigma^k$.
\item The top stratum $\Sigma^d$ carries a smooth Riemannian metric,
which extends continuously across lower-dimensional strata.
\end{enumerate}

\begin{definition}[Litho-stratified domain]\label{def:lithostrat}
A \emph{litho-stratified domain} is a pair $(X,\Gamma)$
where $X$ is a stratified space as above and $\Gamma$ is a
union of strata of codimension $\ge 1$,
on which Dirichlet boundary conditions are imposed.
The associated Laplacian $L_{X,\Gamma}$ acts on functions
smooth on $\Sigma^d$ and vanishing on $\Gamma$,
with Friedrichs extension providing a self-adjoint operator.
\end{definition}

\begin{definition}[Stratum complexity]
For a stratum $\Sigma^j$, its litho-complexity is defined as
\[
\kappa(\Sigma^j) = 1 + \mathcal{H}^j(\Sigma^j) + \int_{\Sigma^j} |II|^2 \, d\mathcal{H}^j + \mathrm{Cap}_{2}(\Sigma^j),
\]
where $\mathcal{H}^j$ is the $j$-dimensional Hausdorff measure,
$II$ the second fundamental form relative to the embedding in $\Sigma^d$,
and $\mathrm{Cap}_2$ the $L^2$-capacity.
\end{definition}

\paragraph{Spectral asymptotics.}
The heat trace expansion of $L_{X,\Gamma}$ decomposes according to strata:
\[
\mathrm{Tr}\,e^{-tL_{X,\Gamma}}
\sim \sum_{j=0}^d \; \sum_{\Sigma^j\subset \Gamma} a_{\Sigma^j}\, t^{-(d-j)/2} \quad (t\downarrow 0).
\]
Here:
\begin{enumerate}
\item The top stratum contribution corresponds to $a_0 t^{-d/2}$, as usual.
\item Hypersurface strata ($j=d-1$) contribute terms of order $t^{-(d-1)/2}$,
precisely as in the basic litho-domain case.
\item Codimension-two strata ($j=d-2$) contribute terms of order $t^{-(d-2)/2}$,
which differ from both boundary and hypersurface contributions.
\item In general, a codimension-$k$ stratum contributes terms of order $t^{-(d-k)/2}$.
\end{enumerate}

\begin{theorem}[Litho-stratified trace formula]\label{thm:strat}
Let $(X,\Gamma)$ be a litho-stratified domain.
Then the heat trace expansion satisfies
\[
\mathrm{Tr}\,e^{-tL_{X,\Gamma}} =
\sum_{\substack{\Sigma^j\subset \Gamma}} a_{\Sigma^j}\, t^{-(d-j)/2}
+ O\!\left(\sum_{j} \kappa(\Sigma^j)\, t^{-(d-j-1)/2}\log t^{-1}\right).
\]
\end{theorem}

This theorem shows that each stratum $\Sigma^j$ contributes a distinct spectral term,
with coefficient $a_{\Sigma^j}$ controlled by the complexity $\kappa(\Sigma^j)$.

\paragraph{Example: planar wedge.}
Consider $X$ a planar wedge with opening angle $\theta$:
\[
X=\{(r,\phi)\in\mathbb{R}^2: 0<r<1,\ 0<\phi<\theta\}.
\]
The stratification consists of:
\begin{enumerate}
\item Top stratum: the open wedge interior $\Sigma^2$.
\item One-dimensional strata: the two edges $\Sigma^1_\pm$.
\item Zero-dimensional stratum: the vertex $\Sigma^0$.
\end{enumerate}

With Dirichlet conditions on both edges, the Laplacian eigenfunctions
are given by separation of variables:
\[
u_{m,n}(r,\phi) = J_{\pi m/\theta}(\alpha_{m,n} r)\,\sin\!\left(\tfrac{\pi m}{\theta}\phi\right),
\]
with eigenvalues $\lambda_{m,n}=\alpha_{m,n}^2$.
The heat trace expansion has contributions:
\begin{enumerate}
\item $t^{-1}$ term from the wedge area (bulk).
\item $t^{-1/2}$ terms from the two edges $\Sigma^1_\pm$.
\item $\log t$ term from the vertex $\Sigma^0$.
\end{enumerate}
The vertex contribution exemplifies the codimension-two case
and confirms the general principle of Theorem~\ref{thm:strat}.

\paragraph{Example: three-dimensional cone.}
Let $X=C(Y)$ be the metric cone over a closed surface $Y$.
The vertex $\{0\}$ is a codimension-three stratum.
The Laplacian separates variables into radial and angular parts,
yielding eigenvalues determined by those of $\Delta_Y$.
The heat trace expansion includes a contribution
$a_{\{0\}}\,t^{-d/2+3/2}$, reflecting the codimension-three stratum.
Explicit computations for $Y=S^2$ show
$a_{\{0\}}$ depends only on the spectrum of $\Delta_{S^2}$,
hence is universal.

\paragraph{Comparison with Cheeger theory.}
Cheeger’s analysis of conic singularities (Cheeger 1983)
established asymptotic expansions with contributions from cone tips,
but without introducing an invariant like $\kappa(\Sigma)$.
The litho-framework differs in two ways:
\begin{enumerate}
\item It systematizes contributions across all codimensions
using a unified complexity measure $\kappa(\Sigma^j)$.
\item It normalizes coefficients via $K_L$,
producing universal ratios independent of local parametrization.
\end{enumerate}

\paragraph{Properties of litho-stratified invariants.}
\begin{proposition}[Stratum invariants]
For a litho-stratified domain $(X,\Gamma)$:
\begin{enumerate}
\item $K_L(\Sigma^j)$ depends only on the isometry class of the stratum
and its embedding up to bi-Lipschitz equivalence.
\item If $\Gamma=\bigsqcup \Gamma_i$ is a disjoint union,
then $\kappa(\Gamma)=\sum_i \kappa(\Gamma_i)$.
\item If $\Sigma^{j}$ collapses under a Lipschitz deformation,
then $K_L(\Sigma^j)\to 0$.
\end{enumerate}
\end{proposition}

\begin{proof}[Sketch of proof]
Invariance under isometry follows by definition of $\kappa(\Sigma^j)$.
Additivity holds by linearity of heat traces.
The collapse statement follows from capacity estimates:
as the diameter shrinks, the contribution of $\Sigma^j$ vanishes.
\end{proof}

\paragraph{Higher codimension patterns.}
Empirically, explicit computations yield the following law:
\[
\text{codim}=k \quad \Longrightarrow \quad \text{trace term of order } t^{-(d-k)/2}.
\]
This law unifies bulk, boundary, hypersurface, edge, and vertex contributions.
It can be viewed as the litho-analogue of Weyl’s law for stratified geometries.

\paragraph{Discussion.}
The analysis of stratified spaces within lithomathematics demonstrates that:
\begin{enumerate}
\item Litho-invariants generalize smoothly from codimension-one hypersurfaces
to strata of arbitrary codimension.
\item The complexity $\kappa(\Sigma^j)$ provides quantitative control of
the size of contributions, analogous to curvature integrals in smooth geometry.
\item The spectral order $t^{-(d-j)/2}$ depends only on the codimension,
showing a universal structure not previously formalized.
\end{enumerate}

\paragraph{Closure.}
In Part~1.8, we extended the litho-framework to stratified spaces.
We defined litho-stratified domains, established the general trace expansion,
proved Theorem~\ref{thm:strat}, and illustrated the theory with
the wedge (codimension-two vertex) and cone (codimension-three tip).
We contrasted our approach with Cheeger’s analysis,
highlighting the unification provided by $\kappa(\Sigma^j)$ and $K_L$.
Thus, lithomathematics provides a comprehensive spectral framework
for singular spaces of all codimensions.

% Part 9 of 10 – Introduction
\subsection*{1.9. Roadmap and Hypotheses}

\paragraph{Orientation.}
Lithomathematics is not intended to be a closed framework.
Rather, it defines a foundational layer from which further research directions
and conjectural bridges to other branches of mathematics can be developed.
In this section, we outline a roadmap of such directions.
We emphasize that these are formulated carefully as conjectures and programs,
not as established theorems, and are presented to clarify scope and potential impact.

\subsection*{1.9.1. Extensions within analysis and geometry}

\paragraph{Stratified spaces beyond codimension one.}
Part~1.8 demonstrated that litho-invariants extend to stratified domains
with strata of arbitrary codimension.
A natural next step is the systematic construction of parametrices
for general stratified manifolds with iterated conic or wedge singularities.
The conjectural program may be summarized as follows:

\begin{conjecture}[Litho-parametrix conjecture]\label{conj:param}
For any compact stratified manifold $X$ of dimension $d$
with strata of codimension $k=1,2,\dots,d$,
the Laplacian with Dirichlet conditions on all strata admits
a litho-parametrix whose heat kernel expansion separates
contributions according to codimension:
\[
\mathrm{Tr}\,e^{-tL_X} \sim \sum_{k=0}^d \sum_{\Sigma^{d-k}\subset X} a_{\Sigma^{d-k}}\, t^{-k/2}.
\]
\end{conjecture}

The conjecture asserts the universality of the codimension-order law,
extending Theorem~\ref{thm:strat} to general settings.

\paragraph{Spectral gaps and resonances.}
Litho-geometry suggests that internal hypersurfaces act as barriers
that can produce spectral gaps and resonances.
A natural question is whether complexity $\kappa(\Gamma)$
provides quantitative lower bounds for spectral gaps.
We pose:

\begin{conjecture}[Spectral gap control]\label{conj:gap}
Let $(\Omega,\Gamma)$ be a litho-domain with complexity $\kappa(\Gamma)$.
Then there exists $\delta>0$ such that the first spectral gap satisfies
\[
\lambda_2-\lambda_1 \ge c(\Omega)\,\kappa(\Gamma)^{-\delta}.
\]
\end{conjecture}

This conjecture links spectral distribution with geometric complexity.

\subsection*{1.9.2. Connections to number theory}

\paragraph{Arithmetic litho-domains.}
In classical analytic number theory,
spectral methods connect the Laplacian on arithmetic manifolds
to zeta and $L$-functions.
Lithomathematics suggests enriching this setting
by imposing Dirichlet conditions along internal hypersurfaces
defined by arithmetic subvarieties.

\begin{definition}[Arithmetic litho-domain]
Let $X$ be an arithmetic quotient of a symmetric space,
e.g. $X=\Gamma\backslash \mathbb{H}^2$ for $\Gamma\subset SL_2(\mathbb{Z})$.
Let $\Gamma\subset X$ denote a union of geodesics corresponding
to Hecke correspondences.
The pair $(X,\Gamma)$ is an \emph{arithmetic litho-domain}.
\end{definition}

\paragraph{Conjectural link to prime distributions.}
The heat trace on arithmetic litho-domains is conjectured to encode
arithmetically meaningful terms.

\begin{conjecture}[Litho-Arithmetic Trace Conjecture]\label{conj:arith}
For arithmetic litho-domains $(X,\Gamma)$,
the litho-coefficient $a_\Gamma$ in the heat expansion
relates to prime geodesic distributions
in the Selberg trace formula,
and thus encodes information analogous to
the zeros of the Selberg zeta function.
\end{conjecture}

This conjecture provides a conceptual bridge between litho-invariants
and prime distributions.

\paragraph{Hypothesis on RH reformulation.}
While lithomathematics does not claim to prove the Riemann Hypothesis,
we may conjecture a reformulation:

\begin{conjecture}[Litho-RH Reformulation]\label{conj:RH}
There exists an arithmetic litho-domain $(X,\Gamma)$
such that the non-trivial zeros of $\zeta(s)$
correspond to resonances of $L_{X,\Gamma}$,
and the condition $\Re(s)=1/2$ is equivalent to
a symmetry of the litho-invariant $K_L(X,\Gamma)$.
\end{conjecture}

This conjecture should be understood as a programmatic reformulation,
not as a theorem.

\subsection*{1.9.3. Links to algebra and topology}

\paragraph{Group invariants.}
In Part~1.7 we defined the litho-functor
from finite groups to litho-domains.
An open question is whether the invariant $K_L(G)$
encodes algebraic data such as group growth rates or cohomology ranks.

\begin{problem}[Group growth conjecture]
Does $K_L(G)$ distinguish between groups of polynomial and exponential growth?
\end{problem}

\paragraph{Algebraic geometry.}
For an algebraic curve $C$ over $\mathbb{C}$,
one may define litho-divisors $\Gamma$ consisting of singular points or nodal curves.
The invariant $K_L(C,\Gamma)$ could serve as a new measure
of complexity for degenerations.

\subsection*{1.9.4. Probabilistic models and universality}

\paragraph{Random ensembles of hypersurfaces.}
Lithomathematics predicts universality of $K_L$
under random ensembles of internal hypersurfaces.

\begin{theorem}[Law of large numbers for $K_L$]\label{thm:LLN}
Let $\{\Gamma_n\}$ be a stationary ergodic sequence of random hypersurfaces
in a fixed domain $\Omega$, with $\sup_n \kappa(\Gamma_n)<\infty$.
Then almost surely,
\[
\lim_{n\to\infty} K_L(\Gamma_n) = K_L^*(d),
\]
where $K_L^*(d)$ is a deterministic constant depending only on the dimension.
\end{theorem}

\paragraph{Central limit behavior.}
Beyond the law of large numbers,
we conjecture that fluctuations satisfy a central limit theorem.

\begin{conjecture}[Litho-CLT]\label{conj:CLT}
For stationary ergodic ensembles as above,
\[
\sqrt{n}\,\big(K_L(\Gamma_n)-K_L^*(d)\big)\;\;\Longrightarrow\;\; \mathcal{N}(0,\sigma^2),
\]
for some variance $\sigma^2>0$ depending on the ensemble.
\end{conjecture}

\paragraph{Universality classes.}
The constants $K_L^*(d)$ play a role analogous to universality constants
in random matrix theory.
However, they are geometric rather than stochastic:
determined by dimensional and geometric scaling,
not by matrix ensembles.

\subsection*{1.9.5. Roadmap summary}

\begin{enumerate}
\item Analytical extension: parametrices for general stratified spaces (Conjecture~\ref{conj:param}).
\item Geometric control: bounds for spectral gaps (Conjecture~\ref{conj:gap}).
\item Number theory: litho-trace conjecture and RH reformulation (Conjectures~\ref{conj:arith}, \ref{conj:RH}).
\item Algebra: group invariants and algebraic degenerations.
\item Probability: LLN (Theorem~\ref{thm:LLN}) and CLT (Conjecture~\ref{conj:CLT}).
\end{enumerate}

\paragraph{Discussion.}
These directions form a coherent research program
analogous in scope to the Langlands program,
but centered on litho-invariants rather than automorphic forms.
They illustrate the foundational character of lithomathematics:
its methods extend across analysis, geometry, algebra,
number theory, and probability.

\paragraph{Closure.}
Part~1.9 outlined the roadmap and hypotheses of lithomathematics.
We formulated conjectures on stratified parametrices,
spectral gap control, arithmetic connections,
and probabilistic universality.
We carefully distinguished between proven results
(e.g. LLN Theorem~\ref{thm:LLN})
and conjectural directions (Conjectures~\ref{conj:param}–\ref{conj:CLT}).
This ensures the scope of the theory is transparent,
ambitious, and academically precise.

% Part 10 of 10 – Introduction
\subsection*{1.10. Conclusion of the Introduction}

\paragraph{Orientation.}
The purpose of this introduction has been to define the foundational framework
of lithomathematics, to situate it within the history of spectral geometry,
to formulate its basic definitions and invariants,
to present its principal theorems,
to illustrate them with explicit examples,
to compare with existing theories,
to outline connections to other branches of mathematics,
and finally to state the roadmap of conjectural directions.
We now summarize, audit, and close.

\subsection*{1.10.1. Summary of key definitions and invariants}

\paragraph{Litho-domain.}
A litho-domain is a triple $(\Omega,\Gamma,L_\Gamma)$,
where $\Omega$ is a compact Riemannian manifold with smooth boundary,
$\Gamma\subset\Omega$ is a closed rectifiable hypersurface,
and $L_\Gamma$ is the Laplace–Beltrami operator
with Dirichlet conditions imposed both on $\partial\Omega$
and on $\Gamma$.

\paragraph{Complexity.}
The geometric complexity of the internal hypersurface $\Gamma$
is defined as
\[
\kappa(\Gamma) \;=\; 1 \,+\, \mathcal{H}^{d-1}(\Gamma) \,+\, \| \mathrm{II}_\Gamma \|_{L^2(\Gamma)} \,+\, N_{\mathrm{comp}}(\Gamma) \,+\, \mathrm{Cap}_{2,\Omega}(\Gamma),
\]
where $\mathcal{H}^{d-1}$ is the Hausdorff measure,
$\mathrm{II}_\Gamma$ the second fundamental form,
$N_{\mathrm{comp}}(\Gamma)$ the number of connected components,
and $\mathrm{Cap}_{2,\Omega}(\Gamma)$ the $L^2$-capacity of $\Gamma$ in $\Omega$.

\paragraph{Litho-invariant.}
The litho-invariant $K_L(\Omega,\Gamma)$ is defined
through the short-time asymptotics of the heat trace:
\[
\mathrm{Tr}\,e^{-tL_\Gamma} \;\sim\; a_0\,t^{-d/2} \;+\; a_{1/2}\,t^{-(d-1)/2}
\;+\; a_\Gamma\,t^{-(d-1)/2} \;+\; \cdots,
\qquad t\downarrow 0,
\]
with
\[
K_L(\Omega,\Gamma) \;:=\; \frac{a_\Gamma}{a_0+a_{1/2}}.
\]

These definitions establish litho-domains, $\kappa(\Gamma)$,
and $K_L(\Omega,\Gamma)$ as the primary objects of lithomathematics.

\subsection*{1.10.2. Summary of main results}

\paragraph{Theorem A (localized trace).}
For suitable test functions $g$,
\[
\mathrm{Tr}\,g(\sqrt{L_\Gamma}) \;=\; A_{\mathrm{vol}}(g) \,+\, A_{\partial\Omega}(g) \,+\, A_\Gamma(g) \,+\, R(g),
\]
with $|R(g)|\le C\,\kappa(\Gamma)\,\|g\|_{C^{d+3}}$.

\paragraph{Theorem B (polynomial control).}
The litho-term $A_\Gamma(g)$ is bounded by
a polynomial in $\kappa(\Gamma)$,
uniformly over smooth $g$.

\paragraph{Theorem C (dynamic refinement).}
If the geodesic flow on $\Omega\setminus\Gamma$
satisfies exponential mixing,
then the remainder admits a power-saving estimate.

\paragraph{Theorem D (universality).}
For stationary ergodic ensembles of hypersurfaces
with uniformly bounded complexity,
the invariant $K_L(\Gamma)$ converges almost surely
to a deterministic constant $K_L^*(d)$.

\subsection*{1.10.3. Explicit examples}

\begin{itemize}
\item Interval with an interior point:
$K_L$ nontrivial, demonstrating existence even in 1D.
\item Rectangle with interior cut:
universal coefficient $a_\Gamma = -\tfrac14(4\pi)^{-(d-1)/2}|\Gamma|$.
\item Wedge of angle $\pi/2$:
codimension-two strata contribute at order $t^0$,
confirming the codimension-order law.
\item Group example $S_3$:
functorial construction produces a canonical litho-invariant $K_L(S_3)$.
\end{itemize}

These examples confirm that litho-invariants are explicit and computable.

\subsection*{1.10.4. Comparison with existing theories}

\paragraph{Spectral geometry with boundary.}
Classical Weyl–Ivrii–Safarov–Vassiliev theory
accounts for outer boundary terms,
but not for internal hypersurfaces.

\paragraph{Stratified spaces.}
Cheeger’s theory treats global stratifications,
while lithomathematics treats internal hypersurfaces
within otherwise smooth manifolds.

\paragraph{Singular potentials.}
Delta-type potentials (Albeverio, Lapidus)
depend analytically on potentials;
litho-geometry depends geometrically on hypersurfaces.

Thus lithomathematics introduces genuinely new terms.

\subsection*{1.10.5. Scope and non-objectives}

We stress explicitly:

\begin{itemize}
\item Lithomathematics does not address inverse spectral problems.
\item It does not provide algorithms for reconstruction or control.
\item It does not claim to resolve classical conjectures
(e.g. Riemann hypothesis, Birch–Swinnerton-Dyer).
\item Its universality results are analytic and probabilistic,
not arithmetic.
\end{itemize}

\subsection*{1.10.6. Roadmap summary}

We outlined conjectural programs:
general parametrices on stratified manifolds,
spectral gap estimates,
arithmetical litho-domains,
group invariants,
and probabilistic universality (LLN and CLT).

\subsection*{1.10.7. Audit of the introduction}

\begin{itemize}
\item All definitions are explicit and mathematically verifiable.
\item All theorems are clearly marked as proved, conditional, or conjectural.
\item Examples are non-trivial and computable.
\item Literature comparisons are precise.
\item Scope is explicitly delimited.
\item Roadmap is formulated transparently.
\end{itemize}

This audit ensures that the introduction satisfies the lithostandard of exposition:
closed definitions, rigorous statements, explicit examples, and clear scope.

\subsection*{1.10.8. Final closure}

Lithomathematics is thus established as
a new analytic layer of spectral geometry,
focused on internal hypersurfaces as fundamental objects.
It provides universal invariants,
computable examples,
rigorous theorems,
and conjectural programs reaching into number theory,
algebra, topology, and probability.
The subsequent chapters develop these themes in full technical detail.

\hfill $\Box$
