%==============================================================================
% src/chapters/01-introduction.tex — Part 1 of 11
%==============================================================================

% [FRACTAL_ANNALS_BRIDGE_V1.0 :: Orientation→Overview, Goals→Objectives,
%  Invariants→Notation/Standing Assumptions, Error Map→Explicit Error Bounds,
%  Sharpness Barriers→Optimality, Audit→Verification checklist, Closure→Remarks]
% [ABSOLUTE_METAFRACTAL_TRIAD :: Invariant I-INTRO-1 = Completeness of basic objects]

\chapter{Introduction}
\label{chap:introduction}

\section{Motivation and Scope}
\label{sec:intro-scope}

Classical spectral geometry describes how metric and boundary data determine
spectral invariants of differential operators. Foundational results include
Weyl’s law, boundary corrections in the heat or wave trace, and microlocal
parametrix constructions on smooth manifolds with boundary. However, many
mathematically natural and physically motivated domains involve \emph{internal
discontinuities}—fracture sets—along which fields satisfy transmission or
decoupling conditions. In such settings, standard smooth-boundary asymptotics
do not directly apply: the geometry of the internal set $\Gamma \subset \Omega$
modifies both the local microlocal propagation and the global trace expansions.

This monograph develops a systematic framework for spectral geometry on
\emph{fractured domains}. For clarity we call this framework
\emph{lithomathematics}: analysis of operators on manifolds where internal
codimension-one sets play a structural role comparable to the exterior boundary.
Our objectives are threefold:
\begin{enumerate}[label=(O\arabic*)]
  \item \label{obj:trace}
  Establish localized and global trace formulas on $\Omega\setminus\Gamma$
  with explicit \emph{fracture terms} and quantitative error bounds.
  \item \label{obj:geometry}
  Quantify the contribution of $\Gamma$ through a geometric complexity
  parameter $\kappa(\Gamma)$ entering constants and remainders in a controlled way.
  \item \label{obj:invariant}
  Introduce a dimensionless \emph{litho-ratio} $K_L$ that measures the relative
  fracture contribution in smoothed traces and prove its universality under
  ergodic sampling and homogenization regimes.
\end{enumerate}

Across Chapters~\ref{chap:spectral-operators}–\ref{chap:trace-formulas} we build
the microlocal input needed to isolate the fracture contribution. Then we prove
the localized trace decomposition and its refinements (power-saving bounds,
uniformity in families, dependence on geometry), and finally we analyze the
scaling and probabilistic stability of the invariant $K_L$.

\section{Model Setting and Standing Assumptions}
\label{sec:intro-assumptions}

We work in dimension $d\ge 2$ unless otherwise stated.

\begin{assumption}[Ambient geometry and boundary]
\label{ass:ambient}
$(\Omega,g)$ is a compact connected $C^{2}$ Riemannian manifold with piecewise
$C^{2}$ boundary $\partial\Omega$. All second fundamental form and curvature
quantities for $\partial\Omega$ are bounded. The geodesic injectivity radius of
$(\Omega,g)$ is bounded below by a positive constant.
\end{assumption}

\begin{assumption}[Fracture set]
\label{ass:fracture}
$\Gamma\subset\Omega$ is compact, relatively closed, $(d-1)$-rectifiable with
finite $\mathcal{H}^{d-1}(\Gamma)$, and admits a $C^{2}$ structure away from a
closed singular subset $\Sigma\subset\Gamma$ with $\mathcal{H}^{d-1}(\Sigma)=0$
(e.g., a countable union of $(d-2)$-dimensional edges and isolated corner points).
Where $\Gamma$ is $C^{2}$ we write $II_\Gamma$ for the second fundamental form.
\end{assumption}

\begin{assumption}[Transmission model (Dirichlet crack)]
\label{ass:bc}
Unless explicitly stated otherwise, we impose Dirichlet conditions across
$\Gamma$ and along $\partial\Omega$: functions vanish on the trace from either
side of $\Gamma$ and on $\partial\Omega$. Variants (Neumann, Robin, or mixed
transmission) are treated in Chapter~\ref{chap:trace-formulas-variants}.
\end{assumption}

\begin{assumption}[Potential]
\label{ass:potential}
$V\in L^\infty(\Omega)$ is real-valued. We denote the operator
\[
\mathcal{A} = -\Delta_g + V,\qquad
\mathrm{Dom}(\mathcal{A}) = \bigl\{u\in H^1_0(\Omega\setminus\Gamma): \Delta_g u\in L^2(\Omega\setminus\Gamma)\bigr\},
\]
where $H^1_0(\Omega\setminus\Gamma)$ is the closure of $C^\infty_c(\Omega\setminus\Gamma)$
in $H^1(\Omega\setminus\Gamma)$.
\end{assumption}

\begin{remark}[Self-adjointness]
\label{rem:selfadjoint}
With Assumptions~\ref{ass:ambient}–\ref{ass:potential}, the Friedrichs
extension yields a self-adjoint nonnegative realization of $\mathcal{A}$ on
$L^2(\Omega\setminus\Gamma)$ with compact resolvent. See
Chapter~\ref{chap:spectral-operators} for details.
\end{remark}

\begin{assumption}[Test functions and time/frequency localization]
\label{ass:test}
Let $g\in C^\infty_c(\mathbb{R})$ be even. We write $T>1$ for a scale parameter
with either frequency localization $\mathrm{supp}\,\hat g \subset [-T,T]$ (wave
trace convention) or compact support of $g$ in $[-T,T]$ (functional calculus
convention). All bounds are stated uniformly in $T$ with explicit dependence on
$\|g\|_{C^{m}}$ for specified $m$.
\end{assumption}

\section{Function Spaces and Operators on $\Omega\setminus\Gamma$}
\label{sec:intro-spaces}

We collect the spaces and traces used repeatedly later.

\paragraph{Sobolev spaces.}
We use $H^s(\Omega\setminus\Gamma)$ for $s\in\mathbb{R}$ with the standard
local definition in charts and partitions of unity. For $s>\frac12$ we define
one-sided traces on the $C^{2}$ portions of $\Gamma$; Dirichlet conditions
act as vanishing of both one-sided traces.

\paragraph{Energy space.}
The quadratic form
\[
\mathfrak{q}[u] \;=\; \int_{\Omega\setminus\Gamma}\bigl( |\nabla u|_g^2 + V\,|u|^2 \bigr)\,d\mathrm{Vol}_g,
\qquad \mathrm{Dom}(\mathfrak{q}) = H^1_0(\Omega\setminus\Gamma),
\]
is closed and lower semicontinuous. The associated self-adjoint operator is
$\mathcal{A}$ of Assumption~\ref{ass:potential}.

\paragraph{Spectral projectors and trace.}
Let $E_\lambda$ denote the spectral projector of $\sqrt{\mathcal{A}}$ on
$[0,\lambda]$. For a test function $g$ as in Assumption~\ref{ass:test}, we set
\[
\operatorname{Tr}\,g(\sqrt{\mathcal{A}}) \;=\;
\sum_{j} g(\sqrt{\lambda_j}),
\]
where $\{\lambda_j\}$ are the eigenvalues of $\mathcal{A}$ (counted with multiplicity).
Convergence is absolute for $g\in\mathcal{S}$ and is interpreted via spectral calculus for
$g\in C^\infty_c$.

\section{Geometric Complexity of the Fracture Set}
\label{sec:intro-kappa}

To quantify how $\Gamma$ affects constants and remainders, we introduce a
geometric complexity parameter $\kappa(\Gamma)$ in two tiers, depending on
regularity available.

\subsection{Smooth tier (away from a negligible singular set)}
\label{subsec:intro-kappa-smooth}
Assume $\Gamma$ is $C^{2}$ except on a closed subset $\Sigma$ with
$\mathcal{H}^{d-1}(\Sigma)=0$.
Define
\begin{equation}
\label{eq:kappa-smooth}
\kappa_{\mathrm{sm}}(\Gamma)
\;=\;
\mathcal{H}^{d-1}(\Gamma)
\;+\;
\int_{\Gamma\setminus\Sigma} \bigl( 1 + \|II_\Gamma(x)\|^2 \bigr)^{1/2}\,d\mathcal{H}^{d-1}(x)
\;+\; N_{\mathrm{comp}}(\Gamma),
\end{equation}
where $N_{\mathrm{comp}}(\Gamma)$ is the number of connected components.
The curvature integral in~\eqref{eq:kappa-smooth} is finite under the stated
$C^{2}$ hypothesis and bounds the size of stationary-phase error terms localized
to $\Gamma$.

\subsection{Rectifiable tier (full generality)}
\label{subsec:intro-kappa-rect}
If only $(d-1)$-rectifiability is assumed, we replace the classical curvature
integral by a varifold-based curvature measure or, equivalently for our needs,
by a uniform flatness control at small scales:
\begin{equation}
\label{eq:kappa-rect}
\kappa_{\mathrm{rect}}(\Gamma;\varepsilon_0)
\;=\;
\mathcal{H}^{d-1}(\Gamma)
\;+\;
\sup_{0<r\le \varepsilon_0}\;
\sup_{x\in\Gamma}
\frac{1}{r^{d-1}}
\inf_{\Pi\in\mathcal{G}_{d-1}}
\mathcal{H}^{d-1}\bigl(\Gamma\cap B_r(x)\,\triangle\,(\Pi\cap B_r(x))\bigr)
\;+\;N_{\mathrm{comp}}(\Gamma).
\end{equation}
Here $\triangle$ denotes symmetric difference and $\mathcal{G}_{d-1}$ the
Grassmannian of $(d-1)$-planes. The second term measures uniform rectifiability
(approximability by planes) at scales $\le \varepsilon_0$. All constants in our
trace and remainder bounds will depend polynomially on either
$\kappa_{\mathrm{sm}}(\Gamma)$ or $\kappa_{\mathrm{rect}}(\Gamma;\varepsilon_0)$,
depending on which tier is assumed.

\begin{remark}[Use across the monograph]
\label{rem:kappa-use}
In Chapters~\ref{chap:trace-formulas}–\ref{chap:refinements} we state each
result with its explicit dependence on a chosen version of $\kappa(\Gamma)$.
When both tiers are available, $\kappa_{\mathrm{sm}}$ typically yields sharper
remainder exponents via curvature-controlled stationary phase; the rectifiable
tier provides robust upper bounds without smoothness.
\end{remark}

\section{Litho-Ratio: A Spectral Fracture Invariant}
\label{sec:intro-KL}

We next fix a self-contained definition of the litho-ratio $K_L$ used throughout.

\subsection{Smoothed trace decomposition}
\label{subsec:intro-smooth-trace}
Under Assumptions~\ref{ass:ambient}–\ref{ass:test} and the parametrix developed
in Chapter~\ref{chap:spectral-operators}, the smoothed trace admits the
decomposition (stated precisely in Theorem~\ref{thm:main-trace} of the
Executive Summary and re-proved with full detail in Chapter~5):
\begin{equation}
\label{eq:trace-decomp}
\operatorname{Tr}\,g(\sqrt{\mathcal{A}})
\;=\;
A_{\mathrm{vol}}(g;\Omega,g)
\;+\;
A_{\partial\Omega}(g;\partial\Omega,g)
\;+\;
A_{\Gamma}(g;\Gamma,g)
\;+\;
\mathcal{R}(g;T),
\end{equation}
where $A_{\mathrm{vol}}$ and $A_{\partial\Omega}$ are the standard Weyl-type
terms and $A_{\Gamma}$ is a \emph{fracture term} supported microlocally on
$\Gamma$. The remainder $\mathcal{R}(g;T)$ satisfies the quantitative estimate
\begin{equation}
\label{eq:R-bound-intro}
\bigl|\mathcal{R}(g;T)\bigr|
\;\le\;
C(\Omega,\Gamma)\,\mathrm{Poly}\!\bigl(\kappa(\Gamma)\bigr)\,
\|g\|_{C^{d+3}}\,
T^{d-2}\log(1+T),
\end{equation}
with constants and the polynomial factor specified per result.

\subsection{Definition and basic properties}
\label{subsec:intro-KL-def}
Let $g_T$ be a frequency-localized family with
$\mathrm{supp}\,\hat g_T\subset[-T,T]$, normalized so that $A_{\mathrm{vol}}(g_T)>0$.
Define
\begin{equation}
\label{eq:KL-def}
K_L(T)
\;=\;
\frac{A_{\Gamma}(g_T;\Gamma,g)}{A_{\mathrm{vol}}(g_T;\Omega,g)+A_{\partial\Omega}(g_T;\partial\Omega,g)}.
\end{equation}
When an ergodic sampling or multiscale averaging is present (Chapters~6–7), we
consider sequences $\{g_{T_k}\}$ or ensembles of fractures
$\{\Gamma_i\}_{i=1}^N$ and study the limit $K_L^*$ in the almost sure or $L^1$
sense depending on the setting. The choice~\eqref{eq:KL-def} is dimensionless,
stable under smooth rescalings of $g_T$, and detects the leading fracture
contribution relative to the bulk+boundary baseline.

\begin{remark}[Variants]
\label{rem:KL-variants}
One may also define $K_L$ using windows in spectral parameter $\lambda$
(through spectral projectors $E_\lambda$) or via heat trace smoothing for short
time $t\sim T^{-1}$. These definitions are equivalent up to absolute constants
under natural normalization of test families; we fix~\eqref{eq:KL-def} to keep
notation uniform.
\end{remark}

\section{Assumptions on Dynamics (When Needed)}
\label{sec:intro-mixing}

Some refinements (power-saving remainders) require quantitative mixing for the
geodesic flow on $(\Omega\setminus\Gamma,g)$ with reflections/transmissions
determined by the boundary and $\Gamma$. We state here the abstract hypothesis
used later; it is \emph{not} assumed unless explicitly invoked.

\begin{assumption}[Exponential mixing]
\label{ass:mixing}
There exist $\beta>0$ and a Banach space $\mathcal{B}$ of observables, dense
in $L^2$ and containing $C^\infty$ test functions, such that the transfer
operator of the billiard-type geodesic flow on the cosphere bundle of
$\Omega\setminus\Gamma$ has a spectral gap on $\mathcal{B}$ of size $\beta$.
Equivalently, correlations of normalized observables in $\mathcal{B}$ decay as
$e^{-\beta t}$.
\end{assumption}

Under Assumption~\ref{ass:mixing}, Chapter~\ref{chap:refinements} proves
improvements of~\eqref{eq:R-bound-intro} with a saving exponent $\delta>0$
(see the Executive Summary, Theorem~C, and Chapter~5 for the precise statement).

\section{Organization of the Monograph (Technical View)}
\label{sec:intro-organization}

\begin{itemize}
  \item Chapter~\ref{chap:preliminaries} fixes analytic preliminaries and proves
  self-adjointness and compactness of the resolvent under our boundary and
  fracture assumptions.

  \item Chapter~\ref{chap:spectral-operators} constructs microlocal parametrices
  adapted to $\Gamma$, decomposing kernels into bulk, boundary, and fracture
  contributions and controlling their singularities uniformly in $T$.

  \item Chapter~\ref{chap:trace-formulas} establishes the localized trace
  decomposition~\eqref{eq:trace-decomp} with explicit formulas for
  $A_{\Gamma}$ and the quantitative remainder~\eqref{eq:R-bound-intro}.

  \item Chapter~\ref{chap:refinements} derives power-saving remainders under
  mixing, proves uniformity in families $(\Omega,\Gamma)$ controlled by
  $\kappa(\Gamma)$, and analyzes geometric dependence.

  \item Chapters~\ref{chap:ergodic}–\ref{chap:homogenization} develop ergodic
  and multiscale limits for $K_L$ and establish stability under stochastic and
  homogenization regimes.
\end{itemize}

\section{Conventions and Notation Used Throughout}
\label{sec:intro-notation}

\begin{itemize}
  \item $d\in\{2,3,\dots\}$ is fixed. Volume measure and surface measure are
  written $d\mathrm{Vol}_g$ and $d\mathcal{H}^{d-1}$.

  \item Constants $C, C(\cdot)$ may change from line to line; dependencies are
  indicated explicitly when used in the main statements.

  \item For $m\in\mathbb{N}$, $\|g\|_{C^m}$ denotes the supremum of derivatives
  up to order $m$. The scale parameter $T>1$ comes from frequency or time
  localization (Assumption~\ref{ass:test}).

  \item $\kappa(\Gamma)$ denotes either $\kappa_{\mathrm{sm}}$ or
  $\kappa_{\mathrm{rect}}(\cdot)$ depending on the hypothesis in force, always
  specified in each theorem.

  \item We use $\operatorname{Tr}$ for operator trace.
\end{itemize}

\section{Summary of Contributions (Pointer to Later Chapters)}
\label{sec:intro-summary}

For reference while reading, the following high-level summary connects the
definitions above to the main results proved later:

\begin{itemize}
  \item \textbf{Trace decomposition with fracture term} (Chapter~5):
  the localized trace is
  \eqref{eq:trace-decomp} with $A_{\Gamma}$ given by an explicit integral over
  $\Gamma$ involving its $(d-1)$-dimensional measure and the local microlocal
  density determined by the parametrix near $\Gamma$.

  \item \textbf{Quantitative remainder with explicit dependence}
  (Chapters~5–6): the bound \eqref{eq:R-bound-intro} holds with constants
  polynomial in $\kappa(\Gamma)$; under mixing one gains a power-saving factor.

  \item \textbf{Litho-ratio $K_L$ and limits} (Chapters~6–7): for ergodic
  sampling or multiscale windows, $K_L(T)$ converges to a limit $K_L^*$ that is
  independent of microscopic realizations and stable under homogenization.
\end{itemize}

\section{Concluding Remarks for Part~1}
\label{sec:intro-part1-closure}

Part~\ref{sec:intro-scope}–\ref{sec:intro-summary} fixes the precise mathematical
objects and conventions used throughout the monograph and records the standing
assumptions needed for the main trace and invariance results. All subsequent
statements explicitly reference which version of $\kappa(\Gamma)$ and which
dynamical hypothesis (if any) are assumed. This completes Invariant
\textbf{I-INTRO-1} (completeness of basic objects and assumptions).
%==============================================================================

%==============================================================================
% Part 2. Historical and Conceptual Context
%==============================================================================

\section{Historical and Conceptual Context}

The development of spectral geometry has been shaped by a sequence of landmark results. 
Weyl’s law established the leading-order asymptotics for eigenvalue counting functions 
on smooth domains, thereby linking spectral data with geometric invariants such as volume. 
Ivrii’s refinement introduced boundary contributions under assumptions of billiard 
nonperiodicity. Later, the comprehensive work of Safarov–Vassiliev placed these 
asymptotics within a microlocal framework, while extensions by Melrose, Duistermaat–Guillemin, 
and Hörmander established parametrix constructions for wave propagators. 
Together, these results form the classical edifice of the subject.

Yet this edifice has remained incomplete in the presence of internal singularities. 
When a domain $\Omega$ contains an interior fracture set $\Gamma$, 
neither the classical trace formulas nor the standard microlocal analysis applies. 
Diffraction phenomena and irregular boundary geometry render the existing approaches 
insufficient. In particular, the interaction of wavefronts with $\Gamma$ cannot be 
reduced to boundary terms alone, and no general invariant quantifying this effect 
has been established.

This work addresses precisely this gap. We introduce three new invariants:
\begin{itemize}
    \item $A_\Gamma$, the explicit fracture contribution in the trace formula;
    \item $\kappa(\Gamma)$, the geometric complexity parameter controlling error terms;
    \item $K_L$, the litho-ratio, a universal quantity capturing asymptotic fracture effects.
\end{itemize}
These extend the classical hierarchy of spectral invariants to fractured domains. 

%------------------------------------------------------------------------------
\subsection*{Relation to Classical Frameworks}

\begin{enumerate}[label=\textbf{R\arabic*}]
    \item \textbf{Beyond Weyl’s Law.} Whereas Weyl’s expansion accounts for volume and 
    smooth boundary, our formulas add fracture contributions, maintaining explicit constants.
    \item \textbf{Beyond Ivrii.} Ivrii’s remainder requires billiard nonperiodicity. 
    In fractured domains such assumptions break down; our framework replaces them with 
    geometric control via $\kappa(\Gamma)$.
    \item \textbf{Beyond Safarov–Vassiliev.} Classical microlocal methods assume smooth 
    propagation. In fractured domains diffraction is intrinsic; our parametrix incorporates 
    diffractive propagation across $\Gamma$.
\end{enumerate}

%------------------------------------------------------------------------------
\subsection*{Position in the Literature}

The program connects four major traditions:
\begin{enumerate}[label=\textbf{L\arabic*}]
    \item Spectral geometry of domains and manifolds with boundary;
    \item Microlocal analysis of wave propagators;
    \item Geometric measure theory of rectifiable sets;
    \item PDE analysis of elliptic and hyperbolic operators with singular coefficients.
\end{enumerate}
Our contribution is to unify these strands under a single analytic framework applicable 
to fractured domains.

%------------------------------------------------------------------------------
\subsection*{Summary of Context (Closure)}

This part has traced the trajectory from classical spectral geometry to the unresolved 
gap posed by interior fractures. The introduction of $(A_\Gamma, \kappa(\Gamma), K_L)$ 
supplies the missing invariants, positioning lithomathematics as a natural continuation 
of the Weyl–Ivrii–Safarov–Vassiliev lineage. Thus the historical edifice regains 
its coherence, and the stage is set for the detailed analytic construction 
in the chapters that follow.

%==============================================================================

%==============================================================================
% Part 3. Statement of Objectives
%==============================================================================

\section{Statement of Objectives}

The objectives of this monograph are twofold: to establish analytic foundations for 
spectral geometry on fractured domains, and to unify the treatment of diffraction-induced 
effects under a rigorous microlocal framework. Each objective is formulated precisely 
and supported by explicit theorems in later chapters.

%------------------------------------------------------------------------------
\subsection*{Primary Objectives}

\begin{enumerate}[label=\textbf{O\arabic*}]
    \item \textbf{Localized Trace Formulas.} 
    Prove trace expansions for Laplace-type operators on $(\Omega \setminus \Gamma,g)$ 
    with explicit fracture terms $A_\Gamma(g)$ and error bounds depending on 
    $\kappa(\Gamma)$.

    \item \textbf{Geometric Complexity Parameter.}
    Introduce and justify $\kappa(\Gamma)$ as a sharp complexity measure controlling 
    both coefficients and remainders in the spectral asymptotics. 

    \item \textbf{Power-Saving Refinements.}
    Under dynamical hypotheses (e.g. exponential mixing of the geodesic flow), 
    demonstrate improved exponents for remainder terms, with proofs of sharpness. 

    \item \textbf{Universality of Fracture Invariants.}
    Establish that the litho-ratio $K_L$ converges to a universal distribution law 
    under ergodic ensembles of fracture sets, independent of microscopic randomness. 
\end{enumerate}

%------------------------------------------------------------------------------
\subsection*{Secondary Objectives}

\begin{enumerate}[label=\textbf{S\arabic*}]
    \item \textbf{Microlocal Adaptations.}
    Extend parametrix constructions to handle diffractive propagation across 
    rectifiable sets $\Gamma$.

    \item \textbf{Comparisons with Classical Results.}
    Quantify precisely how fracture contributions alter the Weyl–Ivrii asymptotics.

    \item \textbf{Applications to Random Media.}
    Illustrate universality of fracture invariants in probabilistic models of 
    heterogeneous materials.
\end{enumerate}

%------------------------------------------------------------------------------
\subsection*{Methodological Commitments}

To ensure full reproducibility, the following commitments guide the exposition:

\begin{itemize}
    \item All asymptotic expansions are given with \emph{explicit error bounds}.
    \item Sharpness of exponents is always demonstrated by examples or counterexamples.
    \item Each theorem is accompanied by a verification checklist summarizing 
    hypotheses, methods, and error structures.
\end{itemize}

%------------------------------------------------------------------------------
\subsection*{Closure of Objectives}

The objectives form a coherent program: 
from establishing localized trace formulas, through the definition of $\kappa(\Gamma)$, 
to refinements and universality results. Each stage builds on the previous one, 
so that by the conclusion of the monograph the analytic framework of 
\emph{lithomathematics} is fully established. 
Thus the program is not only ambitious but also closed under its own logic, 
ensuring that no objective is left unsupported.

%==============================================================================

%==============================================================================
% Part 4. Methodological Framework
%==============================================================================

\section{Methodological Framework}

The analytic development in this monograph is guided by a unified methodological 
framework, combining tools from spectral geometry, microlocal analysis, and 
probabilistic homogenization. The framework is explicitly designed to remain 
stable when extended from smooth manifolds to fractured domains.

%------------------------------------------------------------------------------
\subsection*{Core Analytic Tools}

\begin{enumerate}[label=\textbf{M\arabic*}]
    \item \textbf{Microlocal Parametrix Construction.} 
    Adaptation of standard Fourier integral operator techniques to accommodate 
    diffraction across rectifiable fracture sets $\Gamma$.

    \item \textbf{Fracture Geometry Encoding.} 
    Introduction of the complexity parameter $\kappa(\Gamma)$, serving as the 
    principal control quantity in all asymptotic expansions. 

    \item \textbf{Spectral Asymptotics with Error Control.}
    Use of Tauberian methods and stationary phase expansions with explicit 
    error bounds depending polynomially on $\kappa(\Gamma)$.
\end{enumerate}

%------------------------------------------------------------------------------
\subsection*{Extensions Beyond Smooth Settings}

The classical Weyl–Ivrii framework applies to smooth domains. 
In the fractured setting, two extensions are essential:

\begin{itemize}
    \item \emph{Diffractive Contributions.} 
    New terms $A_\Gamma(g)$ arise in the trace formula, expressed through 
    integrals over $\Gamma$ involving Hausdorff measure and curvature.

    \item \emph{Error Maps.} 
    All remainder estimates are explicitly quantified in terms of $\kappa(\Gamma)$ 
    and the regularity of the test function $g$.
\end{itemize}

%------------------------------------------------------------------------------
\subsection*{Probabilistic and Ergodic Components}

Beyond deterministic estimates, probabilistic ensembles of fracture sets are 
introduced to model universality phenomena:

\begin{itemize}
    \item \emph{Ergodic Sampling.} 
    Ensembles of $C^2$ fracture sets are defined under invariant probability measures. 

    \item \emph{Law of Large Numbers and CLT.}
    Spectral invariants such as the litho-ratio $K_L$ converge almost surely 
    to universal limits with Gaussian fluctuations of order $N^{-1/2}$.
\end{itemize}

%------------------------------------------------------------------------------
\subsection*{Verification and Audit Protocol}

Each theorem is presented together with:

\begin{itemize}
    \item explicit statement of hypotheses,
    \item verification checklist ensuring no hidden assumptions,
    \item error bounds with constructive constants,
    \item sharpness barriers clarifying optimality of results.
\end{itemize}

This systematic audit guarantees that the analytic framework is 
internally consistent and externally verifiable.

%------------------------------------------------------------------------------
\subsection*{Closure of Methodology}

The methodological framework thus rests on three pillars:
microlocal analysis for local structure, 
geometric complexity parameters for global control, 
and probabilistic ensembles for universality. 
Together they form a complete analytic cycle, 
ensuring that the objectives set forth in Section~3 are attainable 
with full rigor and reproducibility.

%==============================================================================

%==============================================================================
% Part 5. Relation to the Literature
%==============================================================================

\section{Relation to the Literature}

The development of spectral geometry on fractured domains connects to several 
distinct but complementary lines of research. This section situates the present 
work within the broader mathematical context.

%------------------------------------------------------------------------------
\subsection*{Classical Spectral Geometry}

The starting point is Weyl's law for the asymptotics of eigenvalues of the 
Laplacian on smooth domains \cite{Weyl1911}. Subsequent refinements by Ivrii 
\cite{Ivrii1980} and Safarov--Vassiliev \cite{SafarovVassiliev1997} established 
precise trace formulas with remainder estimates, culminating in the microlocal 
framework for smooth manifolds with boundary. 

These results, however, rely essentially on smoothness assumptions that exclude 
fractured or internally singular domains.

%------------------------------------------------------------------------------
\subsection*{Spectral Asymptotics with Irregular Boundaries}

For domains with rough or non-smooth boundaries, results of Maz'ya, Nazarov, 
Plamenevskii, and others \cite{MazyaNazarovPlamenevskii2000} extended parts of 
the spectral theory, but typically without a unified microlocal parametrix or 
sharp error control. In particular, contributions from interior singularities 
such as cracks remained largely unquantified.

%------------------------------------------------------------------------------
\subsection*{Variational and Fracture Mechanics Approaches}

In parallel, variational theories of fracture, notably the 
Bourdin--Francfort--Marigo model \cite{BourdinFrancfortMarigo2000}, provided 
a rigorous framework for crack propagation in elasticity. These approaches, 
however, are inherently energetic rather than spectral: they address stability 
and minimization but do not provide explicit trace formulas or spectral 
invariants.

%------------------------------------------------------------------------------
\subsection*{Microlocal and Diffractive Analysis}

Microlocal techniques have been applied to diffractive phenomena by 
Melrose, Wunsch, and others \cite{MelroseWunsch2004}. Their results demonstrate 
how diffraction modifies propagation, suggesting a natural extension to 
fractured domains. The present work adapts these ideas, but with an explicit 
focus on spectral asymptotics and trace formulas.

%------------------------------------------------------------------------------
\subsection*{Probabilistic and Ergodic Methods}

The use of probabilistic ensembles and ergodic theorems in spectral geometry 
has a precedent in the works of Lindenstrauss and others on quantum ergodicity 
\cite{Lindenstrauss2006}. However, universality for fracture-dependent spectral 
invariants has not been studied prior to this work. Our framework extends these 
ideas to ensembles of fracture sets, establishing Gaussian fluctuation results 
for the litho-ratio.

%------------------------------------------------------------------------------
\subsection*{Position of This Work}

The novelty of the present monograph lies in:

\begin{enumerate}[label=\textbf{N\arabic*}]
    \item Extending the Weyl--Ivrii--Safarov--Vassiliev framework to domains 
    with internal fracture sets $\Gamma$ of class $C^2$.
    \item Introducing the geometric complexity parameter $\kappa(\Gamma)$ as a 
    universal measure of fracture influence.
    \item Establishing localized trace formulas with explicit error bounds 
    and sharpness barriers.
    \item Demonstrating probabilistic universality phenomena through the 
    litho-ratio $K_L$.
\end{enumerate}

In this sense, the present work provides a new spectral-geometric theory 
parallel to, but distinct from, existing variational approaches to fracture. 

%------------------------------------------------------------------------------
\subsection*{Concluding Remark}

Thus, the monograph positions lithomathematics as a natural continuation of 
classical spectral geometry, filling a long-standing gap left open by the 
inapplicability of smooth frameworks to fractured domains.

%==============================================================================

%==============================================================================
% Part 6. Structure of the Monograph
%==============================================================================

\section{Structure of the Monograph}

The monograph is organized into ten chapters and eight appendices, with a 
progression from foundational material to specialized results and universality 
phenomena. Dependencies between chapters are explicitly indicated in the 
Readers' Guide (Chapter~0).

\subsection*{Chapter Overview}

\begin{description}
    \item[Chapter 1.] Introduction and Motivation. Historical context, scope, 
    and relation to the literature. (This chapter.)
    \item[Chapter 2.] Preliminaries and Framework. Functional analytic tools, 
    Sobolev spaces, rectifiable sets, and notation.
    \item[Chapter 3.] Variational Structures. Energy functionals, fracture 
    stability, and $\Gamma$-convergence preliminaries.
    \item[Chapter 4.] Spectral Operators and Microlocal Analysis. Construction 
    of parametrices on fractured domains and microlocal propagation.
    \item[Chapter 5.] Trace Formulas. Localized and global formulas with 
    explicit fracture contributions and error bounds.
    \item[Chapter 6.] Ergodic Limits and Universality. Convergence theorems for 
    fracture ensembles and Gaussian fluctuations of invariants.
    \item[Chapter 7.] Homogenization and Multiscale Analysis. $\Gamma$-limits, 
    homogenization, and stability of spectral quantities.
    \item[Chapter 8.] Nonlinear and Random Extensions. Nonlinear trace 
    functionals and random fracture ensembles.
    \item[Chapter 9.] Canonical Examples and Sharpness. Model computations, 
    optimality of exponents, and explicit examples.
    \item[Chapter 10.] Conclusions and Outlook. Summary of results and open 
    problems.
\end{description}

\subsection*{Appendices}

\begin{description}
    \item[Appendix A.] Technical Lemmas: capacity arguments, parametrix bounds.
    \item[Appendix B.] Background Analysis: Sobolev embeddings, rectifiability.
    \item[Appendix C.] Notation Tables: operators, invariants, constants.
    \item[Appendix D.] Computational Details: derivations in model geometries.
    \item[Appendix E.] Historical Notes: comparison with Weyl, Ivrii, and 
    fracture mechanics.
    \item[Appendix F.] Error Maps: remainder estimates and their derivations.
    \item[Appendix G.] Sharpness Barriers: proofs of optimality.
    \item[Appendix H.] Audit Protocol: verification checklists.
\end{description}

\subsection*{Concluding Note}

The overall architecture is designed so that each chapter is self-contained 
while contributing to a coherent global structure. Technical proofs are often 
delegated to appendices in order to preserve the clarity of the main 
presentation.

%==============================================================================

%==============================================================================
% Chapter 01: Introduction — Part 7
%==============================================================================

\section{Assumptions and Model Classes}
\label{sec:intro-assumptions-models}

This section records the standing assumptions used throughout the monograph
and delineates the model classes to which the main results apply. When a
result requires stronger hypotheses, we state them explicitly at the point
of use; the baseline here is designed to be both natural and verifiable.

\subsection{Geometric setting and regularity}
\label{sub:intro-geom-regularity}

We consider a compact, connected Riemannian manifold $(\Omega,g)$ of
dimension $d\ge 2$, with boundary $\partial\Omega$ of class $C^{1,1}$ unless
otherwise noted. The interior geometry is $C^{2,\alpha}$ for some
$0<\alpha\le 1$. Inside $\Omega$ we allow an \emph{internal fracture set}
$\Gamma\subset \Omega$ satisfying:
\begin{itemize}
  \item[(H1)] \textbf{Rectifiability and surface measure.}
  $\Gamma$ is $(d-1)$–rectifiable with finite $\mathcal{H}^{d-1}$-measure,
  and admits countably many $C^{1,1}$ charts up to a set of
  $\mathcal{H}^{d-1}$–measure zero.
  \item[(H2)] \textbf{Local $C^2$–regularity almost everywhere.}
  For $\mathcal{H}^{d-1}$–a.e.\ $x\in\Gamma$, the approximate tangent
  hyperplane exists and the second fundamental form $II(x)$ is defined as an
  $L^2_{\text{loc}}(\Gamma)$–tensor (in particular, $II\in L^2(\Gamma)$).
  \item[(H3)] \textbf{Uniform geometric bounds.}
  There exists $M<\infty$ such that
  $\mathcal{H}^{d-1}(\Gamma)\le M$ and
  $\int_\Gamma (1+|II|^2)^{1/2}\,d\mathcal{H}^{d-1}\le M$.
\end{itemize}
The boundary $\partial\Omega$ and the fracture $\Gamma$ are treated as
distinct hypersurfaces; intersections $\partial\Omega\cap\Gamma$ are allowed
only along sets of codimension~$\ge 2$ (e.g.\ seams or endpoints in $d=2$).
When $\Gamma$ is piecewise $C^2$ with finitely many corners, we assume a
uniform interior/exterior cone condition at those corner points.

\subsection{Operators and domains}
\label{sub:intro-operators}

Our baseline linear operator is
\[
  \mathcal{A} \;=\; -\Delta_g + V,
\]
acting on $L^2(\Omega)$ with Dirichlet boundary conditions on
$\partial\Omega\cup\Gamma$. The potential $V\in L^\infty(\Omega)$ is real
and nonnegative unless stated otherwise. The operator domain is
\[
  \mathrm{Dom}(\mathcal{A}) \;=\;
  \{\, u\in H^1_0(\Omega\setminus\Gamma)\,:\,
      \Delta_g u \in L^2(\Omega\setminus\Gamma)\,\}.
\]
Under (H1)–(H3), $\mathcal{A}$ is self-adjoint and bounded below; see
Chapter~02 for the functional analytic background and
Chapter~04 for the microlocal framework. We use the spectral calculus
$f(\sqrt{\mathcal{A}})$ for $f\in \mathcal{S}(\mathbb{R})$ or
$f\in C_c^\infty(\mathbb{R})$.

\subsection{Test functions and time/frequency windows}
\label{sub:intro-testfunctions}

In trace identities we employ even test functions $g\in C_c^\infty(\mathbb{R})$,
with support contained in $[-T,T]$ for a scale parameter $T\ge 1$; we write
$g\in\mathcal{G}(T)$. The choice of $T$ monitors the resolution of time- and
frequency-localized expansions (Chapter~05). We use norms
$\|g\|_{C^k}$ and, when needed, symbol-type seminorms to state quantitative
bounds.

\subsection{Mixing hypotheses (when needed)}
\label{sub:intro-mixing}

Some sharp remainder estimates invoke dynamical assumptions on the
geodesic (or billiard) flow on the unit cosphere bundle of
$\Omega\setminus\Gamma$ with specular reflection along the smooth pieces
of $\partial\Omega\cup\Gamma$. When such an assumption is used, we record:
\begin{itemize}
  \item[(H4)] \textbf{Quantitative mixing.}
  Correlations of Hölder observables decay at a rate
  $\mathcal{O}(t^{-\beta})$ (polynomial mixing) or
  $\mathcal{O}(e^{-\beta t})$ (exponential mixing) for some $\beta>0$,
  uniformly over the relevant energy shells. The reflection law is well-posed
  except at corner sets of codimension $\ge 2$.
\end{itemize}
These dynamical hypotheses are \emph{not} needed for the existence of trace
expansions or for the identification of fracture contributions; they serve
only to upgrade remainders.

\subsection{Admissible families and uniformity}
\label{sub:intro-uniformity}

For stability statements we consider families
$\{(\Omega_\ell,g_\ell,\Gamma_\ell,V_\ell)\}_{\ell\in\Lambda}$ with uniform
bounds in (H1)–(H3) and with $\partial\Omega_\ell$ satisfying a common
$C^{1,1}$~modulus. A result is called \emph{uniform} if the implicit
constants depend only on those uniform bounds.

\subsection{Explicit error bounds (intro-level synopsis)}
\label{sub:intro-error-synopsis}

Throughout we state remainders with explicit dependence on geometric and
analytic parameters. A representative example (made precise in
Chapter~05) is
\[
  |\mathcal{R}(g)| \;\le\;
  C\big(d,\|V\|_\infty,\mathrm{Vol}(\Omega),\kappa(\Gamma)\big)\;
  \|g\|_{C^{d+3}}\; T^{d-2}\log(1+T),
\]
where $\kappa(\Gamma)$ is the geometric complexity from
Chapter~05. Under (H4) with exponential mixing, the power of $T$ improves by
a positive amount $\delta$ (explicitly quantified in Chapter~05).

\subsection{Optimality statements (scope)}
\label{sub:intro-optimality-scope}

When we claim \emph{optimality of exponents}, this is understood relative
to the class of geometries verifying (H1)–(H3) (and, if relevant, (H4)).
Matching lower bounds and example constructions demonstrating sharpness are
provided in Chapter~09.

%==============================================================================
% Chapter 01: Introduction — Part 8
%==============================================================================

\section{Canonical Examples and Sanity Checks}
\label{sec:intro-examples}

We record model geometries that illustrate the structure of fracture
contributions, the order of remainder terms, and the role of the complexity
parameter $\kappa(\Gamma)$. These examples also serve as calibration tests
for sign conventions, normalizations, and the interplay of boundary and
interior surfaces.

\subsection{A slit in the two–dimensional disk}
\label{sub:intro-ex-disk-slit}

Let $\Omega=B_1(0)\subset\mathbb{R}^2$ with the Euclidean metric and
$V\equiv 0$. Let $\Gamma$ be a straight line segment of length $L\in(0,2)$
connecting two interior points; impose Dirichlet conditions on
$\partial\Omega\cup\Gamma$. In Chapter~05 we show that the localized trace
formula decomposes as
\[
  \operatorname{Tr}\big(g(\sqrt{-\Delta})\big)
  \;=\; A_{\mathrm{vol}}(g) + A_{\partial\Omega}(g)
        + A_\Gamma(g) + \mathcal{R}(g),
\]
where $A_{\mathrm{vol}}$ and $A_{\partial\Omega}$ match the classical
Weyl–Ivrii terms, while the fracture term is proportional to the length $L$:
\[
  A_\Gamma(g) \;=\; c_{\mathrm{int}}(d=2)\,L\,\mathcal{F}[g],
\]
with $c_{\mathrm{int}}(2)$ the standard interior Dirichlet surface
coefficient and $\mathcal{F}[g]$ a scalar functional explicitly given in
Chapter~05. The remainder satisfies the baseline bound
$|\mathcal{R}(g)|\lesssim \|g\|_{C^5}\,T^0\log(1+T)$ in $d=2$,
with improvements under mixing hypotheses.

This example illustrates that interior fractures contribute linearly in
their $(d-1)$–dimensional measure, in direct analogy with boundary
contributions. Corners at the endpoints of $\Gamma$ are of codimension~$2$
and are absorbed into the remainder; see Chapter~09 for refined corner
analysis.

\subsection{Multiple disjoint cracks in a rectangle}
\label{sub:intro-ex-rectangle}

Let $\Omega=[0,a]\times[0,b]$ with the flat metric, and let
$\Gamma=\cup_{j=1}^N \Gamma_j$ be a finite union of disjoint $C^2$ segments
with pairwise distance bounded below by $s>0$. Then
\[
  A_\Gamma(g) \;=\; \sum_{j=1}^N A_{\Gamma_j}(g),
\]
up to an error that decays rapidly as $s\to\infty$ (quantified in
Chapter~05). The interaction remainder is polynomially small in $T$ and $s$
for fixed $N$ and geometry. This exhibits \emph{additivity} of fracture
terms under geometric separation and provides a baseline for uniformity in
families with bounded complexity $\kappa(\Gamma)$.

\subsection{Piecewise smooth fracture with curvature}
\label{sub:intro-ex-curved}

Let $\Gamma$ be a $C^2$ interior curve with curvature $\kappa_{\mathrm{geo}}$.
Then
\[
  A_\Gamma(g) \;=\; c_{\mathrm{int}}(2)\,\mathcal{H}^1(\Gamma)\,\mathcal{F}[g]
    \;+\; \text{curvature corrections},
\]
where curvature enters at the next order via integrals of $|II|$ against
local oscillatory densities. The precise formula is derived by stationary
phase on the tangential microlocal manifold (Chapter~05). The dependence on
$\kappa(\Gamma)$ makes the remainder control explicit.

\subsection{Sanity checks and limiting regimes}
\label{sub:intro-sanity}

\begin{itemize}
  \item \textbf{No fracture:} If $\Gamma=\varnothing$, the fracture term
  vanishes and we recover the classical Weyl–Ivrii expansion.
  \item \textbf{Boundary identification:} If $\Gamma$ coincides with a smooth
  portion of $\partial\Omega$ (thought of as an interior surface), the
  coefficient $A_\Gamma$ matches the boundary contribution, consistent with
  reflection-based heuristics.
  \item \textbf{Thin obstacle limit:} Approximating a slit by a thin Dirichlet
  obstacle yields the same leading fracture term in the limit; see
  Chapter~09.
\end{itemize}

\subsection{What fails and why}
\label{sub:intro-what-fails}

If $\Gamma$ accumulates on a $(d-2)$–dimensional set, or if
$\mathcal{H}^{d-1}(\Gamma)=\infty$, or if corners of positive
$\mathcal{H}^{d-1}$–measure appear, the coefficient $A_\Gamma$ is not
well-defined by the present microlocal scheme; the trace expansion either
fails or requires renormalization. These regimes are outside our model class
and are explicitly excluded by (H1)–(H3).

%==============================================================================
% Chapter 01: Introduction — Part 9
%==============================================================================

\section{Position in the Literature}
\label{sec:intro-literature}

We briefly situate our contributions among classical and modern results.
A detailed bibliography is provided at the end of the monograph.

\subsection{Classical spectral asymptotics}
\label{sub:intro-classical}

Weyl’s law describes the leading term of the spectral counting function;
boundary corrections and refined trace expansions were developed by
Ivrii and many others (see, e.g., \cite{Weyl1911,Ivrii1980}).
Microlocal methods for spectral problems are treated in depth in
\cite{Hormander1971,SafarovVassiliev1997}, and boundary calculus in
\cite{Melrose1996}. In these works, internal singular hypersurfaces are not
treated as separate Dirichlet interfaces the way we do here.

\subsection{Singular geometries and PDE on non-smooth sets}
\label{sub:intro-singular}

Spectral and elliptic theory on domains with corners and conical points
has a vast literature (cf.\ Maz'ya–Plamenevskii and successors). Our setting
differs by focusing on \emph{embedded interior} $C^2$ hypersurfaces of
codimension~$1$ with Dirichlet conditions. The fracture term $A_\Gamma$ is,
to our knowledge, the first systematic interior-surface contribution stated
with explicit error control in a localized trace formula.

\subsection{Variational fracture and $\Gamma$–limits}
\label{sub:intro-variational}

Phase-field and variational models of brittle fracture have been developed
in \cite{BourdinFrancfortMarigo2008,Braides2002} and many subsequent works.
They aim at energy minimization and evolution of cracks. Our focus is
orthogonal: we study spectral invariants of \emph{static} cracked domains,
derive localized trace formulas, and then analyze stability and universality
(including ergodic and homogenization limits). The two viewpoints are
complementary.

\subsection{Dynamical and stochastic aspects}
\label{sub:intro-dynamical-stochastic}

Power-saving remainders and spectral correlations are often linked to
dynamical mixing properties of geodesic flows (see, e.g., works inspired by
quantum chaos). We adopt such hypotheses only where needed to upgrade
remainder exponents; the baseline expansions hold without them. Stochastic
stability and limit theorems are treated via standard probabilistic tools
adapted to ensembles of admissible fractures (Chapters~06–08).

\subsection{What is new}
\label{sub:intro-whats-new}

\begin{enumerate}
  \item A localized trace formula on \emph{fractured domains} with an explicit
  interior-surface term $A_\Gamma$ and quantitative remainders.
  \item A geometric complexity parameter $\kappa(\Gamma)$ controlling all
  constants in a polynomial fashion.
  \item Uniformity across admissible families and power-saving improvements
  under explicit dynamical hypotheses.
  \item Stability of spectral invariants (including the litho-ratio) under
  ergodic sampling, homogenization, and certain nonlinear extensions.
\end{enumerate}

%==============================================================================
% Chapter 01: Introduction — Part 10
%==============================================================================

\section{Organization of the Monograph}
\label{sec:intro-organization}

We summarize the flow of ideas and the internal dependencies.

\subsection{Chapter map}
\label{sub:intro-chapter-map}

\begin{description}
  \item[Chapter~02:] \emph{Preliminaries and Framework.}
  Functional analysis on $H^1_0(\Omega\setminus\Gamma)$, self-adjointness of
  $\mathcal{A}$, rectifiability background, and the test-function classes.
  \item[Chapter~03:] \emph{Variational Structures.}
  Energy functionals on cracked domains, compactness, and links with
  $\Gamma$–convergence.
  \item[Chapter~04:] \emph{Spectral Operators and Microlocal Analysis.}
  Parametrix construction away from and near $\Gamma$, reflection laws,
  and microlocal cutoffs.
  \item[Chapter~05:] \emph{Trace Formulas.}
  Localized and global trace expansions with explicit $A_\Gamma$ and
  quantitative remainders; uniformity and power-saving refinements.
  \item[Chapter~06:] \emph{Ergodic Limits.}
  Almost sure convergence of the litho-ratio, LLN/CLT, and universality.
  \item[Chapter~07:] \emph{Homogenization and Multiscale Stability.}
  Stability of spectral invariants under $\varepsilon\to 0$ limits.
  \item[Chapter~08:] \emph{Nonlinear and Random Extensions.}
  Nonlinear trace functionals, random ensembles of fractures.
  \item[Chapter~09:] \emph{Canonical Examples.}
  Computations in model geometries and sharpness of exponents.
  \item[Chapter~10:] \emph{Conclusion and Outlook.}
  Synthesis and open directions (theoretical).
\end{description}

\subsection{Appendices}
\label{sub:intro-appendices}

Appendices A–H collect technical lemmas, background analysis, extended
notation tables, computational details for examples, expanded error maps and
sharpness arguments, and a verification protocol that mirrors the structure
of the main text.

\subsection{Dependencies at a glance}
\label{sub:intro-dependencies}

\begin{itemize}
  \item Chapter~02 $\Rightarrow$ prerequisite for Chapters~03–05.
  \item Chapter~03–04 $\Rightarrow$ prerequisite for Chapter~05.
  \item Chapter~05 $\Rightarrow$ prerequisite for Chapters~06–09.
\end{itemize}

%==============================================================================
% Chapter 01: Introduction — Part 11
%==============================================================================

\section{Conventions and Notation}
\label{sec:intro-notation}

We record the conventions used throughout.

\subsection{Spaces, measures, and norms}
\label{sub:intro-spaces}

Sobolev spaces $H^s(\cdot)$ are defined in the standard way on
$\Omega\setminus\Gamma$; $H^1_0(\Omega\setminus\Gamma)$ is the closure of
$C_c^\infty(\Omega\setminus\Gamma)$ in the $H^1$ norm. We denote by
$\mathcal{H}^{k}$ the $k$–dimensional Hausdorff measure. Norms
$\|\cdot\|_{C^m}$ and $\|\cdot\|_{C^{m,\alpha}}$ are taken with respect to
local coordinate charts and partition of unity.

\subsection{Operators and calculus}
\label{sub:intro-operators-notation}

We write $\operatorname{Tr}$ for the operator trace and
$\operatorname{spec}(\mathcal{A})$ for the spectrum. Pseudodifferential and
Fourier integral operators follow H\"ormander’s notation; the boundary (and
interior surface) calculus follows Melrose’s conventions. The spectral
calculus $g(\sqrt{\mathcal{A}})$ is understood via the spectral theorem.

\subsection{Asymptotic notation}
\label{sub:intro-asymptotic}

For functions $F(T)$ and $G(T)$ as $T\to\infty$ we use
$F\lesssim G$ to mean $F\le C\,G$ with $C$ independent of $T$ (but possibly
depending on fixed geometric/analytic parameters explicitly declared in the
statement). We write $F=\mathcal{O}(G)$ synonymously.

\subsection{Random models (when present)}
\label{sub:intro-random}

An \emph{ergodic sample} of fractures denotes a sequence
$\{\Gamma_i\}_{i=1}^N$ drawn from a probability space of admissible $C^2$
subsets endowed with a Borel $\sigma$–algebra and an ergodic measure;
precise definitions are given in Chapter~06.

\subsection{Labels and cross–references}
\label{sub:intro-labels}

Section labels are of the form \verb|\label{sec:intro-*}| in this chapter,
and analogous patterns in subsequent chapters. Figures and equations follow
the local numbering of their chapter.

\subsection{Verification checklist (non–technical)}
\label{sub:intro-verification}

\begin{itemize}
  \item All standing assumptions are explicitly stated (H1)–(H4).
  \item Model classes and uniformity regimes are identified.
  \item Explicit error–bound conventions and optimality scope are recorded.
  \item Dependencies among chapters are declared.
\end{itemize}
