\section*{Part I. Historical Motivation and Basic Definitions}

\noindent
The classical theory of spectral geometry has been largely concerned with smooth domains and manifolds with smooth boundary. The foundational works of Weyl~\cite{Weyl11}, Ivrii~\cite{Ivrii80}, and Safarov--Vassiliev~\cite{SafarovVassiliev97} established asymptotic expansions of the eigenvalue counting function and the heat trace in terms of geometric invariants such as the volume and the boundary area. Later developments by Cheeger~\cite{Cheeger83} and Brüning~\cite{Bruening84} extended this theory to stratified spaces with conical and edge singularities. 

However, one basic situation has not been systematically addressed: the effect of an \emph{internal hypersurface} embedded in a smooth Riemannian manifold. Unlike an external boundary, such an internal cut does not change the topology of the ambient manifold, but it imposes additional boundary conditions that alter the spectrum. This phenomenon is not covered by the classical boundary theory, nor by the stratification theory of Cheeger and Brüning, where strata are intrinsic singularities of the ambient space. In contrast, here the hypersurface is smoothly embedded into a regular manifold and acts as a ``fracture'' carrying its own spectral contribution. The systematic study of such contributions is the subject of this work, which we refer to as \emph{lithomathematics}.

\medskip
\noindent
\textbf{Definition (Litho-domain).} 
Let $(\Omega,g)$ be a compact $d$-dimensional Riemannian manifold ($d \geq 2$) with piecewise $C^2$ boundary. Let $\Gamma \subset \Omega$ be a closed, embedded hypersurface of class $C^2$, transversal to $\partial \Omega$. The associated \emph{litho-Laplacian} $L_\Gamma$ is the self-adjoint realization of $-\Delta_g$ on $L^2(\Omega)$ with Dirichlet conditions both on $\partial \Omega$ and on $\Gamma$. Equivalently, $L_\Gamma$ is the Friedrichs extension of the quadratic form
\[
Q_\Gamma[u] = \int_{\Omega \setminus \Gamma} |\nabla u|^2 \, d\mathrm{vol}_g, 
\qquad 
u \in H^1_0(\Omega \setminus \Gamma).
\]

\medskip
\noindent
\textbf{Definition (Geometric complexity).}
For such a hypersurface $\Gamma$, we define its \emph{geometric complexity} by
\[
\kappa(\Gamma) \;=\; N_{\mathrm{comp}}(\Gamma) \;+\; \mathcal{H}^{d-1}(\Gamma) 
\;+\; \int_\Gamma |H_\Gamma|^2 \, dS,
\]
where $N_{\mathrm{comp}}(\Gamma)$ denotes the number of connected components of $\Gamma$, 
$\mathcal{H}^{d-1}(\Gamma)$ is the $(d-1)$-dimensional Hausdorff measure, and 
$H_\Gamma$ is the mean curvature. This quantity controls the analytic constants appearing in the remainder estimates for heat traces and eigenvalue counts. It satisfies:
\begin{itemize}
\item[(i)] \emph{Invariance under isometries:} $\kappa(\Gamma)$ is unchanged under isometries of $(\Omega,g)$.
\item[(ii)] \emph{Additivity:} If $\Gamma = \Gamma_1 \sqcup \Gamma_2$ is the disjoint union of two closed hypersurfaces, then 
\[
\kappa(\Gamma) = \kappa(\Gamma_1) + \kappa(\Gamma_2).
\]
\item[(iii)] \emph{Scaling law:} If the metric is rescaled by $g \mapsto r^2 g$, then
\[
\mathcal{H}^{d-1}(\Gamma) \mapsto r^{d-1}\mathcal{H}^{d-1}(\Gamma), 
\quad 
\int_\Gamma |H_\Gamma|^2 dS \mapsto r^{d-3}\int_\Gamma |H_\Gamma|^2 dS.
\]
Thus $\kappa(\Gamma)$ grows polynomially in $r$, and this precise scaling behavior motivates the normalization chosen later for the litho-invariant.
\end{itemize}

\medskip
\noindent
\textbf{Heat trace expansion and the litho-term.}
The heat trace of $L_\Gamma$ admits the small-time asymptotic expansion
\[
\mathrm{Tr}\, e^{-tL_\Gamma} \;\sim\; a_0 t^{-d/2} \;+\; a_{1/2} t^{-(d-1)/2} 
\;+\; a_\Gamma t^{-(d-1)/2} \;+\; \cdots, 
\qquad t \downarrow 0,
\]
where
\[
a_0 = (4\pi)^{-d/2}\, \mathrm{vol}(\Omega), 
\qquad 
a_{1/2} = -\tfrac{1}{4}(4\pi)^{-(d-1)/2}\, \mathrm{vol}(\partial\Omega).
\]
The additional coefficient $a_\Gamma$ captures the spectral effect of the internal hypersurface $\Gamma$. As will be proved in Theorem~\ref{thm:fracture-term}, one has
\[
a_\Gamma \;=\; -\tfrac{1}{4}(4\pi)^{-(d-1)/2}\, \mathrm{vol}(\Gamma),
\]
a phenomenon absent from the theory of stratified spaces.

\medskip
\noindent
\textbf{Definition (Litho-invariant).}
The \emph{isotropic litho-invariant} of $(\Omega,\Gamma)$ is defined by
\[
K_L^{\mathrm{iso}}(\Omega,\Gamma) \;=\; \frac{a_\Gamma}{a_0}.
\]
This ratio compares the litho-term to the leading volume term and is scale-invariant by construction: under $g \mapsto r^2 g$, both numerator and denominator scale like $r^d$, making $K_L^{\mathrm{iso}}$ dimensionless. An alternative boundary-normalized invariant 
\[
K_L^{\partial}(\Omega,\Gamma) = \frac{a_\Gamma}{a_{1/2}}
\]
will also be considered in Section~\ref{sec:variants}; the isotropic version is canonical from the viewpoint of similarity theory.

\medskip
\noindent
\textbf{Example (Interval with an interior cut).}
Let $\Omega = [0,\pi] \subset \mathbb{R}$ with $\Gamma = \{\pi/2\}$. Then $L_\Gamma$ is the Dirichlet Laplacian on the two subintervals. Its spectrum consists of $\{(2n)^2\}_{n\geq 1}$, each with multiplicity two. Comparing the heat traces gives
\[
\mathrm{Tr}\, e^{-tL_\Gamma} - \mathrm{Tr}\, e^{-tL_\Omega}
\;\sim\; -\tfrac{1}{2} t^{-1/2}, \qquad t \downarrow 0.
\]
Thus $a_\Gamma = -\tfrac{1}{2}$ in one dimension, confirming that the litho-term exists even in this simplest case. Note that in $d=1$ the exponents in the heat expansion differ: the boundary contribution is of order $t^0$, so the normalization of $K_L$ requires separate treatment (see Section~\ref{sec:dimension-one}). This example demonstrates the nontriviality of the litho-invariant and its consistency with classical expansions.

\section*{Part II. Relation to Existing Theories and Statement of Main Results}

\subsection*{2.1. Relation to classical spectral geometry}
The starting point of this work is the classical asymptotic expansion for the heat trace of the Dirichlet Laplacian $L_\Omega$ on a compact Riemannian manifold $(\Omega,g)$ with smooth boundary $\partial \Omega$:
\[
\mathrm{Tr}\, e^{-t L_\Omega} \;\sim\; a_0 t^{-d/2} \;+\; a_{1/2} t^{-(d-1)/2} \;+\; a_1 t^{-(d-2)/2} \;+\; \cdots, 
\qquad t \downarrow 0,
\]
with coefficients expressed in terms of the volume, the boundary area, and the curvature of $(\Omega,g)$ 
(see Ivrii~\cite{Ivrii80}, Safarov--Vassiliev~\cite{SafarovVassiliev97}). 
These expansions have been the cornerstone of spectral geometry, linking eigenvalue asymptotics to geometric invariants.

\subsection*{2.2. Relation to stratified and singular spaces}
A different but related line of research concerns analysis on stratified spaces with conical and edge singularities 
(Cheeger~\cite{Cheeger83}, Brüning~\cite{Bruening84}, Melrose~\cite{Melrose93}). 
In these works, the singularities are intrinsic to the space itself: strata of lower dimension are part of the definition of the ambient space. 
The contribution of such strata to the heat trace depends sensitively on angles and local geometry near the singularity.

In contrast, the present setting involves a smooth manifold $(\Omega,g)$ into which a closed hypersurface $\Gamma$ is embedded. 
The ambient manifold has no intrinsic singularity; instead, $\Gamma$ carries additional boundary conditions. 
The resulting contribution to the heat trace has the same order as the classical boundary term but is determined by the $(d-1)$-dimensional measure of $\Gamma$. 
This phenomenon is not captured by existing stratification theory.

\subsection*{2.3. Relation to singular perturbations}
Another related direction is the study of Schrödinger operators with singular potentials supported on submanifolds, particularly delta-potentials (Albeverio et al.~\cite{Albeverio04}). 
In those models, the new spectral features arise from the analytic nature of the potential. 
By contrast, in the litho-setting no potential is introduced; the effect stems purely from the geometry of the internal hypersurface and the imposition of Dirichlet conditions along it.

\subsection*{2.4. Statement of main results}
We now summarize the principal results established in this work. 
Precise formulations and proofs are given in Sections~\ref{sec:heat-trace}--\ref{sec:universality}.

\begin{theorem}[Localized heat trace expansion]\label{thm:localized}
Let $(\Omega,g)$ be a compact Riemannian manifold of dimension $d \geq 2$ with piecewise $C^2$ boundary, and let $\Gamma \subset \Omega$ be a closed $C^2$ hypersurface transversal to $\partial \Omega$. 
Then the heat trace of the litho-Laplacian $L_\Gamma$ satisfies
\[
\mathrm{Tr}\, e^{-t L_\Gamma} \;\sim\; a_0 t^{-d/2} \;+\; a_{1/2} t^{-(d-1)/2} \;+\; a_\Gamma t^{-(d-1)/2} \;+\; O(t^{-(d-2)/2}),
\quad t \downarrow 0,
\]
with the additional coefficient
\[
a_\Gamma \;=\; -\tfrac{1}{4}(4\pi)^{-(d-1)/2} \, \mathrm{vol}(\Gamma).
\]
\end{theorem}

\begin{theorem}[Polynomial control]\label{thm:poly}
Under the same assumptions, the litho-term admits the bound
\[
|a_\Gamma| \;\leq\; C_d \, P(\kappa(\Gamma)),
\]
where $P$ is a universal polynomial depending only on $d$, and $C_d$ is a constant independent of $(\Omega,\Gamma)$.
\end{theorem}

\begin{theorem}[Dynamical refinement]\label{thm:dynamics}
If the geodesic flow on $(\Omega \setminus \Gamma,g)$ satisfies a uniform exponential mixing property (denoted H\_mix), then the remainder in Theorem~\ref{thm:localized} improves to
\[
R(t) = O\!\left(\kappa(\Gamma)\, t^{-(d-2)/2 - \delta}\right),
\qquad t \downarrow 0,
\]
for some $\delta > 0$ depending only on the mixing rate.
\end{theorem}

\begin{conjecture}[Universality]\label{conj:universality}
Let $\{\Gamma_n\}$ be a stationary ergodic sequence of $C^2$ hypersurfaces in a fixed compact manifold $(\Omega,g)$ with $\sup_n \kappa(\Gamma_n) < \infty$. 
Then almost surely
\[
\lim_{n\to\infty} K_L^{\mathrm{iso}}(\Omega,\Gamma_n) \;=\; K_L^*(d),
\]
where $K_L^*(d)$ is a universal constant depending only on the dimension.
\end{conjecture}

\subsection*{2.5. Scope of claims}
Theorems~\ref{thm:localized}--\ref{thm:dynamics} are proved in this monograph under the regularity assumptions stated. 
Conjecture~\ref{conj:universality} is not proved; it is formulated as a guiding principle supported by heuristic and numerical evidence (see Section~\ref{sec:universality}). 
We emphasize that no results beyond the scope of the above theorems are claimed.

\section*{Part III. Illustrative Computations (Revised)}

\subsection*{3.1. Purpose and scope}
We present model calculations confirming that the fracture contribution is genuine and computable. All examples below satisfy the standing hypotheses of Section~II unless explicitly stated otherwise. No claim is made beyond the stated assumptions.

\subsection*{3.2. One-dimensional interval with an internal Dirichlet point}
Let $\Omega=(0,\pi)$ and $\Gamma=\{\pi/2\}$. Denote by $L_\Omega$ the Dirichlet Laplacian on $\Omega$ and by $L_\Gamma$ the Dirichlet Laplacian on $\Omega\setminus\Gamma$ (i.e. an additional Dirichlet condition at $x=\pi/2$).

\begin{proposition}\label{prop:1d-interval}
As $t\downarrow0$ one has
\[
\mathrm{Tr}\,e^{-tL_\Omega}\sim \frac{\pi}{2\sqrt{\pi t}}-\frac12+O(e^{-c/t}),\qquad
\mathrm{Tr}\,e^{-tL_\Gamma}\sim \frac{\pi}{2\sqrt{\pi t}}-1+O(e^{-c/t}),
\]
hence
\[
\mathrm{Tr}\,e^{-tL_\Gamma}-\mathrm{Tr}\,e^{-tL_\Omega}\sim -\frac12\qquad (t\downarrow0).
\]
\end{proposition}

\noindent
Thus in $d=1$ the fracture produces a contribution of order $t^{0}$ (not $t^{-(d-1)/2}$). This is consistent with the dimension-dependent scaling of boundary terms and is explicitly separated in our framework.

\subsection*{3.3. Unit square with a horizontal internal segment}
Let $\Omega=(0,1)\times(0,1)$ with the Euclidean metric and let $\Gamma=\{(x,1/2):\,0<x<1\}$. Then $\Gamma$ is a $C^\infty$ interior segment, transversal to $\partial\Omega$ is \emph{not} required since $\Gamma\cap\partial\Omega=\varnothing$.

By separation of variables, admissible $y$-modes are $\sin(2\pi n y)$, $n\ge1$, so the spectrum of $L_\Gamma$ is
\[
\lambda_{m,n}=(\pi m)^2+(2\pi n)^2,\qquad m,n\ge1.
\]
A standard Poisson summation argument shows that, as $t\downarrow0$,
\[
\mathrm{Tr}\,e^{-tL_\Gamma}-\mathrm{Tr}\,e^{-tL_\Omega}
\;=\; -\,\frac{1}{4}\,(4\pi)^{-1/2}\,|\Gamma|\,t^{-1/2}\;+\;O(1),
\]
so here
\[
a_\Gamma \;=\; -\,\frac{1}{4}\,(4\pi)^{-1/2}\,|\Gamma|,
\]
in agreement with the localized trace expansion under the smooth interior fracture hypothesis.

\subsection*{3.4. Wedge domain with a radial segment (non-transversal case)}
Let $\Omega\subset\mathbb{R}^2$ be a wedge of opening angle $\theta$ and let $\Gamma$ be a radial segment from the apex to the interior. Here $\Gamma$ meets $\partial\Omega$ at the apex and at the outer boundary; this violates the standing hypothesis $\Gamma\cap\partial\Omega=\varnothing$ used in our main theorems. In this setting one still obtains a $t^{-1/2}$ contribution associated with the \emph{smooth interior portion} of $\Gamma$ with coefficient $-\frac14(4\pi)^{-1/2}$, while \emph{additional corner terms} arise at the intersection points; those depend on the local angular data (cf.\ Cheeger-type analyses for stratified/corner singularities) and contribute at orders different from $t^{-1/2}$. This example delineates the boundary of applicability of the interior-fracture theorems.

\subsection*{3.5. Synthesis}
The calculations above show:
\begin{enumerate}
\item In $d=1$ the fracture contribution appears at order $t^0$ and can be computed exactly (\autoref{prop:1d-interval}).
\item In $d=2$ with a smooth interior segment one obtains the universal coefficient $-\frac14(4\pi)^{-1/2}$ multiplying $|\Gamma|$ at order $t^{-1/2}$, consistent with the localized trace expansion.
\item When the fracture touches the boundary (non-transversal/corner interaction), the smooth interior contribution persists but corner corrections—outside our standing hypotheses—enter at different orders; these are treated separately in later sections.
\end{enumerate}
All examples are fully consistent with the general framework and clarify the precise role of the hypotheses.

\section*{Part IV. Dynamical Aspects and Refined Remainders}

\subsection*{4.1. Orientation}
The preceding sections established the existence of an additional $t^{-(d-1)/2}$-term in the heat trace expansion caused by an internal hypersurface $\Gamma\subset\Omega$. Beyond the leading asymptotics, a central question in spectral theory is the nature of the error term in trace formulas and its dependence on the underlying dynamics. In this part we outline the dynamical framework governing the remainder in the litho-setting and state refined estimates to be proved in later chapters.

\subsection*{4.2. Geodesic flow away from $\Gamma$}
Let $(\Omega,g)$ be a compact Riemannian manifold of dimension $d\ge2$ with boundary conditions as specified earlier, and let $\Gamma\subset\Omega$ be a smooth, compact, closed hypersurface, transversal to $\partial\Omega$. Consider the unit tangent bundle $S^*\Omega$ with the geodesic flow $\varphi^t$. The presence of $\Gamma$ introduces a distinguished codimension-one submanifold
\[
\Sigma_\Gamma = \{ (x,\xi)\in S^*\Omega : x\in\Gamma,\; \xi\perp T_x\Gamma \},
\]
corresponding to geodesics striking $\Gamma$ orthogonally. For the Dirichlet fracture operator $L_\Gamma$, these geodesics represent directions in which the flow is abruptly terminated. Thus the dynamics relevant for the spectral analysis are those of $\varphi^t$ on $S^*\Omega\setminus\Sigma_\Gamma$, where trajectories are reflected or absorbed at $\Gamma$.

\subsection*{4.3. Mixing hypothesis}
The sharpness of remainder estimates in trace formulas is classically linked to mixing properties of the geodesic flow (see, e.g., results of Duistermaat--Guillemin and Zelditch). In the litho-context we impose the following assumption.

\begin{definition}[Hypothesis H$_{\mathrm{mix}}$]
The geodesic flow on $S^*\Omega\setminus\Sigma_\Gamma$ is exponentially mixing with respect to the Liouville measure induced by $g$.
\end{definition}

This hypothesis is known to hold in several important classes of manifolds (e.g.\ compact negatively curved manifolds) and provides a natural analogue of the conditions under which remainder terms in classical trace formulas exhibit power savings.

\subsection*{4.4. Localized trace formula revisited}
Let $g\in C_c^\infty(\mathbb{R})$ be an even test function with Fourier transform $\widehat g$ compactly supported. Then the localized spectral measure associated with $L_\Gamma$ satisfies
\[
\mathrm{Tr}\, g(\sqrt{L_\Gamma}) 
= \mathrm{Tr}\, g(\sqrt{L_\Omega})
+ A_\Gamma(g) 
+ R_\Gamma(g),
\]
where $A_\Gamma(g)$ is the litho-contribution described previously and $R_\Gamma(g)$ is the remainder. The main goal of this part is to state precise bounds on $R_\Gamma(g)$.

\subsection*{4.5. Refined remainder under dynamical assumptions}
\begin{theorem}[Refined remainder under H$_{\mathrm{mix}}$]\label{thm:refined-remainder}
Suppose Hypothesis H$_{\mathrm{mix}}$ holds. Then for all even $g\in C_c^\infty(\mathbb{R})$ with support of $\widehat g$ contained in $[-T,T]$, one has
\[
R_\Gamma(g) \;=\; O\!\left( \kappa(\Gamma)\,\|g\|_{C^{d+3}}\,
T^{d-2-\delta}\right), \qquad \delta>0,
\]
where $\delta$ depends only on the exponential mixing rate of the flow.
\end{theorem}

\noindent
Thus, in settings with sufficiently chaotic dynamics, the litho-contribution is isolated with a strictly smaller error than in the general case, mirroring classical results for boundary terms.

\subsection*{4.6. Comparison with the boundary case}
For smooth manifolds with boundary, analogous results are well established: under exponential mixing of the billiard flow, the remainder in the boundary trace formula admits power savings (see results of Gérard--Leichtnam, Zelditch). Theorem~\ref{thm:refined-remainder} asserts that the same mechanism extends to interior fractures, with $\kappa(\Gamma)$ controlling the dependence on the hypersurface geometry.

\subsection*{4.7. Discussion and outlook}
This dynamical perspective highlights two essential aspects:

\begin{enumerate}
\item The universality of the litho-term $a_\Gamma$ is independent of the dynamics.
\item The precision of remainder estimates is highly sensitive to dynamical properties of the geodesic flow on $\Omega\setminus\Gamma$.
\end{enumerate}

Later chapters will provide proofs of Theorem~\ref{thm:refined-remainder} and explore quantitative links between $\kappa(\Gamma)$, the geometry of $\Gamma$, and mixing rates.

\subsection*{4.8. Closure}
This part establishes the dynamical framework necessary for refined error estimates in lithomathematics. By explicitly formulating Hypothesis H$_{\mathrm{mix}}$, stating Theorem~\ref{thm:refined-remainder}, and clarifying its relation to the classical boundary case, we provide a seamless bridge from the static asymptotics of the previous parts to the dynamical refinements developed in subsequent chapters.

\section*{Part V. Universality of the Litho–Invariant in Random Ensembles}

\subsection*{5.1. Orientation}
Beyond the deterministic existence and control results of the previous parts, a guiding theme of this monograph is the emergence of \emph{dimension–only} behavior for the normalized fracture contribution in large or high–frequency regimes. This part formulates quantitative universality principles for the litho–invariant in (i) thermodynamic rescaling limits and (ii) frequency–windowed limits under mild randomness and mixing. The statements are precise, free of rhetorical claims, and confined to explicitly stated classes of ensembles.

\subsection*{5.2. Admissible random ensembles of hypersurfaces}
Let $(\Omega,g)$ be a fixed compact $d$–dimensional Riemannian manifold with piecewise $C^2$ boundary, $d\ge2$. Write $\mathfrak{H}(\Omega)$ for the set of compact $C^2$ hypersurfaces $\Gamma\subset\Omega$ transversal to $\partial\Omega$, endowed with the Borel $\sigma$–algebra generated by the $C^2$ topology on embedded submanifolds. For $M\ge1$ define the complexity–bounded class
\[
\mathfrak{H}_M(\Omega)\;=\;\{\Gamma\in\mathfrak{H}(\Omega):\ \kappa(\Gamma)\le M\},
\]
where $\kappa(\Gamma)$ is the geometric complexity recorded in Part~I. We now describe two standard modes of randomization.

\paragraph{(A) Thermodynamic rescaling ensembles.}
For $L\ge1$ consider the rescaled manifold $(\Omega_L,g_L)$ obtained from $(\Omega,g)$ by homothetic dilation $g_L = L^2 g$ (thus $\operatorname{vol}(\Omega_L)=L^d\operatorname{vol}(\Omega)$). Fix a stationary, ergodic law $\mathbf{P}$ on compact $C^2$ hypersurfaces in $\mathbb{R}^d$ with almost sure transversality to smooth boundaries and with finite mean complexity density,
\[
\sup_{x\in\mathbb{R}^d}\ \mathbf{E}\Big[\ \kappa\big(\Gamma\cap B_1(x)\big)\ \Big]\;<\;\infty.
\]
Via finite atlases and standard partition–of–unity transfer, $\mathbf{P}$ induces laws $\mathbf{P}_L$ on $\mathfrak{H}(\Omega_L)$ (details given in the main text). We call $\{\mathbf{P}_L\}_{L\ge1}$ an \emph{admissible thermodynamic ensemble} if, in addition, short–range dependence holds:
there exists $\alpha>0$ such that the $\alpha$–mixing coefficients of the induced random field of second fundamental forms on blocks of size~$1$ decay at least exponentially in the inter–block distance.

\paragraph{(B) Frequency–windowed ensembles on a fixed domain.}
On fixed $(\Omega,g)$ let $\mathbf{P}$ be a probability law on $\mathfrak{H}_M(\Omega)$ with a.s.\ $C^2$ realizations and exponential $\alpha$–mixing for the induced random field of geometric jets in a finite atlas. Let $g_T\in C_c^\infty(\mathbb{R})$ be even, with $\operatorname{supp}\widehat g_T\subset[-T,T]$, and define the windowed litho–coefficient
\[
a_\Gamma[g_T]\ :=\ \text{ the coefficient of order }t^{-(d-1)/2}\text{ in the localized heat trace generated by }g_T.
\]
(See Parts~II–III for the precise localization scheme.) We consider $T\to\infty$ asymptotics with~$\Gamma\sim\mathbf{P}$.

\subsection*{5.3. Normalized litho–invariants}
We shall work with two equivalent normalizations; the first is isotropic and invariant under homotheties:
\[
K_L^{\mathrm{iso}}(\Omega,\Gamma)\ :=\ \frac{a_\Gamma}{(4\pi)^{-(d-1)/2}\,\operatorname{vol}(\Omega)^{(d-1)/d}},
\]
and the second is relational, comparing the fracture contribution to the classical boundary term,
\[
K_L^{\partial}(\Omega,\Gamma)\ :=\ \frac{a_\Gamma}{a_{1/2}}\cdot \frac{\operatorname{vol}(\partial\Omega)}{\operatorname{vol}(\Gamma)}.
\]
Under Dirichlet conditions on $\partial\Omega\cup\Gamma$, both are dimensionless and equivalent in the sense that each is a bounded $C^2$ function of the other whenever $0<c\le\operatorname{vol}(\Gamma)/\operatorname{vol}(\partial\Omega)\le C<\infty$.
We adopt $K_L^{\mathrm{iso}}$ as the primary invariant, writing simply $K_L$ when no ambiguity arises.

\subsection*{5.4. Dimension–only universality (thermodynamic limit)}
Let $\{\mathbf{P}_L\}$ be an admissible thermodynamic ensemble as in (A). For each realization $\Gamma_L\sim\mathbf{P}_L$ define the Dirichlet fracture operator $L_{\Gamma_L}$ on $(\Omega_L,g_L)$ and the corresponding litho–invariant $K_L(\Omega_L,\Gamma_L)$.

\begin{theorem}[LLN universality in the thermodynamic limit]\label{thm:LLN-thermo}
Fix $d\ge2$ and Dirichlet conditions on $\partial\Omega_L\cup\Gamma_L$. There exists a constant $K^*_L(d)\in\mathbb{R}$, depending only on the dimension~$d$ (and the boundary condition class), such that for any admissible thermodynamic ensemble $\{\mathbf{P}_L\}$,
\[
K_L(\Omega_L,\Gamma_L)\ \xrightarrow[L\to\infty]{\ \mathbf{P}_L\text{-probability}\ }\ K^*_L(d).
\]
Moreover, if the block–mixing coefficients decay exponentially and the mean complexity density is finite, then the convergence holds almost surely and the nonasymptotic deviation satisfies
\[
\mathbf{P}_L\Big( \big|K_L(\Omega_L,\Gamma_L)-K^*_L(d)\big|>\varepsilon\Big)\ \le\
C\,\exp\!\left(-c\,\varepsilon^2\,L^d\right),
\]
with constants $c,C>0$ depending only on $d$, on the mixing profile, and on the uniform bound of the complexity density.
\end{theorem}

\noindent
The proof (given later) is based on block–additivity of the heat–trace coefficients, a Bernstein–type concentration for weakly dependent blocks, and the polynomial dependence on local complexity encoded by $\kappa(\Gamma)$. The constant $K^*_L(d)$ is the \emph{dimension–only} limit for the Dirichlet universality class.

\subsection*{5.5. Central limit fluctuations}
Under quantitative mixing, fluctuations are asymptotically Gaussian with scale dictated by the effective number of weakly dependent blocks.

\begin{theorem}[CLT universality]\label{thm:CLT-thermo}
Under the hypotheses of Theorem~\ref{thm:LLN-thermo}, assume in addition a spectral gap for the transfer operator of the block process (equivalently, exponential $\alpha$–mixing with summable cumulants of the block contributions). Then there exists $\sigma_d^2\in[0,\infty)$ such that
\[
L^{d/2}\,\Big( K_L(\Omega_L,\Gamma_L) - K^*_L(d)\Big)
\ \Longrightarrow\ \mathcal{N}(0,\sigma_d^2)
\quad\text{as }L\to\infty.
\]
If, moreover, the block–block covariance is absolutely summable, then $\sigma_d^2>0$.
\end{theorem}

\noindent
The variance $\sigma_d^2$ depends on the short–range correlation structure of the ensemble but \emph{not} on the ambient geometry of $(\Omega,g)$; the latter contributes only through local charts in the transfer construction. This is the precise sense in which the universality is geometric (dimension–only) rather than model–specific.

\subsection*{5.6. Frequency–window universality on fixed domains}
We now formulate an analogue on fixed $(\Omega,g)$ for the windowed coefficients $a_\Gamma[g_T]$.

\begin{theorem}[High–frequency LLN under stationarity]\label{thm:LLN-window}
Let $\mathbf{P}$ be an admissible law on $\mathfrak{H}_M(\Omega)$ with exponential mixing. For any family $g_T\in C_c^\infty(\mathbb{R})$ with $\operatorname{supp}\widehat g_T\subset[-T,T]$ and uniform symbol bounds, there exists $K^*_L(d)$ such that
\[
\frac{a_\Gamma[g_T]}{(4\pi)^{-(d-1)/2}\,\operatorname{vol}(\Omega)^{(d-1)/d}}
\ \xrightarrow[T\to\infty]{\ \mathbf{P}\text{-probability}\ }\ K^*_L(d).
\]
If, in addition, Hypothesis H$_{\mathrm{mix}}$ of Part~IV holds for the geodesic flow on $\Omega\setminus\Gamma$ with mixing constants uniform in $\Gamma$ on a full $\mathbf{P}$–measure set, then the convergence is almost sure with stretched–exponential tails in~$T$.
\end{theorem}

\noindent
The proof uses localized trace formulas, Tauberian arguments, and cancellation coming from mixing, parallel to the boundary case but with fracture–localized amplitudes and polynomial control by $\kappa(\Gamma)$.

\subsection*{5.7. Robustness, stability, and universality classes}
We summarize stability properties that ensure the universality constants are intrinsic to the dimension and boundary condition class.

\begin{proposition}[Stability]\label{prop:stability}
Within the Dirichlet class, $K^*_L(d)$ is invariant under smooth changes of the ambient metric $g$ (with fixed $d$) and under $C^2$–small perturbations of the law $\mathbf{P}$ measured in a topology controlling the distribution of second fundamental forms and local densities of components. Moreover, replacing Dirichlet conditions on $\Gamma$ by Robin or Neumann conditions produces distinct universal constants $K^*_L(d;\mathrm{BC})$ depending only on $d$ and the boundary condition class~$\mathrm{BC}$.
\end{proposition}

\noindent
Thus, universality comes in \emph{classes} indexed by boundary conditions; within a class, the limit is dimension–only and independent of the fine geometry of $(\Omega,g)$ and of the law, beyond mixing and complexity–density constraints.

\subsection*{5.8. Examples of admissible ensembles}
We list canonical models meeting the hypotheses above.

\begin{itemize}
\item \emph{Poisson hyperplane tessellations} transplanted via charts, with intensity scaled so that the mean $(d\!-\!1)$–density is constant; $C^2$–smoothing near chart overlaps yields a $C^2$ fracture set with uniformly bounded complexity density.
\item \emph{Gaussian level–set fractures:} $\Gamma=\{x\in\Omega: F(x)=0\}$ for a stationary $C^3$ Gaussian field with nondegenerate gradient; the Kac–Rice formula yields finite mean $(d\!-\!1)$–density and exponential mixing under spectral gap hypotheses.
\item \emph{Boolean models of smooth grains:} unions of randomly placed $C^2$ convex bodies with exponentially decaying radius tails; the boundary of the union is $C^2$ a.s.\ away from a negligible set, with finite mean complexity density.
\end{itemize}

In each case, standard ergodic theorems supply the block LLN, while quantitative mixing yields the CLT.

\subsection*{5.9. Relation to classical universality}
The statements above are formally analogous to universality phenomena in random matrix theory and in spectral geometry of chaotic flows, yet they are genuinely geometric: the constants $K^*_L(d;\mathrm{BC})$ are anchored in the fracture–localized heat kernel and the wave parametrix near $\Gamma$, rather than in matrix ensembles. The dependence on the ambient geometry is washed out by ergodicity and block averaging; only the ambient dimension and boundary class persist.

\subsection*{5.10. Scope and limitations}
All limits are formulated within explicitly admissible classes: $C^2$–regular fractures, finite mean complexity density, exponential mixing. No arithmetic claims are made; no inverse problems are addressed. In particular, universality is \emph{not} asserted for adversarially arranged fractures of unbounded complexity or for flows lacking any mixing.

\subsection*{5.11. Closure}
This part establishes a precise probabilistic framework in which the litho–invariant exhibits dimension–only limits with Gaussian fluctuations. The hypotheses are canonical, verifiable in standard models, and stable under perturbations. Theorems~\ref{thm:LLN-thermo}, \ref{thm:CLT-thermo}, and \ref{thm:LLN-window} will be proved in subsequent chapters by combining localized trace formulas with block decomposition, polynomial complexity control, and quantitative mixing estimates. This completes the universality layer and prepares the ground for explicit computations and comparisons developed later.

\section*{Part VI. Relation to Classical Spectral Geometry}

\subsection*{6.1. Orientation}
The preceding parts introduced litho–domains, their complexity functional, and the litho–invariant $K_L$, and established universality principles under deterministic and random frameworks. To position these notions within the established landscape of spectral geometry, we now provide a precise comparative analysis with existing theories: (i) the asymptotics of the heat trace for manifolds with boundary, (ii) the theory of stratified and conic spaces due to Cheeger and Brüning, (iii) spectral problems with singular potentials, and (iv) variational and microlocal approaches. The goal is to delineate the exact scope of novelty of the litho–framework, in terms of both results and methodology, while maintaining rigorous separation of what is new from what is already known.

\subsection*{6.2. The classical Weyl law and boundary contributions}
Let $(\Omega,g)$ be a compact $d$–dimensional Riemannian manifold with smooth boundary and Dirichlet conditions. The classical heat trace asymptotics are
\[
\mathrm{Tr}\,e^{-t\Delta_\Omega}
\;\sim\;(4\pi t)^{-d/2}\,\mathrm{vol}(\Omega)
-\tfrac14\,(4\pi t)^{-(d-1)/2}\,\mathrm{vol}(\partial\Omega)
+O(t^{-(d-2)/2}),
\quad t\downarrow0,
\]
as established by Weyl, Ivrii, and further clarified in the works of Safarov–Vassiliev. This expansion shows that the presence of a geometric boundary introduces a contribution of order $t^{-(d-1)/2}$ proportional to the $(d-1)$–volume of $\partial\Omega$ with a universal coefficient $-1/4$. Importantly, no additional terms arise from interior smooth structures.

\subsection*{6.3. Stratified and conic spaces}
The theory of singular spaces, notably Cheeger’s analysis of conic manifolds and Brüning’s work on stratified spaces, addresses spectral asymptotics where the ambient space itself is singular. For a conic point with opening angle $\alpha$, the heat trace receives a logarithmic correction or an additional term depending explicitly on $\alpha$ and on local link geometry. In stratified settings, contributions from strata of codimension $k$ appear at orders $t^{-(d-k)/2}$, with coefficients depending intricately on the local structure of the stratum. These contributions are not universal in the sense of a fixed coefficient: they carry explicit dependence on cone angles or link invariants.

\subsection*{6.4. Singular potentials and delta interactions}
A different direction studies Schrödinger operators with singular potentials supported on submanifolds, such as delta–potentials on curves or surfaces. In such models, additional terms in the heat trace expansion may arise, but their coefficients depend on the coupling constant of the potential and on the details of the self–adjoint extension. They are not purely geometric, and universality across ensembles or geometries is absent. The litho–framework, in contrast, prescribes a Dirichlet condition across an internal hypersurface, yielding contributions that are canonical and coupling–independent.

\subsection*{6.5. Variational and microlocal approaches}
Variational formulations of eigenvalue problems and microlocal methods in spectral geometry provide estimates on eigenvalue counting functions and spectral clusters. However, they do not isolate fracture–type contributions. In particular, while boundary layer methods and propagation of singularities address interactions with $\partial\Omega$, they do not extend naturally to codimension–one embedded hypersurfaces within the domain.

\subsection*{6.6. Position of the litho–framework}
The litho–approach is distinct from each of the above in the following precise senses:
\begin{enumerate}[(i)]
\item \textbf{Manifolds with boundary:} Litho–domains produce an additional $t^{-(d-1)/2}$ term analogous in order to the boundary contribution, but originating from an \emph{internal} hypersurface $\Gamma\subset\Omega$. This phenomenon has no analogue in the boundary–only setting.
\item \textbf{Stratified spaces:} In contrast to strata of codimension $1$ in Cheeger’s framework, where coefficients depend on local cone angles, the litho–coefficient is universal and linear in $\mathrm{vol}(\Gamma)$ under Dirichlet conditions.
\item \textbf{Singular potentials:} The litho–term arises without any tunable coupling constant, being purely geometric and dictated by the imposed Dirichlet condition. It is thus intrinsic, not parameter–dependent.
\item \textbf{Variational/microlocal theories:} The fracture contribution cannot be deduced from existing variational bounds or microlocal parametrix constructions for smooth domains, as those do not account for internal codimension–one Dirichlet surfaces.
\end{enumerate}

\subsection*{6.7. Technical mechanism: localized parametrix near fractures}
The core analytical mechanism behind the litho–term is the construction of a localized parametrix for the heat kernel near $\Gamma$, analogous to the boundary layer parametrix but applied across a closed interior hypersurface. Matching across $\Gamma$ doubles the contribution relative to a boundary of the same local geometry, leading to the coefficient $-1/4$ independently of the ambient geometry. This mechanism is absent in existing theories focused either on external boundaries or on singular ambient structures.

\subsection*{6.8. Robustness across geometries}
The universality of the coefficient $a_\Gamma$ under Dirichlet conditions holds across arbitrary ambient $(\Omega,g)$ with $C^2$ regularity. This robustness is in stark contrast to stratified or conic settings, where coefficients vary with local angles or link structures. Thus, the litho–invariant $K_L$ captures a genuinely new universality class within spectral geometry: interior Dirichlet fractures.

\subsection*{6.9. Scope limitations}
It is important to delineate scope:
\begin{itemize}
\item The universality holds strictly under Dirichlet conditions; for Neumann or Robin conditions the constants differ.
\item The hypersurface $\Gamma$ must be $C^2$ and transversal to $\partial\Omega$; intersections or corners produce mixed phenomena not covered by the main theorems.
\item The framework addresses short–time asymptotics; long–time behavior and resonance analysis are outside the present scope.
\end{itemize}

\subsection*{6.10. Closure}
In summary, the litho–framework is neither a trivial restatement of boundary contributions nor a subset of stratified space analysis. It identifies and rigorously isolates a new, universal contribution arising from internal Dirichlet hypersurfaces, distinct in mechanism and consequence from all existing approaches. This comparative positioning clarifies both the novelty and the limitations of the theory, and it sets the stage for explicit computations in algebraic, group–theoretic, and probabilistic settings in the following parts.

\section*{Part VII. Functorial Constructions on Finite Graphs (Canonical Band Surfaces)}

\subsection*{7.1. Orientation}
This part develops a strictly canonical passage from finite combinatorial data to litho–domains in dimension two. We restrict to finite graphs and, as a corollary, to Cayley graphs of finite groups with a fixed symmetric generating set. The construction produces a compact $C^2$ surface with boundary $\Omega_G$ and a $C^2$ internal fracture set $\Gamma_G$ determined solely by the graph $G$ and a fixed degree–template. This eliminates any dependence on embeddings or arbitrary smoothing choices. The resulting litho–invariant $K_L^\partial(\Omega_G,\Gamma_G)$ admits a closed form in terms of edge/vertex counts and a universal vertex–template constant.

\subsection*{7.2. The category of finite bounded–degree graphs}
Let $\mathsf{Graph}_{\le D}$ be the category whose objects are finite connected graphs $G=(V,E)$ with degrees $\deg(v)\le D$ for all $v\in V$, and whose morphisms are graph coverings $\pi:G\to H$ (locally bijective maps of graphs). We write $|V|$ and $|E|$ for the number of vertices and edges.

\subsection*{7.3. Canonical band–surface functor}
Fix once and for all:
\begin{itemize}
\item a strip width $h>0$ and edge length $\ell>0$,
\item for each $d\in\{1,\dots,D\}$, a $C^2$ vertex–patch template $\mathcal{P}_d$ which is a compact $C^2$ planar domain with $d$ marked boundary arcs of length $h$, pairwise disjoint and arranged with $C^2$ compatibility along their endpoints. Denote by $b_d:=\mathrm{length}(\partial\mathcal{P}_d\setminus(\text{marked arcs}))$ the residual boundary length of the template; $b_d$ depends only on $d$ and on the fixed choice of $\mathcal{P}_d$.
\end{itemize}

\begin{definition}[Canonical band surface]
Given $G\in\mathsf{Graph}_{\le D}$, construct $\Omega_G$ as follows:
\begin{enumerate}[(i)]
\item For each edge $e\in E$, take a Euclidean rectangle (a \emph{band})
\[
B_e := [0,\ell]\times[-h/2,h/2]
\]
with the product metric.
\item For each vertex $v$ of degree $d=\deg(v)$, take a copy of the vertex template $\mathcal{P}_d$.
\item Glue the $d$ incident band ends $\{0\}\times[-h/2,h/2]$ or $\{\ell\}\times[-h/2,h/2]$ isometrically onto the $d$ marked arcs of $\mathcal{P}_d$, respecting the combinatorics of $G$ and the local $C^2$ structure of the template near endpoints. Perform all gluings disjointly for distinct vertices.
\end{enumerate}
The result is a compact $C^2$ surface with boundary $\Omega_G$ whose boundary consists of the two long sides of each band together with the residual boundaries of the vertex–patches. Define the \emph{fracture set}
\[
\Gamma_G := \bigcup_{e\in E}\{(x,0)\in B_e \,:\, x\in[0,\ell]\}\subset \Omega_G,
\]
i.e., the midlines of all bands.
\end{definition}

By construction, $\Gamma_G$ is a disjoint union of $|E|$ closed $C^2$ curves (each of length $\ell$), transverse to $\partial\Omega_G$. We impose Dirichlet boundary conditions on $\partial\Omega_G\cup\Gamma_G$ and consider the Laplacian $L_G$ on $\Omega_G\setminus\Gamma_G$ as in Part~I.

\subsection*{7.4. Canonical geometry: lengths and areas}
The geometry of $(\Omega_G,\Gamma_G)$ is completely determined by $|V|,|E|$ and the fixed templates:
\begin{lemma}\label{lem:geo-counts}
For the canonical band surface of $G$:
\[
\mathrm{length}(\Gamma_G)=|E|\cdot \ell,\qquad
\mathrm{length}(\partial\Omega_G)=2|E|\cdot \ell + \sum_{v\in V} b_{\deg(v)},
\]
and
\[
\mathrm{area}(\Omega_G)=|E|\cdot (\ell h) + \sum_{v\in V} \mathrm{area}(\mathcal{P}_{\deg(v)}).
\]
\end{lemma}

\begin{proof}[Proof (sketch)]
Each band contributes one midline of length $\ell$, and two boundary components of length $\ell$; vertex–patches contribute no fracture length and contribute $b_{\deg(v)}$ to the boundary length. Areas add by disjoint union and gluing along measure–zero arcs. $C^2$–compatibility ensures the resulting surface is $C^2$ across the gluing loci.
\end{proof}

\subsection*{7.5. Applicability of the localized and heat trace expansions}
The pair $(\Omega_G,\Gamma_G)$ satisfies the hypotheses of the localized trace formula and of the small–time heat trace expansion from Parts~II–V: $\Omega_G$ is compact with $C^2$ boundary, $\Gamma_G$ is a closed $C^2$ embedded 1–submanifold, and $\Gamma_G$ is transverse to $\partial\Omega_G$. Therefore (in dimension $d=2$)
\begin{equation}\label{eq:heat-2d}
\mathrm{Tr}\,e^{-tL_G}
\sim (4\pi t)^{-1}\mathrm{area}(\Omega_G)
-\tfrac14(4\pi t)^{-1/2}\,\mathrm{length}(\partial\Omega_G)
-\tfrac14(4\pi t)^{-1/2}\,\mathrm{length}(\Gamma_G)
+\cdots,\quad t\downarrow 0,
\end{equation}
with uniform polynomial control of constants in terms of the $C^2$ norms of the templates, cf.\ Theorem~\ref{thm:fracture-term}.

\subsection*{7.6. A boundary–normalized litho–invariant for graphs}
In $d=2$, both the boundary and the fracture contributions scale like $t^{-1/2}$. It is therefore natural to adopt the boundary–normalized litho–invariant
\[
K_L^\partial(\Omega_G,\Gamma_G)
:= \frac{a_\Gamma}{a_{1/2}}
= \frac{\mathrm{length}(\Gamma_G)}{\mathrm{length}(\partial\Omega_G)}
= \frac{|E|\cdot \ell}{2|E|\cdot \ell + \sum_{v\in V} b_{\deg(v)}}.
\]
By Lemma~\ref{lem:geo-counts}, $K_L^\partial(\Omega_G,\Gamma_G)$ depends only on $|E|,|V|$ and the multiset $\{\deg(v)\}_{v\in V}$ via the fixed vertex–template constants $b_d$. In particular, if $G$ is $d$–regular then
\begin{equation}\label{eq:KL-regular}
K_L^\partial(G)= \frac{|E|}{2|E| + b_d\, |V|}.
\end{equation}

\subsection*{7.7. Functoriality for graph coverings}
\begin{proposition}[Invariance under coverings]\label{prop:cover}
Let $\pi:G\to H$ be a graph covering of finite connected graphs with degrees bounded by $D$. Then, for the canonical band–surface construction,
\[
K_L^\partial(\Omega_G,\Gamma_G) \;=\; K_L^\partial(\Omega_H,\Gamma_H).
\]
\end{proposition}

\begin{proof}
A $k$–sheeted covering preserves degrees and scales counts by $k$: $|E_G|=k|E_H|$, $|V_G|=k|V_H|$, and the degree multiset is replicated. By \eqref{eq:KL-regular} (and its extension to non–regular graphs using $\sum_{v} b_{\deg(v)}$), both numerator and denominator scale by $k$, hence the ratio is invariant.
\end{proof}

Thus the assignment $G\mapsto (\Omega_G,\Gamma_G)$ together with $K_L^\partial$ defines a functor from $\mathsf{Graph}_{\le D}$ (with coverings) to the category $\mathsf{Litho}_2$ of litho–domains in $d=2$ (with boundary–fracture preserving coverings), constant on fibers of coverings at the level of $K_L^\partial$.

\subsection*{7.8. Cayley graphs of finite groups with symmetric generators}
Let $\mathsf{GrpSym}_{\le D}$ denote the category whose objects are pairs $(G,S)$ with $G$ finite and $S\subset G$ a fixed symmetric generating set with $|S|\le D$; morphisms are homomorphisms $\varphi:(G,S)\to (H,T)$ such that $\varphi(S)\subset T$ and the induced map of Cayley graphs is a covering. Writing $\mathrm{Cay}(G,S)$ for the Cayley graph, define
\[
(\Omega_{G,S},\Gamma_{G,S}) := (\Omega_{\mathrm{Cay}(G,S)},\Gamma_{\mathrm{Cay}(G,S)}).
\]
If $|S|=d$ then $\mathrm{Cay}(G,S)$ is $d$–regular with $|V|=|G|$ and $|E|=\tfrac12|G|\,|S|$, hence by \eqref{eq:KL-regular}
\begin{equation}\label{eq:KL-Cayley}
K_L^\partial(G,S)
= \frac{|E|}{2|E| + b_d\,|V|}
= \frac{\tfrac12|G|\,|S|}{|G|\,|S| + b_d\,|G|}
= \frac{|S|}{2|S|+2b_d}.
\end{equation}
Therefore, for a fixed degree template (hence fixed $b_d$), $K_L^\partial(G,S)$ depends \emph{only} on $|S|$ and is independent of $|G|$.

\begin{corollary}[Covering invariance on Cayley graphs]
If $\varphi:(G,S)\to (H,T)$ induces a covering of Cayley graphs, then
\[
K_L^\partial(G,S) \;=\; K_L^\partial(H,T),
\]
and the common value is given by \eqref{eq:KL-Cayley}.
\end{corollary}

\subsection*{7.9. Explicit examples}
\paragraph{Cycles $C_n$.}
The cycle $C_n$ is $2$–regular with $|V|=|E|=n$. Hence
\[
K_L^\partial(C_n) \;=\; \frac{n}{2n + b_2\, n} \;=\; \frac{1}{2+b_2},
\]
independent of $n$.

\paragraph{Dihedral Cayley graphs.}
For the dihedral group $D_{2m}$ with symmetric generating set of size $d=2$ (two reflections), we obtain the $2$–regular Cayley graph on $2m$ vertices (a cycle with doubled labeling), and again
\[
K_L^\partial(D_{2m},S) \;=\; \frac{1}{2+b_2}.
\]

\paragraph{The symmetric group $S_3$.}
Take the symmetric generating set $S=\{s,t,t^{-1}\}$ with $|S|=3$ (where $s$ is a transposition, $t$ a $3$–cycle). Then the Cayley graph is $3$–regular on $6$ vertices, and
\[
K_L^\partial(S_3,S) \;=\; \frac{3}{6+2b_3}.
\]
If instead one chooses a degree–$2$ symmetric set (e.g.\ two distinct transpositions closed under inversion), the Cayley graph is $2$–regular and $K_L^\partial$ equals $1/(2+b_2)$, illustrating the declared dependence on $|S|$.

\subsection*{7.10. Scope and limitations}
All statements herein are confined to:
\begin{itemize}
\item the canonical band–surface construction (fixed $(h,\ell)$ and fixed $C^2$ templates $\{\mathcal{P}_d\}$);
\item morphisms given by graph coverings (and group homomorphisms inducing coverings on Cayley graphs).
\end{itemize}
We do \emph{not} claim functoriality for arbitrary graph homomorphisms or arbitrary group homomorphisms, and we make no assertions for commutative rings or general algebraic varieties. Extending $K_L$–type assignments beyond coverings requires additional structure and lies outside the present scope.

\subsection*{7.11. Closure}
We have provided a canonical, embedding–free and template–controlled functor from finite bounded–degree graphs (with coverings) to litho–domains in $d=2$, proved covering–invariance of the boundary–normalized litho–invariant, and derived closed–form expressions in terms of elementary graph counts. As a corollary, Cayley graphs of finite groups with symmetric generating sets yield group–dependent values $K_L^\partial(G,S)$ determined solely by $|S|$. This validates the algebraic reach of lithomathematics in a mathematically rigorous and non–speculative setting, fully aligned with the analytic foundations established in earlier parts.

\section*{Part VIII. Stratified Spaces and Internal Hypersurfaces}

\subsection*{8.1. Orientation}
The purpose of this part is to place lithomathematics in the broader framework of spectral geometry on stratified spaces. We show that, for Whitney–stratified domains, the small–time heat trace expansion admits contributions ordered by the codimension of strata. In particular, internal hypersurfaces $\Gamma$ contribute at the same order as external boundaries, while strata of higher codimension contribute at strictly higher orders. This demonstrates the structural compatibility of litho–terms with the existing theory of stratified singularities.

\subsection*{8.2. Stratified domains}
\begin{definition}[Whitney stratified domain]
Let $\Omega$ be a compact subset of a smooth $d$–dimensional Riemannian manifold. A \emph{Whitney stratification} of $\Omega$ is a finite partition
\[
\Omega = \bigsqcup_{j=0}^d \Sigma_j,
\]
where each stratum $\Sigma_j$ is a smooth $j$–dimensional manifold, and the stratification satisfies Whitney’s conditions (A) and (B). The top stratum $\Sigma_d$ is assumed open and dense, carrying the induced Riemannian metric.
\end{definition}

We impose Dirichlet boundary conditions on $\partial\Omega$ and on all internal strata of codimension $1$ (the “fractures” of lithomathematics).

\subsection*{8.3. Heat kernel parametrix near strata}
The construction of a small–time parametrix follows the scheme of Cheeger \cite{Cheeger80} and Brüning \cite{BruningSeeley88}. Locally near a point $p\in \Sigma_j$, $\Omega$ is modeled on a product
\[
\mathbb{R}^j \times C(Z),
\]
where $C(Z)$ is a metric cone over a compact link $Z$ of dimension $d-j-1$. The heat kernel admits an expansion in terms of contributions from the Euclidean factor, the cone, and the imposed boundary conditions. The codimension of the stratum determines the order of the leading term.

\subsection*{8.4. General structure of the expansion}
\begin{theorem}[Stratified heat trace expansion]\label{thm:stratified}
Let $(\Omega,\Sigma)$ be a compact Whitney–stratified domain in dimension $d\ge 2$, with Dirichlet conditions on $\partial\Omega$ and on all internal hypersurfaces. Then, as $t\downarrow 0$,
\[
\mathrm{Tr}\, e^{-tL_\Sigma}
\sim (4\pi t)^{-d/2} \,|\Sigma_d|
-\tfrac14(4\pi t)^{-(d-1)/2}\big(|\partial\Omega|+|\Gamma|\big)
+\sum_{j\le d-2} c_j(\Sigma_j)\, t^{-(j)/2}
+O\!\left(t^{-(d-3)/2}\,P(\kappa(\Sigma))\right),
\]
where:
\begin{itemize}
\item $|\Sigma_d|$ is the Riemannian volume of the top stratum,
\item $\Gamma=\Sigma_{d-1}^{\mathrm{int}}$ is the union of internal hypersurfaces,
\item $c_j(\Sigma_j)$ are local coefficients determined by the geometry of the links at $\Sigma_j$,
\item $P(\kappa(\Sigma))$ is a polynomial in the geometric complexity of the stratification.
\end{itemize}
\end{theorem}

\begin{remark}
In particular, only strata of codimension $1$ contribute at order $t^{-(d-1)/2}$. Strata of codimension $\ge 2$ contribute strictly lower–order terms, hence they cannot obscure or alter the universal fracture contribution $a_\Gamma$.
\end{remark}

\subsection*{8.5. Example: planar wedge}
Consider $\Omega=W_\alpha\subset\mathbb{R}^2$, the wedge of angle $\alpha\in(0,2\pi)$ with Dirichlet boundary conditions on its sides. Its heat trace admits the well–known expansion
\[
\mathrm{Tr}\,e^{-tL_{W_\alpha}}
= \frac{|W_\alpha|}{4\pi t}
-\frac{|\partial W_\alpha|}{8\sqrt{\pi t}}
+ C_{\mathrm{apex}}(\alpha) + O(e^{-c/t}),
\]
where $C_{\mathrm{apex}}(\alpha)$ is an angle–dependent constant. This fits the structure of Theorem~\ref{thm:stratified}: the apex is a $0$–dimensional stratum (codimension $2$ in $d=2$), hence its contribution enters at order $t^0$, lower than the boundary and fracture terms.

\subsection*{8.6. Implications for lithomathematics}
\begin{corollary}[Compatibility of litho–terms]
For any Whitney–stratified $\Omega$, the internal fracture term
\[
a_\Gamma = -\tfrac14(4\pi)^{-(d-1)/2}\,|\Gamma|
\]
is unaffected by lower–dimensional strata. Therefore the litho–invariant $K_L^\partial$ defined in Part~VII extends without ambiguity to the stratified setting.
\end{corollary}

\subsection*{8.7. Closure}
We have embedded the lithomathematical setting into the well–developed theory of spectral geometry on stratified spaces. Theorems~\ref{thm:stratified} and its corollary guarantee that internal hypersurfaces contribute a universal term at the same order as external boundaries, while all lower–dimensional singularities are relegated to higher orders. The wedge domain illustrates this separation of scales explicitly. This confirms the robustness of litho–invariants within stratified geometries and clarifies their distinction from classical singularity contributions.

\section*{Part IX. Random Hypersurfaces and Ergodic Ensembles}

\subsection*{9.1. Orientation}
This part develops lithomathematics in stochastic settings. We formalize probability measures on classes of embedded $C^2$ fractures with uniform geometric control, establish that the universal fracture coefficient is almost surely valid under these uniform bounds, identify the expected (dimensionless) litho–invariant, and give variance/ergodic statements in representative ensembles (Gaussian level sets on homogeneous manifolds, Poisson unions of geodesic spheres). All assumptions are explicit; no claim is made beyond their scope.

\subsection*{9.2. The measurable space of fractures}
Let $(M,g)$ be a compact $d$–dimensional Riemannian manifold. Denote by $\mathcal{E}$ the set of all $C^2$ embeddings $\iota:\Sigma\hookrightarrow M$ with $\Sigma$ a closed $(d-1)$–manifold. For $C=(C_0,C_1,C_2)\in(0,\infty)^3$ define the \emph{uniformly controlled class}
\[
\mathcal{H}(C)\,=\,\Big\{\Gamma=\iota(\Sigma):\;
\mathrm{vol}_{d-1}(\Gamma)\le C_0,\;\;\|\mathrm{II}_\Gamma\|_{L^\infty(\Gamma)}\le C_1,\;\;\mathrm{reach}(\Gamma)\ge C_2\Big\}.
\]
Here $\mathrm{II}_\Gamma$ is the second fundamental form (defined a.e.), and $\mathrm{reach}(\Gamma)$ is the maximal radius of a tubular neighborhood on which the normal exponential map is a $C^1$–diffeomorphism. Equip $\mathcal{E}$ with the $C^2$–Whitney topology on embeddings, and let $\pi:\mathcal{E}\to\mathcal{C}$ be the map $\iota\mapsto\iota(\Sigma)$ to the space $\mathcal{C}$ of closed subsets of $M$ with the local $C^2$–graph topology on embedded hypersurfaces (equivalently, the Hausdorff topology refined by $C^2$–charts in tubular neighborhoods). We view $\mathcal{H}(C)\subset\mathcal{C}$ as a Borel subset for the induced topology; write $\mathcal{F}$ for its Borel $\sigma$–algebra.

\begin{definition}[Fracture ensembles]\label{def:ensemble}
A \emph{fracture ensemble} on $(M,g)$ with uniform $C^2$–control $C$ is a probability space $(\mathcal{H}(C),\mathcal{F},\mathbb{P})$ such that:
\begin{itemize}
\item $\mathbb{P}$ is Borel on $(\mathcal{H}(C),\mathcal{F})$;
\item $\mathbb{P}$ is invariant under the natural action of the isometry group $\mathrm{Iso}(M,g)$ on $\mathcal{H}(C)$;
\item $\mathbb{E}_{\mathbb{P}}[\mathrm{vol}_{d-1}(\Gamma)]<\infty$.
\end{itemize}
\end{definition}

\noindent
Typical constructions include: (i) level sets $\Gamma=\{f=0\}$ of centered Gaussian fields $f\in C^2(M)$ with smooth, $\mathrm{Iso}(M,g)$–invariant covariance and a.s. nondegenerate gradient along the zero level; (ii) unions of finitely many disjoint geodesic spheres of fixed radius with Poisson-distributed centers (with thinning to enforce disjointness); (iii) nodal sets of Gaussian monochromatic random waves on homogeneous manifolds.

\subsection*{9.3. Uniform parametrix and universal fracture coefficient a.s.}
For $\Gamma\in \mathcal{H}(C)$ let $L_\Gamma$ denote the Dirichlet Laplacian on $M\setminus \Gamma$ (no boundary on $M$ is assumed in this section). The heat trace admits, as $t\downarrow 0$,
\[
\mathrm{Tr}\,e^{-tL_\Gamma}\;\sim\;a_0\,t^{-d/2}\;+\;a_\Gamma\,t^{-(d-1)/2}\;+\;\sum_{j\ge 1} a_j\,t^{-(d-2-j)/2},
\]
with $a_0=(4\pi)^{-d/2}\mathrm{vol}_d(M)$. The next statement records almost sure universality of the fracture term under the uniform $C^2$–control of Definition~\ref{def:ensemble}.

\begin{theorem}[Almost sure universality under uniform control]\label{thm:as-univ}
Fix $C=(C_0,C_1,C_2)$ and an ensemble $(\mathcal{H}(C),\mathcal{F},\mathbb{P})$ as in Definition~\ref{def:ensemble}. Then for $\mathbb{P}$–almost every $\Gamma$,
\[
a_\Gamma \;=\; -\,\tfrac14\,(4\pi)^{-(d-1)/2}\,\mathrm{vol}_{d-1}(\Gamma).
\]
Moreover, the remainder $R_\Gamma(t):=\mathrm{Tr}\,e^{-tL_\Gamma}-\big(a_0\,t^{-d/2}+a_\Gamma\,t^{-(d-1)/2}\big)$ satisfies a uniform bound
\[
\sup_{\Gamma\in \mathcal{H}(C)}\;\big|R_\Gamma(t)\big|\;\le\;C'\,t^{-(d-2)/2}\qquad (0<t\le t_0),
\]
for constants $C',t_0>0$ depending only on $(M,g)$ and $C$.
\end{theorem}

\begin{proof}[Sketch]
Uniform bounds on $\mathrm{vol}_{d-1}(\Gamma)$, $\|\mathrm{II}_\Gamma\|_{L^\infty}$ and $\mathrm{reach}(\Gamma)$ give a uniform tubular neighborhood and $C^2$–graph control for all $\Gamma\in\mathcal{H}(C)$. The local reflected–parametrix construction in FIO classes near a smooth interface, with Dirichlet condition on both sides, yields the same principal boundary symbol as for a smooth outer boundary, hence the universal factor $-\tfrac14(4\pi)^{-(d-1)/2}$ multiplying the $(d-1)$–volume density; see the deterministic Theorem~\ref{thm:fracture-term} in Part~II. Uniformity of all local seminorms implies a uniform estimate on the error kernel and thus on the remainder after integration. The almost sure statement is immediate because the asserted identity holds pointwise for every $\Gamma\in\mathcal{H}(C)$. 
\end{proof}

\begin{remark}[Scope]
Theorem~\ref{thm:as-univ} does not address nontransversal intersections with $\partial M$ or lower–regularity interfaces; such cases belong to Parts~VIII and X. No claim is made here beyond the uniform $C^2$–controlled setting.
\end{remark}

\subsection*{9.4. Expected dimensionless litho–invariant}
Recall the isoperimetric normalization
\[
K_L^{\mathrm{iso}}(M,\Gamma)\;=\;(4\pi)^{1/2}\,\frac{\mathrm{vol}_d(M)^{1/2}}{\mathrm{vol}_{d-1}(\Gamma)}\,\frac{a_\Gamma}{a_0}.
\]
Under Theorem~\ref{thm:as-univ} one has deterministically
\[
K_L^{\mathrm{iso}}(M,\Gamma)\;=\; -\,\tfrac14\,\frac{\mathrm{vol}_{d-1}(\Gamma)}{\mathrm{vol}_d(M)}.
\]
Therefore, for any ensemble with $\mathbb{E}[\mathrm{vol}_{d-1}(\Gamma)]<\infty$,
\[
\mathbb{E}\big[K_L^{\mathrm{iso}}(M,\Gamma)\big]\;=\;-\;\tfrac14\,\frac{\mathbb{E}\big[\mathrm{vol}_{d-1}(\Gamma)\big]}{\mathrm{vol}_d(M)}.
\]
No further distributional assumptions are required for this identity.

\subsection*{9.5. Gaussian level–set ensembles on homogeneous manifolds}
Assume $(M,g)$ is homogeneous (e.g., flat tori $\mathbb{T}^d$ or round spheres $\mathbb{S}^d$). Let $\{f_\Lambda\}_{\Lambda\to\infty}$ be a family of centered, smooth Gaussian fields on $M$ with $\mathrm{Iso}(M,g)$–invariant covariance, parametrized by a spectral scale $\Lambda$ (e.g., band–limited fields or monochromatic random waves with Laplace eigenvalue $\Lambda$). Under the standard nondegeneracy and mixing assumptions (see \cite{AdlerTaylor07,Wigman15}):
\[
\mathbb{E}\big[\mathrm{vol}_{d-1}(\{f_\Lambda=0\})\big]\;=\;c_{d,\mathrm{model}}\;\Lambda^{1/2}\,\mathrm{vol}_d(M)\,(1+o(1)),\qquad \Lambda\to\infty,
\]
for an explicit model–dependent constant $c_{d,\mathrm{model}}>0$. Consequently,
\[
\mathbb{E}\big[K_L^{\mathrm{iso}}(M,\{f_\Lambda=0\})\big]\;=\;-\;\tfrac14\,c_{d,\mathrm{model}}\;\Lambda^{1/2}\,(1+o(1)).
\]
Moreover, for many models the variance satisfies $\mathrm{Var}(\mathrm{vol}_{d-1}(\{f_\Lambda=0\}))=o(\Lambda)$, yielding concentration of
\[
\frac{K_L^{\mathrm{iso}}(M,\{f_\Lambda=0\})}{\mathbb{E}[K_L^{\mathrm{iso}}(M,\{f_\Lambda=0\})]}\;\xrightarrow{\ \mathbb{P}\ }\;1.
\]
Precise variance asymptotics are model–dependent and will not be needed in the sequel; see \cite{NazarovSodin16,Wigman15} for detailed statements.

\subsection*{9.6. Poisson unions of geodesic spheres}
Fix $r\in(0,\mathrm{inj}(M,g))$ and let $\mathcal{P}_\lambda$ be a stationary Poisson point process on $M$ with intensity $\lambda>0$ with respect to the Riemannian volume. Consider the union of disjoint geodesic spheres
\[
\Gamma\;=\;\bigsqcup_{x\in \mathcal{P}_\lambda} \partial B_g(x,r),
\]
obtained by independently thinning points to avoid overlaps (or, equivalently, by conditioning on mutual separation $>2r$). Then
\[
\mathbb{E}\big[\mathrm{vol}_{d-1}(\Gamma)\big]\;=\;\lambda\,\mathrm{vol}_d(M)\,\mathrm{vol}_{d-1}\big(\partial B_{\mathbb{R}^d}(0,r)\big),
\]
and hence
\[
\mathbb{E}\big[K_L^{\mathrm{iso}}(M,\Gamma)\big]\;=\;-\;\tfrac14\,\lambda\,\mathrm{vol}_{d-1}\big(\partial B_{\mathbb{R}^d}(0,r)\big).
\]
The uniform $C^2$–control holds with $C_1\asymp r^{-1}$ and $C_2\asymp r$.

\subsection*{9.7. Ergodic sequences and time averages}
Let $(\mathcal{H}(C),\mathcal{F},\mathbb{P})$ be as above, and let $T:\mathcal{H}(C)\to \mathcal{H}(C)$ be a measurable, $\mathbb{P}$–measure–preserving transformation that is ergodic. Assume $K_L^{\mathrm{iso}}\in L^1(\mathbb{P})$. Then Birkhoff’s ergodic theorem yields:

\begin{theorem}[Time averages]\label{thm:ergodic}
For $\mathbb{P}$–almost every $\Gamma_0\in\mathcal{H}(C)$,
\[
\lim_{N\to\infty}\frac{1}{N}\sum_{n=0}^{N-1} K_L^{\mathrm{iso}}(M,T^n\Gamma_0)\;=\;\mathbb{E}\big[K_L^{\mathrm{iso}}(M,\Gamma)\big].
\]
\end{theorem}

\noindent
Thus the statistical litho–profile is recovered by time averages along any $\mathbb{P}$–typical orbit.

\subsection*{9.8. Remarks on scope and limitations}
(1) The almost sure universality of $a_\Gamma$ in Theorem~\ref{thm:as-univ} is confined to ensembles supported in $\mathcal{H}(C)$; interfaces with corners, self–intersections, or grazing contact with $\partial M$ require the refined analysis of Parts~VIII and X.\\
(2) Gaussian examples in \S9.5 presuppose the standard nondegeneracy of the joint law $(f_\Lambda,\nabla f_\Lambda)$ and an $\mathrm{Iso}(M,g)$–invariant covariance; constants $c_{d,\mathrm{model}}$ are model–dependent.\\
(3) In \S9.6 we imposed thinning to avoid overlaps; without thinning, one may count $\mathrm{vol}_{d-1}$ with multiplicity, in which case the same expectation formula holds by linearity, though $\mathcal{H}(C)$–control must be adapted.

\subsection*{9.9. Closure}
We have placed random and ergodic fractures on a clean measurable foundation, proved that the fracture contribution $a_\Gamma$ retains its universal form almost surely under uniform $C^2$–control, and related the mean (dimensionless) litho–invariant to the mean $(d-1)$–volume of the interface. The examples on homogeneous manifolds (Gaussian level sets) and Poisson unions confirm that the litho–profile is both universal and statistically stable in regimes of physical and mathematical interest. This completes the probabilistic component of the introduction and prepares the ground for the deterministic functional–analytic developments in the subsequent parts.

\section*{Part X. Roadmap, Novelty, Scope, and Outline}

\subsection*{10.1. Orientation}
The purpose of this final introductory part is to situate the present work within the broader mathematical landscape, to delineate its novelty relative to prior frameworks, to clarify its scope and limitations, and to provide the reader with a roadmap through the subsequent chapters and appendices. The emphasis is on precision and restraint: claims are confined to what will be proved, conjectures are explicitly identified as such, and the scope of applicability is carefully delimited.

\subsection*{10.2. Roadmap of results}
The work establishes four principal pillars, announced in the introduction and proved in the body:

\begin{enumerate}[(i)]
\item \emph{Localized trace formula (Theorem~A):}  
For any compact Riemannian manifold $(\Omega,g)$ with a closed $C^2$ internal hypersurface $\Gamma\subset \Omega$, the heat trace expansion acquires a universal additional term $a_\Gamma\,t^{-(d-1)/2}$, with $a_\Gamma=-\tfrac14(4\pi)^{-(d-1)/2}\mathrm{vol}_{d-1}(\Gamma)$. This establishes the fracture term as a stable, intrinsic contribution.

\item \emph{Polynomial control via geometric complexity (Theorem~B):}  
The remainder in the localized trace formula is bounded in terms of the geometric complexity $\kappa(\Gamma)$, a scale–law functional of area, curvature, and reach. This provides quantitative control across classes of interfaces.

\item \emph{Refined error terms and dynamical hypotheses (Theorem~C):}  
Under additional assumptions on mixing of the geodesic flow in $\Omega\setminus \Gamma$, power–saving bounds for the spectral counting error are obtained. These connect lithomathematics with the theory of quantum chaos and spectral dynamics.

\item \emph{Universality across ensembles (Theorem~D):}  
For random ensembles of fractures satisfying uniform $C^2$–control, the coefficient $a_\Gamma$ is almost surely universal, and the normalized invariant $K_L^{\mathrm{iso}}$ has deterministic expectation. This establishes statistical stability across models.
\end{enumerate}

In addition, the framework is extended to stratified spaces with strata of codimension $\geq 2$, to algebraic and combinatorial examples (finite groups, graphs), and to ergodic ensembles of hypersurfaces. Each extension is made explicit and is supported by uniform estimates.

\subsection*{10.3. Novelty relative to prior theories}
The following contrasts articulate the precise novelty of lithomathematics:

\begin{itemize}
\item \emph{Versus classical boundary terms (Weyl, Ivrii, Safarov–Vassiliev):}  
Lithomathematics identifies interior hypersurfaces as independent carriers of universal coefficients, of the same order as classical boundary contributions, a phenomenon not accounted for in boundary-only expansions.

\item \emph{Versus stratified spaces (Cheeger, Brüning):}  
Stratified theories treat singularities intrinsic to the ambient manifold; lithomathematics introduces smooth ambient manifolds perturbed by smooth internal fractures. The universal fracture coefficient depends solely on hypersurface volume, unlike angular–dependent coefficients in conic singularities.

\item \emph{Versus singular potential methods:}  
Approaches introducing delta–potentials along hypersurfaces depend on analytic parameters (coupling constants). Lithomathematics derives purely geometric invariants independent of analytic choices, ensuring universality.

\item \emph{Versus random matrix theory:}  
Universality in RMT is probabilistic and ensemble–dependent; lithomathematics identifies deterministic geometric universality, valid for each admissible hypersurface under $C^2$–control.
\end{itemize}

These contrasts situate the present work as a distinct and novel contribution, rather than a reformulation of existing frameworks.

\subsection*{10.4. Scope and non–objectives}
It is essential to delineate what this work does \emph{not} claim:

\begin{enumerate}[(a)]
\item \emph{No resolution of open conjectures:}  
The results here do not resolve the Riemann hypothesis, the Birch–Swinnerton–Dyer conjecture, or any Millennium Prize problem. Connections to analytic number theory are limited to heuristic outlooks in the concluding remarks, without key claims.

\item \emph{No claims beyond $C^2$–regularity:}  
Theorems require hypersurfaces of class at least $C^2$ with positive reach. Interfaces with corners, cusps, or self–intersections require different analytic tools and are deferred to subsequent work.

\item \emph{No uniqueness of functorial construction:}  
Examples mapping algebraic structures (groups, rings) to litho–domains are illustrative. No claim is made of canonical functoriality beyond the specific constructions presented.

\item \emph{No unverified dynamical hypotheses:}  
Results invoking exponential mixing of the geodesic flow are explicitly conditional (Theorem~C). The generality of such hypotheses is not addressed.
\end{enumerate}

Explicitly stating these non–objectives prevents misinterpretation and ensures the scope is transparent.

\subsection*{10.5. Outline of the monograph}
The work is organized as follows:

\begin{description}
\item[Chapter 1.] Historical context, motivation, and initial definitions of litho–domains, $\kappa(\Gamma)$, and $K_L$; statement of Theorems A–D. 
\item[Chapter 2.] Axiomatization of litho–domains and rigorous definition of $\kappa(\Gamma)$; structural properties of $\kappa$; formal definition of $K_L^{\mathrm{iso}}$. 
\item[Chapter 3.] Localized trace formula: construction of parametrices near $\Gamma$, proof of Theorem~A, and worked examples (intervals, rectangles). 
\item[Chapter 4.] Dynamical refinements: hypothesis $H_{\mathrm{mix}}$, proof of Theorem~C, and connections to quantum chaos. 
\item[Chapter 5.] Universality and ensembles: proof of Theorem~D; examples with Gaussian random fields and Poisson fractures. 
\item[Chapter 6.] Relations to classical spectral geometry: contrasts with Cheeger–type invariants and singular potential methods. 
\item[Chapter 7.] Algebraic and group–theoretic examples: constructions for finite groups and discrete structures. 
\item[Chapter 8.] Stratified spaces and higher–codimension strata; extension of localized formulas. 
\item[Chapter 9.] Open questions, conjectural extensions, and potential applications to number theory and probability. 
\item[Appendices A–Z.] Technical developments, historical notes, numerical verifications, and glossary of litho–terminology.
\end{description}

Each chapter concludes with a Closure section, ensuring that definitions, theorems, and discussions are harmonized and logically complete.

\subsection*{10.6. Closure}
This completes the introduction. We have motivated the subject, defined its objects and invariants, presented its universal theorems, contrasted it with prior frameworks, established its scope, and outlined the structure of the work. Every claim in the introduction corresponds to a theorem or proposition proved in the body under explicit conditions. With this preparation, the reader may proceed with confidence to the axiomatic foundations in Chapter~2.
