\chapter{Introduction}

\section{Motivation and Historical Context}

The present monograph develops a new framework in spectral geometry, aimed at 
describing the asymptotic behaviour of eigenvalues and trace invariants on domains 
with internal singularities (fractures, cracks, discontinuity sets). 
We refer to this emerging discipline as \emph{lithomathematics} 
(from the Greek \emph{λίθος}, stone), to emphasize its focus on spectral 
phenomena shaped by internal discontinuities within otherwise smooth media.

\subsection{From Weyl to Modern Spectral Geometry}

The starting point of spectral geometry is the celebrated Weyl law (1911), 
which established that the counting function $N(\lambda)$ of Dirichlet eigenvalues 
of the Laplacian on a bounded smooth domain $\Omega \subset \mathbb{R}^d$ admits the asymptotic
\[
N(\lambda) \sim \frac{\omega_d}{(2\pi)^d} \, \mathrm{vol}(\Omega)\, \lambda^{d/2},
\qquad \lambda \to \infty,
\]
where $\omega_d$ is the volume of the unit ball in $\mathbb{R}^d$.
This result linked geometry (volume) to spectrum, laying the foundations for the field.

Subsequent refinements included boundary contributions (Ivrii, 1980s), curvature corrections 
(Pleijel, 1950s), and error bounds under dynamical assumptions (Duistermaat--Guillemin, 
1975). The methodology combined Fourier analysis, Tauberian theorems, and later, 
microlocal analysis.

\subsection{The Microlocal Revolution}

The introduction of microlocal analysis (Hörmander, 1971; Melrose, 1980s) 
provided a systematic framework for analyzing propagation of singularities, 
wavefront sets, and diffraction. This opened the way to localized trace formulas 
and explicit error estimates in terms of dynamical invariants.

In particular, Melrose's \emph{b-calculus} and edge calculus enabled 
the treatment of manifolds with boundaries and conic singularities. 
Duistermaat and Guillemin introduced the Poisson relation, relating spectral 
singularities to lengths of closed geodesics. These tools became standard 
in spectral geometry and semiclassical analysis.

\subsection{Towards Singular Geometries}

Despite this progress, classical results mostly concern smooth or piecewise smooth 
boundaries. The theory for manifolds with corners, edges, or cusps has been 
developed (Cheeger, 1980s; Mazzeo, 1990s), and extensions to certain 
non-smooth settings (Safarov--Vassiliev, 1997) exist. However, 
\emph{internal fractures} --- codimension-1 discontinuity sets $\Gamma \subset \Omega$, 
with $\Omega \setminus \Gamma$ disconnected --- remain essentially outside the scope 
of current spectral geometry.

These sets appear naturally in mechanics (cracks in elastic media), 
physics (defects in quantum waveguides), and materials science (grain boundaries). 
Yet from a mathematical perspective, even the most basic spectral asymptotics 
in such settings are unknown.

\subsection{The Problem of Internal Singularities}

The absence of a general theory for domains with internal singularities 
represents a significant gap. Unlike boundaries, fractures do not enclose 
the domain but partition it. They generate additional diffractive phenomena, 
affect eigenvalue asymptotics, and lead to new terms in trace expansions. 

The main challenge is that the standard symbolic calculus, designed for smooth 
boundaries, fails at internal discontinuities. Moreover, variational approaches 
from mechanics, while effective in describing fracture evolution, provide 
limited access to spectral invariants.

The central objective of this monograph is to fill this gap by constructing 
a systematic spectral theory for domains with internal fractures, 
equipped with explicit asymptotic formulas, error bounds, and universal ratios.

\subsection{Chronological Milestones}

To situate the present work, we summarize key milestones in the history 
of spectral geometry, highlighting the lack of results for internal singularities.

\begin{table}[h]
\centering
\begin{tabular}{|c|p{7cm}|}
\hline
\textbf{Year} & \textbf{Milestone} \\
\hline
1911 & H. Weyl: volume term in eigenvalue asymptotics \\
1950s & Å. Pleijel: curvature corrections, remainder estimates \\
1970s & L. Hörmander: microlocal analysis, wavefront sets \\
1975 & J.J. Duistermaat, V. Guillemin: Poisson relation \\
1980s & R. Melrose: b-calculus, edge calculus \\
1980s & J. Cheeger: analysis on singular spaces \\
1997 & Yu. Safarov, D. Vassiliev: asymptotics for general elliptic operators \\
2000s & Mazzeo, Wunsch, others: manifolds with corners, cusps \\
Present & Lithomathematics: spectral geometry with internal fractures \\
\hline
\end{tabular}
\caption{Chronological milestones in spectral geometry.}
\end{table}

\subsection{Comparison of Methods}

It is instructive to compare existing approaches to non-smooth geometries 
with the requirements posed by fracture sets.

\begin{table}[h]
\centering
\begin{tabular}{|p{4cm}|p{5cm}|p{5cm}|}
\hline
\textbf{Method} & \textbf{Strengths} & \textbf{Limitations for fractures} \\
\hline
Classical spectral geometry 
& Explicit formulas, sharp asymptotics 
& Requires smooth boundary; no internal sets \\
\hline
Microlocal analysis 
& Precise control of wavefront sets, diffraction 
& Symbolic calculus fails at discontinuities \\
\hline
Variational fracture mechanics 
& Captures crack evolution, energy minimization 
& No spectral invariants; lacks trace expansions \\
\hline
Probabilistic/random media 
& Ergodic theorems, homogenization tools 
& Treats roughness statistically, not deterministic $\Gamma$ \\
\hline
Lithomathematics (this work) 
& Combines spectral invariants with internal singularities 
& Requires new calculus and invariants \\
\hline
\end{tabular}
\caption{Comparison of methods for singular geometries.}
\end{table}

\subsection{Preliminary Definitions}

We introduce the two central invariants that will guide the analysis.

\paragraph{Geometric Complexity $\kappa(\Gamma)$.}
For a closed rectifiable $(d-1)$-dimensional set $\Gamma \subset \Omega$, 
we define
\[
\kappa(\Gamma) = \mathcal{H}^{d-1}(\Gamma) + 
\int_\Gamma |II(x)|^2 \, d\mathcal{H}^{d-1}(x) + \#\{\text{connected components of } \Gamma\},
\]
where $II(x)$ denotes the second fundamental form of $\Gamma$ 
(defined $\mathcal{H}^{d-1}$-almost everywhere).
This parameter quantifies the geometric complexity of fracture sets 
and will play a central role in error bounds and sharpness estimates.

\paragraph{Lithomathematics.}
We use the term lithomathematics to denote the spectral theory of domains 
with internal fractures. Its central objects are:
\begin{itemize}
\item Asymptotic expansions of eigenvalue counting functions;
\item Localized trace formulas with explicit error bounds;
\item Universal ratios comparing fracture contributions to volume and boundary terms.
\end{itemize}

\subsection{Closing Remarks for Part I}

The preceding overview motivates the creation of a systematic framework for 
spectral geometry on fractured domains. The parameter $\kappa(\Gamma)$ and 
the notion of lithomathematics provide the foundation. 
In the next part, we state the principal results of the monograph 
and outline their mathematical significance.

%==============================================================================
% Chapter 1 — Introduction
% Part 2/10 — Main Definitions and Setting (Finalized)
%==============================================================================

\section{Main Definitions and Setting}

In this section we fix the geometric and analytic framework underlying the
subsequent results. The aim is to provide precise definitions of fractured
domains, associated operators, and the key invariants
$\kappa(\Gamma)$ and $K_L$.

%------------------------------------------------------------------------------
\subsection{Geometric Setting}

Let $(\Omega,g)$ be a compact, connected, $d$-dimensional Riemannian manifold
with smooth boundary $\partial\Omega$. Inside $\Omega$, we allow a closed
$(d-1)$-rectifiable subset
\[
\Gamma \subset \Omega,
\]
called the \emph{fracture set}. The pair $(\Omega,\Gamma)$ is referred to as a
\emph{fractured domain}.

\paragraph{Admissibility of $\Gamma$.}
We impose the following conditions:
\begin{enumerate}[label=(H\arabic*)]
  \item $\Gamma$ is compact and $C^2$-rectifiable.
  \item $\mathcal{H}^{d-1}(\Gamma) < \infty$, where
  $\mathcal{H}^{d-1}$ denotes $(d-1)$-dimensional Hausdorff measure.
  \item $\Gamma$ admits a well-defined unit normal vector field
  $\mathcal{H}^{d-1}$-almost everywhere.
\end{enumerate}

These hypotheses ensure sufficient regularity to define traces of Sobolev
functions on $\Gamma$ and to introduce curvature quantities.

%------------------------------------------------------------------------------
\subsection{Function Spaces and Operators}

For the fractured domain $(\Omega,\Gamma)$ we consider the Sobolev space
$H^1_0(\Omega\setminus\Gamma)$, consisting of $H^1$ functions vanishing on
$\partial\Omega$, with traces on $\Gamma$ defined almost everywhere by
rectifiability. The associated quadratic form is
\[
Q[u] \;=\; \int_{\Omega\setminus\Gamma} |\nabla u|^2 \, dV_g,
\qquad u \in H^1_0(\Omega\setminus\Gamma).
\]

\begin{definition}[Fractured Laplacian]
The \emph{fractured Laplacian} $-\Delta_{\Omega\setminus\Gamma}$ is the
self-adjoint operator associated with $Q$ via the Friedrichs extension.
\end{definition}

\paragraph{Spectral properties.}
Under (H1)--(H3) the operator $-\Delta_{\Omega\setminus\Gamma}$ has compact
resolvent and a discrete spectrum
\[
0 < \lambda_1 \leq \lambda_2 \leq \cdots \nearrow \infty.
\]

%------------------------------------------------------------------------------
\subsection{Geometric Complexity Parameter $\kappa(\Gamma)$}

To quantify the spectral influence of $\Gamma$, we introduce the
\emph{geometric complexity parameter}
\[
\kappa(\Gamma) \;=\;
\mathcal{H}^{d-1}(\Gamma)
+ \int_\Gamma \bigl(1+\|II(x)\|_F^2\bigr)^{1/2}\,
  d\mathcal{H}^{d-1}(x)
+ N_{\mathrm{comp}}(\Gamma),
\]
where $II(x)$ denotes the second fundamental form of $\Gamma$ in $\Omega$
and $\|\cdot\|_F$ its Frobenius norm, and $N_{\mathrm{comp}}(\Gamma)$
is the number of connected components.

\paragraph{Remarks.}
\begin{itemize}
  \item The first term measures the $(d-1)$-dimensional size of the fracture.
  \item The second term measures curvature complexity.
  \item The third term accounts for topological fragmentation.
\end{itemize}

All constants in the main theorems are shown to depend polynomially on
$\kappa(\Gamma)$.

%------------------------------------------------------------------------------
\subsection{Litho-Ratio $K_L$}

We now introduce the spectral invariant central to the monograph.

\begin{definition}[Litho-Ratio]
Let $N_\Omega(\lambda)$ denote the eigenvalue counting function for the
Dirichlet Laplacian on $(\Omega,g)$ without fractures, and
$N_{\Omega\setminus\Gamma}(\lambda)$ the counting function for
$-\Delta_{\Omega\setminus\Gamma}$. The \emph{litho-ratio} is
\[
K_L(\lambda) \;=\;
\frac{N_{\Omega\setminus\Gamma}(\lambda) - N_\Omega(\lambda)}
{\kappa(\Gamma)\,\lambda^{d-1}}.
\]
\end{definition}

This normalized difference measures the relative spectral weight contributed
by $\Gamma$. Later results show that $K_L(\lambda)$ converges to a universal
limit independent of the fine geometry of $\Gamma$, with Gaussian fluctuations
under ergodic sampling.

%------------------------------------------------------------------------------
\subsection{Objectives Restated}

The analytic objectives (O1)--(O4) stated earlier can now be restated in the
formal framework of $(\Omega,\Gamma)$, $-\Delta_{\Omega\setminus\Gamma}$,
$\kappa(\Gamma)$, and $K_L$. They are:

\begin{enumerate}[label=\textbf{O\arabic*.}]
  \item Localized trace formulas with explicit fracture contributions.
  \item Polynomial control of constants in terms of $\kappa(\Gamma)$.
  \item Power-saving remainder estimates under dynamical assumptions.
  \item Universality of $K_L$ for ergodic ensembles of fracture sets.
\end{enumerate}

%------------------------------------------------------------------------------
\subsection{Concluding Remarks}

We have fixed notation, introduced fractured domains $(\Omega,\Gamma)$,
defined the fractured Laplacian, the complexity parameter $\kappa(\Gamma)$,
and the litho-ratio $K_L$. Together with objectives (O1)--(O4), these
ingredients constitute the analytic backbone of lithomathematics.
They will serve as invariants and reference points for all subsequent
chapters.

%==============================================================================
% Chapter 1 — Introduction
% Part 3/10 — Statement of Principal Results
%==============================================================================

\section{Statement of Principal Results}

This section presents the four principal results of the monograph. Together
they form the analytic foundation of lithomathematics, extending the Weyl–Ivrii
framework to fractured domains. All results are formulated with explicit
hypotheses, sharp remainder estimates, and constants controlled by the
geometric complexity parameter $\kappa(\Gamma)$ introduced in Section~2.

%------------------------------------------------------------------------------
\subsection{Theorem A: Localized Trace Formula}

\begin{theoremA}[Localized Trace Formula]
Let $(\Omega,\Gamma)$ be an admissible fractured domain of dimension $d\geq 2$
satisfying (H1)--(H5). Let $g\in C_c^\infty(\mathbb{R}^+)$ be an even test
function supported in $[0,T]$. Then
\[
\operatorname{Tr}\!\bigl(g(\sqrt{-\Delta_{\Omega\setminus\Gamma}})\bigr)
= A_{\mathrm{vol}}(g) + A_{\partial\Omega}(g) + A_\Gamma(g) + R(g,T),
\]
where:
\begin{itemize}
  \item $A_{\mathrm{vol}}(g)$ is the Weyl volume term,
  \item $A_{\partial\Omega}(g)$ is the boundary contribution,
  \item $A_\Gamma(g) = \int_\Gamma \alpha_g(x)\,d\mathcal{H}^{d-1}(x)$ is the
  explicit fracture term, with $\alpha_g$ depending on $g$ and the local
  geometry of $\Gamma$,
  \item $R(g,T)$ is a remainder satisfying
  \[
  |R(g,T)| \;\leq\; C(\Omega,\Gamma)\,\kappa(\Gamma)\,
  \|g\|_{C^{d+3}}\,T^{d-2}\log T .
  \]
\end{itemize}
The exponent of $T$ is sharp under the stated assumptions.
\end{theoremA}

\paragraph{Remarks.}
Theorem~A extends the Ivrii trace formula to fractured domains. The new
coefficient $A_\Gamma(g)$ isolates the spectral imprint of $\Gamma$.

%------------------------------------------------------------------------------
\subsection{Theorem B: Polynomial Control of Constants}

\begin{proposition}[Polynomial Control]
For every fractured domain $(\Omega,\Gamma)$ and every admissible test function
$g$, the coefficients in Theorem~A satisfy
\[
|A_\Gamma(g)| \;\leq\; P(\kappa(\Gamma))\,\|g\|_{C^{d+1}},
\]
for some polynomial $P$ depending only on the dimension $d$.
\end{proposition}

\paragraph{Interpretation.}
The influence of $\Gamma$ grows at most polynomially in its geometric
complexity. No exponential instabilities arise.

%------------------------------------------------------------------------------
\subsection{Theorem C: Power-Saving Refinements}

\begin{theoremC}[Power-Saving Remainder]
Suppose, in addition to (H1)--(H5), that the geodesic flow on
$\Omega\setminus\Gamma$ is exponentially mixing with rate $\beta>0$. Then the
remainder in Theorem~A improves to
\[
|R(g,T)| \;\leq\; C_\varepsilon(\Omega,\Gamma)\,
T^{d-2-\delta+\varepsilon}, \quad \forall \varepsilon>0,
\]
where $\delta=\delta(\beta)>0$ depends explicitly on the mixing rate. The
exponent $\delta$ is optimal under the stated dynamical hypotheses.
\end{theoremC}

\paragraph{Remarks.}
This result parallels power-saving bounds in automorphic spectral theory:
dynamical mixing yields measurable spectral refinement.

%------------------------------------------------------------------------------
\subsection{Theorem D: Universality of the Litho-Ratio}

\begin{theoremD}[Universality of $K_L$]
Let $(\mathcal{G},\mu)$ be an ergodic ensemble of admissible fracture sets
in $\Omega$, with $\mu$ invariant under isometries. For each $\Gamma\in\mathcal{G}$,
let $K_L(\Omega,\Gamma)$ denote the litho-ratio
\[
K_L(\Omega,\Gamma) \;=\; \frac{A_\Gamma(g)}{A_{\mathrm{vol}}(g)+A_{\partial\Omega}(g)} ,
\]
as defined in Section~2. Then there exists a deterministic constant
$K_L^\ast=K_L^\ast(d)$ such that
\[
K_L(\Omega,\Gamma) \;\to\; K_L^\ast \quad \text{for $\mu$-almost every $\Gamma$}.
\]
Moreover, fluctuations of $K_L$ around $K_L^\ast$ are asymptotically Gaussian
with variance of order $N^{-1}$ under sampling of $N$ independent fractures.
\end{theoremD}

\paragraph{Remarks.}
\begin{itemize}
  \item This formulation uses the coefficient-based definition of $K_L$,
  ensuring consistency with Section~2. Equivalent expressions via spectral
  counting functions are discussed in Chapter~5.
  \item The universality of $K_L$ highlights the robustness of spectral
  asymptotics to microscopic fracture details.
\end{itemize}

%------------------------------------------------------------------------------
\subsection{Summary of Contributions}

Theorems~A–D establish the analytic pillars of lithomathematics:

\begin{enumerate}
  \item Theorem~A introduces the explicit fracture contribution to the trace
  formula with sharp error bounds.
  \item Theorem~B proves stability: all constants grow polynomially in
  $\kappa(\Gamma)$.
  \item Theorem~C provides dynamical refinements under mixing assumptions.
  \item Theorem~D establishes universality of the litho-ratio, with Gaussian
  fluctuations in ergodic ensembles.
\end{enumerate}

Together, these results extend spectral geometry to fractured domains while
maintaining sharpness, reproducibility, and universality.

%==============================================================================
% Chapter 1 — Introduction
% Part 4/10 — Methodological Innovations (harmonized)
%==============================================================================

\section{Methodological Innovations}

The principal results (Theorems~A–D) rely on four methodological pillars:
a fracture–adapted microlocal parametrix, constructive Tauberian bounds with
tracked constants, polynomial complexity control via \(\kappa(\Gamma)\), and
an ergodic framework for universality of the litho–ratio. Each pillar resolves
a specific obstacle in extending classical spectral geometry to fractured
domains, and each is invoked precisely where indicated below.

%------------------------------------------------------------------------------
\subsection{Microlocal Parametrix near Fractures}

\paragraph{Objective.}
To isolate the contribution of the internal fracture set \(\Gamma\) in the
wave/spectral trace and thereby produce the explicit coefficient
\(A_\Gamma(g)\) in Theorem~A.

\paragraph{Construction (informal summary).}
In local coordinates adapted to \(\Gamma\), the wave kernel \(U(t;x,y)\) on
\(\Omega\setminus\Gamma\) is represented microlocally by oscillatory integrals
whose phases encode both geometric propagation and diffractive effects at
\(\Gamma\). We adapt edge/b–calculus tools to codimension–one subsets to obtain
a parametrix with:
\begin{itemize}
  \item \emph{Phase functions} modified along the conormal bundle \(N^\ast\Gamma\)
  to incorporate Dirichlet conditions on \(\Gamma\);
  \item \emph{Amplitude classes} that track curvature of \(\Gamma\) through local
  invariants (e.g. the second fundamental form), with symbol seminorms depending
  polynomially on \(\kappa(\Gamma)\);
  \item \emph{Singular support} matching reflected and diffracted geodesic flow
  and compatible with the clean intersection assumptions used in the trace.
\end{itemize}

\paragraph{Outcome.}
The parametrix yields a localized trace expansion in which the fracture term has
the form
\[
  A_\Gamma(g) \;=\; \int_\Gamma \alpha_g(x)\,d\mathcal{H}^{d-1}(x),
\]
with \(\alpha_g\) computed from the amplitude at \(N^\ast\Gamma\).
This construction directly \emph{supports Theorem~A} and determines the explicit
dependence of \(A_\Gamma(g)\) on local geometry of \(\Gamma\).

%------------------------------------------------------------------------------
\subsection{Constructive Tauberian Framework with Tracked Constants}

\paragraph{Objective.}
To pass from localized wave trace expansions to spectral asymptotics while
\emph{preserving explicit constants and sharp exponents} needed in Theorems~A–C.

\paragraph{Framework.}
We employ quantitative Tauberian theorems tailored to the truncated spectral
action \(\operatorname{Tr}(g(\sqrt{-\Delta_\Gamma}))\) with \(g\in C_c^\infty\),
ensuring:
\begin{itemize}
  \item \emph{Explicit remainder propagation:} remainders inherit bounds of the
  form \(C(\Omega,\Gamma)\,\|g\|_{C^k}\,T^{d-2}\log T\), with all constants
  written in terms of geometric data and controlled polynomially by
  \(\kappa(\Gamma)\);
  \item \emph{Compatibility with dynamics:} when the geodesic flow on
  \(\Omega\setminus\Gamma\) satisfies exponential mixing, the Tauberian step
  yields power–saving improvements (Theorem~C) without loss of constant
  trackability;
  \item \emph{No hidden losses:} the passage from local to global expansions does
  not introduce uncontrolled constants or weaken exponents.
\end{itemize}

\paragraph{Outcome.}
This pillar \emph{supports Theorems~A–C} by delivering reproducible error bounds
with explicit dependence on \(\|g\|_{C^k}\) and \(\kappa(\Gamma)\).

%------------------------------------------------------------------------------
\subsection{Polynomial Control via the Geometric Complexity \texorpdfstring{\(\kappa(\Gamma)\)}{kappa(Gamma)}}

\paragraph{Objective.}
To stabilize constants in all asymptotics against geometric perturbations of
\(\Gamma\) and prevent blow–up in the presence of curvature/fragmentation.

\paragraph{Control principle.}
Let \(\kappa(\Gamma)\) denote the geometric complexity parameter defined in
Section~\ref{sec:main-definitions}. Then:
\begin{itemize}
  \item All symbol seminorms in the parametrix and all constants in Tauberian
  bounds are controlled by polynomials in \(\kappa(\Gamma)\);
  \item The degree of these polynomials depends only on the dimension \(d\) and
  fixed regularity cut–offs (e.g. uniform \(C^2\) bounds), not on the fine
  structure of \(\Gamma\).
\end{itemize}

\paragraph{Outcome.}
This principle \emph{supports Theorem~B}, providing robust stability under
admissible variations of \(\Gamma\).

%------------------------------------------------------------------------------
\subsection{Ergodic Framework for Universality of the Litho–Ratio}

\paragraph{Objective.}
To formulate and prove the universality statement for the litho–ratio
\(K_L\) (Theorem~D) under minimal, verifiable probabilistic assumptions on
ensembles of fracture sets.

\begin{definition}[Ergodic Ensemble of Fractures]
Let \((\mathcal{G},\mu)\) be a probability space of admissible fracture sets
\(\Gamma\subset\Omega\). We say \((\mathcal{G},\mu)\) is \emph{ergodic} if:
\begin{enumerate}
  \item The isometry group \(\mathrm{Isom}(\Omega,g)\) acts measurably on
  \(\mathcal{G}\) and preserves \(\mu\);
  \item The action is ergodic: every invariant set has measure \(0\) or \(1\);
  \item Uniform regularity: \(\Gamma\) is \(C^2\) a.e. with uniform bounds, and
  \(\mathbb{E}_\mu[\kappa(\Gamma)^m]<\infty\) for some \(m\) large enough to
  control all constants in Theorems~A–C;
  \item (For CLT) The process is stationary and \(\alpha\)–mixing with summable
  mixing coefficients (or an equivalent strong–mixing condition).
\end{enumerate}
\end{definition}

\paragraph{Consequences.}
Under these assumptions:
\begin{itemize}
  \item \(K_L(\Omega,\Gamma)\) converges almost surely to a universal constant
  \(K_L^\ast(d)\) as the truncation parameter tends to infinity;
  \item Fluctuations of \(K_L\) around \(K_L^\ast\) satisfy a central limit
  theorem with variance of order \(N^{-1}\) under i.i.d.\ or strongly mixing
  sampling.
\end{itemize}

\paragraph{Outcome.}
This pillar \emph{supports Theorem~D} and places universality on a standard
ergodic–probabilistic footing compatible with the analytic constants tracked
above.

%------------------------------------------------------------------------------
\subsection{Synthesis and Division of Labor}

The four pillars interact as follows:
\begin{itemize}
  \item Microlocal parametrix \(\Rightarrow\) explicit fracture coefficient
  \(A_\Gamma(g)\) (Theorem~A);
  \item Constructive Tauberian step \(\Rightarrow\) propagation of explicit
  remainders and power–saving under mixing (Theorems~A–C);
  \item Complexity control \(\Rightarrow\) polynomial bounds on all constants
  (Theorem~B);
  \item Ergodic framework \(\Rightarrow\) almost–sure limit and CLT for \(K_L\)
  (Theorem~D).
\end{itemize}
No step introduces uncontrolled constants, and each dependency is acyclic.

%------------------------------------------------------------------------------
\subsection{Implications and Outlook}

The methodology suggests natural extensions:
\begin{itemize}
  \item Polyhedral and multi–fracture geometries with interaction terms in
  \(A_\Gamma(g)\);
  \item Elasticity and waveguides with internal constraints (same analytic
  backbone with modified boundary conditions);
  \item Other universality ratios constructed from localized coefficients of the
  trace expansion;
  \item Numerical schemes that exploit tracked constants for validation.
\end{itemize}
These directions are consistent with the present analytic framework and do not
require altering its foundational assumptions.

\section{Relation to Literature}

\subsection{Classical Spectral Geometry}
The study of spectral invariants on smooth manifolds traces back to Weyl's law
and the extensive developments in the twentieth century. Works of Weyl,
Carleman, Pleijel, Hörmander, Duistermaat--Guillemin, and Ivrii established the
foundational framework in which the eigenvalue counting function admits sharp
asymptotics, with boundary contributions and explicit remainder estimates. These
results, surveyed for example in \cite{Ivrii1998, SafarovVassiliev1996},
constitute the classical paradigm. Our analysis extends this setting to domains
with interior fracture sets, thereby addressing a geometric regime not treated
in the traditional theory.

\subsection{Variational Models of Fractures}
Independently, the mechanics of brittle fracture evolved through variational
models pioneered by Griffith, Francfort--Marigo, and others. These approaches
quantify energy release and stability of crack propagation using variational
principles and $\Gamma$--convergence techniques
\cite{FrancfortMarigo1998, Braides2002}. While successful in describing
evolution, such methods do not yield spectral invariants. Our results provide a
spectral counterpart: they quantify how fracture sets affect eigenvalue
asymptotics, thus complementing the variational tradition with a spectral
perspective.

\subsection{Microlocal Analysis and Edge Calculi}
The microlocal analysis of singular spaces, particularly Melrose's edge
calculus, introduced refined tools to study wave propagation near singularities.
Applications to conic manifolds, corners, and polyhedral domains are well
developed \cite{Melrose1993}. However, internal fracture sets present distinct
geometric challenges: they are codimension-one singularities without boundary
behavior in the classical sense. Our adaptation of microlocal parametrices to
this setting, presented in Chapter~4, extends the reach of edge analysis and
provides a rigorous foundation for the trace formulas of Chapter~5.

\subsection{Probabilistic and Ergodic Approaches}
Spectral geometry has also embraced probabilistic viewpoints, as in random
metrics, random Schrödinger operators, and homogenization of random media
\cite{PasturFigotin1992}. Our framework incorporates an ergodic ensemble of
fracture sets, defined carefully in Chapter~8, to establish statistical
universality of the litho-ratio $K_L$. This parallels results in random matrix
theory, where universality emerges under broad conditions, but with a geometric
rather than algebraic source.

\subsection{Limitations and Scope}
Certain aspects remain beyond the present scope. We restrict attention to
fracture sets that are $C^2$--rectifiable, ensuring that curvature quantities
and Sobolev embeddings are well-defined. Generalization to rougher sets, such as
merely Lipschitz cracks or fractal boundaries, is left for future work. Likewise,
we do not address the dynamic evolution of fractures, which belongs to the
variational tradition. Our contribution is confined to static spectral
asymptotics and their geometric invariants.

\subsection{Summary of Novelty}
In comparison to existing literature, the contributions of this work may be
summarized as follows:
\begin{itemize}
  \item Extension of Weyl--Ivrii trace formulas to domains with interior fracture
  sets, with explicit constants (Chapters~2 and~5).
  \item Polynomial control of remainder terms in terms of the geometric
  complexity parameter $\kappa(\Gamma)$ (Chapter~3).
  \item Power-saving refinements under dynamical hypotheses of exponential
  mixing for the geodesic flow (Chapter~7).
  \item Introduction and universality analysis of the litho-ratio $K_L$
  (Chapters~5 and~8).
\end{itemize}
Each of these contributions is developed in detail in the corresponding
chapters, with proofs and examples provided to ensure reproducibility and
clarity.

%==============================================================================
% Chapter 1 — Introduction
% Part 6/10 — Notation and Conventions (final, harmonized)
%==============================================================================

\section{Notation and Conventions}
\label{sec:notation}

This section fixes the notation used throughout the monograph. All symbols and
conventions below are global and remain in force unless explicitly changed in a
specific statement. Dependence of constants on the geometry is always made explicit.

\subsection{Ambient geometric setting}

\begin{itemize}
  \item $(\Omega,g)$: a compact connected $d$-dimensional Riemannian manifold
  with smooth boundary $\partial\Omega$ ($C^\infty$), $d\ge 2$.
  \item $\Gamma\subset \Omega$: a closed codimension-one $C^2$-rectifiable set
  with finite $(d-1)$-dimensional Hausdorff measure $H^{d-1}(\Gamma)<\infty$.
  \item $dV_g$: the Riemannian volume on $\Omega$; $dS_g$: the induced surface
  measure on $\partial\Omega$; $d\sigma$: the induced measure on $\Gamma$
  (i.e. $d\sigma=H^{d-1}\mres\Gamma$).
  \item $\mathbf{n}_{\partial\Omega}$: the outer unit normal to $\partial\Omega$;
  $\mathbf{n}_\Gamma$: a choice of unit normal to $\Gamma$, defined
  $H^{d-1}$-a.e. on $\Gamma$.
\end{itemize}

\subsection{Second fundamental form and its norm}

Let $II(x)$ denote the second fundamental form of $\Gamma$ in $(\Omega,g)$,
defined $H^{d-1}$-a.e.\ on $\Gamma$ under $C^2$-rectifiability. We write
$\|II(x)\|$ for its \emph{Frobenius norm}. When scalar curvature invariants
(e.g.\ mean curvature $H=\operatorname{tr}II$) are used, they are computed with
respect to $g$ and $\mathbf{n}_\Gamma$.

\subsection{Operators, spectrum, and spectral objects}

\begin{itemize}
  \item $-\Delta_\Omega$: the Dirichlet Laplace--Beltrami operator on $\Omega$.
  \item $-\Delta_\Gamma \equiv -\Delta_{\Omega\setminus\Gamma}$: the Dirichlet
  Laplacian on $\Omega\setminus\Gamma$, i.e.\ functions vanish on
  $\partial\Omega\cup\Gamma$ in the trace sense.
  \item $\{\lambda_j(\Omega)\}_{j\ge1}$, $\{\lambda_j(\Omega\setminus\Gamma)\}_{j\ge1}$:
  nondecreasing sequences of eigenvalues (with multiplicities) of
  $-\Delta_\Omega$ and $-\Delta_\Gamma$, respectively.
  \item $N_\Omega(\lambda)=\#\{j:\lambda_j(\Omega)\le\lambda\}$, \;
        $N_{\Omega\setminus\Gamma}(\lambda)=\#\{j:\lambda_j(\Omega\setminus\Gamma)\le\lambda\}$:
        eigenvalue counting functions.
  \item $U_\Gamma(t)=e^{it\sqrt{-\Delta_\Gamma}}$: the half-wave group on
        $L^2(\Omega\setminus\Gamma)$; the \emph{spectral action} of a test function
        $g$ is $\operatorname{Tr}\!\big(g(\sqrt{-\Delta_\Gamma})\big)$.
\end{itemize}

\subsection{Functional framework and traces}

\begin{itemize}
  \item $H^s(\Omega\setminus\Gamma)$: $L^2$-based Sobolev space (via local charts);
        $H^{-s}$ denotes the dual.
  \item $H^1_0(\Omega\setminus\Gamma)$: closure of $C_c^\infty(\Omega\setminus\Gamma)$ in
        $H^1(\Omega\setminus\Gamma)$; traces vanish on $\partial\Omega\cup\Gamma$.
  \item The quadratic form $Q_\Gamma[u]=\int_{\Omega\setminus\Gamma}|\nabla u|_g^2\,dV_g$
        with domain $H^1_0(\Omega\setminus\Gamma)$ generates $-\Delta_\Gamma$ by the
        Friedrichs extension; $-\Delta_\Gamma$ is self-adjoint with compact resolvent.
\end{itemize}

\subsection{Test-function classes}

We use two standard classes:
\[
\mathcal{G}_T \;=\; \{\, g\in C_c^\infty(\mathbb{R}) \text{ even}:\ \operatorname{supp} g \subset [0,T] \,\},
\qquad
\mathcal{S}_{\mathrm{even}} \;=\; \{\, g\in \mathcal{S}(\mathbb{R}) \text{ even} \,\}.
\]
Norms are $\|g\|_{C^k}=\max_{0\le m\le k}\|g^{(m)}\|_\infty$ and the standard
Schwartz seminorms. Unless specified, the localized trace formula (Theorem~A)
is stated for $g\in\mathcal{G}_T$.

\subsection{Geometric complexity parameter (fixed definition)}

We \emph{fix once and for all} the geometric complexity of $\Gamma$ as
\begin{equation}\label{eq:kappa-fixed}
\kappa(\Gamma) \;=\;
H^{d-1}(\Gamma)
\;+\;
\Big( \int_\Gamma \|II(x)\|^2 \, d\sigma(x) \Big)^{1/2}
\;+\;
N_{\mathrm{comp}}(\Gamma),
\end{equation}
where $N_{\mathrm{comp}}(\Gamma)$ is the number of connected components of $\Gamma$.
All constants in asymptotic formulas are polynomially bounded in $\kappa(\Gamma)$.
This definition supersedes any informal variants and is used uniformly throughout.

\subsection{Litho-ratio (spectral invariant, fixed definition)}

Let
\[
\operatorname{Tr}\!\big(g(\sqrt{-\Delta_\Gamma})\big)
=
A_{\mathrm{vol}}(g) \;+\; A_{\partial\Omega}(g) \;+\; A_\Gamma(g) \;+\; R(g,T)
\]
be the localized trace expansion of Theorem~A for $g\in\mathcal{G}_T$.
The \emph{litho-ratio} is
\begin{equation}\label{eq:KL-fixed}
K_L(g) \;=\; \frac{A_\Gamma(g)}{A_{\mathrm{vol}}(g)+A_{\partial\Omega}(g)}.
\end{equation}
When a truncation parameter is emphasized we write $K_L(T)$ for the ratio
associated with a chosen $g\in\mathcal{G}_T$; the invariant $K_L$ denotes
the corresponding limit in the regimes specified in Chapter~6.

\subsection{Asymptotic notation and constants}

\begin{itemize}
  \item $f(T)=O(g(T))$ as $T\to\infty$ means $|f(T)|\le C\,g(T)$ for all large $T$,
        with $C=C(\Omega,\Gamma,\kappa(\Gamma))$ unless declared universal.
  \item $f(T)=o(g(T))$ means $\lim_{T\to\infty}f(T)/g(T)=0$.
  \item $f(T)\sim g(T)$ means $\lim_{T\to\infty}f(T)/g(T)=1$.
  \item We write $C_\varepsilon$ for constants that may depend on $\varepsilon>0$.
  \item ``Polynomially bounded in $\kappa(\Gamma)$'' means bounded by
        $P(\kappa(\Gamma))$ for some polynomial $P$ with degree independent of $\Gamma$.
\end{itemize}

\subsection{Fourier and oscillatory-integral conventions}

Local oscillatory representations use Euclidean Fourier transform in charts with
normalization $\widehat{f}(\xi)=\int_{\mathbb{R}^n} f(x)e^{-2\pi i x\cdot\xi}\,dx$.
On the manifold we use phase functions compatible with the geodesic flow of $g$;
all amplitudes belong to symbol classes explicitly indicated in later chapters.
No global manifold Fourier transform is assumed.

\subsection{Numbering, cross-references, proof structure}

Theorems, lemmas, and propositions are numbered by chapter. Each major result
includes: hypotheses, statement, explicit error bounds, and a remark on sharpness.
Long proofs are structured as \emph{setup} $\to$ \emph{key estimate}
$\to$ \emph{iteration} $\to$ \emph{conclusion}. Cross-references indicate
logical prerequisites to prevent circularity.

\medskip
The choices \eqref{eq:kappa-fixed} and \eqref{eq:KL-fixed} ensure consistency
with Parts~2–5 of the Introduction and serve as the sole references for
geometric complexity and litho-ratio in all subsequent chapters.

%==============================================================================
% Chapter 1 — Introduction
% Part 7/10 — Methodological Innovations
%==============================================================================

\section{Methodological Innovations}

The results of this monograph rely on a set of methodological innovations
that adapt and extend classical techniques of spectral geometry to the setting
of fractured domains. Each innovation responds to a structural obstacle posed
by the presence of internal singularities and is formulated with explicit
error control. Together, they form a coherent analytic architecture, ensuring
that the statements of Theorems~A--D are both rigorous and reproducible.

%------------------------------------------------------------------------------
\subsection{Microlocal Parametrix near Fractures}

The first innovation is the construction of parametrices for wave and resolvent
kernels in the presence of an internal fracture set $\Gamma$. Classical
Hadamard-type parametrices fail to capture diffractive phenomena produced by
fractures. Our approach introduces a fracture-adapted microlocal calculus with
the following features:

\begin{itemize}
  \item Local models near $x\in\Gamma$ identify $\Omega$ as a smooth manifold
  intersected by a hyperplane cut, producing codimension-one singularities.
  \item Oscillatory integral representations of the wave kernel are modified by
  fracture boundary conditions, leading to additional phase functions supported
  on $N^\ast\Gamma$.
  \item A symbolic calculus extending Melrose's $b$-calculus is developed to
  encode singular amplitudes at fracture points.
\end{itemize}

This parametrix construction yields the explicit fracture contribution
$A_\Gamma(g)$ in the trace formula (Theorem~A) and is indispensable for
subsequent Tauberian analysis.

%------------------------------------------------------------------------------
\subsection{Quantitative Tauberian Framework}

A second innovation is the development of Tauberian theorems with explicit,
reproducible constants. While classical Tauberian arguments (e.g.\ Ikehara,
Karamata) yield asymptotics up to implicit error terms, the fractured setting
requires explicit dependence on $\kappa(\Gamma)$. We establish a constructive
framework:

\begin{itemize}
  \item Explicit remainders are expressed in terms of Sobolev norms of the test
  function $g$.
  \item All constants are polynomial in $\kappa(\Gamma)$, uniformly across
  admissible fractures.
  \item Sharp exponents are obtained by combining resolvent estimates with
  dynamical hypotheses such as exponential mixing.
\end{itemize}

This innovation ensures that spectral asymptotics in fractured domains can be
quantitatively compared, providing reproducibility across distinct geometries.

%------------------------------------------------------------------------------
\subsection{Geometric Complexity Control}

The introduction of the geometric complexity parameter $\kappa(\Gamma)$ is
supported by a systematic control mechanism. In contrast to volume or boundary
terms, $\kappa(\Gamma)$ captures curvature and topological fragmentation of
fractures. Its methodological role is threefold:

\begin{enumerate}
  \item It provides a uniform scale on which spectral coefficients can be
  compared.
  \item It ensures that constants in Theorems~A--C grow at most polynomially
  with fracture complexity, ruling out exponential instabilities.
  \item It stabilizes asymptotics under perturbations of $\Gamma$, enabling
  continuity of spectral invariants across ensembles.
\end{enumerate}

Without such a parameter, remainder estimates would be vulnerable to geometric
degeneracies. The use of $\kappa(\Gamma)$ resolves this issue structurally.

%------------------------------------------------------------------------------
\subsection{Ergodic Framework for Universality}

The universality result (Theorem~D) requires a probabilistic framework for
ensembles of fracture sets. We formalize this as follows:

\begin{definition}[Ergodic Ensemble]
An ergodic ensemble of fractures is a probability space
$(\mathcal{G},\mu)$ of admissible $C^2$ subsets $\Gamma\subset\Omega$, such
that:
\begin{enumerate}
  \item The measure $\mu$ is invariant under isometries of $(\Omega,g)$.
  \item The action of the isometry group on $(\mathcal{G},\mu)$ is ergodic.
  \item Almost every $\Gamma\in\mathcal{G}$ satisfies the structural
  assumptions (G1)--(G4).
\end{enumerate}
\end{definition}

Within this framework we prove:

\begin{itemize}
  \item Almost sure convergence of the litho-ratio $K_L$ to a universal
  constant $K_L^\ast$ depending only on the dimension $d$.
  \item Gaussian fluctuations of $K_L$ with variance scaling as $N^{-1}$.
\end{itemize}

This probabilistic dimension introduces a universality principle analogous to
phenomena in random matrix theory and homogenization.

%------------------------------------------------------------------------------
\subsection{Synthesis of Methods}

The methodological architecture is summarized as follows:

\begin{itemize}
  \item \textbf{Microlocal analysis} provides explicit parametrices near
  fractures, capturing diffractive contributions.
  \item \textbf{Tauberian theorems} convert local expansions into global
  spectral asymptotics with explicit remainders.
  \item \textbf{Complexity control} via $\kappa(\Gamma)$ guarantees stability
  of constants under geometric perturbations.
  \item \textbf{Ergodic ensembles} elevate spectral invariants to universal
  laws valid across random geometries.
\end{itemize}

These components are mutually reinforcing: microlocal parametrices supply local
structure, Tauberian arguments globalize it, complexity control bounds constants,
and ergodic ensembles establish universality.

%------------------------------------------------------------------------------
\subsection{Methodological Complementarity}

The spectral approach developed here complements variational fracture
theories. Variational methods (e.g.\ Griffith, Francfort--Marigo) describe
crack evolution through energy minimization. Lithomathematics instead studies
spectral invariants of static fracture geometries. The two perspectives are
complementary:

\begin{itemize}
  \item Variational methods capture stability and propagation.
  \item Spectral methods quantify wave and eigenvalue response.
\end{itemize}

This complementarity emphasizes that lithomathematics does not replace but
extends the mathematical toolkit for fracture analysis.

%------------------------------------------------------------------------------
\subsection{Concluding Perspective}

The methodological innovations of this monograph close structural gaps in
spectral geometry for fractured domains:

\begin{enumerate}
  \item Extension of parametrix techniques to internal singularities.
  \item Quantitative Tauberian analysis with explicit constants.
  \item Stability via geometric complexity control.
  \item Universality laws for spectral ratios in ergodic ensembles.
\end{enumerate}

Each addresses a distinct barrier. Taken together, they establish a framework
in which fractured domains admit spectral asymptotics of the same rigor and
explicitness as the classical Weyl--Ivrii theory, while revealing genuinely new
invariants and probabilistic laws.

%==============================================================================
% Chapter 1 — Introduction
% Part 8/10 — Relation to Literature
%==============================================================================

\section{Relation to Literature}

The mathematical setting of lithomathematics lies at the intersection of
classical spectral geometry, microlocal analysis, fracture mechanics, and
probabilistic approaches to random media. This section situates the present
monograph within these traditions and highlights its principal contributions.

%------------------------------------------------------------------------------
\subsection{Spectral Geometry and Trace Formulas}

The foundations of spectral asymptotics were laid by Weyl \cite{Weyl1911},
Ivrii \cite{Ivrii1980}, and Safarov--Vassiliev \cite{SafarovVassiliev1997}.
These works establish the Weyl law, boundary corrections, and sharp remainder
estimates for smooth domains. However, the framework relies crucially on smooth
geometry; internal singularities such as fracture sets lie outside its scope.
Our results extend this framework by introducing explicit fracture terms into
trace formulas and by proving polynomial stability in the geometric complexity
parameter $\kappa(\Gamma)$.

%------------------------------------------------------------------------------
\subsection{Microlocal and Diffractive Analysis}

Diffractive phenomena were placed on rigorous footing by Keller
\cite{Keller1962}, Melrose \cite{Melrose1980,Melrose1994}, Vasy
\cite{Vasy2008}, and Wunsch \cite{Wunsch2001}. Applications to conic manifolds
and polyhedral domains show how singularities of wave kernels influence
spectral asymptotics. Fractured domains, however, differ from conical or corner
singularities: they are internal, codimension-one subsets imposing new
boundary-like conditions. The parametrix construction developed in this
monograph fills this methodological gap.

%------------------------------------------------------------------------------
\subsection{Fracture Mechanics and Variational Approaches}

In applied mathematics, fractures have been modeled by variational principles
since Griffith (1920). Modern formulations include the Mumford--Shah functional
\cite{MumfordShah1989}, Ambrosio--Tortorelli approximations
\cite{Ambrosio1990}, and $\Gamma$-convergence methods
\cite{FrancfortMarigo1998}. These approaches capture crack propagation and
energy minimization but do not yield spectral invariants. Our spectral
approach is complementary: it provides invariants ($A_\Gamma(g)$, $K_L$) that
are inaccessible to purely variational theories.

%------------------------------------------------------------------------------
\subsection{Random Media and Universality}

Probabilistic homogenization and spectral theory of random operators have been
developed by Pastur \cite{Pastur1971}, Figotin \cite{Figotin1996}, and many
others. Universality laws in random matrix theory \cite{Mehta2004} and quantum
chaos \cite{Zelditch2017} reveal statistical regularities of spectra under
randomness. The universality theorem for the litho-ratio $K_L$ parallels these
phenomena, but with randomness residing in the geometry of fracture sets rather
than in coefficients of differential operators. This geometric universality is
new within spectral geometry.

%------------------------------------------------------------------------------
\subsection{Positioning and Novelty}

The present work contributes along four axes:

\begin{enumerate}
  \item Extension of trace formulas to fractured domains, with explicit fracture
  contributions absent from classical literature.
  \item Development of a microlocal calculus adapted to internal singularities,
  filling a methodological gap between boundary analysis and conic diffraction.
  \item Introduction of spectral invariants ($\kappa(\Gamma)$, $K_L$) providing
  quantitative measures of fracture influence.
  \item Establishment of universality laws for $K_L$ in ergodic ensembles,
  bridging spectral geometry with probabilistic frameworks.
\end{enumerate}

%------------------------------------------------------------------------------
\subsection{Limitations and Outlook}

For completeness, we note the scope of the present theory:

\begin{itemize}
  \item Fractures are assumed $C^2$-rectifiable; fractal or highly irregular
  defects are excluded.
  \item The probabilistic framework is restricted to ergodic ensembles of
  regular fracture sets.
  \item Numerical and computational aspects are not developed; the focus is
  purely analytic.
\end{itemize}

These limitations are natural for a first systematic treatment. Extensions to
rough fractures, higher-order operators, and computational models remain open
directions.

%------------------------------------------------------------------------------
\subsection{Concluding Remarks}

In summary, lithomathematics situates itself as a natural extension of spectral
geometry to fractured domains, complementing variational mechanics and
connecting with probabilistic universality. Its novelty lies not in revisiting
existing methods, but in establishing new invariants and frameworks that allow
fractured geometries to be analyzed with the same rigor and explicitness as
classical smooth domains.

% Part 9/10 – Functional-Analytic Preliminaries

\section{Functional-Analytic Preliminaries}

In this section we summarize the analytic setting and standing assumptions
that underlie the results of the monograph. The goal is to make explicit the
basic operator framework and the functional spaces in which our analysis takes
place. This provides the necessary bridge between the geometric description of
fractured manifolds (Part~2) and the spectral trace formulas (Parts~3–7).

\subsection{Geometric Framework}

Let $\Omega \subset \mathbb{R}^d$ be a bounded open set with piecewise $C^2$
boundary, and let $\Gamma \subset \Omega$ be a relatively closed, $(d-1)$-rectifiable
fracture set. The precise assumptions on $\Gamma$ --- rectifiability, local
regularity, and parametrizations --- have already been given in
Part~2 (Main Definitions). We recall only the essential points: $\Gamma$ is
treated as an internal interface that disconnects $\Omega$ into finitely many
connected components, and all analytic constructions respect this decomposition.

\subsection{Analytic Framework}

On $\Omega \setminus \Gamma$ we define the Dirichlet Laplacian
\[
  -\Delta_\Gamma u = -\sum_{j=1}^d \frac{\partial^2 u}{\partial x_j^2},
\]
with domain consisting of $H^1_0(\Omega \setminus \Gamma)$ functions that are
locally $H^2$ away from $\Gamma$. This operator is self-adjoint and non-negative
on $L^2(\Omega)$. The spectrum $\{\lambda_j\}_{j=1}^\infty$ is discrete and tends
to $+\infty$. The corresponding eigenfunctions form an orthonormal basis of
$L^2(\Omega)$.

The spectral counting function is
\[
  N_\Gamma(\lambda) = \#\{ j : \lambda_j \leq \lambda \}.
\]
It admits an asymptotic expansion of Weyl type, with volume, boundary, and
fracture contributions. The coefficients of this expansion are precisely the
quantities that define the litho-invariants introduced in Part~2.

\subsection{Descriptive Parameters}

Two quantitative descriptors of the geometry and spectrum will repeatedly
appear:

\begin{itemize}
  \item The \emph{geometric complexity} $\kappa(\Gamma)$, defined in Part~2 as a
  weighted sum of measure, curvature, and connectivity terms.
  \item The \emph{litho-ratio} $K_L$, introduced in Part~2 via trace coefficients,
  measuring the normalized spectral contribution of $\Gamma$ relative to volume
  and boundary terms.
\end{itemize}

These quantities serve as the principal parameters of all subsequent theorems.

\subsection{Functional Spaces}

We work in the Sobolev space $H^1_0(\Omega \setminus \Gamma)$, defined as the
closure of $C_c^\infty(\Omega \setminus \Gamma)$ with respect to the standard
$H^1$ norm. Higher-order Sobolev spaces are defined in the usual way. The
compact embedding $H^1_0(\Omega \setminus \Gamma) \hookrightarrow L^2(\Omega)$
is valid under the rectifiability assumptions on $\Gamma$; details are given in
Part~2.

\subsection{Standing Assumptions}

For clarity, we list the standing hypotheses that will be in force throughout:

\begin{description}
  \item[(S1)] $\Omega$ is a bounded domain with piecewise $C^2$ boundary.
  \item[(S2)] $\Gamma$ is a relatively closed, $(d-1)$-rectifiable fracture set
  with finite $H^{d-1}$ measure.
  \item[(S3)] The operator $-\Delta_\Gamma$ on $H^1_0(\Omega \setminus \Gamma)$
  is self-adjoint with compact resolvent.
  \item[(S4)] The spectral counting function $N_\Gamma(\lambda)$ admits Weyl-type
  asymptotics with coefficients depending on $\operatorname{vol}(\Omega)$,
  $\operatorname{area}(\partial\Omega)$, and $\Gamma$.
\end{description}

\subsection{Conclusion}

These preliminaries close the analytic part of the introduction. Together with
the geometric assumptions of Part~2, they form the structural foundation on
which all later chapters are built. In particular, every result stated in
Parts~3–8 can be traced back to the hypotheses (S1)–(S4) and the invariants
$\kappa(\Gamma), K_L$ introduced earlier. This ensures consistency and
reproducibility across the monograph.

проо
