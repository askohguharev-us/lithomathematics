%==============================================================================
% Chapter 00: Preface and Orientation
%==============================================================================

\chapter*{Preface and Orientation}
\label{chap:preface}

\section*{Orientation}

This monograph develops the mathematical foundations of
\emph{lithomathematics}, a framework that extends classical spectral
geometry to domains with internal singularities such as fractures or
rectifiable sets of codimension one. 

The central motivation is that traditional trace formulas, developed for
smooth manifolds with or without boundary, do not directly apply to
domains containing internal discontinuities. In fracture mechanics,
variational models (e.g.~Bourdin–Francfort–Marigo) capture energy
minimization, but spectral invariants remain unexplored. This work
establishes a coherent analytic and microlocal theory addressing this
gap.

\section*{Scope and Objectives}

The objectives of this work are:

\begin{enumerate}
    \item To construct localized and global trace formulas on domains with
    fractures, extending Weyl’s law and Ivrii’s boundary asymptotics.
    \item To define explicit coefficients $a_\Gamma(g)$ accounting for
    fracture contributions.
    \item To introduce a geometric complexity parameter $\kappa(\Gamma)$
    that quantifies spectral effects of internal discontinuities.
    \item To establish quantitative remainder bounds with sharp exponents
    under dynamical mixing conditions.
    \item To generalize the theory to nonlinear operators, stochastic
    ensembles, and multiscale homogenization.
    \item To prove ergodic limit theorems demonstrating universality of
    the litho-ratio $K_L$.
\end{enumerate}

\section*{Position in the Literature}

This work builds on several classical strands:

\begin{itemize}
    \item Weyl (1911): spectral asymptotics for smooth compact domains.
    \item Ivrii (1980): boundary contributions under regularity conditions.
    \item Safarov–Vassiliev (1997): microlocal analysis of spectral
    asymptotics.
    \item Bourdin–Francfort–Marigo (2000s): variational approaches to
    fracture mechanics.
\end{itemize}

The novelty of lithomathematics lies in integrating spectral invariants
with the geometry of fractures, producing universal limit laws that
remain valid beyond the smooth or variational settings.

\section*{Methodology}

Throughout, we employ:

\begin{itemize}
    \item Microlocal parametrix construction adapted to rectifiable
    singularities.
    \item Stationary phase analysis with curvature corrections.
    \item Ergodic and probabilistic techniques for random fracture
    ensembles.
    \item Homogenization and $\Gamma$-convergence for multiscale stability.
\end{itemize}

All results are stated with explicit constants, quantitative error
estimates, and sharpness barriers, in compliance with the
\emph{Diamond Standard v3.0} for mathematical rigor and reproducibility.

%==============================================================================
% End of Chapter 00
%==============================================================================
